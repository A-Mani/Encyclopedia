
\preface

The \textbf{Encyclopedia of Proof Systems} aims at providing a reliable, technically accurate, historically informative, concise and convenient central repository of proof systems for various logics. The goal is to facilitate the exchange of information among logicians with mathematical, computational or philosophical backgrounds; in order to foster and accelerate the development of new proof systems and automated deduction tools that rely on them.

Preparatory work for the creation of the Encyclopedia, such as the implementation of the LaTeX template and the setup of the Github repository, started in October 2014, triggered by the call for workshop proposals for the 25th Conference on Automated Deduction (CADE). Christoph Benzm\"uller, CADE's conference chair, and Jasmin Blanchette, CADE's workshop co-chair, encouraged me to submit a workshop proposal and supported my alternative idea to organize instead a special poster session based on encyclopedia entries. I am thankful for their encouragement and support.

In December 2014, Bj\"orn Lellmann, Giselle Reis and Martin Riener kindly accepted my request to beta-test the template and the instructions I had created. They submitted the first few example entries to the encyclopedia and provided valuable feedback, for which I am grateful. Their comments were essential for improving the templates and instructions before the public announcement of the encyclopedia.

Discussions with Lev Beklemishev, Bj\"orn Lellmann, Roman Kuznets, Sergei Soloviev and Anna Zamansky brainstormed many ideas for improving the organization and structure of the encyclopedia. Many of these ideas still need to be fully implemented.



\vspace{\baselineskip}
\begin{flushright}\noindent
May 2015\hfill {\it Bruno Woltzenlogel Paleo}
\end{flushright}


