
% If the calculus has an acronym, define it.
% (e.g. \newcommand{\LK}{\ensuremath{\mathbf{LK}}\xspace})
%\newcommand{\ND}{\ensuremath{\mathbf{ND}}\xspace} % already defined in GentzenLK.tex
%\newcommand{\NJ}{\ensuremath{\mathbf{NJ}}\xspace} 

\calculusName{Multi-conclusion Intuitionistic Sequent Calculus}   % The name of the calculus
\calculusAcronym{\textbf{LJ'}}    % The acronym if defined above, or empty otherwise. 
\calculusLogic{Intuitionistic Predicate Logic}
\calculusType{Sequent Calculus}
\calculusYear{1954}   % The year when the calculus was invented.
\calculusAuthor{Sh\^{o}ji Maehara} % The name(s) of the author(s) of the calculus.

\entryTitle{Multi-conclusion Intuitionistic Sequent Calculus \textbf{LJ'}}
\entryAuthor{Giselle Reis}    % Your name(s). Separate multiple names with "\and"

\maketitle


% If your files are called "<ID>.tex" and "<ID>.bib", 
% then you should write "\begin{entry}{<ID>}" in the line below
\begin{entry}{MultiConclusionLJ}  

% Define here any newcommands you may need:
% e.g. \newcommand{\necessarily}{\Box}
% e.g. \newcommand{\possibly}{\Diamond}

\begin{calculus}

% Add the inference rules of your proof system here.
% The "proof.sty" and "bussproofs.sty" packages are available.
% If you need any other package, please contact the editor (bruno@logic.at)
\[
\begin{array}{cc}
\infer[init]{A \rightarrow A}{}
&
\infer[cut]{\Gamma, \Delta \rightarrow \Theta, \Lambda}{\Gamma \rightarrow
\Theta, A & A, \Delta \rightarrow \Lambda}
\\[7pt]
\infer[\wedge_{li}]{A_1 \wedge A_2, \Gamma \rightarrow \Theta}{A_i, \Gamma \rightarrow \Theta}
&
\infer[\wedge_r]{\Gamma \rightarrow \Theta, A \wedge B}{\Gamma \rightarrow
\Theta, A & \Gamma \rightarrow \Theta, B}
\\[7pt]
\infer[\vee_l]{A \vee B, \Gamma \rightarrow \Theta}{A, \Gamma \rightarrow \Theta & B, \Gamma \rightarrow \Theta}
&
\infer[\vee_{ri}]{\Gamma \rightarrow \Theta, A_1 \vee A_2}{\Gamma \rightarrow
\Theta, A_i}
\\[7pt]
\infer[\supset_l]{A \supset B, \Gamma, \Delta \rightarrow \Theta, \Lambda}{\Gamma
\rightarrow \Theta, A & B, \Delta \rightarrow \Lambda}
&
\infer[\supset_r]{\Gamma \rightarrow A \supset B}{A, \Gamma \rightarrow B}
\\
\end{array}
\]
\[
\begin{array}{cccc}
\infer[\exists_l]{\exists x.Ax, \Gamma \rightarrow \Theta}{A\alpha, \Gamma \rightarrow \Theta}
&
\infer[\exists_r]{\Gamma \rightarrow \Theta, \exists x.Ax}{\Gamma \rightarrow
\Theta, At}
&
\infer[\forall_l]{\forall x.Ax, \Gamma \rightarrow \Theta}{At, \Gamma \rightarrow \Theta}
&
\infer[\forall_r]{\Gamma \rightarrow \forall x.Ax}{\Gamma \rightarrow A\alpha}
\\[7pt]
\infer[\neg_l]{\neg A, \Gamma \rightarrow \Theta}{\Gamma \rightarrow \Theta, A}
&
\infer[\neg_r]{\Gamma \rightarrow \neg A}{A, \Gamma \rightarrow }
&
\infer[e_l]{\Gamma, A, B, \Delta \rightarrow \Theta}{\Gamma, B, A, \Delta \rightarrow \Theta}
&
\infer[e_r]{\Gamma \rightarrow \Theta, A, B, \Lambda}{\Gamma \Delta \rightarrow
\Theta, B, A, \Lambda}
\\[7pt]
\infer[c_l]{A, \Gamma \rightarrow \Theta}{A, A, \Gamma \rightarrow \Theta}
&
\infer[c_r]{\Gamma \rightarrow \Theta, A}{\Gamma \rightarrow \Theta, A, A}
&
\infer[w_l]{A, \Gamma \rightarrow \Theta}{\Gamma \rightarrow \Theta}
&
\infer[w_r]{\Gamma \rightarrow A}{\Gamma \rightarrow}
\\
\end{array}
\]
\end{calculus}

% The following environments ("clarifications", "history", 
% "technicalities") are optional. If you do use them, 
% be very concise and objective.

\begin{clarifications}
% ToDo: write here short remarks that may help the reader to understand 
% the inference rules of the proof system.
In all rules, $A$, $A_i$ and $B$ are arbitrary formulas. In rules $\exists_l$
and $\forall_r$, $\alpha$ is a variable not contained in $A$, $\Gamma$ or
$\Theta$. In rules $\exists_r$ and $\forall_l$, $t$ does not contain variables
bound in $A$.
It is common to consider \textbf{LJ'} without the exchange rules $e_l$ and $e_r$ just by
interpreting $\Gamma$ and $\Theta$ as multi-sets of formulas instead of lists.
While \textbf{LJ}\iref{GentzenLJ} is defined by restricting all rules of
\textbf{LK}\iref{GentzenLK} to single conclusion, in \textbf{LJ'} this
restriction is only for the rules $\neg_r$, $\supset_r$ and $\forall_r$.
\end{clarifications}

\begin{history}
% ToDo: write here short historical remarks about this proof system,
% especially if they relate to other proof systems. 
% Use "\iref{OtherProofSystem}" to refer to another proof system 
% in the Encyclopedia (where "OtherProofSystem" is its ID). 
% Use "\irefmissing{SuggestedIDForOtherProofSystem}" to refer to 
% another proof system that is not yet available in the encyclopedia.
Proposed by Maehara in \cite{Maehara1954} and used to prove the completeness of
\textbf{LJ}\iref{GentzenLJ} in \cite{Takeuti1987}. The same calculus (and other
multi-conclusion calculi for intuitionistic logic) is rediscovered in
\cite{Nadathur1998} while analysing classical and intuitionistic provability.
\end{history}

\begin{technicalities}
% ToDo: write here remarks about soundness, completeness, decidability...
The equivalence of \textbf{LJ'} to \textbf{LJ} is established by translating
sequents $\Gamma \vdash A_1, ..., A_n$ into $\Gamma \vdash A_1 \vee ... \vee
A_n$. Cut is admissible in this calculus via a combination of the rewriting
rules for cut-elimination in \textbf{LJ} and \textbf{LK}.
\end{technicalities}



% Please cite the original paper where the proof system was defined.
% To do so, you may use the \cite command within 
% one of the optional environments above,
% or use the \nocite command otherwise.

% You may also cite a modern paper or book where the 
% proof system is explained in greater depth or clarity.
% Cite parsimoniously.

% Do not cite related work. Instead, use the "\iref" or "\irefmissing" 
% commands to make an internal reference to another entry, 
% as explained within the "history" environment above.

% You do not need to create the "References" section yourself. 
% This is done automatically.

\end{entry}
