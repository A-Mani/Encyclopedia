
\newcommand{\dest}[4]{\mathsf{dest}~#1~\mathsf{as}~(#2,#3)~\mathsf{in}~#4}
\newcommand{\Case}[5]{\mathsf{case}~#1~\mathsf{of}~[#2\,\Rightarrow\,#3~|~#4\,\Rightarrow\,#5]}
\newcommand{\efq}[1]{\texttt{efq}~{#1}}
\newcommand{\lmu}{\lambda\mu}

\calculusName{LambdaMu} % The name of the calculus
\calculusAcronym{$\lmu$} % The acronym if defined above, or empty otherwise.
\calculusLogic{Classical} % Specify the logic (e.g. classical, intuitionistic, ...) for which this calculus is intended.
\calculusType{natural deduction} % Specify the calculus type (e.g. Frege-Hilbert style, tableau, sequent calculus, hypersequent calculus, natural deduction, ...)
\calculusYear{1992} % The year when the calculus was invented.
\calculusAuthor{Parigot} % The name(s) of the author(s) of the calculus.
\entryTitle{Classical Natural Deduction ($\lmu$-calculus)} % Title of the entry (usually coincides with the name of the calculus).
\entryAuthor{Hugo Herbelin} % Your name(s). Separate multiple names with "\and".
% If you wish, use tags to give any other information
% that might be helpful for classifying and grouping this entry:
\tag{Multi-Conclusion Sequents}
% e.g. \tag{Multiset Cedents}
%\tag{List Cedents}
\tag{Term-annotated}
% You are free to invent your own tags.
% The Encyclopedia's coordinator will take care of
% merging semantically similar tags in the future.
\maketitle
% If your files are called "MyProofSystem.tex" and "MyProofSystem.bib",
% then you should write "\begin{entry}{MyProofSystem}" in the line below
\begin{entry}{LambdaMu}
% Define here any newcommands you may need:
% e.g. \newcommand{\necessarily}{\Box}
% e.g. \newcommand{\possibly}{\Diamond}
\begin{calculus}
% Add the inference rules of your proof system here.
% The "proof.sty" and "bussproofs.sty" packages are available.
% If you need any other package, please cpontact the editor (bruno@logic.at)

{\sc Structural subsystem}
\[
\infer[\mathit{Ax}]
      {a : \Gamma \vdash A ~|~ \Delta}
      {A^a \in \Gamma}
\]
\[
\infer[\mathit{Focus}]
      {\mu\alpha.c : \Gamma \vdash A ~|~ \Delta}
      {c : \Gamma \vdash A^\alpha, \Delta}
\qquad
\infer[\mathit{Unfocus}]
      {[\alpha]p : \Gamma \vdash \Delta}
      {p : \Gamma \vdash A ~|~ \Delta \qquad A^\alpha \in \Delta}
\]
{\sc Introduction rules}
\[
\infer[\wedge_E^i]
      {\pi_1(p) : \Gamma \vdash A_i ~|~ \Delta}
      {p : \Gamma \vdash A_1 \wedge A_2 ~|~ \Delta}
\qquad
\infer[\wedge_I]
      {(p_1,p_2) : \Gamma \vdash A_1 \wedge A_2 ~|~ \Delta}
      {p_1 : \Gamma \vdash A_1 ~|~ \Delta & p_2 : \Gamma \vdash A_2 ~|~ \Delta}
\]
\[
\infer[\vee_E]
      {\Case{p}{a_1}{p_1}{a_2}{p_2} : \Gamma \vdash C ~|~ \Delta}
      {p : \Gamma \vdash A_1 \vee A_2 ~|~ \Delta & p_1 : \Gamma, A_1^{a_1} \vdash C ~|~ \Delta & p_2 : \Gamma, A_2^{a_2} \vdash C ~|~ \Delta}
\]
\[
\infer[\vee_I^i]
      {\iota_i(q) : \Gamma \vdash A_1 \vee A_2 ~|~ \Delta}
      {q : \Gamma \vdash A_i ~|~ \Delta}
\]
\[
\infer[\rightarrow_E]
      {p\,q: \Gamma \vdash B ~|~ \Delta}
      {p : \Gamma \vdash A\rightarrow B ~|~ \Delta & q : \Gamma \vdash A ~|~ \Delta}
\qquad
\infer[\rightarrow_I]
      {\lambda a.p : \Gamma \vdash A \rightarrow B ~|~\Delta}
      {p : \Gamma, A^a \vdash B ~|~ \Delta}
\]
\[
\infer[\exists_E]
      {\dest{p}{y}{a}{q} : \Gamma \vdash C ~|~ \Delta}
      {p : \Gamma \vdash \exists x\,A ~|~ \Delta & q : \Gamma, A[y/x]^a \vdash C~|~ \Delta}
\qquad
\infer[\exists_I]
      {(t,p) : \Gamma \vdash \exists x\,A ~|~ \Delta}
      {p :  \Gamma \vdash A[t/x] ~|~ \Delta}
\]
\[
\infer[\forall_E]
      {p\,t : \Gamma \vdash A[t/x] ~|~ \Delta}
      {p: \Gamma \vdash \forall x\, A ~|~ \Delta}
\qquad
\infer[\forall_I]
      {\lambda y.p : \Gamma \vdash \forall x\,A ~|~ \Delta}
      {p : \Gamma \vdash A[y/x] ~|~ \Delta}
\]
\[
\infer[\bot_E]
      {\efq{p} : \Gamma \vdash C ~|~ \Delta}
      {p : \Gamma \vdash \bot ~|~ \Delta}
\qquad
\infer[\top_I]
      {\Gamma \vdash () : \top ~|~ \Delta}
      {}
\]

\end{calculus}
% The following sections ("clarifications", "history",
% "technicalities") are optional. If you use them,
% be very concise and objective. Nevertheless, do write full sentences.
% Try to have at most one paragraph per section, because line breaks
% do not look nice in a short entry.
\begin{clarifications}
There are two kinds of sequents: first $p: \Gamma \vdash A ~|~
\Delta$ with a distinguished formula on the right for typing the
so-called {\em unnamed} term $p$, second $c : \Gamma \vdash \Delta$ with no
distinguished formula for typing the so-called {\em named} term $c$.
The syntax of the underlying $\lambda\mu$-calculus 
is:
$$
\begin{array}{lll}
c & ::= & [\alpha]p\\
p,q  & ::= & a ~|~ \mu\alpha.c ~|~ (p,p) ~|~ \pi_i(p) ~|~ \iota_i(p) ~|~ \Case{p}{a_1}{p_1}{a_2}{p_2} \\
& ~|~ & \lambda a.p ~|~ p\,q 
~|~ \lambda x.p ~|~ p\,t 
~|~ (t,p) ~|~ \dest{p}{x}{a}{q}
 ~|~ () ~|~ \efq{p}\\
\end{array}
$$ The variables used for referring to assumptions in $\Gamma$ and to
conclusions in $\Delta$ range over distinct classes (denoted by Latin
and Greek letters respectively). In the rules
$\exists_E$ (resp. $\forall_I$), $y$ is assumed fresh in $\Gamma$,
$\Delta$ and $\exists x\, A$ (resp. $\forall x\, A$).
\end{clarifications}

\begin{history}
This system, defined in Parigot~\cite{Parigot92},
% and derived from free deduction~\cite{Parigot91},
highlights that classical logic in natural deduction can be obtained
from allowing several conclusions with contraction and weakening on
the right of the sequent, as in Gentzen's LK.  Additionally, the
system assigns to this form of classical reasoning a computational
content, based on the $\mu$ and bracket operator which provides with a
fine-grained decomposition
%~\cite{AriolaHerbelin07}
of the operators {\tt call-cc} (from Scheme/ML) or ${\cal C}$
(from~\cite{FelFriKohDub86}) that were known at this time to provide
computational content to classical logic~\cite{Griffin90}, as well as
a decomposition of Prawitz's classical elimination rule of
negation~\cite{Prawitz65}.

The original presentation~\cite{Parigot92} only contains implication
as well as first-order and second-order universal quantification \`a
la Curry (i.e. without leaving trace of the quantification in the
proof-term, what corresponds to computationally interpreting
quantification as an intersection type). The presentation above has
quantification \`a la Church (i.e. with an explicit trace in the proof
term) what makes the calculus compatible with several reduction
strategies such as both call-by-name or call-by-value (see
e.g.~\cite{HerbelinHdR}). Variants with multiplicative disjunctions
can be found in~\cite{Selinger01} or \cite{PymRitter01}, or
multiplicative conjunctions in \cite{HerbelinHdR}.

A standard variant originating in~\cite{deGroote94} uses only one kind
of sequents, interpreting $c : \Gamma \vdash \Delta$ as $c : \Gamma
\vdash \bot ~|~ \Delta$ (and hence removing $\bot_E$ and merging the
syntactic categories $c$ and $p$ into one). This variant is logically
equivalent to the original presentation (in the presence of $\bot$),
but not computationally equivalent~\cite{HerbelinSaurin10}.

\end{history}

% Use "\irefmissing{SuggestedIDForOtherProofSystem}" to refer to
% another proof system that is not yet available in the encyclopedia.
% \end{history}
%\begin{technicalities}
%The system is obviously logically equivalent to Gentzen's $\LK$ when
%equipped with the corresponding connectives and observed through the
%sequents of the form $\Gamma \vdash \Delta$. 
%\end{technicalities}

% General Instructions:
% =====================
% The preferred length of an entry is 1 page.
% Do the best you can to fit your proof system in one page.
%
% If you are finding it hard to fit what you want in one page, remember:
%
% * Your entry needs to be neither self-contained nor fully understandable
% (the interested reader may consult the cited full paper for details)
%
% * If you are describing several proof systems in one entry,
% consider splitting your entry.
%
% * You may reduce the size of your entry by ommitting inference rules
% that are already described in other entries.
%
% * Cite parsimoniously (see detailed citation instructions below).
%
%
% If you do not manage to fit everything in one page,
% it is acceptable for an entry to have 2 pages.
%
% For aesthetical reasons, it is preferable for an entry to have
% 1 full page or 2 full pages, in order to avoid unused blank space.
% Citation Instructions:
% ======================
% Please cite the original paper where the proof system was defined.
% To do so, you may use the \cite command within
% one of the optional environments above,
% or use the \nocite command otherwise.
% You may also cite a modern paper or book where the
% proof system is explained in greater depth or clarity.
% Cite parsimoniously.
% Do not cite related work. Instead, use the "\iref" or "\irefmissing"
% commands to make an internal reference to another entry,
% as explained within the "history" environment above.
% You do not need to create the "References" section yourself.
% This is done automatically.
% Leave an empty line above "\end{entry}".
\end{entry}
