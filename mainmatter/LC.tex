
% If the calculus has an acronym, define it.
% (e.g. \newcommand{\LK}{\ensuremath{\mathbf{LK}}\xspace})


\calculusName{\LC}   % The name of the calculus
\calculusAcronym{\LC}    % The acronym if defined above, or empty otherwise. 
\calculusLogic{Classical Logic}  % Specify the logic (e.g. classical, intuitionistic, ...) for which this calculus is intended.
\calculusType{sequent calculus}   % Specify the calculus type (e.g. Frege-Hilbert style, tableau, sequent calculus, hypersequent calculus, natural deduction, ...)
\calculusYear{1991}   % The year when the calculus was invented.
\calculusAuthor{Jean-Yves Girard} % The name(s) of the author(s) of the calculus.


\entryTitle{Constructive Classical Logic \LC}     % Title of the entry (usually coincides with the name of the calculus).
\entryAuthor{Olivier Laurent}    % Your name(s). Separate multiple names with "\and".


% If you wish, use tags to give any other information 
% that might be helpful for classifying and grouping this entry:
% e.g. \tag{Two-Sided Sequents}
% e.g. \tag{Multiset Cedents}
% e.g. \tag{List Cedents}
% You are free to invent your own tags. 
% The Encyclopedia's coordinator will take care of 
% merging semantically similar tags in the future.
\tag{One-Sided Sequents}
\tag{List Cedents}

\maketitle


% If your files are called "MyProofSystem.tex" and "MyProofSystem.bib", 
% then you should write "\begin{entry}{MyProofSystem}" in the line below
\begin{entry}{LC}  

% Define here any newcommands you may need:
% e.g. \newcommand{\necessarily}{\Box}
% e.g. \newcommand{\possibly}{\Diamond}

\newcommand{\stp}{\mathrel{;}}

\begin{calculus}
% Add the inference rules of your proof system here.
% The "proof.sty" and "bussproofs.sty" packages are available.
% If you need any other package, please contact the editor (bruno@logic.at)
\begin{center}
\AxiomC{}
\UnaryInfC{$\seq \neg P\stp P$}
\DisplayProof
\qquad
\AxiomC{$\seq \Gamma\stp P$}
\AxiomC{$\seq \Delta,\neg P\stp\Pi$}
\BinaryInfC{$\seq \Gamma,\Delta\stp\Pi$}
\DisplayProof
\qquad
\AxiomC{$\seq \Gamma,N\stp{}$}
\AxiomC{$\seq \Delta,\neg N\stp\Pi$}
\BinaryInfC{$\seq \Gamma,\Delta\stp\Pi$}
\DisplayProof
\\[2ex]
\AxiomC{$\seq \Gamma\stp\Pi$}
\UnaryInfC{$\seq \sigma(\Gamma)\stp\Pi$}
\DisplayProof
\qquad
\AxiomC{$\seq \Gamma\stp P$}
\UnaryInfC{$\seq \Gamma,P\stp{}$}
\DisplayProof
\qquad
\AxiomC{$\seq \Gamma,A,A\stp\Pi$}
\UnaryInfC{$\seq \Gamma,A\stp\Pi$}
\DisplayProof
\qquad
\AxiomC{$\seq \Gamma\stp\Pi$}
\UnaryInfC{$\seq \Gamma,A\stp\Pi$}
\DisplayProof
\\[2ex]
\AxiomC{}
\UnaryInfC{$\seq {}\stp V$}
\DisplayProof
\qquad
\AxiomC{}
\UnaryInfC{$\seq \Gamma,\neg F\stp\Pi$}
\DisplayProof
\\[2ex]
\AxiomC{$\seq \Gamma\stp P$}
\AxiomC{$\seq \Delta\stp Q$}
\BinaryInfC{$\seq \Gamma,\Delta\stp P\wedge Q$}
\DisplayProof
\qquad
\AxiomC{$\seq \Gamma,M\stp\Pi$}
\AxiomC{$\seq \Delta,N\stp\Pi$}
\BinaryInfC{$\seq \Gamma,\Delta,M\wedge N\stp\Pi$}
\DisplayProof
\\[2ex]
\AxiomC{$\seq \Gamma\stp P$}
\AxiomC{$\seq \Delta,N\stp {}$}
\BinaryInfC{$\seq \Gamma,\Delta\stp P\wedge N$}
\DisplayProof
\qquad
\AxiomC{$\seq \Gamma,M\stp {}$}
\AxiomC{$\seq \Delta\stp Q$}
\BinaryInfC{$\seq \Gamma,\Delta\stp M\wedge Q$}
\DisplayProof
\\[2ex]
\AxiomC{$\seq \Gamma,A,B\stp\Pi$}
\UnaryInfC{$\seq \Gamma,A\vee B\stp\Pi$}
\DisplayProof
$A\vee B$ negative
\qquad
\AxiomC{$\seq \Gamma\stp P$}
\UnaryInfC{$\seq \Gamma\stp P\vee Q$}
\DisplayProof
\qquad
\AxiomC{$\seq \Gamma\stp Q$}
\UnaryInfC{$\seq \Gamma\stp P\vee Q$}
\DisplayProof
\\[1ex]
\[
\neg\neg X = X \qquad \neg(A\wedge B) = \neg A\vee \neg B \qquad \neg(A\vee B) = \neg A\wedge\neg B
\]
\[
\begin{array}{lrlccccccccccccc}
\textrm{Formulas:} & A,B & ::= & P & \mid & N \\
\textrm{Positive formulas:} & P,Q & ::= & X & \mid & V & \mid & F & \mid & P\wedge Q & \mid & P\wedge N & \mid & M\wedge Q & \mid & P\vee Q \\
\textrm{Negative formulas:} & M,N & ::= & \neg X & \mid & \neg V & \mid & \neg F & \mid & M\vee N & \mid & M\vee Q & \mid & P\vee N & \mid & M\wedge N \\
\end{array}
\]
$\Gamma$ and $\Delta$ are lists of formulas, and
$\Pi$ consists of 0 or 1 positive formula.\\
$\sigma$ is a permutation.
\end{center}
\end{calculus}

% The following sections ("clarifications", "history", 
% "technicalities") are optional. If you use them, 
% be very concise and objective. Nevertheless, do write full sentences. 
% Try to have at most one paragraph per section, because line breaks 
% do not look nice in a short entry.

\begin{clarifications}
% ToDo: write here short remarks that may help the reader to understand 
% the inference rules of the proof system.
Negation is not a connective. It is defined using De Morgan's laws so that $\neg\neg A=A$. There are two atomic formulas for truth (a positive one $V$ and a negative one $\neg F$) and two atomic formulas for falsity (a positive one $F$ and a negative one $\neg V$). Sequents have the shape $\vdash\Gamma\stp\Pi$ where $\Pi$ is called the stoup.
\end{clarifications}

\begin{history}
% ToDo: write here short historical remarks about this proof system,
% especially if they relate to other proof systems. 
% Use "\iref{OtherProofSystem}" to refer to another proof system 
% in the Encyclopedia (where "OtherProofSystem" is its ID). 
% Use "\irefmissing{SuggestedIDForOtherProofSystem}" to refer to 
% another proof system that is not yet available in the encyclopedia.
\LC~\cite{lc} comes from the analysis of classical logic inside the coherent semantics of linear logic~\cite{ll} together with the use of the focusing property~\cite{focal}.
\end{history}

\begin{technicalities}
% ToDo: write here remarks about soundness, completeness, decidability...
Cut elimination holds.
\LK\iref{GentzenLK} can be translated into \LC, but not in a canonical manner.
\LC{} satisfies constructive properties such as the disjunction property: if $\vdash{}\stp P\vee Q$ is provable then $\vdash{}\stp P$ or $\vdash{}\stp Q$ as well.
\LC{} admits a denotational semantics through correlation spaces~\cite{lc} (a variant of coherence spaces~\cite{ll}).
\end{technicalities}


% General Instructions:
% =====================

% The preferred length of an entry is 1 page. 
% Do the best you can to fit your proof system in one page.
%
% If you are finding it hard to fit what you want in one page, remember:
%
%   * Your entry needs to be neither self-contained nor fully understandable
%     (the interested reader may consult the cited full paper for details)
%
%   * If you are describing several proof systems in one entry, 
%     consider splitting your entry.
%
%   * You may reduce the size of your entry by ommitting inference rules
%     that are already described in other entries.
%
%   * Cite parsimoniously (see detailed citation instructions below).
%
% 
% If you do not manage to fit everything in one page, 
% it is acceptable for an entry to have 2 pages.
%
% For aesthetical reasons, it is preferable for an entry to have
% 1 full page or 2 full pages, in order to avoid unused blank space.



% Citation Instructions:
% ======================

% Please cite the original paper where the proof system was defined.
% To do so, you may use the \cite command within 
% one of the optional environments above,
% or use the \nocite command otherwise.

% You may also cite a modern paper or book where the 
% proof system is explained in greater depth or clarity.
% Cite parsimoniously.

% Do not cite related work. Instead, use the "\iref" or "\irefmissing" 
% commands to make an internal reference to another entry, 
% as explained within the "history" environment above.

% You do not need to create the "References" section yourself. 
% This is done automatically.




% Leave an empty line above "\end{entry}".

\end{entry}



%%% Local Variables: 
%%% mode: latex
%%% TeX-master: "main"
%%% End: 
