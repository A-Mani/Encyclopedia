
% If the calculus has an acronym, define it.
% (e.g. \newcommand{\LK}{\ensuremath{\mathbf{LK}}\xspace})
%\newcommand{\ND}{\ensuremath{\mathbf{ND}}\xspace} % already defined in GentzenLK.tex
\newcommand{\NJ}{\ensuremath{\mathbf{NJ}}\xspace} 

\calculusName{Intuitionistic Sequent Calculus}   % The name of the calculus
\calculusAcronym{\LJ}    % The acronym if defined above, or empty otherwise. 
\calculusLogic{Intuitionistic Predicate Logic}
\calculusType{Sequent Calculus}
\calculusYear{1935}   % The year when the calculus was invented.
\calculusAuthor{Gerhard Karl Erich Gentzen} % The name(s) of the author(s) of the calculus.

\entryTitle{Intuitionistic Sequent Calculus \LJ}
\entryAuthor{Giselle Reis}    % Your name(s). Separate multiple names with "\and"

\maketitle


% If your files are called "<ID>.tex" and "<ID>.bib", 
% then you should write "\begin{entry}{<ID>}" in the line below
\begin{entry}{GentzenLJ}  

% Define here any newcommands you may need:
% e.g. \newcommand{\necessarily}{\Box}
% e.g. \newcommand{\possibly}{\Diamond}

\begin{calculus}

% Add the inference rules of your proof system here.
% The "proof.sty" and "bussproofs.sty" packages are available.
% If you need any other package, please contact the editor (bruno@logic.at)
\[
\begin{array}{cc}
\infer[init]{A \rightarrow A}{}
&
\infer[cut]{\Gamma, \Delta \rightarrow \Theta}{\Gamma \rightarrow A & A, \Delta \rightarrow \Theta}
\\[8pt]
\infer[\neg_l]{\neg A, \Gamma \rightarrow }{\Gamma \rightarrow A}
&
\infer[\neg_r]{\Gamma \rightarrow \neg A}{A, \Gamma \rightarrow }
\\[8pt]
\infer[\&_{li}]{A_1 \& A_2, \Gamma \rightarrow \Theta}{A_i, \Gamma \rightarrow \Theta}
&
\infer[\&_r]{\Gamma \rightarrow A \& B}{\Gamma \rightarrow A & \Gamma \rightarrow B}
\\[8pt]
\infer[\vee_l]{A \vee B, \Gamma \rightarrow \Theta}{A, \Gamma \rightarrow \Theta & B, \Gamma \rightarrow \Theta}
&
\infer[\vee_{ri}]{\Gamma \rightarrow A_1 \vee A_2}{\Gamma \rightarrow A_i}
\\[8pt]
\infer[\supset_l]{A \supset B, \Gamma, \Delta \rightarrow \Theta}{\Gamma \rightarrow A & B, \Delta \rightarrow \Theta}
&
\infer[\supset_r]{\Gamma \rightarrow A \supset B}{A, \Gamma \rightarrow B}
\\[8pt]
\infer[\exists_l]{\exists x.Ax, \Gamma \rightarrow \Theta}{A\alpha, \Gamma \rightarrow \Theta}
&
\infer[\exists_r]{\Gamma \rightarrow \exists x.Ax}{\Gamma \rightarrow At}
\\[8pt]
\infer[\forall_l]{\forall x.Ax, \Gamma \rightarrow \Theta}{At, \Gamma \rightarrow \Theta}
&
\infer[\forall_r]{\Gamma \rightarrow \forall x.Ax}{\Gamma \rightarrow A\alpha}
\\[8pt]
\infer[e_l]{\Gamma, A, B, \Delta \rightarrow \Theta}{\Gamma, B, A, \Delta \rightarrow \Theta}
&
\infer[c_l]{A, \Gamma \rightarrow \Theta}{A, A, \Gamma \rightarrow \Theta}
\\[8pt]
\infer[w_l]{A, \Gamma \rightarrow \Theta}{\Gamma \rightarrow \Theta}
&
\infer[w_r]{\Gamma \rightarrow A}{\Gamma \rightarrow}
\\
\end{array}
\]
\end{calculus}

% The following environments ("clarifications", "history", 
% "technicalities") are optional. If you do use them, 
% be very concise and objective.

\begin{clarifications}
% ToDo: write here short remarks that may help the reader to understand 
% the inference rules of the proof system.
In all rules, $A$, $A_i$ and $B$ are arbitrary formulas and $\Theta$ is a set
with at most one formula. In rules $\exists_l$
and $\forall_r$, $\alpha$ is a variable not contained in $A$, $\Gamma$ or
$\Theta$. In rules $\exists_r$ and $\forall_l$, $t$ does not contain variables
bound in $A$.
It is common to consider \textbf{LJ} without the exchange rule $e_l$ just by
interpreting $\Gamma$ and $\Theta$ as multi-sets of formulas instead of lists.
Also, the conjunction $\&$ is usually denoted by $\wedge$.
\end{clarifications}

\begin{history}
% ToDo: write here short historical remarks about this proof system,
% especially if they relate to other proof systems. 
% Use "\iref{OtherProofSystem}" to refer to another proof system 
% in the Encyclopedia (where "OtherProofSystem" is its ID). 
% Use "\irefmissing{SuggestedIDForOtherProofSystem}" to refer to 
% another proof system that is not yet available in the encyclopedia.
Proposed by Gentzen in \cite{Gentzen1935} by restricting the
succedent of sequents in \irefmissing{SequentCalculusLK} to have at most one
formula. In the original paper, he notes that this restriction is equivalent to
removing the principle of excluded middle from the natural deduction system
\iref{NaturalDeduction} in order to obtain \irefmissing{NJ}.
% NOTE Assuming that Natural Deduction is the classical version.
%The cut is admissible in \LJ and this result is known as \emph{Hauptsatz}.
\end{history}

\begin{technicalities}
% ToDo: write here remarks about soundness, completeness, decidability...
Soundness and completeness of \LJ can be proved using a translation of \LJ
derivations into \NJ\irefmissing{NJ}.
Decidability of the propositional fragment and consistency of intuitionistic
logic follows from cut admissibility in this calculus (\emph{Hauptsatz}).
\end{technicalities}



% Please cite the original paper where the proof system was defined.
% To do so, you may use the \cite command within 
% one of the optional environments above,
% or use the \nocite command otherwise.

% You may also cite a modern paper or book where the 
% proof system is explained in greater depth or clarity.
% Cite parsimoniously.

% Do not cite related work. Instead, use the "\iref" or "\irefmissing" 
% commands to make an internal reference to another entry, 
% as explained within the "history" environment above.

% You do not need to create the "References" section yourself. 
% This is done automatically.

\end{entry}
