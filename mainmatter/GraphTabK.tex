
% If the calculus has an acronym, define it.
% (e.g. \newcommand{\LK}{\ensuremath{\mathbf{LK}}\xspace})

\calculusName{Graph-based tableaux for modal logics}   % The name of the calculus
\calculusAcronym{}    % The acronym if defined above, or empty otherwise. 
\calculusLogic{modal logics}  % Specify the logic (e.g. classical, intuitionistic, ...) for which this calculus is intended.
\calculusType{tableau}   % Specify the calculus type (e.g. Frege-Hilbert style, tableau, sequent calculus, hypersequent calculus, natural deduction, ...)
\calculusYear{1997}   % The year when the calculus was invented.
\calculusAuthor{Marcos A. Castilho \and Luis Fari\~nas del Cerro \and Olivier Gasquet \and Andreas Herzig} % The name(s) of the author(s) of the calculus.


\entryTitle{Graph-based tableaux for modal logics}     % Title of the entry (usually coincides with the name of the calculus).
\entryAuthor{Joseph Boudou \and Olivier Gasquet}  % Your name(s). Separate multiple names with "\and".


% If you wish, use tags to give any other information 
% that might be helpful for classifying and grouping this entry:
% e.g. \tag{Two-Sided Sequents}
% e.g. \tag{Multiset Cedents}
% e.g. \tag{List Cedents}
% You are free to invent your own tags. 
% The Encyclopedia's coordinator will take care of 
% merging semantically similar tags in the future.


\maketitle


% If your files are called "MyProofSystem.tex" and "MyProofSystem.bib", 
% then you should write "\begin{entry}{MyProofSystem}" in the line below
\begin{entry}{GraphTabK}

% Define here any newcommands you may need:
% e.g. \newcommand{\necessarily}{\Box}
% e.g. \newcommand{\possibly}{\Diamond}

\newenvironment{infruleset}[1]{%
  \sc{#1} %\vspace{-.5ex} %
  \[ %
%  $$
}{%
  \] %
%  $$
}
\newcommand{\infrule}[3]{\vcenter{\infer[\mbox{\textit{(#1)}}]{\tikz{ #3 }}{\tikz{ #2 }}}}
\newcommand{\sepproof}{\hskip 2em plus 6em\relax}
\newcommand{\sepline}{$$ $$}

\usetikzlibrary{backgrounds}
\tikzset{ lstate/.pic={\draw[fill] (-.1,0) circle (1.5pt) node [left]  () {\tikzpictext};} }
\tikzset{ rstate/.pic={\draw[fill] (.1,0)  circle (1.5pt) node [right] () {\tikzpictext};} }
\tikzset{ ustate/.pic={\draw[fill] (.1,0)  circle (1.5pt) node [above] () {\tikzpictext};} }
\tikzset{ dstate/.pic={\draw[fill] (.1,0)  circle (1.5pt) node [below] () {\tikzpictext};} }

\begin{calculus}

% Add the inference rules of your proof system here.
% The "proof.sty" and "bussproofs.sty" packages are available.
% If you need any other package, please contact the editor (bruno@logic.at)

\begin{infruleset}{Boolean rules}
  \infrule{$\bot$}{
    \draw pic ["${\Gamma, A, \neg A}$"] {lstate};}{
    \draw pic ["${\Gamma, A, \neg A, \bot}$"] {lstate};}
  \sepproof
%  \infrule{$\neg$}{
%    \draw pic ["${\Gamma, \neg\neg A}$"] {lstate};}{
%    \draw pic ["${\Gamma, \neg\neg A, A}$"] {lstate};}
%  \sepproof
  \infrule{$\wedge$}{
    \draw pic ["${\Gamma, A \wedge B}$"] {lstate} ;}{
    \draw pic ["${\Gamma, A \wedge B, A, B}$"] {lstate} ;}
  \sepproof
  \infrule{$\vee$}{
    \draw pic ["${\Gamma, A_1 \vee A_2}$"] {lstate} ;}{
    \draw pic ["${\Gamma, A_1 \vee A_2, A_i}$"] {lstate} ;}
\end{infruleset}

\begin{infruleset}{Diamond rule}
  \infrule{$\Diamond$}{
    \draw pic ["${\Gamma, \Diamond A}$"] {lstate} ;}{
    \draw[->] (0,0) pic ["${\Gamma, \Diamond A}$"] {lstate} -- (.5,0) pic ["$A$"] {rstate} ;}
\end{infruleset}

\begin{infruleset}{Propagation rules}
  \infrule{K}{
    \draw[->] (0,0)  pic ["${\Gamma, \Box A}$"] {lstate} --
              (.5,0) pic ["${\Delta}$"] {rstate} ;}{
    \draw[->] (0,0)  pic ["${\Gamma, \Box A}$"] {lstate} --
              (.5,0) pic ["${\Delta, A}$"] {rstate} ;}
  \sepline
  \infrule{T}{
    \draw pic ["${\Gamma, \Box A}$"] {lstate} ;}{
    \draw pic ["${\Gamma, \Box A, A}$"] {lstate} ;}
  \sepproof
  \infrule{4}{
    \draw[->] (0,0)  pic ["${\Gamma, \Box A}$"] {lstate} --
              (.5,0) pic ["${\Delta}$"] {rstate} ;}{
    \draw[->] (0,0)  pic ["${\Gamma, \Box A}$"] {lstate} --
              (.5,0) pic ["${\Delta, \Box A}$"] {rstate} ;}
  \sepproof
  \infrule{B}{
    \draw[->] (0,0)  pic ["${\Gamma}$"] {lstate} --
              (.5,0) pic ["${\Delta, \Box A}$"] {rstate} ;}{
    \draw[->] (0,0)  pic ["${\Gamma, A}$"] {lstate} --
              (.5,0) pic ["${\Delta, \Box A}$"] {rstate} ;}
  \sepline
  \infrule{$5_{\mathord\rightarrow}$}{
    \draw[->] (-.5,0) pic ["${\Gamma}$"] {lstate} --
              (0, .2) pic ["${\Delta_1, \Box A}$"] {rstate} ;
    \draw[->] (-.5,0) --
              (0,-.2) pic ["${\Delta_2}$"] {rstate} ;}{
    \draw[->] (-.5,0) pic ["${\Gamma}$"] {lstate} --
              (0, .2) pic ["${\Delta_1, \Box A}$"] {rstate} ;
    \draw[->] (-.5,0) --
              (0,-.2) pic ["${\Delta_2, \Box A}$"] {rstate} ;}
  \sepproof
  \infrule{$5_{\mathord\uparrow}$}{
    \draw[->] (0,0)  pic ["${\Gamma}$"] {lstate} --
              (.4,0) pic ["${\Delta, \Box A}$"] {rstate} ;}{
    \draw[->] (0,0)  pic ["${\Gamma, \Box A}$"] {lstate} --
              (.4,0) pic ["${\Delta, \Box A}$"] {rstate} ;}
  \sepproof
  \infrule{$5_{\mathord\downarrow}$}{
    \draw[->] (0,0)  pic ["${\Gamma}$"] {lstate} --
              (.4,0) pic ["${\Delta_1, \Box A}$"] {ustate};
    \draw[->] (.6,0) --
              (1,0)  pic ["${\Delta_2}$"] {rstate}; }{
    \draw[->] (0,0)  pic ["${\Gamma}$"] {lstate} --
              (.4,0) pic ["${\Delta_1, \Box A}$"] {ustate};
    \draw[->] (.6,0) --
              (1,0)  pic ["${\Delta_2, \Box A}$"] {rstate}; }
\end{infruleset}

\begin{infruleset}{Structural rules}
  \infrule{D}{
    \draw pic ["${\Gamma}$"] {lstate} ;}{
    \draw[->] (0,0)  pic ["${\Gamma}$"] {lstate} --
              (.5,0) pic ["${\emptyset}$"] {rstate};}
  \sepproof
  \infrule{De}{
    \draw[->] (0,0) pic ["${\Gamma}$"] {lstate} --
              (1.2,0) pic ["${\Delta}$"] {rstate} ;}{
    \draw[->] (0,0) pic ["${\Gamma}$"] {lstate} --
              (1.2,0) pic ["${\Delta}$"] {rstate} ;
    \draw[->] (-.02,-.07) --
              (.5,-.2) pic ["${\emptyset}$"] {dstate};
    \draw[->] (.7,-.2) -- (1.22,-.07);}
  \sepline
  \infrule{$C_0$}{
    \draw[->] (-.5,0) pic ["${\Gamma}$"] {lstate} --
              (0, .2) pic ["${\Delta_1}$"] {rstate} ;
    \draw[->] (-.5,0) --
              (0,-.2) pic ["${\Delta_2}$"] {rstate} ;}{
    \draw[->] (-.5,0) pic ["${\Gamma}$"] {lstate} --
              (0, .2) pic ["${\Delta_1}$"] {ustate} ;
    \draw[->] (-.5,0) --
              (0,-.2) pic ["${\Delta_2}$"] {dstate} ;
    \draw[->] (.2,.2) --
              (.7,0)  pic ["${\emptyset}$"] {rstate};
    \draw[->] (.2,-.2) -- (.7,0);}
  \sepproof
  \infrule{$C_1$}{
    \draw[->] (0,0)  pic ["${\Gamma}$"] {lstate} --
              (.5,0) pic ["${\Delta}$"] {rstate} ;}{
    \draw[->] (0,0)  pic ["${\Gamma}$"] {lstate} --
              (.5,0) pic ["${\Delta}$"] {dstate} ;
    \draw[->] (.7,0)  --
              (1.2,0) pic ["${\emptyset}$"] {rstate} ;}
\end{infruleset}

\end{calculus}

% The following sections ("clarifications", "history", 
% "technicalities") are optional. If you use them, 
% be very concise and objective. Nevertheless, do write full sentences. 
% Try to have at most one paragraph per section, because line breaks 
% do not look nice in a short entry.

\begin{clarifications}
  The method constructs a collection of rooted directed acyclic graphs with vertices labeled with sets of formulas.
  \tikz[baseline=-.7ex]{\draw pic ["${\Gamma}$"] {lstate};} denotes a vertex labeled with $\Gamma$.
  To each branch of the tableau corresponds a graph.
  The only branching rule is $(\vee)$, for which $i$ must be choosen among $\{1,2\}$.
  A branch is closed if the corresponding graph contains a vertex of the form
  \tikz[baseline=-.7ex]{\draw pic ["${\Gamma, \bot}$"] {lstate};}.
%  To prove $A$, graphs are constructed from the initial graph containing only one vertex
%  \tikz[baseline=-.7ex]{\draw pic ["${\neg A}$"] {lstate};} and no edges.
  For modal logic K, only rules $(\bot)$, $(\mathord\wedge)$, $(\mathord\vee)$, $(\Diamond)$ and
  $(K)$ are used.
  To each additional axiom T, 4, B, 5, D, De and C corresponds a set of rules
  to add, as detailed in Table~\ref{tab:GraphTabK} below.

  \begin{table}
  \center
  \begin{tabular}{l l l}
    Axiom & Model's property & Rules \\
    \hline
    $T = \Box A \imp A$ & reflexivity & $(T)$ \\
    $4 = \Box A \imp \Box\Box A$ & transitivity & $(4)$ \\
    $B = \Diamond\Box A \imp A$ & symmetry & $(B)$ \\
    $5 = \Diamond\Box A \imp \Box A$ & euclideanity &
      $(5_{\mathord\rightarrow})$, $(5_{\mathord\uparrow})$, $(5_{\mathord\downarrow})$ \\
    $D = \Box A \imp \Diamond A$ & seriality & $(D)$ \\
    $De = \Diamond A \imp \Diamond\Diamond A$ & density & $(De)$ \\
    $C = \Diamond\Box A \imp \Box\Diamond A$ & confluence & $(C_0)$, $(C_1)$
  \end{tabular}
  \caption{Correspondences between modal axioms and graph-based tableaux rules.}
  \label{tab:GraphTabK}
  \end{table}
\end{clarifications}

\begin{history}
  Tableaux methods for modal logics have a long history started back by Fitting~\cite{fitting.1972}.
%  F. Massacci~\cite{massacci.1994} provided labeled tableaux methods for a range of modal logics.
  The present method, introduced in~\cite{castilho-farinas-gasquet-herzig.1997} and extended in~\cite{farinas-gasquet.2002},
  distinguish itself by its ability to deal with properties like confluence or density.
  Moreover, it can be easily adapted to multimodal logics.
  The method has been enhanced and implemented in the LoTREC prover~\cite{gasquet-herzig-said-schwarzentruber.2014}.
% ToDo: write here short historical remarks about this proof system,
% especially if they relate to other proof systems. 
% Use "\iref{OtherProofSystem}" to refer to another proof system 
% in the Encyclopedia (where "OtherProofSystem" is its ID). 
% Use "\irefmissing{SuggestedIDForOtherProofSystem}" to refer to 
% another proof system that is not yet available in the encyclopedia.
\end{history}

\begin{technicalities}
  The method is sound and complete for any combination of axioms.
  Termination is more problematic
  and has been investigated in~\cite{farinas-gasquet.2002,gasquet-herzig-sahade.2006,gasquet-herzig-said-schwarzentruber.2014}.
% ToDo: write here remarks about soundness, completeness, decidability...
\end{technicalities}


% General Instructions:
% =====================

% The preferred length of an entry is 1 page. 
% Do the best you can to fit your proof system in one page.
%
% If you are finding it hard to fit what you want in one page, remember:
%
%   * Your entry needs to be neither self-contained nor fully understandable
%     (the interested reader may consult the cited full paper for details)
%
%   * If you are describing several proof systems in one entry, 
%     consider splitting your entry.
%
%   * You may reduce the size of your entry by ommitting inference rules
%     that are already described in other entries.
%
%   * Cite parsimoniously (see detailed citation instructions below).
%
% 
% If you do not manage to fit everything in one page, 
% it is acceptable for an entry to have 2 pages.
%
% For aesthetical reasons, it is preferable for an entry to have
% 1 full page or 2 full pages, in order to avoid unused blank space.



% Citation Instructions:
% ======================

% Please cite the original paper where the proof system was defined.
% To do so, you may use the \cite command within 
% one of the optional environments above,
% or use the \nocite command otherwise.

% You may also cite a modern paper or book where the 
% proof system is explained in greater depth or clarity.
% Cite parsimoniously.

% Do not cite related work. Instead, use the "\iref" or "\irefmissing" 
% commands to make an internal reference to another entry, 
% as explained within the "history" environment above.

% You do not need to create the "References" section yourself. 
% This is done automatically.




% Leave an empty line above "\end{entry}".

\end{entry}
