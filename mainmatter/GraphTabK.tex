\calculusName{Graph-based tableaux for modal logics}
\calculusAcronym{}
\calculusLogic{Modal logics}
\calculusType{Tableau Calculus}
\calculusYear{1997}
\calculusAuthor{Marcos A. Castilho \and Luis Fari\~nas del Cerro \and Olivier Gasquet \and Andreas Herzig}



%
\entryTitle{Graph-based tableaux for modal logics}
\entryAuthor{Joseph Boudou \and Olivier Gasquet}

\maketitle

\begin{entry}{GraphTabK}

\newenvironment{infruleset}[1]{%
  \sc{#1}
  \[ %
}{%
  \] %
}
\newcommand{\infrule}[3]{\vcenter{\infer[\mbox{\textit{(#1)}}]{\tikz{ #3 }}{\tikz{ #2 }}}}
\newcommand{\sepproof}{\hskip 1.5em plus 6em\relax}
\newcommand{\sepline}{$$ $$}

\usetikzlibrary{backgrounds}
\tikzset{ lstate/.pic={\draw[fill] (-.1,0) circle (1.5pt) node [left]  () {\tikzpictext};} }
\tikzset{ rstate/.pic={\draw[fill] (.1,0)  circle (1.5pt) node [right] () {\tikzpictext};} }
\tikzset{ ustate/.pic={\draw[fill] (.1,0)  circle (1.5pt) node [above] () {\tikzpictext};} }
\tikzset{ dstate/.pic={\draw[fill] (.1,0)  circle (1.5pt) node [below] () {\tikzpictext};} }

\begin{calculus}

\begin{infruleset}{Boolean rules}
  \infrule{$\bot$}{
    \draw pic ["${\Gamma, A, \neg A}$"] {lstate};}{
    \draw pic ["${\Gamma, A, \neg A, \bot}$"] {lstate};}
  \sepproof
  \infrule{$\wedge$}{
    \draw pic ["${\Gamma, A \wedge B}$"] {lstate} ;}{
    \draw pic ["${\Gamma, A \wedge B, A, B}$"] {lstate} ;}
  \sepproof
  \infrule{$\vee$}{
    \draw pic ["${\Gamma, A_1 \vee A_2}$"] {lstate} ;}{
    \draw pic ["${\Gamma, A_1 \vee A_2, A_i}$"] {lstate} ;}
\end{infruleset}

\begin{infruleset}{Diamond rule}
  \infrule{$\Diamond$}{
    \draw pic ["${\Gamma, \Diamond A}$"] {lstate} ;}{
    \draw[->] (0,0) pic ["${\Gamma, \Diamond A}$"] {lstate} -- (.5,0) pic ["$A$"] {rstate} ;}
\end{infruleset}

\begin{infruleset}{Propagation rules}
  \infrule{K}{
    \draw[->] (0,0)  pic ["${\Gamma, \Box A}$"] {lstate} --
              (.5,0) pic ["${\Delta}$"] {rstate} ;}{
    \draw[->] (0,0)  pic ["${\Gamma, \Box A}$"] {lstate} --
              (.5,0) pic ["${\Delta, A}$"] {rstate} ;}
  \sepline
  \infrule{T}{
    \draw pic ["${\Gamma, \Box A}$"] {lstate} ;}{
    \draw pic ["${\Gamma, \Box A, A}$"] {lstate} ;}
  \sepproof
  \infrule{4}{
    \draw[->] (0,0)  pic ["${\Gamma, \Box A}$"] {lstate} --
              (.5,0) pic ["${\Delta}$"] {rstate} ;}{
    \draw[->] (0,0)  pic ["${\Gamma, \Box A}$"] {lstate} --
              (.5,0) pic ["${\Delta, \Box A}$"] {rstate} ;}
  \sepproof
  \infrule{B}{
    \draw[->] (0,0)  pic ["${\Gamma}$"] {lstate} --
              (.5,0) pic ["${\Delta, \Box A}$"] {rstate} ;}{
    \draw[->] (0,0)  pic ["${\Gamma, A}$"] {lstate} --
              (.5,0) pic ["${\Delta, \Box A}$"] {rstate} ;}
  \sepline
  \infrule{$5_{\mathord\rightarrow}$}{
    \draw[->] (-.5,0) pic ["${\Gamma}$"] {lstate} --
              (0, .2) pic ["${\Delta_1, \Box A}$"] {rstate} ;
    \draw[->] (-.5,0) --
              (0,-.2) pic ["${\Delta_2}$"] {rstate} ;}{
    \draw[->] (-.5,0) pic ["${\Gamma}$"] {lstate} --
              (0, .2) pic ["${\Delta_1, \Box A}$"] {rstate} ;
    \draw[->] (-.5,0) --
              (0,-.2) pic ["${\Delta_2, \Box A}$"] {rstate} ;}
  \sepproof
  \infrule{$5_{\mathord\uparrow}$}{
    \draw[->] (0,0)  pic ["${\Gamma}$"] {lstate} --
              (.4,0) pic ["${\Delta, \Box A}$"] {rstate} ;}{
    \draw[->] (0,0)  pic ["${\Gamma, \Box A}$"] {lstate} --
              (.4,0) pic ["${\Delta, \Box A}$"] {rstate} ;}
  \sepproof
  \infrule{$5_{\mathord\downarrow}$}{
    \draw[->] (0,0)  pic ["${\Gamma}$"] {lstate} --
              (.4,0) pic ["${\Delta_1, \Box A}$"] {ustate};
    \draw[->] (.6,0) --
              (1,0)  pic ["${\Delta_2}$"] {rstate}; }{
    \draw[->] (0,0)  pic ["${\Gamma}$"] {lstate} --
              (.4,0) pic ["${\Delta_1, \Box A}$"] {ustate};
    \draw[->] (.6,0) --
              (1,0)  pic ["${\Delta_2, \Box A}$"] {rstate}; }
\end{infruleset}

\begin{infruleset}{Structural rules}
  \infrule{D}{
    \draw pic ["${\Gamma}$"] {lstate} ;}{
    \draw[->] (0,0)  pic ["${\Gamma}$"] {lstate} --
              (.5,0) pic ["${\emptyset}$"] {rstate};}
  \sepproof
  \infrule{De}{
    \draw[->] (0,0) pic ["${\Gamma}$"] {lstate} --
              (1.2,0) pic ["${\Delta}$"] {rstate} ;}{
    \draw[->] (0,0) pic ["${\Gamma}$"] {lstate} --
              (1.2,0) pic ["${\Delta}$"] {rstate} ;
    \draw[->] (-.02,-.07) --
              (.5,-.2) pic ["${\emptyset}$"] {dstate};
    \draw[->] (.7,-.2) -- (1.22,-.07);}
  \sepline
  \infrule{$C_0$}{
    \draw[->] (-.5,0) pic ["${\Gamma}$"] {lstate} --
              (0, .2) pic ["${\Delta_1}$"] {rstate} ;
    \draw[->] (-.5,0) --
              (0,-.2) pic ["${\Delta_2}$"] {rstate} ;}{
    \draw[->] (-.5,0) pic ["${\Gamma}$"] {lstate} --
              (0, .2) pic ["${\Delta_1}$"] {ustate} ;
    \draw[->] (-.5,0) --
              (0,-.2) pic ["${\Delta_2}$"] {dstate} ;
    \draw[->] (.2,.2) --
              (.7,0)  pic ["${\emptyset}$"] {rstate};
    \draw[->] (.2,-.2) -- (.7,0);}
  \sepproof
  \infrule{$C_1$}{
    \draw[->] (0,0)  pic ["${\Gamma}$"] {lstate} --
              (.5,0) pic ["${\Delta}$"] {rstate} ;}{
    \draw[->] (0,0)  pic ["${\Gamma}$"] {lstate} --
              (.5,0) pic ["${\Delta}$"] {dstate} ;
    \draw[->] (.7,0)  --
              (1.2,0) pic ["${\emptyset}$"] {rstate} ;}
\end{infruleset}

\end{calculus}

\begin{clarifications}
  The method constructs a collection of rooted directed acyclic graphs with vertices labeled with sets of formulas.
  \tikz[baseline=-.7ex]{\draw pic ["${\Gamma}$"] {lstate};} denotes a vertex labeled with $\Gamma$.
  To each branch of the tableau corresponds a graph.
  The only branching rule is $(\vee)$, for which $i$ must be choosen among $\{1,2\}$.
  A branch is closed if the corresponding graph contains a vertex of the form
  \tikz[baseline=-.7ex]{\draw pic ["${\Gamma, \bot}$"] {lstate};}.
  For modal logic K, only rules $(\bot)$, $(\mathord\wedge)$, $(\mathord\vee)$, $(\Diamond)$ and
  $(K)$ are used.
  To each additional axiom T, 4, B, 5, D, De and C corresponds a set of rules
  to add, as detailed in Table~\ref{tab:GraphTabK} below.

  \begin{table}
  \center
  \begin{tabular}{l l l}
    Axiom & Model's property & Rules \\
    \hline
    $T = \Box A \imp A$ & reflexivity & $(T)$ \\
    $4 = \Box A \imp \Box\Box A$ & transitivity & $(4)$ \\
    $B = \Diamond\Box A \imp A$ & symmetry & $(B)$ \\
    $5 = \Diamond\Box A \imp \Box A$ & euclideanity &
      $(5_{\mathord\rightarrow})$, $(5_{\mathord\uparrow})$, $(5_{\mathord\downarrow})$ \\
    $D = \Box A \imp \Diamond A$ & seriality & $(D)$ \\
    $De = \Diamond A \imp \Diamond\Diamond A$ & density & $(De)$ \\
    $C = \Diamond\Box A \imp \Box\Diamond A$ & confluence & $(C_0)$, $(C_1)$
  \end{tabular}
  \caption{Correspondences between modal axioms and graph-based tableaux rules.}
  \label{tab:GraphTabK}
  \end{table}
\end{clarifications}

\begin{history}
  Tableaux methods for modal logics have a long history started by Kripke~\cite{kripke.1959}.
  The present method, introduced in~\cite{castilho-farinas-gasquet-herzig.1997} and extended in~\cite{farinas-gasquet.2002},
  distinguish itself by its ability to deal with properties like confluence or density.
  Moreover, it can be easily adapted to multimodal logics.
  The method has been enhanced and implemented in the LoTREC prover~\cite{gasquet-herzig-said-schwarzentruber.2014}.
\end{history}

\begin{technicalities}
  The method is sound and complete for any combination of axioms.
  Termination is more problematic
  and has been investigated in~\cite{farinas-gasquet.2002,gasquet-herzig-sahade.2006,gasquet-herzig-said-schwarzentruber.2014}.
\end{technicalities}


\end{entry}
