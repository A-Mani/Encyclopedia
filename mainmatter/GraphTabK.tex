
% If the calculus has an acronym, define it.
% (e.g. \newcommand{\LK}{\ensuremath{\mathbf{LK}}\xspace})

\calculusName{Graph-based tableaux for modal logics}   % The name of the calculus
\calculusAcronym{}    % The acronym if defined above, or empty otherwise. 
\calculusLogic{modal logics}  % Specify the logic (e.g. classical, intuitionistic, ...) for which this calculus is intended.
\calculusType{tableau}   % Specify the calculus type (e.g. Frege-Hilbert style, tableau, sequent calculus, hypersequent calculus, natural deduction, ...)
\calculusYear{1997}   % The year when the calculus was invented.
\calculusAuthor{Marcos A. Castilho \and Luis Fari\~nas del Cerro \and Olivier Gasquet \and Andreas Herzig} % The name(s) of the author(s) of the calculus.


\entryTitle{Graph-based tableaux for modal logics}     % Title of the entry (usually coincides with the name of the calculus).
\entryAuthor{ToDo}    % Your name(s). Separate multiple names with "\and".


% If you wish, use tags to give any other information 
% that might be helpful for classifying and grouping this entry:
% e.g. \tag{Two-Sided Sequents}
% e.g. \tag{Multiset Cedents}
% e.g. \tag{List Cedents}
% You are free to invent your own tags. 
% The Encyclopedia's coordinator will take care of 
% merging semantically similar tags in the future.


\maketitle


% If your files are called "MyProofSystem.tex" and "MyProofSystem.bib", 
% then you should write "\begin{entry}{MyProofSystem}" in the line below
\begin{entry}{GraphTabK}

% Define here any newcommands you may need:
% e.g. \newcommand{\necessarily}{\Box}
% e.g. \newcommand{\possibly}{\Diamond}

\newenvironment{infruleset}[1]{%
%  \sc{#1} \vspace{-1ex} %
  \[ %
%  $$
}{%
  \] %
%  $$
}
\newcommand{\infrule}[3]{\vcenter{\infer[\mbox{\textit{(#1)}}]{\tikz{ #3 }}{\tikz{ #2 }}}}
%\newcommand{\infrule}[3]{\centerAlignProof \AxiomC{\tikz{#2}} \RightLabel{\textit{(#1)}} \UnaryInfC{\tikz{#3}} \DisplayProof }
\newcommand{\sepproof}{\hskip 2em plus 6em\relax}
%\newcommand{\sepline}{\]\[}
\newcommand{\sepline}{$$ $$}

\usetikzlibrary{backgrounds}
\tikzset{ lstate/.pic={\draw[fill] (-.1,0) circle (1.5pt) node [left]  () {\tikzpictext};} }
\tikzset{ rstate/.pic={\draw[fill] (.1,0)  circle (1.5pt) node [right] () {\tikzpictext};} }
\tikzset{ ustate/.pic={\draw[fill] (.1,0)  circle (1.5pt) node [above] () {\tikzpictext};} }
\tikzset{ dstate/.pic={\draw[fill] (.1,0)  circle (1.5pt) node [below] () {\tikzpictext};} }

\begin{calculus}

% Add the inference rules of your proof system here.
% The "proof.sty" and "bussproofs.sty" packages are available.
% If you need any other package, please contact the editor (bruno@logic.at)

\[
  \infrule{$\bot$}{
    \draw pic ["${\Gamma, A, \neg A}$"] {lstate};}{
    \draw pic ["${\Gamma, A, \neg A, \bot}$"] {lstate};}
  \sepproof
%  \infrule{$\neg$}{
%    \draw pic ["${\Gamma, \neg\neg A}$"] {lstate};}{
%    \draw pic ["${\Gamma, \neg\neg A, A}$"] {lstate};}
%  \sepproof
  \infrule{$\wedge$}{
    \draw pic ["${\Gamma, A \wedge B}$"] {lstate} ;}{
    \draw pic ["${\Gamma, A \wedge B, A, B}$"] {lstate} ;}
  \sepproof
  \infrule{$\vee_i$}{
    \draw pic ["${\Gamma, A_1 \vee A_2}$"] {lstate} ;}{
    \draw pic ["${\Gamma, A_1 \vee A_2, A_i}$"] {lstate} ;}
  \sepline
  \infrule{$\Diamond$}{
    \draw pic ["${\Gamma, \Diamond A}$"] {lstate} ;}{
    \draw[->] (0,0) pic ["${\Gamma, \Diamond A}$"] {lstate} -- (.5,0) pic ["$A$"] {rstate} ;}
  \sepproof
  \infrule{K}{
    \draw[->] (0,0)  pic ["${\Gamma, \Box A}$"] {lstate} --
              (.5,0) pic ["${\Delta}$"] {rstate} ;}{
    \draw[->] (0,0)  pic ["${\Gamma, \Box A}$"] {lstate} --
              (.5,0) pic ["${\Delta, A}$"] {rstate} ;}
  \sepline
  \infrule{T}{
    \draw pic ["${\Gamma, \Box A}$"] {lstate} ;}{
    \draw pic ["${\Gamma, \Box A, A}$"] {lstate} ;}
  \sepproof
  \infrule{4}{
    \draw[->] (0,0)  pic ["${\Gamma, \Box A}$"] {lstate} --
              (.5,0) pic ["${\Delta}$"] {rstate} ;}{
    \draw[->] (0,0)  pic ["${\Gamma, \Box A}$"] {lstate} --
              (.5,0) pic ["${\Delta, \Box A}$"] {rstate} ;}
  \sepproof
  \infrule{B}{
    \draw[->] (0,0)  pic ["${\Gamma}$"] {lstate} --
              (.5,0) pic ["${\Delta, \Box A}$"] {rstate} ;}{
    \draw[->] (0,0)  pic ["${\Gamma, A}$"] {lstate} --
              (.5,0) pic ["${\Delta, \Box A}$"] {rstate} ;}
  \sepline
  \infrule{$5_{\mathord\rightarrow}$}{
    \draw[->] (-.5,0) pic ["${\Gamma}$"] {lstate} --
              (0, .2) pic ["${\Delta_1, \Box A}$"] {rstate} ;
    \draw[->] (-.5,0) --
              (0,-.2) pic ["${\Delta_2}$"] {rstate} ;}{
    \draw[->] (-.5,0) pic ["${\Gamma}$"] {lstate} --
              (0, .2) pic ["${\Delta_1, \Box A}$"] {rstate} ;
    \draw[->] (-.5,0) --
              (0,-.2) pic ["${\Delta_2, \Box A}$"] {rstate} ;}
  \sepproof
  \infrule{$5_{\mathord\uparrow}$}{
    \draw[->] (0,0)  pic ["${\Gamma}$"] {lstate} --
              (.4,0) pic ["${\Delta, \Box A}$"] {rstate} ;}{
    \draw[->] (0,0)  pic ["${\Gamma, \Box A}$"] {lstate} --
              (.4,0) pic ["${\Delta, \Box A}$"] {rstate} ;}
  \sepproof
  \infrule{$5_{\mathord\downarrow}$}{
    \draw[->] (0,0)  pic ["${\Gamma}$"] {lstate} --
              (.4,0) pic ["${\Delta_1, \Box A}$"] {ustate};
    \draw[->] (.6,0) --
              (1,0)  pic ["${\Delta_2}$"] {rstate}; }{
    \draw[->] (0,0)  pic ["${\Gamma}$"] {lstate} --
              (.4,0) pic ["${\Delta_1, \Box A}$"] {ustate};
    \draw[->] (.6,0) --
              (1,0)  pic ["${\Delta_2, \Box A}$"] {rstate}; }
  \sepline
  \infrule{D}{
    \draw pic ["${\Gamma}$"] {lstate} ;}{
    \draw[->] (0,0)  pic ["${\Gamma}$"] {lstate} --
              (.5,0) pic ["${\emptyset}$"] {rstate};}
  \sepproof
  \infrule{De}{
    \draw[->] (0,0) pic ["${\Gamma}$"] {lstate} --
              (1.2,0) pic ["${\Delta}$"] {rstate} ;}{
    \draw[->] (0,0) pic ["${\Gamma}$"] {lstate} --
              (1.2,0) pic ["${\Delta}$"] {rstate} ;
    \draw[->] (-.02,-.07) --
              (.5,-.2) pic ["${\emptyset}$"] {dstate};
    \draw[->] (.7,-.2) -- (1.22,-.07);}
  \sepline
  \infrule{$C_0$}{
    \draw[->] (-.5,0) pic ["${\Gamma}$"] {lstate} --
              (0, .2) pic ["${\Delta_1}$"] {rstate} ;
    \draw[->] (-.5,0) --
              (0,-.2) pic ["${\Delta_2}$"] {rstate} ;}{
    \draw[->] (-.5,0) pic ["${\Gamma}$"] {lstate} --
              (0, .2) pic ["${\Delta_1}$"] {ustate} ;
    \draw[->] (-.5,0) --
              (0,-.2) pic ["${\Delta_2}$"] {dstate} ;
    \draw[->] (.2,.2) --
              (.7,0)  pic ["${\emptyset}$"] {rstate};
    \draw[->] (.2,-.2) -- (.7,0);}
  \sepproof
  \infrule{$C_1$}{
    \draw[->] (0,0)  pic ["${\Gamma}$"] {lstate} --
              (.5,0) pic ["${\Delta}$"] {rstate} ;}{
    \draw[->] (0,0)  pic ["${\Gamma}$"] {lstate} --
              (.5,0) pic ["${\Delta}$"] {dstate} ;
    \draw[->] (.7,0)  --
              (1.2,0) pic ["${\emptyset}$"] {rstate} ;}
\]

\end{calculus}

% The following sections ("clarifications", "history", 
% "technicalities") are optional. If you use them, 
% be very concise and objective. Nevertheless, do write full sentences. 
% Try to have at most one paragraph per section, because line breaks 
% do not look nice in a short entry.

\begin{clarifications}
  Rooted directed acyclic graphs with vertices labeled with set of formulas are constructed.
  \tikz[baseline=-.7ex]{\draw pic ["${\Gamma}$"] {rstate};} denotes a vertex labeled with $\Gamma$.
%  The obtained graphs correspond to Kripke models.
  To prove $A$, graphs are constructed from the initial graph containing only one vertex
  \tikz[baseline=-.7ex]{\draw pic ["${\neg A}$"] {rstate};} and no edges.
  For the modal logic K only rules $(\bot)$, $(\mathord\wedge)$, $(\mathord\vee_i)$, $(\Diamond)$ and
  $(K)$ are used.
  Then to each additional standart axiom T, 4, B, 5, D, De or C corresponds a set of rules with
  corresponding names.
%  For other normal modal logics, rules are added as follows:
%
%  \begin{tabular}{l l l}
%    Axiom & Model's property & Rules \\
%    \hline
%    $T = \Box A \imp A$ & reflexivity & $(T)$ \\
%    $4 = \Box A \imp \Box\Box A$ & transitivity & $(4)$ \\
%    $B = \Diamond\Box A \imp A$ & symmetry & $(B)$ \\
%    $5 = \Diamond\Box A \imp \Box A$ & euclideanity &
%      $(5_{\mathord\rightarrow})$, $(5_{\mathord\uparrow})$, $(5_{\mathord\downarrow})$ \\
%    $D = \Box A \imp \Diamond A$ & seriality & $(D)$ \\
%    $De = \Diamond A \imp \Diamond\Diamond A$ & density & $(De)$ \\
%    $C = \Diamond\Box A \imp \Box\Diamond A$ & confluence & $(C_0)$, $(C_1)$
%  \end{tabular}
\end{clarifications}

\begin{history}
  Introduced in\cite{castilho-farinas-gasquet-herzig.1997}.
% ToDo: write here short historical remarks about this proof system,
% especially if they relate to other proof systems. 
% Use "\iref{OtherProofSystem}" to refer to another proof system 
% in the Encyclopedia (where "OtherProofSystem" is its ID). 
% Use "\irefmissing{SuggestedIDForOtherProofSystem}" to refer to 
% another proof system that is not yet available in the encyclopedia.
\end{history}

\begin{technicalities}
  The method terminates and is sound complete for any combination of axioms.
% ToDo: write here remarks about soundness, completeness, decidability...
\end{technicalities}


% General Instructions:
% =====================

% The preferred length of an entry is 1 page. 
% Do the best you can to fit your proof system in one page.
%
% If you are finding it hard to fit what you want in one page, remember:
%
%   * Your entry needs to be neither self-contained nor fully understandable
%     (the interested reader may consult the cited full paper for details)
%
%   * If you are describing several proof systems in one entry, 
%     consider splitting your entry.
%
%   * You may reduce the size of your entry by ommitting inference rules
%     that are already described in other entries.
%
%   * Cite parsimoniously (see detailed citation instructions below).
%
% 
% If you do not manage to fit everything in one page, 
% it is acceptable for an entry to have 2 pages.
%
% For aesthetical reasons, it is preferable for an entry to have
% 1 full page or 2 full pages, in order to avoid unused blank space.



% Citation Instructions:
% ======================

% Please cite the original paper where the proof system was defined.
% To do so, you may use the \cite command within 
% one of the optional environments above,
% or use the \nocite command otherwise.

% You may also cite a modern paper or book where the 
% proof system is explained in greater depth or clarity.
% Cite parsimoniously.

% Do not cite related work. Instead, use the "\iref" or "\irefmissing" 
% commands to make an internal reference to another entry, 
% as explained within the "history" environment above.

% You do not need to create the "References" section yourself. 
% This is done automatically.




% Leave an empty line above "\end{entry}".

\end{entry}
