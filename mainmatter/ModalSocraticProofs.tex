
% If the calculus has an acronym, define it.
\newcommand{\EL}{\mathbf{E}^\mathsf{L}}

\calculusName{Socratic Proofs for Modal Propositional Logics}   % The name of the calculus
\calculusAcronym{\EL}    % The acronym if defined above, or empty otherwise. 
\calculusLogic{basic normal modal propositional logics}  % Specify the logic (e.g. classical, intuitionistic, ...) for which this calculus is intended.
\calculusType{other}   % Specify the calculus type (e.g. Frege-Hilbert style, tableau, sequent calculus, hypersequent calculus, natural deduction, ...)
\calculusYear{2004}   % The year when the calculus was invented.
\calculusAuthor{Dorota Leszczy\'nska-Jasion} % The name(s) of the author(s) of the calculus.


\entryTitle{Socratic Proofs for Modal Propositional Logics}     % Title of the entry (usually coincides with the name of the calculus).
\entryAuthor{Dorota Leszczy\'nska-Jasion}    % Your name(s). Separate multiple names with "\and".


\tag{Right-Sided Sequents}
\tag{Sequence Cedents}
\tag{Labelled System}


\maketitle

\begin{entry}{ModalSocraticProofs}  

\begin{calculus}

The rules of $\EL$ are the rules of $\EK$ (see \iref{ModalSocraticProofsK}), where the proviso of applicability of $\mathbf{R}_\mu$ depends on the logic $\mathsf{L}$ and is a combination of some of the following clauses:

\begin{enumerate}
\item\label{1} $\langle i, j \rangle$ is present in the premise sequent
\item\label{2} $i = j$
\item\label{3} $\langle j, i \rangle$ is present in the premise sequent
\item\label{4} there is a sequence $i_1, \ldots, i_n$ such that $i_1 = i$, $i_n = j$ and each $\langle i_k, i_{k+1} \rangle$, where $1 \geq k \geq n-1$, is present in the premise sequent
\item\label{5} there is a sequence $i_1, \ldots, i_n$ such that $i_1 = i$, $i_n = j$ and for each $\langle i_k, i_{k+1} \rangle$, where $1 \geq k \geq n-1$, $\langle i_k, i_{k+1} \rangle$ or $\langle i_{k+1}, i_{k} \rangle$ is present in the premise sequent
\item\label{6} there are sequences $i_1, \ldots, i_n$ and $j_1, \ldots, j_m$ such that $i_1 = 1$, $i_n = i$, $j_1 = 1$, $j_m = j$ and for each $\langle i_k, i_{k+1} \rangle$, where $1 \geq k \geq n-1$, $\langle i_k, i_{k+1} \rangle$ is present in the premise sequent and for each $\langle j_l, j_{l+1} \rangle$, where $1 \geq l \geq m-1$, $\langle j_l, j_{l+1} \rangle$ is present in the premise sequent
\end{enumerate}

\smallskip

%where the relevant proviso is:

\begin{center}
\begin{tabular}{c|l||c|l||c|l}
$\mathsf{L}$ 				& proviso & $\mathsf{L}$ 		& proviso & $\mathsf{L}$ 		& proviso \\
\hline
$\mathsf{K}$,$\mathsf{KD}$	& (\ref{1}) & $\mathsf{K4}$, $\mathsf{KD4}$ & (\ref{4}) & $\mathsf{K5}$, $\mathsf{D5}$ & (\ref{6}) \\
$\mathsf{KT}$				& (\ref{1}) or (\ref{2}) & $\mathsf{S4}$		& (\ref{2}) or (\ref{4}) & $\mathsf{K45}$, $\mathsf{D45}$ & (\ref{4}) or (\ref{6}) \\
$\mathsf{KB}$, $\mathsf{KDB}$ & (\ref{1}) or (\ref{3}) & $\mathsf{KB4}$	& (\ref{5}) && \\
$\mathsf{KTB}$				& (\ref{1}) or (\ref{2}) or (\ref{3}) & $\mathsf{S5}$	& (\ref{2}) or (\ref{5})
\end{tabular}
\end{center}

\smallskip

Calculi for logics: $\mathsf{KD}$, $\mathsf{KDB}$, $\mathsf{KD4}$, $\mathsf{KD5}$, $\mathsf{KD45}$ have also the following rule, where $j$ is new:

$$
\infer[\mathbf{R}_{\pi\mathsf{D}}]{?(\Phi ~;~ \vdash S ~'~ (\pi)^{\phi(i)} ~'~ (\pi_0)^{j} ~'~ T ~;~ \Psi)}{?(\Phi ~;~ \vdash S ~'~ (\pi)^{\phi(i)} ~'~ T ~;~ \Psi)}
$$
\end{calculus}

\begin{clarifications}
See \iref{ModalSocraticProofsK}, \iref{SocraticProofsCPL}, \iref{SocraticProofsFOL} for more comments.
\end{clarifications}

\begin{history}
The proof system has been presented in \cite{DLJ:2004}, the completeness proof may be found in \cite{DLJ:2007}, and extensions to some non-basic modal logics in \cite{DLJ:2008}.
\end{history}

\begin{technicalities}
A sequent $\vdash (A)^1$ has a Socratic proof in $\EL$ iff $A$ is $\mathsf{L}$-valid.
\end{technicalities}


% General Instructions:
% =====================

% The preferred length of an entry is 1 page. 
% Do the best you can to fit your proof system in one page.
%
% If you are finding it hard to fit what you want in one page, remember:
%
%   * Your entry needs to be neither self-contained nor fully understandable
%     (the interested reader may consult the cited full paper for details)
%
%   * If you are describing several proof systems in one entry, 
%     consider splitting your entry.
%
%   * You may reduce the size of your entry by ommitting inference rules
%     that are already described in other entries.
%
%   * Cite parsimoniously (see detailed citation instructions below).
%
% 
% If you do not manage to fit everything in one page, 
% it is acceptable for an entry to have 2 pages.
%
% For aesthetical reasons, it is preferable for an entry to have
% 1 full page or 2 full pages, in order to avoid unused blank space.



% Citation Instructions:
% ======================

% Please cite the original paper where the proof system was defined.
% To do so, you may use the \cite command within 
% one of the optional environments above,
% or use the \nocite command otherwise.

% You may also cite a modern paper or book where the 
% proof system is explained in greater depth or clarity.
% Cite parsimoniously.

% Do not cite related work. Instead, use the "\iref" or "\irefmissing" 
% commands to make an internal reference to another entry, 
% as explained within the "history" environment above.

% You do not need to create the "References" section yourself. 
% This is done automatically.




% Leave an empty line above "\end{entry}".

\end{entry}
