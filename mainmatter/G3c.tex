
% If the calculus has an acronym, define it.
% (e.g. \newcommand{\LK}{\ensuremath{\mathbf{LK}}\xspace})
\newcommand{\Gtc}{\ensuremath{\mathbf{G3c}}\xspace}

\calculusName{G3c}   % The name of the calculus
\calculusAcronym{\Gtc}    % The acronym if defined above, or empty otherwise. 
\calculusYear{1996}   % The year when the calculus was invented.
\calculusAuthor{A.S.~Troelstra \and H.~Schwichtenberg} % The name(s) of the author(s) of the calculus.
\entryAuthor{}    % Your name(s). Separate multiple names with "\and"

\maketitle


% If your files are called "<ID>.tex" and "<ID>.bib", 
% then you should write "\begin{entry}{<ID>}" in the line below
\begin{entry}{G3c}  

% Define here any newcommands you may need:
% e.g. \newcommand{\necessarily}{\Box}
% e.g. \newcommand{\possibly}{\Diamond}

\newcommand{\rarr}{\rightarrow}

\begin{calculus}
% Add the inference rules of your proof system here.
% The "proof.sty" and "bussproofs.sty" packages are available.
% If you need any other package, please contact the editor (bruno@logic.at)
  \[
  \begin{array}{l@{\qquad}l}
      \infer[\mathrm{Ax}]{P, \Gamma \seq \Delta, P
  }{}
  &
  \infer[\mathrm{L}\bot]{\bot, \Gamma \seq \Delta}{}
  \medskip\\
  \infer[\mathrm{L}\land]{A \land B, \Gamma \seq \Delta}{A, B, \Gamma \seq \Delta}
  &
  \infer[\mathrm{R}\land]{\Gamma \seq \Delta, A\land B}{\Gamma \seq \Delta,A &
    \Gamma \seq \Delta, B}
  \medskip\\
  \infer[\mathrm{L}\lor]{A\lor B, \Gamma \seq \Delta}{A, \Gamma \seq \Delta & B, \Gamma \seq \Delta}
  &
  \infer[\mathrm{R}\lor]{\Gamma \seq \Delta, A\lor B}{\Gamma \seq \Delta, A, B}
  \medskip\\
  \infer[\mathrm{L}\rarr]{A \rarr B, \Gamma \seq \Delta}{\Gamma \seq \Delta, A
    & B, \Gamma \seq \Delta}
  &
  \infer[\mathrm{R}\rarr]{\Gamma \seq \Delta, A \rarr B}{A, \Gamma \seq \Delta,
  B}
  \medskip\\
  \infer[\mathrm{L}\forall]{\forall x A, \Gamma \seq \Delta}{\forall x A,
    A[x/t], \Gamma \seq \Delta}
  &
  \infer[\mathrm{R}\forall]{\Gamma \seq \Delta, \forall x A}{\Gamma \seq
    \Delta, A[x/y]}
  \medskip\\
  \infer[\mathrm{L}\exists]{\exists x A, \Gamma \seq \Delta}{A [ x/ y], \Gamma \seq \Delta}
  &
  \infer[\mathrm{R}\exists]{\Gamma \seq \Delta, \exists x A}{\Gamma \seq
    \Delta, A[x/t], \exists x A}
  \end{array}
  \]
\centerline{\small with $P$ in $\mathrm{Ax}$ atomic and $y$ not free
  in the conclusion in $\mathrm{R}\forall, \;\mathrm{L}\exists$}
\end{calculus}

% The following environments ("clarifications", "history", 
% "technicalities") are optional. If you do use them, 
% be very concise and objective.

\begin{clarifications}
% ToDo: write here short remarks that may help the reader to understand 
% the inference rules of the proof system.
  Sequents are based on multisets. A formula $A[x/t]$ is the result of
  uniformly substituting the term $t$ for the variable $x$ in
  $A$.\nocite{Troelstra:2000}
\end{clarifications}

% \begin{history}
% ToDo: write here short historical remarks about this proof system,
% especially if they relate to other proof systems. 
% Use "\iref{OtherProofSystem}" to refer to another proof system 
% in the Encyclopedia (where "OtherProofSystem" is its ID). 
% Use "\irefmissing{SuggestedIDForOtherProofSystem}" to refer to 
% another proof system that is not yet available in the encyclopedia.
% \end{history}

\begin{technicalities}
% ToDo: write here remarks about soundness, completeness, decidability...
  Sound and complete wrt.\ classical first-order logic.  Weakening and
  contraction are depth-preserving admissible and all the rules are
  depth-preserving invertible.
\end{technicalities}



% Please cite the original paper where the proof system was defined.
% To do so, you may use the \cite command within 
% one of the optional environments above,
% or use the \nocite command otherwise.

% You may also cite a modern paper or book where the 
% proof system is explained in greater depth or clarity.
% Cite parsimoniously.

% Do not cite related work. Instead, use the "\iref" or "\irefmissing" 
% commands to make an internal reference to another entry, 
% as explained within the "history" environment above.

% You do not need to create the "References" section yourself. 
% This is done automatically.

\end{entry}

%%% Local Variables: 
%%% mode: latex
%%% TeX-master: "../main"
%%% End: 
