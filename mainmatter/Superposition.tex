
\calculusName{Superposition}   % The name of the calculus
\calculusAcronym{}    % The acronym if defined above, or empty otherwise. 
\calculusLogic{classical}  % Specify the logic (e.g. classical, intuitionistic, ...) for which this calculus is intended.
\calculusType{saturation}   % Specify the calculus type (e.g. Frege-Hilbert style, tableau, sequent calculus, hypersequent calculus, natural deduction, ...)
\calculusYear{1990/1994}   % The year when the calculus was invented.
\calculusAuthor{Leo Bachmair, Harald Ganzinger} % The name(s) of the author(s) of the calculus.


\entryTitle{Superposition}     % Title of the entry (usually coincides with the name of the calculus).
\entryAuthor{Uwe Waldmann}    % Your name(s). Separate multiple names with "\and".


% If you wish, use tags to give any other information 
% that might be helpful for classifying and grouping this entry:
% e.g. \tag{Two-Sided Sequents}
% e.g. \tag{Multiset Cedents}
% e.g. \tag{List Cedents}
% You are free to invent your own tags. 
% The Encyclopedia's coordinator will take care of 
% merging semantically similar tags in the future.


\maketitle


% If your files are called "MyProofSystem.tex" and "MyProofSystem.bib", 
% then you should write "\begin{entry}{MyProofSystem}" in the line below
\begin{entry}{Superposition}  

% Define here any newcommands you may need:
% e.g. \newcommand{\necessarily}{\Box}
% e.g. \newcommand{\possibly}{\Diamond}


\begin{calculus}

% Add the inference rules of your proof system here.
% The "proof.sty" and "bussproofs.sty" packages are available.
% If you need any other package, please contact the editor (bruno@logic.at)

\[
\infer[\textit{Equality Resolution}]
{C\sigma}{C \lor \neg u \approx v\vphantom{[]}}
\]
\[
\infer[\textit{Negative Superposition}]
{(D \lor C \lor \neg t[u'] \approx t')\sigma}{D \lor u \approx u'
& C \lor \neg t[v] \approx t'}
\]
\[
\infer[\textit{Positive Superposition}]
{(D \lor C \lor t[u'] \approx t')\sigma}{D \lor u \approx u'
& C \lor t[v] \approx t'}
\]
plus either
\[
\infer[\textit{Merging Paramodulation}]
{(D \lor C \lor s \approx s' \lor t \approx t'[u'])\sigma}{D \lor u \approx u'
& C \lor s \approx s' \lor t \approx t'[v]}
\]
\[
\infer[\textit{Ordered Factoring}]
{(C \lor s \approx u)\sigma}{C \lor s \approx u \lor t \approx v}
\]
or
\[
\infer[\textit{Equality Factoring}]
{(C \lor \neg u' \approx v' \lor v \approx v')\sigma}{C \lor v \approx v' \lor u \approx u'\vphantom{[]}}
\]
$C,D$ are (possibly empty) equational clauses,
$s,s',t,t',u,u',v,v'$ are terms,
$u$ and $v$ (and, for \textit{Ordered Factoring}
and \textit{Merging Paramodulation}, $s$ and $t$)
are unifiable with most general unifier~$\sigma$.
In all binary inferences, $v$ is not a variable.

Except for the last but one literal in
\textit{Equality Factoring} and \textit{Merging Paramodulation} inferences,
every literal involved in some inference
is maximal in the respective premise
(strictly maximal, if the literal is positive and the inference is binary).
In every literal involved in some inference
(except \textit{Equality Resolution}),
the lhs is strictly maximal.
Optionally,
ordering restrictions can be overridden by \emph{selection functions}.

For simplicity, it is assumed that the equality predicate $\approx$
is the only predicate symbol in the signature.
Non-equational atoms $P(t_1,\dots,t_n)$ can be encoded as
equations $P(t_1,\dots,t_n) \approx \mathit{true}$.

% \bigskip
% 
% \textbf{Saturation}
% \[
% \infer
% {N \cup \{C\}}{N & C \textrm{~is the conclusion of a Resolution inference
% from clauses in~} N}
% \]
% \mbox{}\quad $N$ is a finite set of clauses,
% $C$ is a clause.
\end{calculus}

% The following sections ("clarifications", "history", 
% "technicalities") are optional. If you use them, 
% be very concise and objective. Nevertheless, do write full sentences. 
% Try to have at most one paragraph per section, because line breaks 
% do not look nice in a short entry.

\begin{clarifications}
Superposition is a refutational saturation calculus for
first-order clauses (disjunctions of possibly negated atoms)
with equality.
The inference rules are supplemented by a redundancy criterion
that permits to delete clauses that are unnecessary for
deriving a contradiction during the saturation, see \iref{SaturationWithRed}.
\end{clarifications}

\begin{history}
The superposition
calculus~\cite{BachmairGanzinger1990CTRS,BachmairGanzinger1994JLC}
by Bachmair and Ganzinger
refines the paramodulation calculus~\iref{Paramodulation}.
It uses a syntactic ordering on terms and
literals to restrict the paramodulation inference rules
in such a way that only (strictly) maximal sides of (strictly) maximal
literals participate in inferences.
In order to preserve refutational completeness, one new
inference rule must be added -- either
the \textit{Merging Paramodulation} rule~\cite{BachmairGanzinger1990CTRS}
or the \textit{Equality Factoring} rule originally due to Nieuwenhuis
(which then subsumes \textit{Ordered Factoring}).

The superposition calculus is the basis of most current automated
theorem provers for full first-order logic with equality,
such as E, SPASS, or Vampire.
The calculus and the \emph{model construction} technique used
to prove its refutational completeness have been a prototype
for numerous refinements, such as
constraint superposition~\iref{ConstraintSup},
theory superposition~\iref{CancellativeSup},
% ordered chaining~\irefmissing{OrderedChaining},
or hierarchic superposition~\iref{HierarchicSup}.

\end{history}

\begin{technicalities}
The superposition calculus is refutationally complete for
first-order logic with equality.
For certain fragments of first-order logic with equality,
there exist strategies that guarantee termination of the calculus,
turning superposition into a decision procedure for these fragments.
\end{technicalities}


% General Instructions:
% =====================

% The preferred length of an entry is 1 page. 
% Do the best you can to fit your proof system in one page.
%
% If you are finding it hard to fit what you want in one page, remember:
%
%   * Your entry needs to be neither self-contained nor fully understandable
%     (the interested reader may consult the cited full paper for details)
%
%   * If you are describing several proof systems in one entry, 
%     consider splitting your entry.
%
%   * You may reduce the size of your entry by ommitting inference rules
%     that are already described in other entries.
%
%   * Cite parsimoniously (see detailed citation instructions below).
%
% 
% If you do not manage to fit everything in one page, 
% it is acceptable for an entry to have 2 pages.
%
% For aesthetical reasons, it is preferable for an entry to have
% 1 full page or 2 full pages, in order to avoid unused blank space.



% Citation Instructions:
% ======================

% Please cite the original paper where the proof system was defined.
% To do so, you may use the \cite command within 
% one of the optional environments above,
% or use the \nocite command otherwise.

% You may also cite a modern paper or book where the 
% proof system is explained in greater depth or clarity.
% Cite parsimoniously.

% Do not cite related work. Instead, use the "\iref" or "\irefmissing" 
% commands to make an internal reference to another entry, 
% as explained within the "history" environment above.

% You do not need to create the "References" section yourself. 
% This is done automatically.




% Leave an empty line above "\end{entry}".

\end{entry}
