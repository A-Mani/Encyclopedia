
% If the calculus has an acronym, define it.
% (e.g. \newcommand{\LK}{\ensuremath{\mathbf{LK}}\xspace})

\newcommand{\LK}{\ensuremath{\mathbf{LK}}\xspace}

\calculusName{Sequent Calculus for Classical Logic }   % The name of the calculus
\calculusAcronym{\LK}    % The acronym if defined above, or empty otherwise. 
\calculusYear{1935}   % The year when the calculus was invented.
\calculusAuthor{Gerhard Gentzen} % The name(s) of the author(s) of the calculus.
\entryAuthor{Martin Riener}    % Your name(s). Separate multiple names with "\and"

\maketitle


% If your files are called "<ID>.tex" and "<ID>.bib", 
% then you should write "\begin{entry}{<ID>}" in the line below
\begin{entry}{GentzenLK}  

% Define here any newcommands you may need:
% e.g. \newcommand{\necessarily}{\Box}
% e.g. \newcommand{\possibly}{\Diamond}

\newcommand{\lkproves}{\ensuremath{\vdash}}
\renewcommand{\fCenter}{\lkproves}

\newcommand{\twocolumns}[2]{
\begin{minipage}[b]{0.5\linewidth}
\centering
#1
\end{minipage}
\hspace{0.5cm}
\begin{minipage}[b]{0.5\linewidth}
\centering
#2
\end{minipage}
}

\newcommand{\threecolumns}[3]{
\begin{minipage}[b]{0.3\linewidth}
\centering
#1
\end{minipage}
\hspace{0.5cm}
\begin{minipage}[b]{0.3\linewidth}
\centering
#2
\end{minipage}
\hspace{0.5cm}
\begin{minipage}[b]{0.3\linewidth}
\centering
#3
\end{minipage}
}

\newcommand{\fourcolumns}[4]{
\begin{minipage}[b]{0.22\linewidth}
\centering
#1
\end{minipage}
\hspace{0.5cm}
\begin{minipage}[b]{0.22\linewidth}
\centering
#2
\end{minipage}
\hspace{0.5cm}
\begin{minipage}[b]{0.22\linewidth}
\centering
#3
\end{minipage}
\begin{minipage}[b]{0.22\linewidth}
\centering
#4
\end{minipage}
}



\newcommand{\LKAX}[2]{\AxiomC{\ensuremath{#1} \fCenter \ensuremath{#2}}}
\newcommand{\LKUI}[2]{\UnaryInfC{\ensuremath{#1} \fCenter \ensuremath{#2}}}
\newcommand{\LKBI}[2]{\BinaryInfC{\ensuremath{#1} \fCenter \ensuremath{#2}}}
\newcommand{\LKLL}[1]{\LeftLabel{\footnotesize \ensuremath{#1}}}
\newcommand{\LKRL}[1]{\RightLabel{\footnotesize \ensuremath{#1}}}
\newcommand{\LKRLN}[1]{\RightLabel{#1}}

\newcommand{\SALLL}{\LKRL{\forall:l}}
\newcommand{\SALLR}{\LKRL{\forall:r\,(*)}}
\newcommand{\SEXL}{\LKRL{\exists:l\,(*)}}
\newcommand{\SEXR}{\LKRL{\exists:r}}
\newcommand{\SANDL}{\LKRL{\land:l}}
\newcommand{\SANDR}{\LKRL{\land:r}}
\newcommand{\SORL}{\LKRL{\lor:l}}
\newcommand{\SORR}{\LKRL{\lor:r}}
\newcommand{\SIMPL}{\LKRL{\imp:l}}
\newcommand{\SIMPR}{\LKRL{\imp:r}}
\newcommand{\SNEGL}{\LKRL{\neg:l}}
\newcommand{\SNEGR}{\LKRL{\neg:r}}
\newcommand{\SWEAKL}{\LKRL{w:l}}
\newcommand{\SWEAKR}{\LKRL{w:r}}
\newcommand{\SCONTRL}{\LKRL{c:l}}
\newcommand{\SCONTRR}{\LKRL{c:r}}
\newcommand{\SEXCHL}{\LKRL{e:l}}
\newcommand{\SEXCHR}{\LKRL{e:r}}
\newcommand{\SCUT}{\LKRL{cut}}
\newcommand{\SDEF}{\LKRL{def}}

\newcommand{\ALLL}     [3]{\SALLL \LKUI{#2}{#3} }
\newcommand{\ALLR}     [3]{\SALLR \LKUI{#2}{#3} }
\newcommand{\EXL}      [3]{\SEXL  \LKUI{#2}{#3} }
\newcommand{\EXR}      [3]{\SEXR  \LKUI{#2}{#3} }
\newcommand{\ANDL}     [2]{\SANDL \LKUI{#1}{#2} }
\newcommand{\ANDR}     [2]{\SANDR \LKBI{#1}{#2} }
\newcommand{\ORL}      [2]{\SORL  \LKBI{#1}{#2} }
\newcommand{\ORR}      [2]{\SORR  \LKUI{#1}{#2} }
\newcommand{\IMPL}     [2]{\SIMPL \LKBI{#1}{#2}}
\newcommand{\IMPR}     [2]{\SIMPR \LKUI{#1}{#2}}
\newcommand{\NEGL}     [2]{\SNEGL \LKUI{#1}{#2}}
\newcommand{\NEGR}     [2]{\SNEGR \LKUI{#1}{#2}}
\newcommand{\EQL}      [2]{\SEQL  \LKBI{#1}{#2}}
\newcommand{\EQR}      [2]{\SEQR  \LKBI{#1}{#2}}
\newcommand{\WEAKL}    [2]{\SWEAKL \LKUI{#1}{#2}}
\newcommand{\WEAKR}    [2]{\SWEAKR \LKUI{#1}{#2}}
\newcommand{\EXCHL}    [2]{\SEXCHL \LKUI{#1}{#2}}
\newcommand{\EXCHR}    [2]{\SEXCHR \LKUI{#1}{#2}}
\newcommand{\CONTRL}   [2]{\SCONTRL \LKUI{#1}{#2}}
\newcommand{\CONTRR}   [2]{\SCONTRR \LKUI{#1}{#2}}
\newcommand{\CUT}      [2]{\SCUT    \LKBI{#1}{#2}}


\begin{calculus}

% Add the inference rules of your proof system here.
% The "proof.sty" and "bussproofs.sty" packages are available.
% If you need any other package, please contact the editor (bruno@logic.at)
\textbf{Structural rules:}\\
\twocolumns{
  \AxiomC{}
  \LKRL{Axiom}
  \LKUI{D}{D}
  \DisplayProof
}{
  \LKAX{\Gamma}{\Theta,D}
  \LKAX{\Gamma,D}{\Theta}
  \CUT{\Gamma}{\Theta}
  \DisplayProof
}
\\

\fourcolumns{
  \LKAX{\Gamma}{\Theta}
  \WEAKL{D, \Gamma}{\Theta}
  \DisplayProof
}{
  \LKAX{\Gamma}{\Theta}
  \WEAKR{\Gamma}{\Theta, D}
  \DisplayProof
}{
  \LKAX{D,D,\Gamma}{\Theta}
  \CONTRL{D, \Gamma}{\Theta}
  \DisplayProof
}{
  \LKAX{\Gamma}{\Theta,D,D}
  \CONTRR{\Gamma}{\Theta, D}
  \DisplayProof
}
\\


\twocolumns{
  \LKAX{\Gamma,D,E,\Delta}{\Theta}
  \EXCHL{\Gamma, E, D, \Delta}{\Theta}
  \DisplayProof
}{
  \LKAX{\Gamma}{\Theta,D,E, \Lambda}
  \EXCHR{\Gamma}{\Theta, E, D, \Lambda}
  \DisplayProof
 
}


\textbf{Logical rules:}\\
\twocolumns{
\LKAX{A,\Gamma}{\Theta}
\ANDL{A \land B,\Gamma}{\Theta}
\DisplayProof
}{
\LKAX{B,\Gamma}{\Theta}
\ANDL{A \land B,\Gamma}{\Theta}
\DisplayProof
}


\twocolumns{
\LKAX{\Gamma}{\Theta,A}
\ORR{\Gamma}{\Theta,A\lor B}
\DisplayProof
}{
\LKAX{\Gamma}{\Theta,B}
\ORR{\Gamma}{\Theta,A\lor B}
\DisplayProof
}

\twocolumns{
\LKAX{A,  \Gamma}{\Theta}
\LKAX{B,  \Gamma}{\Theta}
\ORL{A \lor B\Gamma}{\Theta}
\DisplayProof
}{
\LKAX{\Gamma}{\Theta,A}
\LKAX{\Gamma}{\Theta,B}
\ANDR{\Gamma}{\Theta,A \land B}
\DisplayProof
}

\twocolumns{
\LKAX{\Gamma}{\Theta, A}
\LKAX{B,  \Delta}{\Lambda}
\IMPL{A \imp B\Gamma}{\Theta}
\DisplayProof
}{
\LKAX{A, \Gamma}{\Theta, B}
\IMPR{\Gamma}{\Theta,A \imp B}
\DisplayProof
}

\twocolumns{
\LKAX{A,\Gamma}{\Theta}
\NEGR{\Gamma}{\Theta, \neg A}
\DisplayProof
}{
\LKAX{\Gamma}{\Theta,A}
\NEGL{ \neg A,\Gamma}{\Theta}
\DisplayProof
}

\twocolumns{
  \LKAX{F\,a, \Gamma}{\Theta}
  \ALLL{}{\forall x\,F x, \Gamma}{\Theta}
  \DisplayProof
}{
  \LKAX{\Gamma}{\Theta,F\,a}
  \EXR{}{\Gamma}{\Theta,\exists x\,F x}
  \DisplayProof
}

\twocolumns{
  \LKAX{F\,a, \Gamma}{\Theta}
  \EXL{}{\exists x\,F x, \Gamma}{\Theta}
  \DisplayProof
}{
  \LKAX{\Gamma}{\Theta,F\,a}
  \ALLR{}{\Gamma}{\Theta,\forall x\,F x}
  \DisplayProof
}

$(*)$: Eigenvariable condition

%} % end centering
\end{calculus}

% The following environments ("clarifications", "history", 
% "technicalities") are optional. If you do use them, 
% be very concise and objective.

 \begin{clarifications}
   In all rules, $A,B,D,E,F$ are arbitrary formulas, $\Gamma,\Theta,\Delta,\Lambda$ are lists of arbitrary formulas, $a$ is a free variable and $x$ a bound variable. Within the quantifier rules, $F a$ is obtained from $F x$ by applying the substitution $\{ a/x \} $. Additionally, the rules $\forall:r$ and $\exists:l$ must fulfill the so called Eigenvariable condition $(*)$. That means that the quantified variable $x$ may not occur in the conclusion $\Gamma$, $\Delta$ and $\forall x\, F x$ respectively $\exists x\, F x$.


% ToDo: write here short remarks that may help the reader to understand 
% the inference rules of the proof system.
 \end{clarifications}

 \begin{history}
This is G. Gentzen's formulation of sequent calculus as defined in his PhD thesis which was later on published in the journal ``Mathematische Zeitschrift''\cite{Gentzen1934}. The term language in this formulation only has free and bound variables. A version of \LK with terms is presented in G. Takeuti's book on proof theory\cite{Takeuti1975}. Restricting the right hand side to only one formula, we obtain the intuitionistic calculus $LJ$\irefmissing{GentzenLJ}. The variant $LK'$ has only invertible non-quantifier rules\irefmissing{LKprime} allowing better proof search. J{-}Y. Girard 's linear logic\cite{Girard1987} allows to formula purely additive or multiplicative rules, wheras Gentzen defined a mixed calculus.
% ToDo: write here short historical remarks about this proof system,
% especially if they relate to other proof systems. 
% Use "\iref{OtherProofSystem}" to refer to another proof system 
% in the Encyclopedia (where "OtherProofSystem" is its ID). 
% Use "\irefmissing{SuggestedIDForOtherProofSystem}" to refer to 
% another proof system that is not yet available in the encyclopedia.
\end{history}

% \begin{technicalities}
% ToDo: write here remarks about soundness, completeness, decidability...
% \end{technicalities}



% Please cite the original paper where the proof system was defined.
% To do so, you may use the \cite command within 
% one of the optional environments above,
% or use the \nocite command otherwise.

% You may also cite a modern paper or book where the 
% proof system is explained in greater depth or clarity.
% Cite parsimoniously.

% Do not cite related work. Instead, use the "\iref" or "\irefmissing" 
% commands to make an internal reference to another entry, 
% as explained within the "history" environment above.

% You do not need to create the "References" section yourself. 
% This is done automatically.

\end{entry}
