
% If the calculus has an acronym, define it.
% (e.g. \newcommand{\LK}{\ensuremath{\mathbf{LK}}\xspace})

\newcommand{\LK}{\ensuremath{\mathbf{LK}}\xspace}

\calculusName{Gentzen Sequent }   % The name of the calculus
\calculusAcronym{}    % The acronym if defined above, or empty otherwise. 
\calculusYear{1934}   % The year when the calculus was invented.
\calculusAuthor{Gerhard Gentzen} % The name(s) of the author(s) of the calculus.
\entryAuthor{Martin Riener}    % Your name(s). Separate multiple names with "\and"

\maketitle


% If your files are called "<ID>.tex" and "<ID>.bib", 
% then you should write "\begin{entry}{<ID>}" in the line below
\begin{entry}{GentzenLK}  

% Define here any newcommands you may need:
% e.g. \newcommand{\necessarily}{\Box}
% e.g. \newcommand{\possibly}{\Diamond}

\newcommand{\lkproves}{\ensuremath{\rightarrow}}
\renewcommand{\fCenter}{\lkproves}

\newcommand{\LKAX}[2]{\AxiomC{\ensuremath{#1} \fCenter \ensuremath{#2}}}
\newcommand{\LKUI}[2]{\UnaryInfC{\ensuremath{#1} \fCenter \ensuremath{#2}}}
\newcommand{\LKBI}[2]{\BinaryInfC{\ensuremath{#1} \fCenter \ensuremath{#2}}}
\newcommand{\LKLL}[1]{\LeftLabel{\footnotesize \ensuremath{#1}}}
\newcommand{\LKRL}[1]{\RightLabel{\footnotesize \ensuremath{#1}}}
\newcommand{\LKRLN}[1]{\RightLabel{#1}}

\newcommand{\SALLL}{\RL{\forall:l}}
\newcommand{\SALLR}{\RL{\forall:r}}
\newcommand{\SEXL}{\RL{\exists:l}}
\newcommand{\SEXR}{\RL{\exists:r}}
\newcommand{\SANDL}{\RL{\land:l}}
\newcommand{\SANDR}{\RL{\land:r}}
\newcommand{\SORL}{\RL{\lor:l}}
\newcommand{\SORR}{\RL{\lor:r}}
\newcommand{\SIMPL}{\RL{\impl:l}}
\newcommand{\SIMPR}{\RL{\impl:r}}
\newcommand{\SNEGL}{\RL{\neg:l}}
\newcommand{\SNEGR}{\RL{\neg:r}}
\newcommand{\SWEAKL}{\RL{w:l}}
\newcommand{\SWEAKR}{\RL{w:r}}
\newcommand{\SCONTRL}{\RL{c:l}}
\newcommand{\SCONTRR}{\RL{c:r}}
\newcommand{\SCUT}{\RL{cut}}
\newcommand{\SDEF}{\RL{def}}

\begin{calculus}

% Add the inference rules of your proof system here.
% The "proof.sty" and "bussproofs.sty" packages are available.
% If you need any other package, please contact the editor (bruno@logic.at)





\end{calculus}

% The following environments ("clarifications", "history", 
% "technicalities") are optional. If you do use them, 
% be very concise and objective.

% \begin{clarifications}
% ToDo: write here short remarks that may help the reader to understand 
% the inference rules of the proof system.
% \end{clarifications}

% \begin{history}
% ToDo: write here short historical remarks about this proof system,
% especially if they relate to other proof systems. 
% Use "\iref{OtherProofSystem}" to refer to another proof system 
% in the Encyclopedia (where "OtherProofSystem" is its ID). 
% Use "\irefmissing{SuggestedIDForOtherProofSystem}" to refer to 
% another proof system that is not yet available in the encyclopedia.
% \end{history}

% \begin{technicalities}
% ToDo: write here remarks about soundness, completeness, decidability...
% \end{technicalities}



% Please cite the original paper where the proof system was defined.
% To do so, you may use the \cite command within 
% one of the optional environments above,
% or use the \nocite command otherwise.

% You may also cite a modern paper or book where the 
% proof system is explained in greater depth or clarity.
% Cite parsimoniously.

% Do not cite related work. Instead, use the "\iref" or "\irefmissing" 
% commands to make an internal reference to another entry, 
% as explained within the "history" environment above.

% You do not need to create the "References" section yourself. 
% This is done automatically.

\end{entry}
