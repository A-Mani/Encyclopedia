
\calculusName{Classical Sequent Calculus}  
\calculusAcronym{\LK}  
\calculusLogic{Classical Predicate Logic}
\calculusType{Sequent Calculus}
\calculusYear{1935}   
\calculusAuthor{Gerhard Gentzen} 

\entryTitle{Classical Sequent Calculus \LK}
\entryAuthor{Martin Riener}   

\tag{Two-Sided Sequents}
\tag{Multiset Cedents}
\tag{Multi-Conclusion Succedent}

\maketitle

\begin{entry}{GentzenLK}  

\begin{calculus}

\[
\begin{array}{cc}
\infer{A \vdash A}{}
&
\infer[cut]{\Gamma, \Delta \vdash \Lambda, \Theta}{\Gamma \vdash \Lambda, A & A, \Delta \vdash \Theta}
\\[8pt]
\infer[w_l]{A, \Gamma \vdash \Theta}{\Gamma \vdash \Theta}
&
\infer[w_r]{\Gamma \vdash \Theta, A}{\Gamma \vdash \Theta}
\\[8pt]
\infer[e_l]{\Gamma, A, B, \Delta \vdash \Theta}{\Gamma, B, A, \Delta \vdash \Theta}
\quad%&
\infer[c_l]{A, \Gamma \vdash \Theta}{A, A, \Gamma \vdash \Theta}
\quad
&%\\[8pt]
\quad
\infer[e_r]{\Gamma \vdash \Theta, A, B, \Delta}{\Gamma \vdash \Theta, B, A, \Delta}
\quad%&
\infer[c_r]{\Gamma \vdash \Theta, A}{\Gamma \vdash \Theta, A, A}
\\[8pt]
\infer[\neg_l]{\neg A, \Gamma \vdash \Theta}{\Gamma \vdash \Theta, A}
&
\infer[\neg_r]{\Gamma \vdash \Theta, \neg A}{A, \Gamma \vdash \Theta}
\\[8pt]
\infer[\wedge_{l}]{A_1 \wedge A_2, \Gamma \vdash \Theta}{A_i, \Gamma \vdash \Theta}
&
\infer[\wedge_r]{\Gamma \vdash \Theta, A \wedge B}{\Gamma \vdash \Theta, A & \Gamma \vdash \Theta,  B}
\\[8pt]
\infer[\vee_l]{A \vee B, \Gamma \vdash \Theta}{A, \Gamma \vdash \Theta & B, \Gamma \vdash \Theta}
&
\infer[\vee_{r}]{\Gamma \vdash \Theta, A_1 \vee A_2}{\Gamma \vdash \Theta, A_i}
\\[8pt]
\infer[\rightarrow_l]{A \rightarrow B, \Gamma, \Delta \vdash \Lambda, \Theta}{\Gamma \vdash \Lambda, A & B, \Delta \vdash \Theta}
&
\infer[\rightarrow_r]{\Gamma \vdash \Theta, A \rightarrow B}{A, \Gamma \vdash \Theta, B}
\\[8pt]
\infer[\exists_l]{\exists x.A[x], \Gamma \vdash \Theta}{A[\alpha], \Gamma \vdash \Theta}
\quad
\infer[\forall_l]{\forall x.A[x], \Gamma \vdash \Theta}{A[t], \Gamma \vdash \Theta}
\ \
&
\ \
\infer[\forall_r]{\Gamma \vdash \Theta, \forall x.A[x]}{\Gamma \vdash \Theta, A[\alpha]}
\quad
\infer[\exists_r]{\Gamma \vdash \Theta, \exists x.A[x]}{\Gamma \vdash \Theta, A[t]}
\\
\end{array}
\]

\centering
The eigenvariable $\alpha$ should not occur in $\Gamma$, $\Theta$ or $A[x]$. \\ 
The term $t$ should not contain variables bound in $A[t]$.
\end{calculus}


\begin{history}
This is a modern presentation of Gentzen's original \LK calculus\cite{lk:Gentzen1935}, using modern notations and rule names.
\end{history}

\newcommand{\LHK}{\ensuremath{\mathbf{LHK}}\xspace}
\newcommand{\NK}{\ensuremath{\mathbf{NK}}\xspace}


\begin{technicalities}
\LK is complete relative to \NK (i.e. \NJ \iref{GentzenNJ} with the axiom of excluded middle) and sound relative to a Hilbert-style calculus \LHK \cite{lk:Gentzen1935a}. Cut is eliminable (\emph{Hauptsatz} \cite{lk:Gentzen1935}), and hence classical predicate logic is consistent. Any \emph{prenex} cut-free proof may be further transformed into a shape with only propositional inferences above and only quantifier and structural inferences below a \emph{midsequent} \cite{lk:Gentzen1935a}.
\end{technicalities}

\end{entry}

