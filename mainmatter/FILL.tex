
% If the calculus has an acronym, define it.
% (e.g. \newcommand{\LK}{\ensuremath{\mathbf{LK}}\xspace})

\calculusName{Full Intuitionistic Linear Logic}   % The name of the calculus
\calculusAcronym{\FILL}    % The acronym if defined above, or empty otherwise. 
\calculusLogic{Intuitionistic,Linear}  % Specify the logic (e.g. classical, intuitionistic, ...) for which this calculus is intended.
\calculusType{Sequent Calculus}   % Specify the calculus type (e.g. Frege-Hilbert style, tableau, sequent calculus, hypersequent calculus, natural deduction, ...)
\calculusYear{1990}   % The year when the calculus was invented.
\calculusAuthor{Martin Hyland \and Valeria de Paiva} % The name(s) of the author(s) of the calculus.

\entryTitle{Full Intuitionistic Linear Logic (FILL)}        % Title of the entry (usually coincides with the name of the calculus).
\entryAuthor{Harley Eades III \and Valeria de Paiva} % Your name(s). Separate multiple names with "\and".

% If you wish, use tags to give any other information 
% that might be helpful for classifying and grouping this entry:
% e.g. \tag{Two-Sided Sequents}
% e.g. \tag{Multiset Cedents}
% e.g. \tag{List Cedents}
% You are free to invent your own tags. 
% The Encyclopedia's coordinator will take care of 
% merging semantically similar tags in the future.
\tag{Two-Sided Sequents}
\tag{Term Assignment}


\maketitle


% If your files are called "MyProofSystem.tex" and "MyProofSystem.bib", 
% then you should write "\begin{entry}{MyProofSystem}" in the line below
\begin{entry}{FILL}  

% Define here any newcommands you may need:
% e.g. \newcommand{\necessarily}{\Box}
% e.g. \newcommand{\possibly}{\Diamond}
\newcommand{\letbe}[3]{\mathsf{let}\,#1\,\mathsf{be}\,#2\,\mathsf{in}\,#3}
\newcommand{\letpat}[3]{\mathsf{let}\mbox{-}\mathsf{pat}\,#1\,#2\,#3}

\begin{calculus}
\[
\begin{array}{ccc}
\infer[Ax]{x : A \vdash x : A}{}
&
\quad
&
\infer[cut]{\Gamma, \Gamma' \vdash \Delta \mid [t/y]\Delta'}{\Gamma \vdash t : A \mid \Delta & \Gamma', y : A \vdash \Delta'}
\\[8pt]
\infer[IL]{\Gamma,x : \top \vdash \letbe{x}{*}{\Delta}}{\Gamma \vdash \Delta}
&
\quad
&
\infer[IR]{\cdot \vdash * : \top}{}
\\[8pt]
\infer[PL]{x : \perp \vdash \cdot}{}
& \quad & 
\infer[PR]{\Gamma \vdash \circ : \perp \mid \Delta}{\Gamma \vdash \Delta}
\\[8pt]
\infer[TL]{\Gamma,z : A \otimes B \vdash \letbe{z}{x \otimes y}{\Delta}}{\Gamma,x : A,y : B \vdash \Delta}
&
\quad
&
\infer[TR]{\Gamma,\Gamma' \vdash t_1 \otimes t_2 : A \otimes B \mid \Delta \mid \Delta'}{\Gamma \vdash t_1 : A \mid \Delta & \Gamma' \vdash t_2 : B \mid \Delta'}
\\[8pt]
  \infer[ImpL]{\Gamma,y:A \multimap B,\Gamma' \vdash [y\,t/x]t_i : C_i \mid \Delta}{\Gamma \vdash t : A \mid \Delta & \Gamma',x : B \vdash t_i : C_i}
  &
  \quad
  &
  \infer[ImpR]{\Gamma \vdash \lambda x.t : A \multimap B \mid \Delta}{\Gamma, x : A \vdash t : B & x \not\in \mathsf{FV}(\Delta)}
  \\[8pt]
  \infer[ParR]{\Gamma \vdash \Delta \mid t_1 \oplus t_2 : A \oplus B \mid \Delta'}{\Gamma \vdash \Delta \mid t_1 : A \mid t_2 : B \mid \Delta'}
  \\
\end{array}
\]
\[
\begin{array}{ccc}
  \infer[ParL]{\Gamma,\Gamma',z : A \oplus B \vdash \letpat{z}{(x \oplus -)}{t_i} : C_i \mid \letpat{z}{(- \oplus y)}{t_j} : D_j}{\Gamma,x : A \vdash t_i : C_i & \Gamma', y : B \vdash t_j : D_j}\\
\end{array}
\]


\end{calculus}

% The following sections ("clarifications", "history", 
% "technicalities") are optional. If you use them, 
% be very concise and objective. Nevertheless, do write full sentences. 
% Try to have at most one paragraph per section, because line breaks 
% do not look nice in a short entry.

\begin{clarifications}
The left-hand side and right-hand side of sequents are multisets of
formulas denoted $\Gamma$ and $\Delta$.  The terms annotating formulas
are standard terms used in the simply typed $\lambda$-calculus.
Capture avoiding substitution is denoted by $[t/x]t'$, and
uniformly replaces every occurrence of $x$ in $t'$ with $t$.  The
definition of the let-pattern function used in the rule $ParL$ is
defined as follows:
\begin{center}
    \begin{math}
      \begin{array}{lll}      
        \begin{array}{lll}
          \letpat{z}{(x \oplus -)}{t} = t\\
          \,\,\,\,\,\,\mbox{where } x \not\in \mathsf{FV}(t)\\
        \end{array}
        & 
          \begin{array}{lll}
            \letpat{z}{(- \oplus y)}{t} = t\\
        \,\,\,\,\,\,\mbox{where } y \not\in \mathsf{FV}(t)\\
          \end{array}
        & 
          \begin{array}{lll}
            \letpat{z}{p}{t} = \letbe{z}{p}{t}\\
            & \\
          \end{array}
      \end{array}
    \end{math}
\end{center}
We denote vectors of terms (resp. types) by $t_i$ (resp. $A_j$).  The
function $\mathsf{FV}(\Delta)$ constructs the set of all free
variables in each term found in $\Delta$.
\end{clarifications}

\begin{history}
%% ToDo: write here short historical remarks about this proof system,
%% especially if they relate to other proof systems. 
%% Use "\iref{OtherProofSystem}" to refer to another proof system 
%% in the Encyclopedia (where "OtherProofSystem" is its ID). 
%% Use "\irefmissing{SuggestedIDForOtherProofSystem}" to refer to 
%% another proof system that is not yet available in the encyclopedia.
  The formulation of $\FILL$ given here was defined by Martin Hyland
  and Valeria de Paiva \cite{Hyland:1993}.  This version was an
  improvement over an early version first defined by Valeria de Paiva
  in her thesis \cite{dePaiva:1990}.  The improvement was the addition
  of the term assignment which was necessary to gain cut elimination.
\end{history}

%% \begin{technicalities}
%%   $\FILL$ enjoys cut elimination.  It also has a categorical model in
%%   dialectica categories \cite{dePaiva:1990}.
%% \end{technicalities}

% General Instructions:
% =====================

% The preferred length of an entry is 1 page. 
% Do the best you can to fit your proof system in one page.
%
% If you are finding it hard to fit what you want in one page, remember:
%
%   * Your entry needs to be neither self-contained nor fully understandable
%     (the interested reader may consult the cited full paper for details)
%
%   * If you are describing several proof systems in one entry, 
%     consider splitting your entry.
%
%   * You may reduce the size of your entry by ommitting inference rules
%     that are already described in other entries.
%
%   * Cite parsimoniously (see detailed citation instructions below).
%
% 
% If you do not manage to fit everything in one page, 
% it is acceptable for an entry to have 2 pages.
%
% For aesthetical reasons, it is preferable for an entry to have
% 1 full page or 2 full pages, in order to avoid unused blank space.



% Citation Instructions:
% ======================

% Please cite the original paper where the proof system was defined.
% To do so, you may use the \cite command within 
% one of the optional environments above,
% or use the \nocite command otherwise.

% You may also cite a modern paper or book where the 
% proof system is explained in greater depth or clarity.
% Cite parsimoniously.

% Do not cite related work. Instead, use the "\iref" or "\irefmissing" 
% commands to make an internal reference to another entry, 
% as explained within the "history" environment above.

% You do not need to create the "References" section yourself. 
% This is done automatically.




% Leave an empty line above "\end{entry}".

\end{entry}
