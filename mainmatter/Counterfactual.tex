
% If the calculus has an acronym, define it.
% (e.g. \newcommand{\LK}{\ensuremath{\mathbf{LK}}\xspace})

\calculusName{Calculi for Lewis' Counterfactual Logics I} % The name of the calculus
\calculusAcronym{} % The acronym if defined above, or empty otherwise.
\calculusYear{1983,1992,2012,2013} % The year when the calculus was invented.
\calculusAuthor{de Swart, Gent, Lellmann, Pattinson} % The name(s) of the author(s) of the calculus.

\entryTitle{Calculi for Lewis' Counterfactual Logics I}
\entryAuthor{Bj{\"o}rn Lellmann} % Your name(s). Separate multiple names with "\and"
\maketitle

% If your files are called "<ID>.tex" and "<ID>.bib",
% then you should write "\begin{entry}{<ID>}" in the line below

\begin{entry}{Counterfactual}

% Define here any newcommands you may need:
% e.g. \newcommand{\necessarily}{\Box}
% e.g. \newcommand{\possibly}{\Diamond}

\newcommand{\nc}{\newcommand}
\nc{\cless}{\preccurlyeq}
\nc{\rarr}{\rightarrow}
\nc{\CC}{\mathbb{C}}
\nc{\NN}{\mathbb{N}}
\newcommand{\Rules}{\mathcal{R}}
\nc{\TT}{\mathbb{T}}
\nc{\VA}{\mathbb{VA}}
\nc{\VNA}{\mathbb{VNA}}
\nc{\VV}{\mathbb{V}}
\nc{\WW}{\mathbb{W}}


\begin{calculus}
\[
  \vcenter{\infer[R_{n,m}]{\Gamma,(C_1\cless D_1),\dots,(C_m\cless
    D_m)\seq\Delta,(A_1\cless B_1),\dots,(A_n\cless B_n)}
    {\begin{array}{c}\{\;B_k\seq A_1,\dots,A_n,D_1,\dots,D_m\mid k\leq n\;\} \\\;\cup\;
    \{\;C_k\seq A_1,\dots,A_n,D_1,\dots,D_{k-1}\mid k\leq m\;\} \end{array}
  }}
\]
\[
\infer[T_{m}]{\Gamma, (C_1\cless D_1),\dots, (C_m\cless D_m) \seq \Delta
}
{\{\; C_k \seq D_1,\dots, D_{k-1} \mid k \leq m\;\}  \quad & \quad \Gamma \seq
  \Delta, D_1,\dots, D_m
}
\]
\[
  \vcenter{\infer[W_{n,m}]{\Gamma,(C_1\cless
    D_1),\dots,(C_m\cless D_m)\seq \Delta,(A_1\cless
    B_1),\dots,(A_n\cless B_n)}
    {\{\;C_k\seq
    A_1,\dots,A_n,D_1,\dots,D_{k-1}\mid k \leq m\;\} \quad&\quad
    \Gamma\seq\Delta,A_1,\dots,A_n,D_1,\dots,D_m}} 
\]
\[
\vcenter{\infer[R_{C1}]{\Gamma\seq\Delta,(A\cless
    B)}{\Gamma\seq\Delta,A}} \quad 
 \vcenter{\infer[R_{C2}]{\Gamma,(A\cless
     B)\seq\Delta}{\Gamma,A\seq\Delta \quad&\quad \Gamma\seq \Delta,
     B}}
\]
\[
\infer[A_{n,m}]{\Gamma, (C_1\cless D_1), \dots,
  (C_m\cless D_m) \seq \Delta, ( A_1,\cless B_1), \dots,
  (A_n\cless B_n)
}
{\begin{array}{c}\{\;\Gamma^\cless,B_k\seq \Delta^\cless, A_1,\dots,A_n,D_1,\dots,D_m\mid k\leq n\;\} \\\;\cup\;
    \{\; \Gamma^\cless,C_k\seq \Delta^\cless, A_1,\dots,A_n,D_1,\dots,D_{k-1}\mid k\leq m\;\}\end{array}
}
\]
\medskip

\begin{center}
\begin{tabular}{c@{\qquad}c}
  \multicolumn{2}{c}{$\Rules_{\VV_\cless} = \{R_{n,m} \mid n\geq 1, m\geq 0\}$
    }\\
\begin{tabular}{lll}
$\Rules_{\VV\NN_\cless}$ & = & $\{ R_{n,m} \mid n+m \geq 1\}$\\
$\Rules_{\VV\TT_\cless}$ & = & $\Rules_{\VV_\cless}\cup \{T_m \mid m \geq 1\}$\\
$\Rules_{\VV\WW_\cless}$ & = &$\Rules_{\VV_\cless}\cup \{W_{n,m}\mid n
+ m \geq 1\}\quad$
\end{tabular} &
\begin{tabular}{lll}
$\Rules_{\VV\CC_\cless}$ & = & $\Rules_\VV\cup\{R_{C1},R_{C2}\}$\\
$\Rules_{\VA_\cless}$ & = & $\{ A_{n,m} \mid n\geq 1, m\geq 0 \}$\\
$\Rules_{\VNA_\cless}$ & = & $\{ A_{n,m} \mid n + m\geq 1 \}$
\end{tabular}
\end{tabular}
\end{center}

% Add the inference rules of your proof system here.
% The "proof.sty" and "bussproofs.sty" packages are available.
% If you need any other package, please contact the editor (bruno@logic.at)

\end{calculus}

% The following environments ("clarifications", "history",
% "technicalities") are optional. If you do use them,
% be very concise and objective.
\begin{clarifications}
% ToDo: write here short remarks that may help the reader to understand
% the inference rules of the proof system.
  Sequents are based on multisets.  The propositional part is that of
  \Gtc \iref{G3c}. Also include the contraction rules.  Rules
  $\Rules_{\mathcal{L}_\cless}$ are for the logic $\mathcal{L}$ in
  terms of the \emph{comparative plausibility} operator $\cless$ from
  \cite{Lewis:1973uq}.  The contexts $\Gamma^\cless$ resp.\
  $\Delta^\cless$ contain all formulae of $\Gamma$ resp.\ $\Delta$ of
  the form $A \cless B$.
\end{clarifications}

\begin{history}
% ToDo: write here short historical remarks about this proof system,
% especially if they relate to other proof systems.
% Use "\iref{OtherProofSystem}" to refer to another proof system
% in the Encyclopedia (where "OtherProofSystem" is its ID).
% Use "\irefmissing{SuggestedIDForOtherProofSystem}" to refer to
% another proof system that is not yet available in the encyclopedia.
  The calculus for $\VV\CC$ was introduced in the tableaux setting
  \cite{Swart:1983uq,Gent:1992p3090}.  The remaining calculi were
  introduced in \cite{Lellmann:2012fk,Lellmann:2013fk} and corrected
  in \cite{Lellmann:2013}.
\end{history}

\begin{technicalities}
% ToDo: write here remarks about soundness, completeness, decidability...
  Soundness and completeness via equivalence to Hilbert-style calculi
  and (syntactical) cut elimination. Yield $\mathsf{PSPACE}$ decision
  procedures (resp.\ $\mathsf{EXPTIME}$ for $\VA_\cless$ and
  $\VNA_\cless$) and in most cases Craig Interpolation. Contraction
  can be made admissible. \cite{Lellmann:2012fk,Lellmann:2013}
\end{technicalities}
% Please cite the original paper where the proof system was defined.
% To do so, you may use the \cite command within
% one of the optional environments above,
% or use the \nocite command otherwise.
% You may also cite a modern paper or book where the
% proof system is explained in greater depth or clarity.
% Cite parsimoniously.
% Do not cite related work. Instead, use the "\iref" or "\irefmissing"
% commands to make an internal reference to another entry,
% as explained within the "history" environment above.
% You do not need to create the "References" section yourself.
% This is done automatically.
%

\clearpage

\end{entry}

%%% Local Variables: 
%%% mode: latex
%%% TeX-master: "../main"
%%% End: 
