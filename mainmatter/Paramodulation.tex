
% If the calculus has an acronym, define it.
% (e.g. \newcommand{\LK}{\ensuremath{\mathbf{LK}}\xspace})

\calculusName{Paramodulation}   % The name of the calculus
\calculusAcronym{}    % The acronym if defined above, or empty otherwise. 
\calculusLogic{classical}  % Specify the logic (e.g. classical, intuitionistic, ...) for which this calculus is intended.
\calculusType{saturation}   % Specify the calculus type (e.g. Frege-Hilbert style, tableau, sequent calculus, hypersequent calculus, natural deduction, ...)
\calculusYear{1969}   % The year when the calculus was invented.
\calculusAuthor{George Robinson, Larry Wos} % The name(s) of the author(s) of the calculus.


\entryTitle{Paramodulation}     % Title of the entry (usually coincides with the name of the calculus).
\entryAuthor{Uwe Waldmann}    % Your name(s). Separate multiple names with "\and".


% If you wish, use tags to give any other information 
% that might be helpful for classifying and grouping this entry:
% e.g. \tag{Two-Sided Sequents}
% e.g. \tag{Multiset Cedents}
% e.g. \tag{List Cedents}
% You are free to invent your own tags. 
% The Encyclopedia's coordinator will take care of 
% merging semantically similar tags in the future.


\maketitle


% If your files are called "MyProofSystem.tex" and "MyProofSystem.bib", 
% then you should write "\begin{entry}{MyProofSystem}" in the line below
\begin{entry}{Paramodulation}  

% Define here any newcommands you may need:
% e.g. \newcommand{\necessarily}{\Box}
% e.g. \newcommand{\possibly}{\Diamond}


\begin{calculus}

% Add the inference rules of your proof system here.
% The "proof.sty" and "bussproofs.sty" packages are available.
% If you need any other package, please contact the editor (bruno@logic.at)

\[
\infer[\textit{Paramodulation}]
{(D \lor C \lor L[u'])\sigma}{D \lor u \approx u'
& C \lor L[v]}
\]
\[
\infer[\textit{Resolution}]
{(D \lor C)\sigma}{D \lor B\vphantom{[]}
& C \lor \neg A}
\]
\[
\infer[\textit{Factoring}]
{(C \lor L_1)\sigma}{C \lor L_1 \lor \dots \lor L_n}
\]
\[
\infer[\textit{Reflexivity}]
{x \approx x\vphantom{[]}}{\vphantom{[]}}
\]

\medskip

$C,D$ are (possibly empty) equational clauses,
$L,L_1,\dots,L_n$ are literals,
$A,B$ are atoms,
$u,u',v$ are terms;
$u$ and $v$, $A$ and $B$, or $L_1,\dots,L_n$, respectively,
are unifiable with most general unifier~$\sigma$.

\end{calculus}

% The following sections ("clarifications", "history", 
% "technicalities") are optional. If you use them, 
% be very concise and objective. Nevertheless, do write full sentences. 
% Try to have at most one paragraph per section, because line breaks 
% do not look nice in a short entry.

\begin{clarifications}
Paramodulation is a refutational saturation calculus for
first-order clauses (disjunctions of possibly negated atoms)
with equality (denoted by~$\approx$).
It works on a set $N$ of clauses that is saturated
by successively computing inferences
with premises in $N$ and adding the conclusion of the inference to $N$,
until the empty clause (i.\,e., false) is derived.
\end{clarifications}

\begin{history}
Handling the equality axioms in the resolution calculus~\iref{Resolution}
is impractical due to the huge search space generated in particular
by the transitivity axiom.
The paramodulation calculus
developed by Robinson and Wos~\cite{RobinsonWos1969} extends
resolution by specific inference rules that
render explicit inferences with the equality axioms unnecessary.
The original completeness proof also assumed the
presence of so-called functional-reflexive axioms of the
form
$f(x_1,\dots,x_n) \approx f(x_1,\dots,x_n)$;
this was later shown to be superfluous by Brand~\cite{Brand1975SIAMJC}.
Many refinements were developed in the sequel,
aiming in particular at reducing the number of possible inferences,
see~\iref{Superposition}.

\end{history}

\begin{technicalities}
The paramodulation calculus is refutationally complete for
first-order logic with equality.
\end{technicalities}


% General Instructions:
% =====================

% The preferred length of an entry is 1 page. 
% Do the best you can to fit your proof system in one page.
%
% If you are finding it hard to fit what you want in one page, remember:
%
%   * Your entry needs to be neither self-contained nor fully understandable
%     (the interested reader may consult the cited full paper for details)
%
%   * If you are describing several proof systems in one entry, 
%     consider splitting your entry.
%
%   * You may reduce the size of your entry by ommitting inference rules
%     that are already described in other entries.
%
%   * Cite parsimoniously (see detailed citation instructions below).
%
% 
% If you do not manage to fit everything in one page, 
% it is acceptable for an entry to have 2 pages.
%
% For aesthetical reasons, it is preferable for an entry to have
% 1 full page or 2 full pages, in order to avoid unused blank space.



% Citation Instructions:
% ======================

% Please cite the original paper where the proof system was defined.
% To do so, you may use the \cite command within 
% one of the optional environments above,
% or use the \nocite command otherwise.

% You may also cite a modern paper or book where the 
% proof system is explained in greater depth or clarity.
% Cite parsimoniously.

% Do not cite related work. Instead, use the "\iref" or "\irefmissing" 
% commands to make an internal reference to another entry, 
% as explained within the "history" environment above.

% You do not need to create the "References" section yourself. 
% This is done automatically.




% Leave an empty line above "\end{entry}".

\end{entry}
