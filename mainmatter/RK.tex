
% If the calculus has an acronym, define it.
% (e.g. \newcommand{\LK}{\ensuremath{\mathbf{LK}}\xspace})

\calculusName{Resolution for Modal Logic K}   % The name of the calculus
\calculusAcronym{\RK}    % The acronym if defined above, or empty otherwise. 
\calculusLogic{Modal logics}  % Specify the logic (e.g. classical, intuitionistic, ...) for which this calculus is intended.
\calculusType{Resolution}   % Specify the calculus type (e.g. Frege-Hilbert style, tableau, sequent calculus, hypersequent calculus, natural deduction, ...)
\calculusYear{1982}   % The year when the calculus was invented.
\calculusAuthor{Luis Fari\~nas del Cerro} % The name(s) of the author(s) of the calculus.


\entryTitle{Resolution for Modal Logic K (\RK)}     % Title of the entry (usually coincides with the name of the calculus).
\entryAuthor{Joseph Boudou}    % Your name(s). Separate multiple names with "\and".


% If you wish, use tags to give any other information 
% that might be helpful for classifying and grouping this entry:
% e.g. \tag{Two-Sided Sequents}
% e.g. \tag{Multiset Cedents}
% e.g. \tag{List Cedents}
% You are free to invent your own tags. 
% The Encyclopedia's coordinator will take care of 
% merging semantically similar tags in the future.


\maketitle


% If your files are called "MyProofSystem.tex" and "MyProofSystem.bib", 
% then you should write "\begin{entry}{MyProofSystem}" in the line below
\begin{entry}{RK}  

\newcommand{\sepproof}{\hskip 2em plus 6em\relax}
\newcommand{\sepline}{\]\[}

\def\fCenter{\longrightarrow}
\newenvironment{infruleset}[1]{%
  \sc{#1} \[
%  \begin{center}
}{%
  \]
%  \end{center}
}

\begin{calculus}

% Add the inference rules of your proof system here.
% The "proof.sty" and "bussproofs.sty" packages are available.
% If you need any other package, please contact the editor (bruno@logic.at)

\begin{infruleset}{Rules for computing resolvents}
    \AxiomC{}
    \LeftLabel{(A1)}
    \UnaryInfC{$\Sigma(p, \neg p) \fCenter \bot$}
    \DisplayProof
  \sepproof
    \AxiomC{}
    \LeftLabel{(A2)}
    \UnaryInfC{$\Sigma(\bot, A) \fCenter \bot$}
    \DisplayProof
  \sepline
    \Axiom$\Sigma(A, B) \fCenter C$
    \LeftLabel{($\Sigma\vee$)}
    \UnaryInf$\Sigma(A \vee D_1, B \vee D_2) \fCenter C \vee D_1 \vee D_2$
    \DisplayProof
  \sepline
    \Axiom$\Sigma(A, B) \fCenter C$
    \LeftLabel{($\Sigma\Box\Diamond$)}
    \UnaryInf$\Sigma(\Box A, \Diamond(B \wedge E)) \fCenter \Diamond(B \wedge C \wedge E)$
    \DisplayProof
  \sepproof
    \Axiom$\Sigma(A, B) \fCenter C$
    \LeftLabel{($\Sigma\Box\Box$)}
    \UnaryInf$\Sigma(\Box A, \Box B) \fCenter \Box C$
    \DisplayProof
  \sepline
    \Axiom$\Gamma(A) \fCenter B$
    \LeftLabel{($\Gamma\vee$)}
    \UnaryInf$\Gamma(A \vee C) \fCenter B \vee C$
    \DisplayProof
  \sepproof
    \Axiom$\Gamma(A) \fCenter B$
    \LeftLabel{($\Gamma\Box$)}
    \UnaryInf$\Gamma(\Box A) \fCenter \Box B$
    \DisplayProof
  \sepline
    \Axiom$\Sigma(A, B) \fCenter C$
    \LeftLabel{($\Gamma\Diamond1$)}
    \UnaryInf$\Gamma(\Diamond(A \wedge B \wedge E)) \fCenter \Diamond(A \wedge B \wedge C \wedge E)$
    \DisplayProof
  \sepline
    \Axiom$\Gamma(A) \fCenter B$
    \LeftLabel{($\Gamma\Diamond2$)}
    \UnaryInf$\Gamma(\Diamond(A \wedge E)) \fCenter \Diamond(A \wedge B \wedge E)$
    \DisplayProof
\end{infruleset}

\sc{Simplification rules}

\begin{center}
\begin{tabular}{rlrl}
  $(S_1)$ &~ $\Diamond\bot \approx \bot       $ ~&\quad
  $(S_3)$ &~ $\bot \wedge E \approx \bot      $ \\
  $(S_2)$ &~ $\bot \vee A \approx A           $ ~&\quad
  $(S_4)$ &~ $A \vee A \vee B \approx A \vee B$ 
\end{tabular}
\end{center}
\vspace{2ex}

\begin{infruleset}{Inference rules}
    \AxiomC{$C$}
    \LeftLabel{(R1)}
    \RightLabel{if $\Gamma(C) \Rightarrow D$}
    \UnaryInfC{$D$}
    \DisplayProof
  \sepline
    \AxiomC{$C_1$}
    \AxiomC{$C_2$}
    \LeftLabel{(R2)}
    \RightLabel{if $\Sigma(C_1, C_2) \Rightarrow D$}
    \BinaryInfC{$D$}
    \DisplayProof
\end{infruleset}

\end{calculus}

% The following sections ("clarifications", "history", 
% "technicalities") are optional. If you use them, 
% be very concise and objective. Nevertheless, do write full sentences. 
% Try to have at most one paragraph per section, because line breaks 
% do not look nice in a short entry.

\begin{clarifications}
  $A, B, C$ and $D$ denote formulas in disjunctive normal form (DNF)
  whereas $E$ denotes a formula in conjunctive normal form (CNF).
  A formula is in DNF if it has the general form
  $ L_1 \vee \ldots \vee L_n \vee
    \Box A_1 \vee \ldots \vee \Box A_p \vee
    \Diamond E_1 \vee \ldots \vee \Diamond E_q $,
  where $L_i$ are literals, $A_i$ are in DNF and $E_i$ are in CNF.
  A formula is in CNF if it is a conjunction of formulas in DNF.
  The relation $\approx$ is the least congruence satisfying all the simplification rules.
  The normal form $A$ of a formula $A'$ is the least formula such that $A' \approx A$.
  We write $\Sigma(A,B) \Rightarrow C$ (respectively $\Gamma(A) \Rightarrow C$)
  if there exist $C'$ such that $\Sigma(A,B) \fCenter C'$ (respectively $\Gamma(A) \fCenter C'$)
  and $C$ is the normal form of $C'$.
\end{clarifications}

\begin{history}
  Introduced in~\cite{farinas.1982}.
  The current presentation is inspired by~\cite{enjalbert-farinas.1989}.
  Variants exist for other modal logics like S4 or S5.
\end{history}

% \begin{technicalities}
% ToDo: write here remarks about soundness, completeness, decidability...
% \end{technicalities}


% General Instructions:
% =====================

% The preferred length of an entry is 1 page. 
% Do the best you can to fit your proof system in one page.
%
% If you are finding it hard to fit what you want in one page, remember:
%
%   * Your entry needs to be neither self-contained nor fully understandable
%     (the interested reader may consult the cited full paper for details)
%
%   * If you are describing several proof systems in one entry, 
%     consider splitting your entry.
%
%   * You may reduce the size of your entry by ommitting inference rules
%     that are already described in other entries.
%
%   * Cite parsimoniously (see detailed citation instructions below).
%
% 
% If you do not manage to fit everything in one page, 
% it is acceptable for an entry to have 2 pages.
%
% For aesthetical reasons, it is preferable for an entry to have
% 1 full page or 2 full pages, in order to avoid unused blank space.



% Citation Instructions:
% ======================

% Please cite the original paper where the proof system was defined.
% To do so, you may use the \cite command within 
% one of the optional environments above,
% or use the \nocite command otherwise.

% You may also cite a modern paper or book where the 
% proof system is explained in greater depth or clarity.
% Cite parsimoniously.

% Do not cite related work. Instead, use the "\iref" or "\irefmissing" 
% commands to make an internal reference to another entry, 
% as explained within the "history" environment above.

% You do not need to create the "References" section yourself. 
% This is done automatically.




% Leave an empty line above "\end{entry}".

\end{entry}
