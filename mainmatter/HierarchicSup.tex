
% If the calculus has an acronym, define it.
% (e.g. \newcommand{\LK}{\ensuremath{\mathbf{LK}}\xspace})

\calculusName{Hierarchic Superposition}   % The name of the calculus
\calculusAcronym{}    % The acronym if defined above, or empty otherwise. 
\calculusLogic{classical}  % Specify the logic (e.g. classical, intuitionistic, ...) for which this calculus is intended.
\calculusType{saturation}   % Specify the calculus type (e.g. Frege-Hilbert style, tableau, sequent calculus, hypersequent calculus, natural deduction, ...)
\calculusYear{1992/2013}   % The year when the calculus was invented.
\calculusAuthor{Leo Bachmair, Harald Ganzinger, Uwe Waldmann} % The name(s) of the author(s) of the calculus.


\entryTitle{Hierarchic Superposition}     % Title of the entry (usually coincides with the name of the calculus).
\entryAuthor{Uwe Waldmann}    % Your name(s). Separate multiple names with "\and".


% If you wish, use tags to give any other information 
% that might be helpful for classifying and grouping this entry:
% e.g. \tag{Two-Sided Sequents}
% e.g. \tag{Multiset Cedents}
% e.g. \tag{List Cedents}
% You are free to invent your own tags. 
% The Encyclopedia's coordinator will take care of 
% merging semantically similar tags in the future.


\maketitle


% If your files are called "MyProofSystem.tex" and "MyProofSystem.bib", 
% then you should write "\begin{entry}{MyProofSystem}" in the line below
\begin{entry}{HierarchicSup}

% Define here any newcommands you may need:
% e.g. \newcommand{\necessarily}{\Box}
% e.g. \newcommand{\possibly}{\Diamond}


\begin{calculus}

% Add the inference rules of your proof system here.
% The "proof.sty" and "bussproofs.sty" packages are available.
% If you need any other package, please contact the editor (bruno@logic.at)

Abstraction
\[
\infer[\textit{Abstraction}]
{C[x] \lor \neg x \approx t}{C[t]\vphantom{[]}}
\]
applied exhaustively until no literal contains operator symbols
from both $\Sigma_{\mathrm{Base}}$ and $\Sigma_{\mathrm{Ext}}$,
followed by saturation under
\[
\infer[\textit{Constraint Refutation}]
{\bot}{M & M \models_{\mathrm{Base}} \bot}
\]
and the rules of the standard superposition calculus~\iref{Superposition},
where the latter are restricted in such a way that only extension
literals participate in inferences and that all unifying substitutions
must be simple.

$C$ is an equational clause,
$t$ is a term,
$x$ is a fresh variable,
$M$ is a finite set of clauses over $\Sigma_{\mathrm{Base}}$.
\end{calculus}

% The following sections ("clarifications", "history", 
% "technicalities") are optional. If you use them, 
% be very concise and objective. Nevertheless, do write full sentences. 
% Try to have at most one paragraph per section, because line breaks 
% do not look nice in a short entry.

\begin{clarifications}
Hierarchic superposition is a refutational saturation calculus for
first-order clauses with equality
modulo a base specification
(e.\,g., some kind of arithmetic),
for which a decision procedure is available
that can be used as a ``black-box''
in the \textit{Constraint Refutation} rule.
The inference rules are supplemented by a redundancy criterion
that permits to delete clauses that are unnecessary for
deriving a contradiction during the saturation, see \iref{SaturationWithRed}.
\end{clarifications}

\begin{history}
The hierarchic superposition calculus~\cite{BachmairGanzingerWaldmann1992ALP,BachmairGanzingerWaldmann1994AAECC}
works in the framework of hierarchic specifications
consisting of a base part and an extension,
where the models of the hierarchic specification
are those models of the extension clauses that
are conservative extensions of some base model.
The calculus is refutationally complete,
provided that the set of clauses is sufficiently
complete after abstraction and that the base specification is compact.
An improved variant of the calculus
was given in \cite{BaumgartnerWaldmann2013CADE};
this calculus uses a weaker form of abstraction that is
guaranteed to preserve sufficient completeness
\looseness=-1
but requires an additional abstraction step after each inference.

\end{history}

% \begin{technicalities}
% ToDo: write here remarks about soundness, completeness, decidability...
% \end{technicalities}


% General Instructions:
% =====================

% The preferred length of an entry is 1 page. 
% Do the best you can to fit your proof system in one page.
%
% If you are finding it hard to fit what you want in one page, remember:
%
%   * Your entry needs to be neither self-contained nor fully understandable
%     (the interested reader may consult the cited full paper for details)
%
%   * If you are describing several proof systems in one entry, 
%     consider splitting your entry.
%
%   * You may reduce the size of your entry by ommitting inference rules
%     that are already described in other entries.
%
%   * Cite parsimoniously (see detailed citation instructions below).
%
% 
% If you do not manage to fit everything in one page, 
% it is acceptable for an entry to have 2 pages.
%
% For aesthetical reasons, it is preferable for an entry to have
% 1 full page or 2 full pages, in order to avoid unused blank space.



% Citation Instructions:
% ======================

% Please cite the original paper where the proof system was defined.
% To do so, you may use the \cite command within 
% one of the optional environments above,
% or use the \nocite command otherwise.

% You may also cite a modern paper or book where the 
% proof system is explained in greater depth or clarity.
% Cite parsimoniously.

% Do not cite related work. Instead, use the "\iref" or "\irefmissing" 
% commands to make an internal reference to another entry, 
% as explained within the "history" environment above.

% You do not need to create the "References" section yourself. 
% This is done automatically.




% Leave an empty line above "\end{entry}".

\end{entry}
