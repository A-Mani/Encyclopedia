
% If the calculus has an acronym, define it.
% (e.g. \newcommand{\LK}{\ensuremath{\mathbf{LK}}\xspace})

\calculusName{Constructive Modal Logic S4}
\calculusAcronym{\CSFour}
\calculusLogic{Constructive Modal Logic}
\calculusType{Sequent Calculus}
\calculusYear{2000}
\calculusAuthor{Gavin Bierman \and Valeria de Paiva}


\entryTitle{Constructive Modal Logic S4 (\CSFour)}
\entryAuthor{Harley Eades III \and Valeria de Paiva}


\maketitle


\begin{entry}{CSFour} 

\newcommand{\llaconj}{\binampersand}
\newcommand{\lladisj}{\oplus}
\newcommand{\llimp}{\multimap}
\newcommand{\llmconj}{\otimes}
\newcommand{\llzero}{0}
\newcommand{\llone}{1}

\newcommand{\sepproof}{\hskip 2em plus 6em\relax}
\newcommand{\sepseq}{\quad}
\newcommand{\sepline}{\]\[}

\newenvironment{infruleset}[1]{%
  \sc{#1} \vspace{-1ex} \[ %
}{%
  \] %
}

\begin{calculus}
\[
\begin{array}{ccccccc}

  \infer[ax]{\Delta,A \vdash A}{}
  &
  \quad
  &
  \infer[cut]{\Gamma, \Delta \vdash B}{\Gamma \vdash A & A, \Delta \vdash B}
  &
  \quad
  &
  \infer[\perp\mathcal{L}]{\Gamma,\perp \vdash A}{}\\
  \\
  \infer[\lor\mathcal{L}]{\Gamma,A \lor B \vdash C}{\Gamma, A \vdash C & \Gamma, B \vdash C}
  &
  \quad
  &
  \infer[\lor\mathcal{R}]{\Gamma \vdash A \lor B}{\Gamma \vdash A}
  &
  \quad
  &
  \infer[\lor\mathcal{R}]{\Gamma \vdash A \lor B}{\Gamma \vdash B}\\
  \\
  \infer[\land\mathcal{L}]{\Gamma, A \land B \vdash C}{\Gamma, A \vdash C}
  &
  \quad
  &
  \infer[\land\mathcal{L}]{\Gamma, A \land B \vdash C}{\Gamma, B \vdash C}
  &
  \quad
  &
  \infer[\land\mathcal{R}]{\Gamma \vdash A \land B}{\Gamma \vdash A & \Gamma \vdash B}\\
  \\
  \infer[\to\mathcal{L}]{\Gamma, A \to B \vdash C}{\Gamma \vdash A & \Gamma, B \vdash C}
  &
  \quad
  &
  \infer[\to\mathcal{R}]{\Gamma \vdash A \to B}{\Gamma, A \vdash B}
  &
  \quad
  &
  \infer[\Box\mathcal{L}]{\Gamma, \Box A \vdash B}{\Gamma, A \vdash B}\\
  \\
  \infer[\Box\mathcal{R}]{\Box \Gamma,\Delta \vdash A}{\Box \Gamma \vdash \Box A}
  &
  \quad
  &
  \infer[\Diamond\mathcal{L}]{\Delta,\Box\Gamma, \Diamond A \vdash \Diamond B}{\Box \Gamma, A \vdash \Diamond B}
  &
  \quad
  &
  \infer[\Diamond\mathcal{R}]{\Gamma \vdash \Diamond A}{\Gamma \vdash A}\\
\end{array}
\]
\vspace{-1em}

\end{calculus}

\begin{clarifications}
  Left contexts, denoted $\Gamma$ or $\Delta$, are multisets of
  formulas. Furthermore, if $\Gamma = A_1,\ldots,A_n$, then
  $\Box \Gamma = \Box A_1,\ldots,\Box A_n$.
\end{clarifications}

\begin{history}
 The intuitionistic system for S4 that we are calling constructive S4
 (CS4) here, was originally described by Prawitz in his Natural
 Deduction book\cite{prawitznatural} in 1965. This system differs from
 what is more widely called now IS4, originally defined by
 Fisher-Servi \cite{Fisher-Servi:1981} and thoroughly studied in
 Simpson's PhD thesis \cite{simpson1994phd} in that it does not
 satisfy the distribution of possibility over disjunctions, either
 binary ($\Diamond (A\lor B) \to \Diamond A \lor \Diamond B$) or
 nullary ($\Diamond \bot \to \bot$). The calculus for CS4 was
 thoroughly investigated by Bierman and de Paiva
 in \cite{bierman2000}.
\end{history}

\end{entry}
