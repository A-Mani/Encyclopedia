
\newcommand{\LKT}{\ensuremath{\mathbf{LKT}\xspace}}
\newcommand{\LJT}{\ensuremath{\mathbf{LJT}\xspace}}
\newcommand{\LKQ}{\ensuremath{\mathbf{LKQ}\xspace}}
\calculusName{} % The name of the calculus
\calculusAcronym{\LKmumutilde} % The acronym if defined above, or empty otherwise.
\calculusLogic{Classical or intuitionistic} % Specify the logic (e.g. classical, intuitionistic, ...) for which this calculus is intended.
\calculusType{sequent calculus} % Specify the calculus type (e.g. Frege-Hilbert style, tableau, sequent calculus, hypersequent calculus, natural deduction, ...)
\calculusYear{2000} % The year when the calculus was invented.
\calculusAuthor{Curien-Herbelin} % The name(s) of the author(s) of the calculus.
\entryTitle{\LKmumutilde} % Title of the entry (usually coincides with the name of the calculus).
\entryAuthor{Hugo Herbelin} % Your name(s). Separate multiple names with "\and".
% If you wish, use tags to give any other information
% that might be helpful for classifying and grouping this entry:
\tag{Two-Sided Sequents}
% e.g. \tag{Multiset Cedents}
\tag{List Cedents}
\tag{Term-annotated}
% You are free to invent your own tags.
% The Encyclopedia's coordinator will take care of
% merging semantically similar tags in the future.
\maketitle
% If your files are called "MyProofSystem.tex" and "MyProofSystem.bib",
% then you should write "\begin{entry}{MyProofSystem}" in the line below
\begin{entry}{LKMuMuTilde}
% Define here any newcommands you may need:
% e.g. \newcommand{\necessarily}{\Box}
% e.g. \newcommand{\possibly}{\Diamond}
\newcommand{\cut}[2]{\langle #1 | #2 \rangle}
\begin{calculus}
% Add the inference rules of your proof system here.
% The "proof.sty" and "bussproofs.sty" packages are available.
% If you need any other package, please contact the editor (bruno@logic.at)

{\sc Structural subsystem}
\[
\infer[\mathit{Ax_R}]
      {\Gamma \vdash a:A ~|~ \Delta}
      {(a:A) \in \Gamma}
\qquad
\infer[\mathit{Cut}]
      {\cut{v}{e} : (\Gamma \vdash \Delta)}
      {\Gamma \vdash v:A ~|~ \Delta & \Gamma ~|~ e:A \vdash \Delta}
\qquad
\infer[\mathit{Ax_L}]
      {\Gamma ~|~ \alpha:A \vdash \Delta}
      {(\alpha:A) \in \Delta}
\]
\[
\infer[\mathit{Focus}_L]
      {\Gamma~|~ \tilde\mu a.c:A \vdash \Delta}
      {c : (\Gamma, a:A \vdash \Delta)}
\qquad
\infer[\mathit{Focus}_R]
      {\Gamma \vdash \mu \alpha.c:A ~|~ \Delta}
      {c : (\Gamma \vdash \alpha:A, \Delta)}
\]
{\sc Introduction rules}
\[
\infer[\wedge_L^i]
      {\Gamma ~|~ \pi_i\cdot e :  A_1 \wedge A_2 \vdash \Delta}
      {\Gamma ~|~ e : A_i \vdash \Delta}
\qquad
\infer[\wedge_R]
      {\Gamma \vdash ~|~ (v_1,v_2) : A_1 \wedge A_2 ~|~ \Delta}
      {\Gamma \vdash v_1:A_1 ~|~ \Delta & \Gamma \vdash v_2 : A_2 ~|~ \Delta}
\]
\[
\infer[\vee_L]
      {\Gamma ~|~ [e_1,e_2] : A_1 \vee A_2 \vdash \Delta}
      {\Gamma ~|~ e_1:A_1 \vdash \Delta & \Gamma ~|~ e_2 : A_2 \vdash \Delta}
\qquad
\infer[\vee_R^i]
      {\Gamma \vdash \iota_i(v) :  A_1 \vee A_2 ~|~ \Delta}
      {\Gamma \vdash v : A_i ~|~ \Delta}
\]
\[
\infer[\rightarrow_L]
      {\Gamma ~|~ v\cdot e : A\rightarrow B \vdash \Delta}
      {\Gamma \vdash v:A ~|~ \Delta & \Gamma ~|~ e:B \vdash \Delta}
\qquad
\infer[\rightarrow_R]
      {\Gamma \vdash \lambda a.v : A \rightarrow B ~|~\Delta}
      {\Gamma, a:A \vdash v:B ~|~ \Delta}
\]
\[
\infer[\exists_L]
      {\Gamma ~|~ \tilde\lambda x.e : \exists x\,A[x] \vdash \Delta}
      {\Gamma ~|~ e: A[y] \vdash \Delta}
\qquad
\infer[\exists_R]
      {\Gamma \vdash t \cdot v  : \exists x\,A[x] ~|~ \Delta}
      {\Gamma \vdash v :  A[t] ~|~ \Delta}
\]
\[
\infer[\forall_L]
      {\Gamma ~|~ t \cdot e : \forall x\, A[x] \vdash \Delta}
      {\Gamma \vdash e: A[t] ~|~ \Delta}
\qquad
\infer[\forall_R]
      {\Gamma \vdash \lambda x.v  : \forall x\,A[x] ~|~ \Delta}
      {\Gamma \vdash v : A[y] ~|~ \Delta}
\]
\[
\infer[\bot_L]
      {\Gamma ~|~ [~] : \bot \vdash \Delta}
      {}
\qquad
\infer[\top_R]
      {\Gamma \vdash () : \top ~|~ \Delta}
      {}
\]

\end{calculus}
% The following sections ("clarifications", "history",
% "technicalities") are optional. If you use them,
% be very concise and objective. Nevertheless, do write full sentences.
% Try to have at most one paragraph per section, because line breaks
% do not look nice in a short entry.
\begin{clarifications}
There are three kinds of sequents: first $\Gamma \vdash v:A ~|~
\Delta$ with a distinguished formula on the right for typing the
program $v$, second $\Gamma ~|~ e:A \vdash \Delta$ with a
distinguished formula on the left for typing the evaluation context
$e$, and finally $c : (\Gamma \vdash \Delta)$ with no distinguished
formula for typing command $c$, i.e. the interaction of a program
within an evaluation context. The typing contexts $\Gamma$ and
$\Delta$ are lists of named formulas so that a non-ambiguous
correspondence with $\lambda$-calculus is possible (if it were sets or
multisets, there were e.g. no way to distinguish the two distinct
proofs of $x:A, x:A \vdash x:A ~|~$). Weakening rules are implemented
implicitly at the level of axioms. Contraction rules are derived,
using a cut against an axiom. No exchange rule is needed.  Not all
cuts are eliminable: only those not involving an axiom rule are.
Negation $\neg A$ can be defined as $A \rightarrow \bot$. In the rules
$\exists_E$ and $\forall_R$, $y$ is assumed fresh in $\Gamma$,
$\Delta$ and $A[x]$. The syntax of the underlying $\lambda$-calculus
is:
$$
\begin{array}{lll}
c & ::= & \cut{v}{e}\\
e & ::= & \alpha ~|~ \tilde\mu a.c ~|~ \pi_i\cdot e ~|~ [e,e] ~|~ v \cdot e ~|~ (t,e) ~|~ \tilde\lambda x.e ~|~ []\\
v & ::= & a ~|~ \mu\alpha.c ~|~ (v,v) ~|~ \iota_i(v) ~|~ \lambda a.v ~|~ \lambda x.v ~|~ (t,v) ~|~ ()\\
\end{array}
$$
\end{clarifications}

\begin{history}
The purpose of this system is to provide with a
$\lambda$-calculus-style computational meaning to Gentzen's LK
\iref{GentzenLK} and to highlight how the symmetries of sequent
calculus show computationally. Seeing the rules as typing rules, the
left/right symmetry is a symmetry between programs and their
evaluation contexts. At the level of cut elimination, giving priority
to the left-hand side relates to call-by-name evaluation while giving
priority to the right-hand side relates to call-by-value
evaluation~\cite{CurienHerbelin00}. Thanks to the presence of two dual
axiom rules and implicit contraction rules, the system supports a
tree-like sequent-free presentation like originally presented by
Gentzen for natural deduction~\cite{HerbelinHdR} (see
\iref{LKMuMuTildeTree}).

The structural subsystem can be adapted to various sequent calculi,
such as $\LK_{\mathit{pol}}$~\cite{Munch-Maccagnoni09}, and in
particular $\LKT$ (emphasizing call-by-name) and $\LKQ$ (emphasizing
call-by-value)~\cite{DanJoiSch95,DanJoiSch97}. Restriction to
intuitionistic logic can be obtained by demanding that the right-hand
side has exactly one formula.

The presentation of this calculus with conjunctive and disjunctive
additive connectives has been studied in
\cite{Wadler03,Wadler05,HerbelinHdR}. A variant with only commands,
called ${\cal X}$, has been studied in \cite{BakLenLes05}, based on
previous work in~\cite{UrbanPhD}.  Various extensions of the system
emphasizing different symmetries can be found in~\cite{DownenAriola14}
(general connectives), recursion/corecursion~\cite{KimuraTatsuta13},
...  An asymmetric variant of $\LKT$ with sequents of the form $\Gamma
~|~ A \vdash B ~|~ \Delta$, $\Gamma \vdash A ~|~ \Delta$ and $\Gamma
\vdash \Delta$ can be found in \cite{HerbelinPhD}, with the
intuitionistic restriction $\LJT$ studied in \cite{Herbelin94}.

\end{history}

% Use "\irefmissing{SuggestedIDForOtherProofSystem}" to refer to
% another proof system that is not yet available in the encyclopedia.
% \end{history}
\begin{technicalities}
The system is obviously logically equivalent to Gentzen's $\LK$ when
equipped with the corresponding connectives and observed through the
sequents of the form $\Gamma \vdash \Delta$. 
\end{technicalities}

% General Instructions:
% =====================
% The preferred length of an entry is 1 page.
% Do the best you can to fit your proof system in one page.
%
% If you are finding it hard to fit what you want in one page, remember:
%
% * Your entry needs to be neither self-contained nor fully understandable
% (the interested reader may consult the cited full paper for details)
%
% * If you are describing several proof systems in one entry,
% consider splitting your entry.
%
% * You may reduce the size of your entry by ommitting inference rules
% that are already described in other entries.
%
% * Cite parsimoniously (see detailed citation instructions below).
%
%
% If you do not manage to fit everything in one page,
% it is acceptable for an entry to have 2 pages.
%
% For aesthetical reasons, it is preferable for an entry to have
% 1 full page or 2 full pages, in order to avoid unused blank space.
% Citation Instructions:
% ======================
% Please cite the original paper where the proof system was defined.
% To do so, you may use the \cite command within
% one of the optional environments above,
% or use the \nocite command otherwise.
% You may also cite a modern paper or book where the
% proof system is explained in greater depth or clarity.
% Cite parsimoniously.
% Do not cite related work. Instead, use the "\iref" or "\irefmissing"
% commands to make an internal reference to another entry,
% as explained within the "history" environment above.
% You do not need to create the "References" section yourself.
% This is done automatically.
% Leave an empty line above "\end{entry}".
\end{entry}
