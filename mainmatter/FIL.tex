
% If the calculus has an acronym, define it.
% (e.g. \newcommand{\LK}{\ensuremath{\mathbf{LK}}\xspace})

\calculusName{Full Intuitionistic Logic}   % The name of the calculus
\calculusAcronym{\FIL}    % The acronym if defined above, or empty otherwise. 
\calculusLogic{Intuitionistic}  % Specify the logic (e.g. classical, intuitionistic, ...) for which this calculus is intended.
\calculusType{Sequent Calculus}   % Specify the calculus type (e.g. Frege-Hilbert style, tableau, sequent calculus, hypersequent calculus, natural deduction, ...)
\calculusYear{2005} % The year when the calculus was invented.
\calculusAuthor{Valeria de Paiva \and Luiz Pereira} % The name(s) of the author(s) of the calculus.

\entryTitle{Full Intuitionistic Logic (FIL)}        % Title of the entry (usually coincides with the name of the calculus).
\entryAuthor{Harley Eades III \and Valeria de Paiva} % Your name(s). Separate multiple names with "\and".

% If you wish, use tags to give any other information 
% that might be helpful for classifying and grouping this entry:
% e.g. \tag{Two-Sided Sequents}
% e.g. \tag{Multiset Cedents}
% e.g. \tag{List Cedents}
% You are free to invent your own tags. 
% The Encyclopedia's coordinator will take care of 
% merging semantically similar tags in the future.
\tag{Two-Sided Sequents}
\tag{Dependency Tracking}


\maketitle


% If your files are called "MyProofSystem.tex" and "MyProofSystem.bib", 
% then you should write "\begin{entry}{MyProofSystem}" in the line below
\begin{entry}{FIL}  

% Define here any newcommands you may need:
% e.g. \newcommand{\necessarily}{\Box}
% e.g. \newcommand{\possibly}{\Diamond}

\begin{calculus}
\[
\begin{array}{ccc}
  \infer[ax]{A(n) \Rightarrow A/\{n\}}{}
  & \quad &
  \infer[\perp \Rightarrow]{\perp (n) \Rightarrow A_1/\{n\},\ldots,A_k/\{n\}}{}\\
  \\[-5pt]
  \infer[cut]{\Gamma_1,\Gamma_1 \Rightarrow \Delta_1,\Delta^*_1}{\Gamma_1 \Rightarrow \Delta_1,A/S & A(n),\Gamma_1 \Rightarrow \Delta_1}
  & \quad &
  \infer[perm \Rightarrow]{\Gamma_1,B(n),A(m),\Gamma_1 \Rightarrow \Delta}{\Gamma_1,A(m),B(n),\Gamma_1 \Rightarrow \Delta}\\
  \\[-5pt]
  \infer[\Rightarrow perm]{\Gamma \Rightarrow \Delta_1,B/S_1,A/S_1,\Delta_1}{\Gamma \Rightarrow \Delta_1,A/S_1,B/S_1,\Delta_1}
  & \quad &
  \infer[weak \Rightarrow]{A(n),\Gamma \Rightarrow \Delta^*}{\Gamma \Rightarrow \Delta}\\
  \\[-0.5pt]
  \infer[\Rightarrow weak]{\Gamma \Rightarrow \Delta,A/\{\}}{\Gamma \Rightarrow \Delta}
  & \quad &
  \infer[cont \Rightarrow]{\Gamma,A(k) \Rightarrow \Delta^*}{\Gamma,A(n),A(m) \Rightarrow \Delta}\\
  \\[-0.5pt]
  \infer[\Rightarrow cont]{\Gamma \Rightarrow \Delta,A/S_1 \cup S_1}{\Gamma \Rightarrow \Delta,A/S_1,A/S_1}
  & \quad &
  \infer[\lor \Rightarrow]{\Gamma_1,\Gamma_1,(A \lor B)(k) \Rightarrow \Delta^*_1,\Delta^*_1}{\Gamma_1,A(n) \Rightarrow \Delta_1 & \Gamma_1,B(m) \Rightarrow \Delta_1}\\
  \\[-0.5pt]
  \infer[\Rightarrow \lor]{\Gamma \Rightarrow \Delta,(A \lor B)/S_1 \cup S_1}{\Gamma \Rightarrow \Delta,A/S_1,B/S_1}
  & \quad &
  \infer[\land \Rightarrow]{\Gamma,(A \land B)(k) \Rightarrow \Delta^*}{\Gamma,A(n),B(m) \Rightarrow \Delta}\\
  \\[-0.5pt]
  \infer[\Rightarrow \land]{\Gamma \Rightarrow \Delta,(A \land B)/S_1 \cup S_1}{\Gamma \Rightarrow \Delta,A/S_1 & \Gamma \Rightarrow \Delta,B/S_1}
  & \quad &
  \infer[\to \Rightarrow]{(A \to B)(n),\Gamma_1,\Gamma_1 \Rightarrow \Delta_1,\Delta^*_1}{\Gamma_1 \Rightarrow \Delta_1,A/S & B(n),\Gamma_1 \Rightarrow \Delta_1}\\
  \\[-0.5pt]
  \infer[\Rightarrow \to]{\Gamma \Rightarrow \Delta, (A \to B)/S - \{n\}}{\Gamma, A(n) \Rightarrow \Delta,B/S}\\
\end{array}
\]
\end{calculus}

% The following sections ("clarifications", "history", 
% "technicalities") are optional. If you use them, 
% be very concise and objective. Nevertheless, do write full sentences. 
% Try to have at most one paragraph per section, because line breaks 
% do not look nice in a short entry.

\begin{clarifications}
Sequents are of the form $\Gamma \Rightarrow \Delta$ where $\Gamma$ is
a multiset of pairs of formulas and natural number indicies, and
$\Delta$ is a multiset of pairs of formulas and sets of natural number
indices.  The set of natural number indices for a particular
conclusion, formula on the right, indicates which hypotheses the
conclusion depends on.  This dependency tracking is used to inforce
intuitionism in the rule $\Rightarrow \to$.  See \cite{dePaiva:2005}
for more details.
\end{clarifications}

\begin{history}
%% ToDo: write here short historical remarks about this proof system,
%% especially if they relate to other proof systems. 
%% Use "\iref{OtherProofSystem}" to refer to another proof system 
%% in the Encyclopedia (where "OtherProofSystem" is its ID). 
%% Use "\irefmissing{SuggestedIDForOtherProofSystem}" to refer to 
%% another proof system that is not yet available in the encyclopedia.
The system FIL was announced in the abstract \cite{dePaiva:1995} but
only published officially ten years later in \cite{dePaiva:2005}.  The
system was conceived after the remark in the paper describing
FILL \iref{FILL} that intuitionism is about proofs that resemble
functions, not about a cardinality constraint in the sequent
calculus. The system shows we can use a notion of \em{dependency
between formulae} to enforce the constructive character of
derivations. This is similar to an impoverished Curry-Howard term
assignment.
\end{history}

% General Instructions:
% =====================

% The preferred length of an entry is 1 page. 
% Do the best you can to fit your proof system in one page.
%
% If you are finding it hard to fit what you want in one page, remember:
%
%   * Your entry needs to be neither self-contained nor fully understandable
%     (the interested reader may consult the cited full paper for details)
%
%   * If you are describing several proof systems in one entry, 
%     consider splitting your entry.
%
%   * You may reduce the size of your entry by ommitting inference rules
%     that are already described in other entries.
%
%   * Cite parsimoniously (see detailed citation instructions below).
%
% 
% If you do not manage to fit everything in one page, 
% it is acceptable for an entry to have 2 pages.
%
% For aesthetical reasons, it is preferable for an entry to have
% 1 full page or 2 full pages, in order to avoid unused blank space.



% Citation Instructions:
% ======================

% Please cite the original paper where the proof system was defined.
% To do so, you may use the \cite command within 
% one of the optional environments above,
% or use the \nocite command otherwise.

% You may also cite a modern paper or book where the 
% proof system is explained in greater depth or clarity.
% Cite parsimoniously.

% Do not cite related work. Instead, use the "\iref" or "\irefmissing" 
% commands to make an internal reference to another entry, 
% as explained within the "history" environment above.

% You do not need to create the "References" section yourself. 
% This is done automatically.




% Leave an empty line above "\end{entry}".

\end{entry}
