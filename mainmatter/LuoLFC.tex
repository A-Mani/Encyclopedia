% If the calculus has an acronym, define it.
	% (e.g. \newcommand{\LK}{\ensuremath{\mathbf{LK}}\xspace})
	\newcommand{\LFC}{\ensuremath{\mathbf{LF_{c}}}\xspace}
	\calculusName{LF with Coercive Subtyping} % The name of the calculus
	\calculusAcronym{LFc} % The acronym if defined above, or empty otherwise.
	\calculusLogic{Type Theory} % Specify the logic (e.g. classical, intuitionistic, ...) for which this calculus is intended.
	\calculusType{Extension of a Logical Framework} % Specify the calculus type (e.g. Frege-Hilbert style, tableau, sequent calculus, hypersequent calculus, natural deduction, ...)
	\calculusYear{1996} % The year when the calculus was invented.
	\calculusAuthor{Zhaohui Luo} % The name(s) of the author(s) of the calculus.
	
	
	\entryTitle{T[C] (extension of LF with Coercive Subtyping)} 
% Title of the entry (usually coincides with the name of the calculus).
	\entryAuthor{Zhaohui Luo \and Sergei Soloviev} % Your name(s). Separate multiple names with "\and".
	
	
	% If you wish, use tags to give any other information
	% that might be helpful for classifying and grouping this entry:
	% e.g. \tag{Two-Sided Sequents}
	% e.g. \tag{Multiset Cedents}
	% e.g. \tag{List Cedents}
	% You are free to invent your own tags.
	% The Encyclopedia's coordinator will take care of
	% merging semantically similar tags in the future.
	
	
	\maketitle
	
	
	% If your files are called "<ID>.tex" and "<ID>.bib",
	% then you should write "\begin{entry}{<ID>}" in the line below
	\begin{entry}{LuoLFC}
	
	% Define here any newcommands you may need:
	% e.g. \newcommand{\necessarily}{\Box}
	% e.g. \newcommand{\possibly}{\Diamond}
	
	
	\begin{calculus}
	
\centering
Basic subkinding rule
$$
\infer{\Gamma\seq El(A)<_cEl(B)}{\Gamma\seq A<_cB:{\bf Type}}
$$
Subkinding for dependent product kinds
$$\infer{\Gamma\seq (x:K_1)K_2<_{[f:(x:K_1)K_2][x:K'_1]c(fx)}(x:K'_1)K'_2}
{\Gamma\seq K'_1=K_1\!\quad\!\Gamma, x:K'_1\seq K_2<_cK'_2
\!\quad\!\Gamma, x:K_1\seq K_2:{\bf kind}}$$
$$\infer{\Gamma\seq (x:K_1)K_2<_{[f:(x:K_1)K_2][x:K'_1]f(cx)}(x:K'_1)K'_2}
{\Gamma\seq K'_1<_cK_1\!\quad\!\Gamma, x:K'_1\seq [cx/x]K_2=K'_2
\!\quad\!\Gamma, x:K_1\seq K_2:{\bf kind}}$$
$$\infer{\Gamma\seq (x:K_1)K_2<_{[f:(x:K_1)K_2][x:K'_1]c_2(f(c_1x))}(x:K'_1)K'_2}
{\Gamma\seq K'_1<_{c_1}K_1\!\quad\!\Gamma, x:K'_1\seq [c_1x/x]K_2<_{c_2}K'_2
\!\quad\!\Gamma, x:K_1\seq K_2:{\bf kind}}$$
Coercive application rules
$$\infer{\Gamma\seq f(k_0):[c(k_0)/x]K'}{\Gamma\seq f:(x:K)K'\!\quad\!\Gamma\seq k_0:K_0\!\quad\!
\Gamma\seq K_0<_cK}$$
$$\infer{\Gamma\seq f(k_0)=f'(k'_0):[c(k_0)/x]K'}
{\Gamma\seq f(k_0)=f(ck_0):(x:K)K'\!\quad\!\Gamma\seq k_0:K_0\!\quad\!
\Gamma\seq K_0<_cK}$$
Coercive definition rule
$$\infer{\Gamma\seq f(k_0):[c(k_0)/x]K'}
{\Gamma\seq f:(x:K)K'\!\quad\!\Gamma\seq k_0:K_0\!\quad\!\Gamma\seq K_0<_cK}$$

	
	\end{calculus}
	
	% The following environments ("clarifications", "history",
	% "technicalities") are optional. If you do use them,
	% be very concise and objective.
	
	\begin{clarifications}
We follow~\cite{LuoSolXue:13}. In this entry the extensions $T[C]$ of the logical framework $\LuoLF$\iref{LuoLF}
are considered. Here $T$ is a type theory specified in $\LuoLF$ (formally, an extension
of $\LuoLF$) and
 $C$ is a (possibly infinite)
set of subtyping judgements of the form $\Gamma\seq A<_cB:Type$. 
The set $C$ itself
may be generated by some user-defined rules. As coercive definition rule
above shows, coercive subtyping is considered as an 
{\em abbreviation mechanism}, the expressions without coercions are 
considered as ``abbreviations'' of the expressions where coercions are 
inserted. For corcive subtyping as an abbreviation mechanism, one of central
questions is the concervativity of the extension $T[C]$ over $T$.

The system $T[C]$ is built by ``layers''
and in this sense may be considered as {\em hybrid}. Above the rules
(except structural rules) of the {\em subkinding} level are given. 
The structure (and rules)
of the subtyping level, as well as its connection with the 
subkinding level, are explained below. 

First the intermediate system $T[C]_0$ is defined. 
The syntax of $T[C]_0$ is the same
as the syntax of $T$ ({\em i.e.}, type theory specified in $\LuoLF$). The rule
 $$\infer{\Gamma\seq A<_cB:Type}{\Gamma\seq A<_cB:Type\in C}$$ is added, 
and the structural subtyping rules
given below.  They state that the subtyping relation $<$ (annotated
by coercion terms $c$) is congruent, transitive, and closed under
substitution, and satisfies the rules of weakening and contextual equality.

{\bf Remark.} Similar structural rules are included in the subkinding level 
above. 
 
\end{clarifications}
	
\begin{calculus}
	
\centering
Structural rules
$$
\infer{\Gamma\seq A'<_{c'}B'}{\Gamma\seq A<_cB\!\quad\!\Gamma\seq A=A':Type
\!\quad\!\Gamma\seq B=B'\!\quad\!\Gamma\seq c=c':(El(A))El(B)}
$$
$$\infer{\Gamma\seq A_{c'\circ c}A''}{\Gamma\seq A<_cA'\!\quad\!\Gamma\seq A'<_{c'}A''}$$
$$\infer{\Gamma, [k/x]\Gamma'\seq [k/x]A<_{[k/x]c}[k/x]B}{\Gamma, x:K, \Gamma'\seq A<_c B
\!\quad\!\Gamma\seq k:K}\,\,\,\,\,\infer{\Gamma, \Gamma'',\Gamma'\seq A<_cB}{\Gamma, \Gamma'\seq A<_cB\!\quad\!\Gamma, 
\Gamma''\seq {\bf valid}}$$
$$\infer{\Gamma, x:K', \Gamma'\seq A<_c B}{\Gamma, x:K, \Gamma'\seq A<_c B
\!\quad\!\Gamma \seq K=K'}$$
	
	\end{calculus}

Main requirement to the set $C$ (expressed in terms of $T[C]_0$) is {\em coherence}:
\begin{itemize}
\item If $\Gamma\seq A<_cB:Type$ then $\Gamma\seq A:Type$, $\Gamma\seq B:Type$ and $\Gamma\seq c: (El(A))El(B)$.
\item $\Gamma\not\seq A<_cA:Type$ for any $\Gamma, A, c$.
\item If $\Gamma\seq A<_c B:Type$ and $\Gamma\seq A<_{c'}B:Type$ then $\Gamma\seq c=c':(El(A))El(B)$.
\end{itemize}

	\begin{history}
Coercive subtiping as an abbreviation mechanism
was introduced in a conference paper~\cite{Luo96:CSL}.  
It was described for type theories specified in Z. Luo's typed $\LuoLF$
(extensions of $\LuoLF$)~\iref{LuoLF}, but
the idea itself is much more general and may apply
to other type theories. The approach was further
developed in~\cite{jls:TYPES96, Luo:99, SolLuo:02, LuoSolXue:13}.

	% ToDo: write here short historical remarks about this proof system,
	% especially if they relate to other proof systems.
	% Use "\iref{OtherProofSystem}" to refer to another proof system
	% in the Encyclopedia (where "OtherProofSystem" is its ID).
	% Use "\irefmissing{SuggestedIDForOtherProofSystem}" to refer to
	% another proof system that is not yet available in the encyclopedia.
	 \end{history}
	
	\begin{technicalities}
	Main result
(that justifies the view of coercive
subtyping as an abbreviation mechanism)
is the conservativity theorem (conservativity of $T[C]$ w.r.t. the
type theory $T$), see~\cite{Luo:99, SolLuo:02, LuoSolXue:14}.
	\end{technicalities}	
	
	% Please cite the original paper where the proof system was defined.
	% To do so, you may use the \cite command within
	% one of the optional environments above,
	% or use the \nocite command otherwise.
	
	% You may also cite a modern paper or book where the
	% proof system is explained in greater depth or clarity.
	% Cite parsimoniously.
	
	% Do not cite related work. Instead, use the "\iref" or "\irefmissing"
	% commands to make an internal reference to another entry,
	% as explained within the "history" environment above.
	
	% You do not need to create the "References" section yourself.
	% This is done automatically.
	
	
	% Leave an empty line above "\end{entry}".
	
	\end{entry}
