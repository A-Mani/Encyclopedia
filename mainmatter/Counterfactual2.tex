
% If the calculus has an acronym, define it.
% (e.g. \newcommand{\LK}{\ensuremath{\mathbf{LK}}\xspace})

\calculusName{Calculi for Lewis' Counterfactual Logics II} % The name of the calculus
\calculusAcronym{} % The acronym if defined above, or empty otherwise.
\calculusYear{2012, 2013} % The year when the calculus was invented.
\calculusAuthor{Lellmann, Pattinson} % The name(s) of the author(s) of the calculus.

\entryTitle{Calculi for Lewis' Counterfactual Logics II}
\entryAuthor{Bj{\"o}rn Lellmann} % Your name(s). Separate multiple names with "\and"

\maketitle

% If your files are called "<ID>.tex" and "<ID>.bib",
% then you should write "\begin{entry}{<ID>}" in the line below

\begin{entry}{Counterfactual2}

% Define here any newcommands you may need:
% e.g. \newcommand{\necessarily}{\Box}
% e.g. \newcommand{\possibly}{\Diamond}

\newcommand{\nc}{\newcommand}
\nc{\rarr}{\rightarrow}
\nc{\scimp}{\boxRight} % uses package txfonts
\nc{\CC}{\mathbb{C}}
\nc{\NN}{\mathbb{N}}
\newcommand{\Rules}{\mathcal{R}}
\nc{\TT}{\mathbb{T}}
\nc{\VV}{\mathbb{V}}
\nc{\WW}{\mathbb{W}}


\begin{calculus}
\[
\vcenter{
  \infer[R_{n,m}]
   {\Gamma,(A_1\scimp B_1),\dots,(A_n\scimp B_n)\seq
        \Delta,(C_1\scimp D_1),\dots,(C_m\scimp D_m)}
   {\begin{array}{c}
           \big\{\;C_k, \vec{B}^I \seq \vec{A}^{[n]\smallsetminus I},
           \vec{C}^J, \vec{D}^{[k-1]\smallsetminus J} \mid 1\leq k\leq m,\,
           I\subseteq [n],\, J\subseteq [k-1]\;\big\}\\
           \cup\;\big\{\; A_k,B_k, \vec{B}^I \seq
           \vec{A}^{[n]\smallsetminus I}, \vec{C}^J,
           \vec{D}^{[m]\smallsetminus J}\mid k\leq n, I\subseteq [n],
           J\subseteq [m]\;\big\} 
         \end{array}
  }
}
\]
\[
\vcenter{
  \infer[T_m]
  {\Gamma \seq
        \Delta,(C_1\scimp D_1),\dots,(C_m\scimp D_m)
  }
  {\big\{\;
    \Gamma \seq \Delta,\vec{C}^J, \vec{D}^{[m]\smallsetminus J} \mid J
    \subseteq [m]
    \;\big\}
    \;\cup\;
    \big\{\;
    C_k \seq D_k, \vec{C}^J, \vec{D}^{[k-1]\smallsetminus J} \mid 1
    \leq k \leq m, J \subseteq [k-1]
    \;\big\}
  }
}
\]
\[
\vcenter{
  \infer[W_{n,m}]
    {\Gamma,(A_1\scimp B_1),\dots,(A_n\scimp
      B_n)\seq\Delta,(C_1\scimp D_1),\dots,(C_m\scimp D_m)}
    {\begin{array}{c}
      \big\{\;C_k,\vec{B}^I \seq \vec{A}^{[n]\smallsetminus I},
           \vec{C}^{J},\vec{D}^{[k-1]\smallsetminus J} \mid 1 \leq k\leq m,\,
           I\subseteq [n],\, J\subseteq [k-1]\;\big\} \\
        \cup \; \big\{\;\Gamma,\vec{B}^I \seq
        \vec{A}^{[n]\smallsetminus I},
        \vec{C}^J,\vec{D}^{[m]\smallsetminus J} \mid I\subseteq
        [n], J\subseteq [m]\;\big\}
      \end{array}
  }
}
\]
\[
\vcenter{
  \infer[R_{C1}]
    {\Gamma,(A\scimp B)\seq\Delta
    }
    {\Gamma\seq\Delta,A \quad&\quad \Gamma,B\seq\Delta
    }
}
\quad
\vcenter{
  \infer[R_{C2}]
      {\Gamma\seq\Delta,(A\scimp B)
      }
      {\Gamma\seq\Delta,A\quad &\quad
        \Gamma,A\seq\Delta,B
      }
}
\]
\centerline{\small For $n>0$ the set $[n]$ is $\{1,
  \dots, n \}$ and $[0]$ is $\emptyset $. For a set $I$ of indices,
  $\vec{A}^I$ contains all $A_i$ with $i \in I$.
  }\\

\begin{center}
\begin{tabular}{c@{\qquad}c}
\multicolumn{2}{c}{
    $\Rules_{\VV_\scimp} = \{R_{n,m} \mid n\geq 1, m\geq 0\}$
 }\\
\begin{tabular}{lll}
$\Rules_{\VV\NN_\scimp}$ & = & $\{ R_{n,m} \mid n+m \geq 1\}$\\
$\Rules_{\VV\TT_\scimp}$ & = & $\Rules_{\VV_\scimp} \cup \{ T_m \mid m
\geq 1\}$\\
\end{tabular}
&
\begin{tabular}{lll}
$\Rules_{\VV\WW_\scimp}$ & = &$\Rules_{\VV\TT_\scimp}\cup
\{W_{n,m}\mid n+m \geq 1\}$\\
$\Rules_{\VV\CC_\scimp}$ & = &
$\Rules_{\VV_\scimp}\cup\{R_{C1},R_{C2}\}$\\
\end{tabular}
\end{tabular}
\end{center}

% Add the inference rules of your proof system here.
% The "proof.sty" and "bussproofs.sty" packages are available.
% If you need any other package, please contact the editor (bruno@logic.at)
% ToDo

\end{calculus}

% The following environments ("clarifications", "history",
% "technicalities") are optional. If you do use them,
% be very concise and objective.
\begin{clarifications}
% ToDo: write here short remarks that may help the reader to understand
% the inference rules of the proof system.
  Sequents are based on multisets.  The propositional part is that of
  \Gtc \iref{G3c}. Also includes the contraction rules. Rules
  $\Rules_{\mathcal{L}_\scimp}$ are for the logic $\mathcal{L}$ in
  terms of the \emph{strong counterfactual implication} $\scimp$ from
  \cite{Lewis:1973uq}.
\end{clarifications}

\begin{history}
% ToDo: write here short historical remarks about this proof system,
% especially if they relate to other proof systems.
% Use "\iref{OtherProofSystem}" to refer to another proof system
% in the Encyclopedia (where "OtherProofSystem" is its ID).
% Use "\irefmissing{SuggestedIDForOtherProofSystem}" to refer to
% another proof system that is not yet available in the encyclopedia.
  Introduced in \cite{Lellmann:2012fk}, corrected in
  \cite{Lellmann:2013}.
\end{history}

\begin{technicalities}
% ToDo: write here remarks about soundness, completeness, decidability...
  Translations of the calculi \iref{counterfactual} to the language
  with $\scimp$. Inherit cut elimination and yield $\mathsf{PSPACE}$
  decision procedures. Contraction can be made admissible.
\end{technicalities}
% Please cite the original paper where the proof system was defined.
% To do so, you may use the \cite command within
% one of the optional environments above,
% or use the \nocite command otherwise.
% You may also cite a modern paper or book where the
% proof system is explained in greater depth or clarity.
% Cite parsimoniously.
% Do not cite related work. Instead, use the "\iref" or "\irefmissing"
% commands to make an internal reference to another entry,
% as explained within the "history" environment above.
% You do not need to create the "References" section yourself.
% This is done automatically.

\end{entry}

%%% Local Variables: 
%%% mode: latex
%%% TeX-master: "../main"
%%% End: 
