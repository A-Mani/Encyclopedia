
% If the calculus has an acronym, define it.
% (e.g. \newcommand{\LK}{\ensuremath{\mathbf{LK}}\xspace})

\calculusName{Sequent Calculus LJ}   % The name of the calculus
\calculusAcronym{}    % The acronym if defined above, or empty otherwise. 
\calculusYear{1935}   % The year when the calculus was invented.
\calculusAuthor{Gerhard Karl Erich Gentzen} % The name(s) of the author(s) of the calculus.
\entryAuthor{Giselle Reis}    % Your name(s). Separate multiple names with "\and"

\maketitle


% If your files are called "<ID>.tex" and "<ID>.bib", 
% then you should write "\begin{entry}{<ID>}" in the line below
\begin{entry}{SequentCalculusLJ}  

% Define here any newcommands you may need:
% e.g. \newcommand{\necessarily}{\Box}
% e.g. \newcommand{\possibly}{\Diamond}


\begin{calculus}

% Add the inference rules of your proof system here.
% The "proof.sty" and "bussproofs.sty" packages are available.
% If you need any other package, please contact the editor (bruno@logic.at)
\[
\begin{array}{cc}
\infer[init]{A \vdash A}{}
&
\infer[cut]{\Gamma_1, \Gamma_2 \vdash C}{\Gamma_1 \vdash P & \Gamma_2, P \vdash C}
\\[8pt]
\infer[\neg_l]{\Gamma, \neg P \vdash }{\Gamma \vdash P}
&
\infer[\neg_r]{\Gamma \vdash \neg P}{\Gamma, P \vdash }
\\[8pt]
\infer[\wedge_{li}]{P_1 \wedge P_2, \Gamma \vdash C}{P_i, \Gamma \vdash C}
&
\infer[\wedge_r]{\Gamma \vdash P \wedge Q}{\Gamma \vdash P & \Gamma \vdash Q}
\\[8pt]
\infer[\vee_l]{P \vee Q, \Gamma \vdash C}{P, \Gamma \vdash C & Q, \Gamma \vdash C}
&
\infer[\vee_{ri}]{\Gamma \vdash P_1 \vee P_2}{\Gamma \vdash P_i}
\\[8pt]
\infer[\rightarrow_l]{P \rightarrow Q, \Gamma_1, \Gamma_2 \vdash C}{\Gamma_1
\vdash P & Q, \Gamma_2 \vdash C}
&
\infer[\rightarrow_r]{\Gamma \vdash P \rightarrow Q}{\Gamma, P \vdash Q}
\\[8pt]
\infer[\exists_l]{\exists x.P, \Gamma \vdash C}{P\{x \leftarrow
\alpha\}, \Gamma \vdash C}
&
\infer[\exists_r]{\Gamma \vdash \exists x.P}{\Gamma \vdash P\{x
\leftarrow t\}}
\\[8pt]
\infer[\forall_l]{\forall x.P, \Gamma \vdash C}{P\{x \leftarrow t\}, \Gamma \vdash C}
&
\infer[\forall_r]{\Gamma \vdash \forall x.P}{\Gamma \vdash P\{x
\leftarrow \alpha \}}
\\
\end{array}
\]
$$
\infer[c_l]{P, \Gamma \vdash C}{P, P, \Gamma \vdash C}
\qquad
\infer[w_l]{P, \Gamma \vdash C}{\Gamma \vdash C}
\qquad
\infer[w_r]{\Gamma \vdash P}{\Gamma \vdash}
$$

\end{calculus}

% The following environments ("clarifications", "history", 
% "technicalities") are optional. If you do use them, 
% be very concise and objective.

\begin{clarifications}
% ToDo: write here short remarks that may help the reader to understand 
% the inference rules of the proof system.
Assuming that $\alpha$ is a variable not contained in $P$, $\Gamma$ or $C$,
$t$ does not contain variables bound in $P$ and $C$ stands for
one formula or the empty set.
\end{clarifications}

\begin{history}
% ToDo: write here short historical remarks about this proof system,
% especially if they relate to other proof systems. 
% Use "\iref{OtherProofSystem}" to refer to another proof system 
% in the Encyclopedia (where "OtherProofSystem" is its ID). 
% Use "\irefmissing{SuggestedIDForOtherProofSystem}" to refer to 
% another proof system that is not yet available in the encyclopedia.
Proposed by Gentzen in \cite{Gentzen1935} by restricting the
succedent of sequents in \irefmissing{SequentCalculusLK} to have at most one
formula. In the original paper, he notes that this restriction is equivalent to
removing the principle of excluded middle from the natural deduction system
\iref{NaturalDeduction} in order to obtain \irefmissing{NJ}.
% NOTE Assuming that Natural Deduction is the classical version.
The cut is admissible in LJ and this result is known as \emph{Hauptsatz}.
\end{history}

\begin{technicalities}
% ToDo: write here remarks about soundness, completeness, decidability...
Soundness and completeness of LJ can be proved using a translation of LJ
derivations into \irefmissing{NJ}.
Decidability of the propositional fragment and consistency of intuitionistic
logic follows from the cut admissibility in this calculus.
\end{technicalities}



% Please cite the original paper where the proof system was defined.
% To do so, you may use the \cite command within 
% one of the optional environments above,
% or use the \nocite command otherwise.

% You may also cite a modern paper or book where the 
% proof system is explained in greater depth or clarity.
% Cite parsimoniously.

% Do not cite related work. Instead, use the "\iref" or "\irefmissing" 
% commands to make an internal reference to another entry, 
% as explained within the "history" environment above.

% You do not need to create the "References" section yourself. 
% This is done automatically.

\end{entry}
