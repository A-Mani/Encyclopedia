
\calculusName{Polarized Linear Sequent Calculus}   % The name of the calculus
\calculusAcronym{\LLP}    % The acronym if defined above, or empty otherwise. 
\calculusLogic{Polarized Linear Logic}  % Specify the logic (e.g. classical, intuitionistic, ...) for which this calculus is intended.
\calculusType{sequent calculus}   % Specify the calculus type (e.g. Frege-Hilbert style, tableau, sequent calculus, hypersequent calculus, natural deduction, ...)
\calculusYear{2000}   % The year when the calculus was invented.
\calculusAuthor{Olivier Laurent} % The name(s) of the author(s) of the calculus.


\entryTitle{Polarized Linear Sequent Calculus \LLP}     % Title of the entry (usually coincides with the name of the calculus).
\entryAuthor{Olivier Laurent}    % Your name(s). Separate multiple names with "\and".


% If you wish, use tags to give any other information 
% that might be helpful for classifying and grouping this entry:
% e.g. \tag{Two-Sided Sequents}
% e.g. \tag{Multiset Cedents}
% e.g. \tag{List Cedents}
% You are free to invent your own tags. 
% The Encyclopedia's coordinator will take care of 
% merging semantically similar tags in the future.
\tag{One-Sided Sequents}
\tag{List Cedents}

\maketitle


% If your files are called "MyProofSystem.tex" and "MyProofSystem.bib", 
% then you should write "\begin{entry}{MyProofSystem}" in the line below
\begin{entry}{LLP}  

% Define here any newcommands you may need:
% e.g. \newcommand{\necessarily}{\Box}
% e.g. \newcommand{\possibly}{\Diamond}


\begin{calculus}
% Add the inference rules of your proof system here.
% The "proof.sty" and "bussproofs.sty" packages are available.
% If you need any other package, please contact the editor (bruno@logic.at)
\begin{center}
\AxiomC{}
\UnaryInfC{$\seq P^\bot,P$}
\DisplayProof
\qquad
\AxiomC{$\seq \Gamma,P$}
\AxiomC{$\seq \Delta,P^\bot,\Pi$}
\BinaryInfC{$\seq \Gamma,\Delta,\Pi$}
\DisplayProof
\qquad
\AxiomC{$\seq \Gamma,\Pi$}
\UnaryInfC{$\seq \sigma(\Gamma),\Pi$}
\DisplayProof
\\[2ex]
\AxiomC{$\seq \Gamma,P$}
\AxiomC{$\seq \Delta,Q$}
\BinaryInfC{$\seq \Gamma,\Delta,P\otimes Q$}
\DisplayProof
\qquad
\AxiomC{$\seq \Gamma,N,M,\Pi$}
\UnaryInfC{$\seq \Gamma,N\parr M,\Pi$}
\DisplayProof
\qquad
\AxiomC{}
\UnaryInfC{$\seq 1$}
\DisplayProof
\qquad
\AxiomC{$\seq \Gamma,\Pi$}
\UnaryInfC{$\seq \Gamma,\bot,\Pi$}
\DisplayProof
\\[2ex]
\AxiomC{$\seq \Gamma,P$}
\UnaryInfC{$\seq \Gamma,P\oplus Q$}
\DisplayProof
\qquad
\AxiomC{$\seq \Gamma,Q$}
\UnaryInfC{$\seq \Gamma,P\oplus Q$}
\DisplayProof
\qquad
\AxiomC{$\seq \Gamma,M,\Pi$}
\AxiomC{$\seq \Gamma,N,\Pi$}
\BinaryInfC{$\seq \Gamma,M\with N,\Pi$}
\DisplayProof
\qquad
\AxiomC{}
\UnaryInfC{$\seq \Gamma,\top,\Pi$}
\DisplayProof
\\[2ex]
\AxiomC{$\seq \Gamma,P$}
\UnaryInfC{$\seq \Gamma,\wn P$}
\DisplayProof
\qquad
\AxiomC{$\seq \Gamma,N$}
\UnaryInfC{$\seq \Gamma,\oc N$}
\DisplayProof
\qquad
\AxiomC{$\seq \Gamma,N,N,\Pi$}
\UnaryInfC{$\seq \Gamma,N,\Pi$}
\DisplayProof
\qquad
\AxiomC{$\seq \Gamma,\Pi$}
\UnaryInfC{$\seq \Gamma,N,\Pi$}
\DisplayProof
\\[2ex]
\[
\begin{array}{c@{\quad}c@{\quad}c}
& (P\otimes Q)^\bot = P^\bot\parr Q^\bot
& 1^\bot = \bot
\\
(\oc N)^\bot = \wn (N^\bot)
& (P\oplus Q)^\bot = P^\bot\with Q^\bot
& 0^\bot = \top
\\
(X^\bot)^\bot = X
& (N\parr M)^\bot = N^\bot\otimes M^\bot
& \bot^\bot = 1
\\
(\wn P)^\bot = \oc (P^\bot)
& (N\with M)^\bot = N^\bot\oplus M^\bot
& \top^\bot = 0
\end{array}
\]
\[
\begin{array}{lrlccccccccccc}
\textrm{Positive formulas:} & P,Q & ::= & X & \mid & P\otimes Q & \mid & 1 & \mid & P\oplus Q & \mid & 0 & \mid & \oc N \\
\textrm{Negative formulas:} & N,M & ::= & X^\bot & \mid & N\parr M & \mid & \bot & \mid & N\with M & \mid & \top & \mid & \wn P \\
\end{array}
\]
$\Gamma$ and $\Delta$ are lists of negative formulas.\\
$\Pi$ consists of 0 or 1 positive formula.\\
$\sigma$ is a permutation.
\end{center}
\end{calculus}

% The following sections ("clarifications", "history", 
% "technicalities") are optional. If you use them, 
% be very concise and objective. Nevertheless, do write full sentences. 
% Try to have at most one paragraph per section, because line breaks 
% do not look nice in a short entry.

\begin{clarifications}
% ToDo: write here short remarks that may help the reader to understand 
% the inference rules of the proof system.
Negation is not a connective. It is defined using De Morgan's laws so that $(A^\bot)^\bot=A$.
Negative connectives which turn negative formulas into negative formulas ($\parr$, $\bot$, $\with$ and $\top$) are the reversible connectives of \LL~\iref{LL}.
Their dual, the positive connectives ($\otimes$, $1$, $\oplus$, $0$) have the focusing property~\cite{focal}, related here with the ``at most one positive formula'' property of sequents.
\end{clarifications}

\begin{history}
% ToDo: write here short historical remarks about this proof system,
% especially if they relate to other proof systems. 
% Use "\iref{OtherProofSystem}" to refer to another proof system 
% in the Encyclopedia (where "OtherProofSystem" is its ID). 
% Use "\irefmissing{SuggestedIDForOtherProofSystem}" to refer to 
% another proof system that is not yet available in the encyclopedia.
\LLP~\cite{phdlaurent} comes from the natural embedding of Girard's \LC~\iref{LC} into linear logic~\iref{LL}. It is obtained by restricting \LL{} to polarized formulas and then by generalizing the structural rules (contraction, weakening and context of promotion) to arbitrary negative formulas, not only those starting with a $\wn$-connective.
\end{history}

\begin{technicalities}
% ToDo: write here remarks about soundness, completeness, decidability...
Cut elimination holds.
In the categorical models of \LLP, positive formulas are interpreted as $\otimes$-comonoids while negative formulas are interpreted as $\parr$-monoids.
\end{technicalities}


% General Instructions:
% =====================

% The preferred length of an entry is 1 page. 
% Do the best you can to fit your proof system in one page.
%
% If you are finding it hard to fit what you want in one page, remember:
%
%   * Your entry needs to be neither self-contained nor fully understandable
%     (the interested reader may consult the cited full paper for details)
%
%   * If you are describing several proof systems in one entry, 
%     consider splitting your entry.
%
%   * You may reduce the size of your entry by ommitting inference rules
%     that are already described in other entries.
%
%   * Cite parsimoniously (see detailed citation instructions below).
%
% 
% If you do not manage to fit everything in one page, 
% it is acceptable for an entry to have 2 pages.
%
% For aesthetical reasons, it is preferable for an entry to have
% 1 full page or 2 full pages, in order to avoid unused blank space.



% Citation Instructions:
% ======================

% Please cite the original paper where the proof system was defined.
% To do so, you may use the \cite command within 
% one of the optional environments above,
% or use the \nocite command otherwise.

% You may also cite a modern paper or book where the 
% proof system is explained in greater depth or clarity.
% Cite parsimoniously.

% Do not cite related work. Instead, use the "\iref" or "\irefmissing" 
% commands to make an internal reference to another entry, 
% as explained within the "history" environment above.

% You do not need to create the "References" section yourself. 
% This is done automatically.




% Leave an empty line above "\end{entry}".

\end{entry}



%%% Local Variables: 
%%% mode: latex
%%% TeX-master: "main"
%%% End: 
