
% If the calculus has an acronym, define it.
% (e.g. \newcommand{\LK}{\ensuremath{\mathbf{LK}}\xspace})

\calculusName{Intuinistic Linear Logic Sequent Calculus}   % The name of the calculus
\calculusAcronym{ILL}    % The acronym if defined above, or empty otherwise. 
\calculusLogic{ILL}  % Specify the logic (e.g. classical, intuitionistic, ...) for which this calculus is intended.
\calculusType{sequent calculus}   % Specify the calculus type (e.g. Frege-Hilbert style, tableau, sequent calculus, hypersequent calculus, natural deduction, ...)
\calculusYear{1987}   % The year when the calculus was invented.
\calculusAuthor{Jean-Yves Girard \and Yves Lafont} % The name(s) of the author(s) of the calculus.


\entryTitle{Intuitionistic Linear Logic Sequent Calculus}     % Title of the entry (usually coincides with the name of the calculus).
\entryAuthor{Joseph Boudou \and Sergei Soloviev}    % Your name(s). Separate multiple names with "\and".


% If you wish, use tags to give any other information 
% that might be helpful for classifying and grouping this entry:
% e.g. \tag{Two-Sided Sequents}
% e.g. \tag{Multiset Cedents}
% e.g. \tag{List Cedents}
% You are free to invent your own tags. 
% The Encyclopedia's coordinator will take care of 
% merging semantically similar tags in the future.


\maketitle


% If your files are called "MyProofSystem.tex" and "MyProofSystem.bib", 
% then you should write "\begin{entry}{MyProofSystem}" in the line below
\begin{entry}{ILL} 

% Define here any newcommands you may need:
% e.g. \newcommand{\necessarily}{\Box}
% e.g. \newcommand{\possibly}{\Diamond}
\newcommand{\llaconj}{\binampersand}
\newcommand{\lladisj}{\oplus}
\newcommand{\llimp}{\multimap}
\newcommand{\llmconj}{\otimes}
\newcommand{\llmdisj}{\bindnasrepma}
\newcommand{\llneg}[1]{{#1}^\bot}
\newcommand{\llzero}{0}
\newcommand{\llone}{1}

\newcommand{\sepproof}{\hskip 2em plus 6em\relax}
\newcommand{\sepseq}{\quad}
\newcommand{\sepline}{\]\[}

\newenvironment{infruleset}[1]{%
  \sc{#1} \vspace{-1ex} \[ %
}{%
  \] %
}

\begin{calculus}

\begin{infruleset}{Structural}
  \infer{A \vdash A}{}
  \sepproof
  \infer[\mbox{\textit{(cut)}}]{\Gamma, \Delta \vdash B}{\Gamma \vdash A \sepseq A, \Delta \vdash B}
  \sepproof
  \infer{\Gamma, B, A, \Delta \vdash C}{\Gamma, A, B, \Delta \vdash C}
\end{infruleset}

\begin{infruleset}{Multiplicative}
  \infer{\vdash \llone}{}
  \sepproof
  \infer{\Gamma, \llone \vdash A}{\Gamma \vdash A}
  \sepproof
  \infer{\Gamma, \Delta \vdash A \llmconj B}{\Gamma \vdash A \sepseq \Delta \vdash B}
  \sepproof
  \infer{\Gamma, A \llmconj B \vdash C}{\Gamma, A, B \vdash C}
  \sepline
%
  \infer{\Gamma \vdash A \llimp B}{\Gamma, A \vdash B}
  \sepproof
  \infer{\Gamma, A \llimp B, \Delta \vdash C}{\Gamma \vdash A \sepseq B, \Delta \vdash C}
\end{infruleset}

\begin{infruleset}{Additive}
  \infer{\Gamma \vdash \top}{}
  \sepproof
  \infer{\Gamma \vdash A \llaconj B}{\Gamma \vdash A \sepseq \Gamma \vdash B}
  \sepproof
  \infer{\Gamma, A_1 \llaconj A_2 \vdash B}{\Gamma, A_i \vdash B}
  \sepline
%
  \infer{\Gamma, \llzero \vdash A}{}
  \sepproof
  \infer{\Gamma \vdash A_1 \lladisj A_2}{\Gamma \vdash A_i}
  \sepproof
  \infer{\Gamma, A \lladisj B \vdash C}{\Gamma, A \vdash C \sepseq \Gamma, B \vdash C}
\end{infruleset}

\begin{infruleset}{Exponential}
  \infer{!\Gamma \vdash !A}{!\Gamma \vdash A}
  \sepproof
  \infer{\Gamma, !A \vdash B}{\Gamma \vdash B}
  \sepproof
  \infer{\Gamma, !A \vdash B}{\Gamma, A \vdash B}
  \sepproof
  \infer{\Gamma, !A \vdash B}{\Gamma, !A, !A \vdash B}
\end{infruleset}

%\begin{infruleset}{First order}
%  \infer{\Gamma \vdash \exists x. A}{\Gamma \vdash A[t]}
%  \sepproof
%  \infer{\Gamma, \exists x. A[x] \vdash B}{\Gamma, A[y] \vdash B}
%  \sepproof
%  \infer{\Gamma \vdash \forall x. A}{\Gamma \vdash A[y]}
%  \sepproof
%  \infer{\Gamma, \forall x. A[x] \vdash B}{\Gamma, A[t] \vdash B}
%\end{infruleset}
\vspace{-1em}


\end{calculus}

\begin{clarifications}
  Succedents are sole formulas.
  Antecedents are ordered list of formulas.
  If $\Gamma$ is the list $A_1, \ldots, A_n$ of formulas, $!\Gamma$ denotes the list $!A_1, \ldots, !A_n$.
  First order quantifier can be added with rules similar to LJ \iref{GentzenLJ}.
  Conversely, removing the exponential rules leads to the intuitionistic multiplicative additive linear logic (IMALL).
  And by further removing the additive rules, the intuitionistic multiplicative linear logic (IMLL) \cite{mints1977closed} is obtained.
\end{clarifications}

\begin{history}
  Introduced by Girard and Lafont in~\cite{lafont1987tapsoft} as intuitionistic variant of LL~\iref{LL}.
  By contrast with the full intuitionistic linear logic~\iref{FILL}, there is no multiplicative disjunction in ILL.
\end{history}

\begin{technicalities}
  Enjoys cut elimination~\cite{lafont1987tapsoft}.
\end{technicalities}


% General Instructions:
% =====================

% The preferred length of an entry is 1 page. 
% Do the best you can to fit your proof system in one page.
%
% If you are finding it hard to fit what you want in one page, remember:
%
%   * Your entry needs to be neither self-contained nor fully understandable
%     (the interested reader may consult the cited full paper for details)
%
%   * If you are describing several proof systems in one entry, 
%     consider splitting your entry.
%
%   * You may reduce the size of your entry by ommitting inference rules
%     that are already described in other entries.
%
%   * Cite parsimoniously (see detailed citation instructions below).
%
% 
% If you do not manage to fit everything in one page, 
% it is acceptable for an entry to have 2 pages.
%
% For aesthetical reasons, it is preferable for an entry to have
% 1 full page or 2 full pages, in order to avoid unused blank space.



% Citation Instructions:
% ======================

% Please cite the original paper where the proof system was defined.
% To do so, you may use the \cite command within 
% one of the optional environments above,
% or use the \nocite command otherwise.

% You may also cite a modern paper or book where the 
% proof system is explained in greater depth or clarity.
% Cite parsimoniously.

% Do not cite related work. Instead, use the "\iref" or "\irefmissing" 
% commands to make an internal reference to another entry, 
% as explained within the "history" environment above.

% You do not need to create the "References" section yourself. 
% This is done automatically.




% Leave an empty line above "\end{entry}".

\end{entry}
