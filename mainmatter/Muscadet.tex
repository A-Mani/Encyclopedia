
\calculusName{Natural Knowledge Bases}
\calculusAcronym{\Muscadet}
\calculusLogic{Classical Predicate Logic}
\calculusType{Natural Deduction}
\calculusYear{1984}
\calculusAuthor{Dominique Pastre}


\entryTitle{Natural Knowledge Bases - \Muscadet}
\entryAuthor{Dominique Pastre}

\tag{natural methods}
\tag{knowledge-based systems}

\maketitle

\begin{entry}{Muscadet}  

\begin{calculus}

\textbf {Some of the rules given to the system} :\\
\hspace*{0.1cm} 
Basic rules of Natural Deduction (similar to Bledsoe's SPLIT rules \iref{Bledsoe}).\\
\hspace*{0.1cm} 
Flatten : Replace $P(f(x))$ by $\exists y(y:f(x) \land P(y))$ or by $\forall y (y:f(x) \Rightarrow P(y))$ depending on the position (positive or negative)
of the formula in the theorem to be proved and in the definitions and lemmas.

\textbf {Rules automatically built by metarules from  definitions} :\\
\begin{tabular}{llll}
\hspace{0.1cm} 
& If $A \subset B$ and $x \in A$ then $x \in B$  & \hspace{0.2cm} &
If $x \in \sigma E$, then $\exists y (y \in E \land x \in y)$\\
& If $C:A \cap B$ and $x \in C$, then $x \in A$  &                &
If $C : A \cap B$, $x \in A$ and $x \in B$, then $x \in C$ \\
\end{tabular}\\
\hspace*{0.15cm} in place of (and more general than) given 
REDUCE conversion rules of \iref{Bledsoe}. \\
\textbf {and from universal hypotheses} :\\
\hspace*{0.1cm}Universal hypotheses are removed and replaced by local rules (for a sub-theorem).\\ 
\hspace*{0.1cm}This replaces and generalize PEEK forward-chaining of \iref{Bledsoe}. 
\end{calculus}

\begin{clarifications}
``If $C:A \cap B$'' expresses that $C$ is 
$A \cap B$ which has already been introduced.
Flattening is used to recursively create and name objects such as $f(x)$,
and in a certain manner to ``eliminate'' functional symbols since
the expression ${y:f(x)}$ will be handled as if 
it was a predicate expression $F(x)$. \\
Rules are conditional actions. Actions may be defined by packs of rules. 
Metarules build rules from definitions, lemmas and universal hypotheses. 
\end{clarifications}

\begin{history}
{\sc \Muscadet} \cite{pastre:1989,pastre:1993} is a knowledge-based system.
Facts are hypotheses and the conclusion 
of a theorem or a sub-theorem to be proved,
and all sorts of facts which give relevant information
during the proof searching process.
Universal hypotheses are handled as local definitions (no skolemization). 
\Muscadet worked in set theory, mappings and relations, 
topology and topological linear spaces, elementary geometry, discrete geometry, 
cellular automata, and TPTP problems. 
It attended CASC competitions.
It is open software, freely available.\\
\Muscadet is efficient for everyday mathematical problems 
which are expressed in a natural manner,
and problems which involve many axioms, definitions or lemmas, 
but not  for problems with only one large conjecture and few definitions.
\end{history}

\begin{technicalities}
The system is sound but not complete (because of the use of many 
selective rules and heuristics).
It displays proofs easily readable by a human reader.
\end{technicalities}
\nocite{pastre:2011}
% Leave an empty line above "\end{entry}".

\end{entry}

