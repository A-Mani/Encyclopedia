
\calculusName{Conditional Nested Sequent Calculi $\mathcal{N}S$}   % The name of the calculus
\calculusAcronym{$\mathcal{N}S$}    % The acronym if defined above, or empty otherwise. 
\calculusLogic{Propositional Conditional Logics}  % Specify the logic (e.g. classical, intuitionistic, ...) for which this calculus is intended.
\calculusType{Sequent Calculus}   % Specify the calculus type (e.g. Frege-Hilbert style, tableau, sequent calculus, hypersequent calculus, natural deduction, ...)
\calculusYear{2012-2014}   % The year when the calculus was invented.
\calculusAuthor{R\'egis Alenda, Nicola Olivetti, Gian Luca Pozzato} % The name(s) of the author(s) of the calculus.


\entryTitle{Conditional Nested Sequents $\mathcal{N}S$}     % Title of the entry (usually coincides with the name of the calculus).
\entryAuthor{Nicola Olivetti \and Gian Luca Pozzato}    % Your name(s). Separate multiple names with "\and".


% If you wish, use tags to give any other information 
% that might be helpful for classifying and grouping this entry:
% e.g. \tag{Two-Sided Sequents}
% e.g. \tag{Multiset Cedents}
% e.g. \tag{List Cedents}
% You are free to invent your own tags. 
% The Encyclopedia's coordinator will take care of 
% merging semantically similar tags in the future.


\maketitle


% If your files are called "MyProofSystem.tex" and "MyProofSystem.bib", 
% then you should write "\begin{entry}{MyProofSystem}" in the line below
\begin{entry}{NestedConditionals}  

% Define here any newcommands you may need:
% e.g. \newcommand{\necessarily}{\Box}
% e.g. \newcommand{\possibly}{\Diamond}


\begin{calculus}
\begin{footnotesize}
\[
\begin{array}{lclr}
\Gamma(P, \lnot P) \quad (\mathit{AX})
& 
\Gamma(\top) \quad (\mathit{AX}_\top)
& 
\quad \Gamma(\lnot \bot) \quad (\mathit{AX}_\bot) & \\
\mbox{\tiny $P$ atomic} & & &  \\ \\ 
\begin{prooftree}
\Gamma(A) \justifies \Gamma(\lnot \lnot A) \using (\lnot)
\end{prooftree}
& 
\quad
\begin{prooftree}
   \Gamma(\lnot(A \Rightarrow B),[A': \Delta, \lnot B]) \quad A, \lnot A' \quad A', \lnot A
   \justifies \Gamma(\lnot(A \Rightarrow B),[A': \Delta]) \using (\Rightarrow^-)
\end{prooftree} 
 &
\quad \begin{prooftree}
  \Gamma([A: B]) \justifies \Gamma(A \Rightarrow B) \using (\Rightarrow^+)
\end{prooftree}
&    \\ \\ 
\begin{prooftree}
  \Gamma([A: \Delta, \lnot A]) \justifies \Gamma([A: \Delta]) \using (\mathit{ID})
\end{prooftree}
& 
\quad
\begin{prooftree}
   \Gamma([A: \Delta, \Sigma],[B: \Sigma]) \qquad A, \lnot B \qquad B, \lnot A
   \justifies \Gamma([A: \Delta],[B: \Sigma]) \using (\mathit{CEM})
\end{prooftree}
 &
& \qquad  \\ \\ 
\end{array}
\]
\[
\begin{array}{l}
\begin{prooftree}
   \Gamma(\lnot(A \Rightarrow B),A) \qquad \Gamma(\lnot(A \Rightarrow B),\lnot B)
   \justifies \Gamma(\lnot(A \Rightarrow B)) \using (\mathit{MP})
\end{prooftree}
\\ \\ 
\begin{prooftree}
\Gamma,\lnot(C \Rightarrow D),[A: \Delta, \lnot D] \quad \Gamma, \lnot(C \Rightarrow D), [A: C]
\quad \Gamma, \lnot(C \Rightarrow D), [C: A] \justifies  \Gamma, \lnot(C \Rightarrow D), [A: \Delta]
\using (\mathit{CSO})
\end{prooftree}
\end{array}
\]
\end{footnotesize}
\end{calculus}

% The following sections ("clarifications", "history", 
% "technicalities") are optional. If you use them, 
% be very concise and objective. Nevertheless, do write full sentences. 
% Try to have at most one paragraph per section, because line breaks 
% do not look nice in a short entry.

 \begin{clarifications}
% ToDo: write here short remarks that may help the reader to understand 
% the inference rules of the proof system.
Conditional logics extend classical logic with formulas of the form $A \Rightarrow B$: intuitively, $A \Rightarrow B$ is true in a world $x$ if $B$ is true in the set of worlds where $A$ is true and that are most similar to $x$. The calculi $\mathcal{N}S$ manipulate \emph{nested} sequents, a generalization of ordinary sequent calculi where sequents are allowed to occur within sequents. A nested sequent $$A_1, \dots, A_m, [B_1: \Gamma_1], \dots, [B_n: \Gamma_n]$$ is inductively defined by the formula $$\mathcal{F}(\Gamma)=A_1 \vee \dots \vee A_m \vee (B_1 \Rightarrow \mathcal{F}(\Gamma_1)) \vee \dots \vee (B_n \Rightarrow \mathcal{F}(\Gamma_n)).$$ $\Gamma(\Delta)$ represents a sequent $\Gamma$ containing a \emph{context} (a unique empty position) filled by the (nested) sequent $\Delta$.
  Besides the rules shown above, the calculi $\mathcal{N}S$ also include standard 
   rules for propositional connectives.
 \end{clarifications}

 \begin{history}
  The calculi $\mathcal{N}S$ have been introduced in 
  \cite{jelia2012pozz} and extended  in \cite{jlcpozz}. The theorem prover NESCOND, implementing $\mathcal{N}S$ in Prolog, has been presented in \cite{ijcarpozz}.
 \end{history}

 \begin{technicalities}
Completeness is a consequence of the admissibility of cut. The calculi $\mathcal{N}S$ can be used to obtain a \textsc{PSpace} decision procedure for the respective conditional logics (optimal for CK and extensions with ID and MP).
 \end{technicalities}


% General Instructions:
% =====================

% The preferred length of an entry is 1 page. 
% Do the best you can to fit your proof system in one page.
%
% If you are finding it hard to fit what you want in one page, remember:
%
%   * Your entry needs to be neither self-contained nor fully understandable
%     (the interested reader may consult the cited full paper for details)
%
%   * If you are describing several proof systems in one entry, 
%     consider splitting your entry.
%
%   * You may reduce the size of your entry by ommitting inference rules
%     that are already described in other entries.
%
%   * Cite parsimoniously (see detailed citation instructions below).
%
% 
% If you do not manage to fit everything in one page, 
% it is acceptable for an entry to have 2 pages.
%
% For aesthetical reasons, it is preferable for an entry to have
% 1 full page or 2 full pages, in order to avoid unused blank space.



% Citation Instructions:
% ======================

% Please cite the original paper where the proof system was defined.
% To do so, you may use the \cite command within 
% one of the optional environments above,
% or use the \nocite command otherwise.

% You may also cite a modern paper or book where the 
% proof system is explained in greater depth or clarity.
% Cite parsimoniously.

% Do not cite related work. Instead, use the "\iref" or "\irefmissing" 
% commands to make an internal reference to another entry, 
% as explained within the "history" environment above.

% You do not need to create the "References" section yourself. 
% This is done automatically.




% Leave an empty line above "\end{entry}".

\end{entry}
