\calculusName{Two-sided Linear Sequent Calculus}   % The name of the calculus
\calculusAcronym{\TLL}    % The acronym if defined above, or empty otherwise. 
\calculusLogic{classical, linear}  % Specify the logic (e.g. classical, intuitionistic, ...) for which this calculus is intended.
\calculusType{Sequent Calculus}   % Specify the calculus type (e.g. Frege-Hilbert style, tableau, sequent calculus, hypersequent calculus, natural deduction, ...)
\calculusYear{1987}   % The year when the calculus was invented.
\calculusAuthor{Jean-Yves Girard} % The name(s) of the author(s) of the calculus.

\entryTitle{Two-sided Linear Sequent Calculus}        % Title of the entry (usually coincides with the name of the calculus).
\entryAuthor{Elaine Pimentel \and Harley Eades III} % Your name(s). Separate multiple names with "\and".

% If you wish, use tags to give any other information 
% that might be helpful for classifying and grouping this entry:
% e.g. \tag{Two-Sided Sequents}
% e.g. \tag{Multiset Cedents}
% e.g. \tag{List Cedents}
% You are free to invent your own tags. 
% The Encyclopedia's coordinator will take care of 
% merging semantically similar tags in the future.
\tag{Two-Sided Sequents}

\maketitle

% If your files are called "MyProofSystem.tex" and "MyProofSystem.bib", 
% then you should write "\begin{entry}{MyProofSystem}" in the line below
\begin{entry}{TLL} 
\newcommand{\one}{\mathbf{1}}
\newcommand{\zero}{\mathbf{0}}
\newcommand\bang{\mathop{!}}
\newcommand\quest{\mathord{?}}
\newcommand\limp{\mathbin{-\hspace{-0.70mm}\circ}}
\newcommand\tensor\otimes
%% \newcommand\with{\mathbin{\&}}


% Define here any newcommands you may need:
% e.g. \newcommand{\necessarily}{\Box}
% e.g. \newcommand{\possibly}{\Diamond}
\newcommand{\llimp}{\multimap}

\begin{calculus}
\small
\[
\begin{array}{cccccccc}
  \infer[Init]{B \vdash B}{}
  &
  \quad
  &
  \infer[Cut]{\Gamma, \Gamma' \vdash \Delta,\Delta'}{\Gamma \vdash B \mid \Delta & \Gamma', B \vdash \Delta'}
  &
  \quad
  &
  \infer[\one_L]{\Gamma,\one \vdash\Delta}{\Gamma \vdash \Delta}\\
  \\
  \infer[\one_R]{ \vdash \one}{}
  &&
  \infer[\bot_L]{\bot\vdash}{}
  &&
  \infer[\bot_R]{\Gamma\vdash\bot, \Delta}{\Gamma \vdash  \Delta}\\
  \\
  \infer[\llimp_L]{\Gamma, A \llimp B, \Delta \vdash C}{\Gamma \vdash A \vdash B, \Delta \vdash C}
  &&
  \infer[\llimp_R]{\Gamma \vdash A \llimp B}{\Gamma, A \vdash B}
  &&
  \infer[\otimes_L]{\Gamma,B\otimes C\vdash \Delta}{\Gamma,B,C\vdash \Delta}\\
  \\
  \infer[\otimes_R]{\Gamma_1,\Gamma_2 \vdash B\otimes C, \Delta_1,\Delta_2}{\Gamma_1 \vdash B,\Delta_1\quad\Gamma_2 \vdash C,\Delta_2}
  &&
  \infer[\bindnasrepma_L]{\Gamma_1,\Gamma_2,B\bindnasrepma C \vdash \Delta_1,\Delta_2}{\Gamma_1,B\vdash \Delta_1 \quad \Gamma_2,C\vdash \Delta_2}
  &&
  \infer[\bindnasrepma_R]{\Gamma \vdash B \bindnasrepma C, \Delta}{\Gamma \vdash B,C,\Delta}\\
  \\
  \infer[\zero_L]{\Gamma,\zero \vdash\Delta}{}
  &&
  \infer[\top_R]{\Gamma \vdash\top, \Delta}{}
  &&
  \infer[\with_L\;(i=1,2)]{\Gamma,B_1\with B_2\vdash \Delta}{\Gamma,B_i\vdash \Delta}\\
  \\
  \infer[\with_R]{\Gamma \vdash B\with C, \Delta}{\Gamma \vdash B,\Delta\quad\Gamma \vdash C,\Delta}
  &&
  \infer[\oplus_L]{\Gamma,B\oplus C \vdash \Delta}{\Gamma,B\vdash \Delta \quad \Gamma,C\vdash \Delta}
  &&
  \infer[\oplus_R\;(i=1,2)]{\Gamma \vdash B_1\oplus B_2, \Delta}{\Gamma \vdash B_i,\Delta}\\
  \\
  \infer[\forall_L]{\Gamma,\forall x.B\vdash \Delta}{\Gamma,B[t/x]\vdash \Delta}\quad
  &&
  \infer[\forall_R]{\Gamma \vdash \forall x.B, \Delta}{\Gamma \vdash B[y/x],\Delta}
  &&
  \infer[\exists_L]{\Gamma,\exists x.B\vdash \Delta}{\Gamma,B[y/x]\vdash \Delta}\quad\\
  \\
  \infer[\exists_R]{\Gamma \vdash \exists x.B, \Delta}{\Gamma \vdash B[t/x],\Delta}
  &&
  \infer[\quest_L]{\bang\Gamma,\quest B\vdash \quest\Delta}{\bang\Gamma,B\vdash \quest\Delta}\quad
  &&
  \infer[\bang_R]{\bang\Gamma\vdash \bang B,\quest\Delta}{\bang\Gamma\vdash B, \quest\Delta}\\
  \\
  \infer[\quest_W]{\Gamma\vdash \quest B,\Delta}{\Gamma\vdash \Delta}\quad
  &&
  \infer[\quest_C]{\Gamma\vdash \quest B,\Delta}{\Gamma\vdash \quest B,\quest B, \Delta}\quad
  &&
  \infer[\quest_D]{\Gamma\vdash \quest B,\Delta}{\Gamma\vdash B, \Delta}\\
  \\
  \infer[\bang_W]{\Gamma,\bang B\vdash \Delta}{\Gamma\vdash \Delta}\quad
  &&
  \infer[\bang_C]{\Gamma,\bang B\vdash \Delta}{\Gamma,\bang B,\bang B\vdash \Delta}\quad
  &&
  \infer[\bang_D]{\Gamma,\bang B\vdash \Delta}{\Gamma, B\vdash \Delta}
\end{array}
\]
\end{calculus}

% The following sections ("clarifications", "history", 
% "technicalities") are optional. If you use them, 
% be very concise and objective. Nevertheless, do write full sentences. 
% Try to have at most one paragraph per section, because line breaks 
% do not look nice in a short entry.

\begin{clarifications}
This is an alternate formalization of the sequent style formalization
of Linear Logic \iref{LL}.
\end{clarifications}

\begin{history}
This formalization first appeared in \cite{Troelstra:1992}.
\end{history}

%% \begin{technicalities}
%%   $\FILL$ enjoys cut elimination.  It also has a categorical model in
%%   dialectica categories \cite{dePaiva:1990}.
%% \end{technicalities}

% General Instructions:
% =====================

% The preferred length of an entry is 1 page. 
% Do the best you can to fit your proof system in one page.
%
% If you are finding it hard to fit what you want in one page, remember:
%
%   * Your entry needs to be neither self-contained nor fully understandable
%     (the interested reader may consult the cited full paper for details)
%
%   * If you are describing several proof systems in one entry, 
%     consider splitting your entry.
%
%   * You may reduce the size of your entry by ommitting inference rules
%     that are already described in other entries.
%
%   * Cite parsimoniously (see detailed citation instructions below).
%
% 
% If you do not manage to fit everything in one page, 
% it is acceptable for an entry to have 2 pages.
%
% For aesthetical reasons, it is preferable for an entry to have
% 1 full page or 2 full pages, in order to avoid unused blank space.



% Citation Instructions:
% ======================

% Please cite the original paper where the proof system was defined.
% To do so, you may use the \cite command within 
% one of the optional environments above,
% or use the \nocite command otherwise.

% You may also cite a modern paper or book where the 
% proof system is explained in greater depth or clarity.
% Cite parsimoniously.

% Do not cite related work. Instead, use the "\iref" or "\irefmissing" 
% commands to make an internal reference to another entry, 
% as explained within the "history" environment above.

% You do not need to create the "References" section yourself. 
% This is done automatically.




% Leave an empty line above "\end{entry}".

\end{entry}
