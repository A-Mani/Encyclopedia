
\calculusName{Synthetic Tableaux}   % The name of the calculus
\calculusAcronym{}    % The acronym if defined above, or empty otherwise. 
\calculusLogic{Classical Propositional Logic}  % Specify the logic (e.g. classical, intuitionistic, ...) for which this calculus is intended.
\calculusType{tableau}   % Specify the calculus type (e.g. Frege-Hilbert style, tableau, sequent calculus, hypersequent calculus, natural deduction, ...)
\calculusYear{2000}   % The year when the calculus was invented.
\calculusAuthor{Mariusz Urba\'nski} % The name(s) of the author(s) of the calculus.


\entryTitle{Synthetic Tableaux}     % Title of the entry (usually coincides with the name of the calculus).
\entryAuthor{Dorota Leszczy\'nska-Jasion}    % Your name(s). Separate multiple names with "\and".

\maketitle


% If your files are called "MyProofSystem.tex" and "MyProofSystem.bib", 
% then you should write "\begin{entry}{MyProofSystem}" in the line below
\begin{entry}{SyntheticTableaux}

\begin{calculus}

The synthesizing rules are:

\begin{center}
\begin{tabular}{ccccccc}
&&&&$A~~~~$&&  \\

$$
\infer[\mathbf{r^1_{\rightarrow}}]{A \rightarrow B}{\lnot A}
$$
&~~~~&
$$
\infer[\mathbf{r^2_{\rightarrow}}]{A \rightarrow B}{B}
$$
&~~~~&
$$
\infer[\mathbf{r^3_{\rightarrow}}]{\lnot (A \rightarrow B)}{\lnot B}
$$

&&\\
&&&&&&\\

&&&&$\lnot A~~~~$&& \\

$$
\infer[\mathbf{r^1_{\lor}}]{A \lor B}{A}
$$
&&
$$
\infer[\mathbf{r^2_{\lor}}]{A \lor B}{B}
$$
&&
$$
\infer[\mathbf{r^3_{\lor}}]{\lnot (A \lor B)}{\lnot B}
$$

&&\\
&&&&&&\\

&&&&$A~~~~$&& \\

$$
\infer[\mathbf{r^1_{\land}}]{\lnot (A \land B)}{\lnot A}
$$
&&
$$
\infer[\mathbf{r^2_{\land}}]{\lnot (A \land B)}{\lnot B}
$$
&&
$$
\infer[\mathbf{r^3_{\land}}]{A \land B}{B}
$$

&&

$$
\infer[\mathbf{r_{\neg}}]{\lnot \lnot A}{A}
$$
\\

\end{tabular}
\end{center}

\smallskip

The premises of rules $\mathbf{r^3_{\rightarrow}}$, $\mathbf{r^3_{\lor}}$, $\mathbf{r^3_{\land}}$ may occur in any order.

\smallskip

The branching rule:

\begin{center}

\Tree[.{} {$p_i$} {$\lnot p_i$} ]

\end{center}

\end{calculus}

\begin{clarifications}
A Synthetic Tableau for a formula $A$ is a finite tree with the following properties: the tree is generated by the above rules (the root is empty), each formula labelling a node of the tree is a subformula of $A$ or the negation of a subformula of $A$, each leaf is labelled with $A$ or $\lnot A$. The tableau is a proof of $A$ if each leaf is labelled with $A$.\end{clarifications}

\begin{history}
The method has been first presented in \cite{Urbanski2001a}, \cite{Urbanski2001b}, \cite{Urbanski2002a}. In \cite{Urbanski2002a}, \cite{Urbanski2002b} and \cite{Urbanski2004} it is also presented for some extensional many-valued logics and for some paraconsistent logics.
\end{history}

\begin{technicalities}
The method is sound and complete with respect to Classical Propositional Logic and constitutes a decision procedure for CPL. The same holds with respect to the non-classical logics for which the method has been described, see \cite{Urbanski2002a}, \cite{Urbanski2002b}, \cite{Urbanski2004}.
\end{technicalities}


% General Instructions:
% =====================

% The preferred length of an entry is 1 page. 
% Do the best you can to fit your proof system in one page.
%
% If you are finding it hard to fit what you want in one page, remember:
%
%   * Your entry needs to be neither self-contained nor fully understandable
%     (the interested reader may consult the cited full paper for details)
%
%   * If you are describing several proof systems in one entry, 
%     consider splitting your entry.
%
%   * You may reduce the size of your entry by ommitting inference rules
%     that are already described in other entries.
%
%   * Cite parsimoniously (see detailed citation instructions below).
%
% 
% If you do not manage to fit everything in one page, 
% it is acceptable for an entry to have 2 pages.
%
% For aesthetical reasons, it is preferable for an entry to have
% 1 full page or 2 full pages, in order to avoid unused blank space.



% Citation Instructions:
% ======================

% Please cite the original paper where the proof system was defined.
% To do so, you may use the \cite command within 
% one of the optional environments above,
% or use the \nocite command otherwise.

% You may also cite a modern paper or book where the 
% proof system is explained in greater depth or clarity.
% Cite parsimoniously.

% Do not cite related work. Instead, use the "\iref" or "\irefmissing" 
% commands to make an internal reference to another entry, 
% as explained within the "history" environment above.

% You do not need to create the "References" section yourself. 
% This is done automatically.




% Leave an empty line above "\end{entry}".

\end{entry}
