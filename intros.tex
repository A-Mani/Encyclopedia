
\begin{partbacktext}
\part{\emph{Introductions}}

\noindent \begin{large}\textbf{ToDo:} \end{large}

\noindent We plan to add a few short introductory chapters addressing various aspects that are orthogonal to several entries, such as: 
\begin{itemize}
\item basic technical notions for each type of proof system (e.g. tableaux, natural deduction systems, sequent calculi, resolution calculi \ldots),
\item logical languages
\item logics (classical, intuitionistic, modal, substructural, linear, paraconsistent, \ldots)
\item application domains
\end{itemize}

\noindent
The goals of these introductory chapters will be to:
\begin{itemize} 
\item provide a global technical and historical view of the entries,
\item reduce repetition of basic notions in the entries,
\item increase the understandability of the entries,
\item make the encyclopedia more self-contained.
\end{itemize}

\noindent
The exact structure and content of these chapters will be decided later, after the collection of sufficiently many entries.

For now, entries are sorted in chronological order only, and various indexes are provided in the backmatter. If the need arises, we might consider grouping entries according to various criteria.


\end{partbacktext}