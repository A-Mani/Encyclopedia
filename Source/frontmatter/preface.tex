
\preface

The \textbf{Encyclopedia of Proof Systems} aims at providing a reliable, technically accurate, historically informative, concise and convenient central repository of proof systems for various logics. The goal is to facilitate the exchange of information among logicians, in order to foster and accelerate the development of proof theory and automated deduction.

Preparatory work for the creation of the Encyclopedia, such as the implementation of the LaTeX template and the setup of the Github repository, started in October 2014, triggered by the call for workshop proposals for the 25th Conference on Automated Deduction (CADE). Christoph Benzm\"uller, CADE's conference chair, and Jasmin Blanchette, CADE's workshop co-chair, encouraged me to submit a workshop proposal and supported my alternative idea to organize instead a special poster session based on encyclopedia entries. I am thankful for their encouragement and support.

In December 2014, Bj\"orn Lellmann, Giselle Reis and Martin Riener kindly accepted my request to beta-test the template and the instructions I had created. They submitted the first few example entries to the encyclopedia and provided valuable feedback, for which I am grateful. Their comments were essential for improving the templates and instructions before the public announcement of the encyclopedia.

Discussions with Lev Beklemishev, Bj\"orn Lellmann, Tomer Libal, Roman Kuznets, Sergei Soloviev and Anna Zamansky brainstormed many ideas for improving the organization and structure of the encyclopedia. Many of these ideas still need to be fully implemented.

In July 2015, Julian R\"oder's assistance was essential for the successful organization of the poster session at CADE. Cezary Kaliszyk and Andrei Paskevitch kindly allowed me to organize a discussion session as part of the Proof Exchange for Theorem Proving (PxTP) workshop, where the participants provided useful feedback and many ideas for improvements.

Based on the feedback received at CADE, the coordination's current focus is the production of introductory chapters on a few selected topics (to be expanded in the future) and the filling of gaps in certain topics. Towards these aims, particularly in topics related to linear logics, Valeria de Paiva has been providing very helpful coordination assistance.

(To be continued...)


\vspace{\baselineskip}
\begin{flushright}\noindent
December 2015\hfill {\it Bruno Woltzenlogel Paleo}
\end{flushright}


