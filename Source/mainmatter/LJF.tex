
\calculusName{Focused LJ}
\calculusAcronym{\LJF}
\calculusLogic{Intuitionistic Logic}
\calculusLogicOrder{First-Order}
\calculusType{Sequent Calculus}
\calculusYear{2007}
\calculusAuthor{\iCA{Chuck Liang and Dale Miller}}

\entryTitle{Focused LJ}
\entryAuthor{\iA{Dale Miller}}

\tag{Two-Sided Sequents}
\tag{Focused Proof System}

\maketitle

\begin{entry}{LJF}  

\newcommand{\truen}{t^-}
\newcommand{\truep}{t^+}
\newcommand{\falsen}{f^-}
\newcommand{\falsep}{f^+}
\newcommand{\wedgep}{\wedge^{\!+}}
\newcommand{\wedgen}{\wedge^{\!-}}
\newcommand{\veep}{\vee^{\!+}}
\newcommand{\veen}{\vee^{\!-}}

\newcommand{\Rscr}{{\cal R}}                                   % Used for an ambiguous rhs
%\newcommand{\Rscr}{\Delta_1\Downarrow\Delta_2}                % Used for an ambiguous rhs
\newcommand{\jUnf    }[4]{#1\mathbin\Uparrow#2\vdash#3\mathbin\Uparrow #4} % unfocused sequent
\newcommand{\jUnfG   }[2]{\jUnf{\Gamma}{#1}{#2}{{}}}           % unf sequ with \Gamma
\newcommand{\jUnfamb }[3]{#1\mathbin\Uparrow#2\vdash#3 \Rscr}  % unfocused sequent
\newcommand{\jUnfGamb}[1]{\Gamma\mathbin\Uparrow#1\vdash \Rscr}% unf sequ with \Gamma
\newcommand{\jLf     }[3]{#1\Downarrow#2\vdash#3}              % left focused sequent
\newcommand{\jLfG    }[1]{\jLf{\Gamma}{#1}{E}}                 % left foc seq with \Gamma 
\newcommand{\jRf     }[2]{#1\vdash #2\Downarrow}               % right focused sequent
\newcommand{\jRfG    }[1]{\jRf{\Gamma}{#1}}                    % right foc seq with \Gamma

\begin{calculus}
{\sc Asynchronous Introduction Rules}
\[
  \infer{\jUnfG{\cdot}{B_1 \supset B_2}}{\jUnfG{B_1}{B_2}} %[\supset_r]
  \qquad
  \infer{\jUnfG{\cdot}{B_1 \wedgen B_2}} %[\wedgen_r]
                   {\jUnfG{\cdot}{B_1} \quad \jUnfG{\cdot}{B_2}}
  \qquad
  \infer{\jUnfG{\cdot}{\truen}}{} %[\truen_r]
\]
\[
  \infer{\jUnfG{\cdot}{\forall x.B}}{\jUnfG{\cdot}{[y/x]B}} %[\forall_r]
  \qquad
  \infer{\jUnfGamb{\exists x.B, \Theta}}{\jUnfGamb{[y/x]B,\Theta}} %[\exists_l]
  \qquad
  \infer{\jUnfGamb{\falsep, \Theta}}{} %[\falsep_l]
\]
\[
  \infer{\jUnfGamb{B_1\wedgep B_2,\Theta}}{\jUnfGamb{B_1,B_2,\Theta}} %[\wedgep_l]
  \qquad
  \infer{\jUnfGamb{\truep, \Theta}}{\jUnfGamb{\Theta}} %[\truep_l]
  \qquad
  \infer{\jUnfGamb{B_1\veep B_2,\Theta}} %[\veep_l]
                 {\jUnfGamb{B_1,\Theta}\quad \jUnfGamb{B_2,\Theta}}
\]
{\sc Synchronous Introduction Rules}
\[ 
  \infer{\jLfG{B_1 \supset B_2}}{\jRfG{B_1}\quad \jLfG{B_2}} %[\supset_l]
  \qquad
  \infer{\jLfG{\forall x.B}}{\jLfG{[t/x]B}} %[\forall_l]
  \qquad
  \infer[i \in \{1,2\}]{\jLfG {B_1 \wedge^- B_2}}{\jLfG{B_i}} %[\wedgen_l]
\]
\[
  \infer{\jRfG{B_1 \veep B_2}}{\jRfG{B_i}} %[\veep_r]
  \qquad
  \infer{\jRfG{\truep}}{} %[\truep_r]
  \qquad
  \infer{\jRfG{B_1 \wedge^+ B_2}}{\jRfG{B_1}\quad\jRfG{B_2}} %[\wedgep_r]
  \qquad
  \infer{\jRfG{\exists x.B}}{\jRfG{[t/x]B}} %[\exists_r]
\]
{\sc Identity rules}
\[
  \infer[I_l]{\jLf{\Gamma}{N}{N}}{N \; \mbox{atomic}}
  \qquad
  \infer[I_r]{\jRf{\Gamma,P}{P}}{P \; \mbox{atomic}}
  \qquad
  \infer[Cut]{\jUnf{\Gamma}{\cdot}{\cdot}{E}}
             {\jUnf{\Gamma}{\cdot}{B}{\cdot}\qquad \jUnf{\Gamma}{B}{\cdot}{E}}
\]
{\sc Structural rules}
\[
  \infer[\kern -2pt D_l]{\jUnf{\Gamma,N}{\cdot}{\cdot}{E}}{\jLf{\Gamma,N}{N}{E}}
  \qquad
  \infer[\kern -2 pt D_r]{\jUnf{\Gamma}{\cdot}{\cdot}{P}}{\jRf{\Gamma}{P}}
  \qquad
  \infer[R_l]{\jLf{\Gamma}{P}{E}}{\jUnf{\Gamma}{P}{\cdot}{E}}
  \qquad
  \infer[R_r]{\jRf{\Gamma}{N}}{ \jUnf{\Gamma}{\cdot}{N}{\cdot}}
\]
\[
  \infer[S_l]{\jUnfamb{\Gamma}{C,\Theta}{\null}}{\jUnfamb{C,\Gamma}{\Theta}{\null}}
  \qquad
  \infer[S_r]{\jUnf{\Gamma}{\cdot}{E}{\cdot}}{\jUnf{\Gamma}{\cdot}{\cdot}{E}}
\]
Here,
$\Theta$ ranges over multisets of polarized formulas;
$\Gamma$ ranges over lists of polarized formulas;
$P$ denotes a positive formula;
$N$ denotes a negative formula;
$C$ denotes either a negative formula or a positive atom; and
$E$ denotes either a positive formula or a negative atom; and
$B$ denotes an unrestricted polarized formula.
The introduction rule for $\forall$ has the usual eigenvariable
restriction that $y$ is not free in any formula in the conclusion
sequent. 
\end{calculus}

\begin{clarifications}
This proof system involves \emph{polarized} formulas of first-order
intuitionistic logic: in order to polarize a formula $B$, one must
assign the status of ``positive'' or ``negative'' bias to all atomic
formulas and replace all occurrences of truth with either $\truep$ or
$\truen$ and all occurrences of conjunction with either $\wedgep$ or
$\wedgen$.  If there are $n$ occurrences of truth and conjunction in
$B$, there are $2^n$ ways to do this replacement.  Similarly, we
replace the false and disjunction with $\falsep$ and $\veep$ since
only the positive polarization for these connectives are available in
\LJF.  (Assigning polarization in classical logic is different: see
the \LKF proof system \iref{LKF}.) 
The \emph{positive connectives} are $\falsep$, $\veep$, $\truep$,
$\wedgep$, and $\exists$ while the \emph{negative connectives} are
$\truen$, $\wedgen$, $\supset$, and $\forall$.
A formula is \emph{positive} if it is a positive atom or has a top-level
positive connective; similarly a formula is \emph{negative} if it is a
negative atom or has a top-level negative connective.

There are two kinds of sequents in this proof system. 
%
One kind contains a single $\Downarrow$ on either
the right or the left of the turnstyle ($\vdash$) and are of
the form $\jLf{\Gamma}{B}{E}$ or $\jRf{\Gamma}{B}$: in both of these
cases, the formula $B$ 
%(the formula between the $\Downarrow$ and the turnstyle) 
is the \emph{focus} of the sequent.
%
The other kind of sequent has an occurrence of $\Uparrow$ on each side
of the turnstyle, eg., $\jUnf{\Gamma}{\Theta}{\Delta_1}{\Delta_2}$, 
and is such that the union of the two multisets $\Delta_1$ and
$\Delta_2$ contains exactly one formula: that is, one of these
multisets is empty and the other is a singleton. 
%
When writing asynchronous rules that introduce a connective on the
left-hand side, we write $\Rscr$ to denote
$\Delta_1\Downarrow\Delta_2$.

Note that in the asynchronous phase, a right introduction rule is
applied only when the left asynchronous zone $\Gamma$ is empty.
Similarly, a left-introduction rule in the async phase introduces the
connective at the top-level of the first formula in that context.  The
scheduling of introduction rules during this phase can be assigned
arbitrarily and the zone $\Gamma$ can be interpreted as a multiset
instead of a list.
% In contrast to \LKF, the focused version of LK \iref{LKF}, the two
% sided presentation of \LJF forces there to be two copies each of the 
% decide, release, store, and initial rules.

The choice of how to polarize an
unpolarized formula does not affect provability in LJF but can make a
big impact on the structure of LJF proofs that can be built.  
\end{clarifications}

\begin{history}
This focused proof system is a slight variation of the proof system in
\cite{liang09tcs,liang07csl}.  
\LJF can be seen as a generalization to the MJ sequent system of Howe
\cite{howe98phd}. 
Other focused proof systems, such as LJT \cite{herbelin95phd},
LJQ/LJQ' \cite{dyckhoff06cie}, and $\lambda$RCC
\cite{jagadeesan05fsttcs} can be directly emulated within \LJF by
making the appropriate choice of polarization.
\end{history}

% \begin{history}
% ToDo: write here short historical remarks about this proof system,
% especially if they relate to other proof systems. 
% Use "\iref{OtherProofSystem}" to refer to another proof system 
% in the Encyclopedia (where "OtherProofSystem" is its ID). 
% Use "\irefmissing{SuggestedIDForOtherProofSystem}" to refer to 
% another proof system that is not yet available in the encyclopedia.
% \end{history}

% \begin{technicalities}
% ToDo: write here remarks about soundness, completeness, decidability...
% \end{technicalities}



% Please cite the original paper where the proof system was defined.
% To do so, you may use the \cite command within 
% one of the optional environments above,
% or use the \nocite command otherwise.

% You may also cite a modern paper or book where the 
% proof system is explained in greater depth or clarity.
% Cite parsimoniously.

% Do not cite related work. Instead, use the "\iref" or "\irefmissing" 
% commands to make an internal reference to another entry, 
% as explained within the "history" environment above.

% You do not need to create the "References" section yourself. 
% This is done automatically.


% Leave an empty line above "\end{entry}".

\end{entry}
