% If the calculus has an acronym, define it.
% (e.g. \newcommand{\LK}{\ensuremath{\mathbf{LK}}\xspace})

\newcommand{\ttrue}{{\mathrm{t}\!\mathrm{t}}}
\newcommand{\ffalse}{{\mathrm{f}\!\mathrm{f}}}
\newcommand{\rl}[1]{\textsf{#1}} %typeface for rules
\newcommand{\unify}[2]{#1 \neq^? #2} %unification problem
\newcommand{\ChoiceFuns}{\textsf{CFs}}


\calculusName{Extensional HO RUE-Resolution}   % The name of the calculus
\calculusAcronym{LEO-II}    % The acronym if defined above, or empty otherwise. 
\calculusLogic{Classical Logic}  % Specify the logic (e.g. classical, intuitionistic, ...) for which this calculus is intended.
\calculusLogicOrder{Higher-Order}
\calculusType{Resolution}   % Specify the calculus type (e.g. Frege-Hilbert style, tableau, sequent calculus, hypersequent calculus, natural deduction, ...)
\calculusYear{1999-2013}   % The year when the calculus was invented.
\calculusAuthor{Christoph Benzm{\"u}ller} % The name(s) of the author(s) of the calculus.


\entryTitle{Extensional HO RUE-Resolution}     % Title of the entry (usually coincides with the name of the calculus).
\entryAuthor{Christoph Benzm{\"u}ller}     


% If you wish, use tags to give any other information 
% that might be helpful for classifying and grouping this entry:
% e.g. \tag{Two-Sided Sequents}
% e.g. \tag{Multiset Cedents}
% e.g. \tag{List Cedents}
% You are free to invent your own tags. 
% The Encyclopedia's coordinator will take care of 
% merging semantically similar tags in the future.


\maketitle


% If your files are called "MyProofSystem.tex" and "MyProofSystem.bib", 
% then you should write "\begin{entry}{MyProofSystem}" in the line below
\begin{entry}{ExtResLEO2}  

% Define here any newcommands you may need:
% e.g. \newcommand{\necessarily}{\Box}
% e.g. \newcommand{\possibly}{\Diamond}


\begin{calculus}

% Add the inference rules of your proof system here.
% The "proof.sty" and "bussproofs.sty" packages are available.
% If you need any other package, please contact the editor (bruno@logic.at)
Normalisation Rules

\quad
\begin{prooftree}
\mathbf{C} \vee [\mathbf{A} \vee \mathbf{B}]^\ttrue
\justifies
\mathbf{C} \vee [\mathbf{A}]^\ttrue \vee [\mathbf{B}]^\ttrue
\thickness=0.08em
\using{\vee^\ttrue}
\label{extcnf-or-pos}
\end{prooftree}
\quad
\begin{prooftree}
\mathbf{C} \vee [\mathbf{A} \vee \mathbf{B}]^\ffalse
\justifies
\begin{array}{c}
\mathbf{C} \vee [\mathbf{A}]^\ffalse \\
\mathbf{C} \vee [\mathbf{B}]^\ffalse
\end{array}
\thickness=0.08em
\using{\vee^\ffalse}
\label{extcnf-or-neg}
\end{prooftree}
\quad
\begin{prooftree}
\mathbf{C} \vee [\neg\mathbf{A}]^\ttrue
\justifies
\mathbf{C} \vee [\mathbf{A}]^\ffalse
\thickness=0.08em
\using{\neg^\ttrue}
\label{extcnf-not-pos}
\end{prooftree}
\quad 
\begin{prooftree}
\mathbf{C} \vee [\neg\mathbf{A}]^\ffalse
\justifies
\mathbf{C} \vee [\mathbf{A}]^\ttrue
\thickness=0.08em
\using{\neg^\ffalse}
\label{extcnf-not-neg}
\end{prooftree}

\vskip.2em

\quad 
\begin{prooftree}
\mathbf{C} \vee [\Pi^\tau \mathbf{A}]^\ttrue
\quad
X^\tau \textrm{ fresh variable}
\justifies
\mathbf{C} \vee [\mathbf{A} \, X]^\ttrue
\thickness=0.08em
\using{\Pi^\ttrue}
\label{extcnf-forall-pos}
\end{prooftree}
\quad
\begin{prooftree}
\mathbf{C} \vee [\Pi^\tau \mathbf{A}]^\ffalse
\quad
\mathtt{sk}^\tau \textrm{ Skolem term}
\justifies
\mathbf{C} \vee [\mathbf{A} \, \mathtt{sk}^\tau]^\ffalse
\thickness=0.08em
\using{\Pi^\ffalse}
\label{extcnf-forall-neg}
\end{prooftree}

\vskip.3em

Resolution, Factorisation and Primitive Substitution

\vskip.2em

\quad 
\begin{prooftree}
[\mathbf{A}]^{p_1} \vee \mathbf{C}
\quad
[\mathbf{B}]^{p_2} \vee \mathbf{D}
\quad
p_1\not=p_2
\justifies
\mathbf{C} \vee \mathbf{D} \vee [\mathbf{A}=\mathbf{B}]^\ffalse
\thickness=0.08em
\using{\mathrm{\rl{res}}}
\end{prooftree}
\quad 
\begin{prooftree}
\mathbf{C} \vee [\mathbf{A}]^p \vee [\mathbf{B}]^p
\justifies
\mathbf{C} \vee [\mathbf{A}]^p \vee [\mathbf{A}=\mathbf{B}]^\ffalse
\thickness=0.08em
\using{\mathrm{\rl{fac}}}
\end{prooftree}

\vskip.2em

\qquad \qquad
\begin{prooftree}
[Q_\tau \overline{\mathbf{A}}^n]^p \vee \mathbf{C}
\quad
\mathbf{P}\in\mathcal{AB}_\tau^{(k)} \textrm{ for logic connective }  k 
\justifies
([Q_\tau \overline{\mathbf{A}}^n]^p \vee \mathbf{C}) [\mathbf{P}/Q]
\thickness=0.08em
\using{\mathrm{\rl{prim\_subst}}}
\end{prooftree}
\vfill

\vskip.3em

Extensionality and Pre-unification

\vskip.2em

\quad
\begin{prooftree}
\mathbf{C} \vee [\mathbf{A}^{\sigma\tau} = \mathbf{B}^{\sigma\tau}]^\ttrue
\quad
X^\tau \textrm{ fresh variable}
\justifies
\mathbf{C} \vee [\mathbf{A} \, X = \mathbf{B} \, X]^\ttrue
\thickness=0.08em
\using{\textsc{FuncPos}}
\label{func-pos}
\end{prooftree}
\quad
\begin{prooftree}
\mathbf{C} \vee [\mathbf{A}^o = \mathbf{B}^o]^\ttrue
\justifies
\mathbf{C} \vee [\mathbf{A}^o \longleftrightarrow \mathbf{B}^o]^\ttrue
\thickness=0.08em
\using{\textsc{BoolPos}}
\end{prooftree}

\vskip.2em

\quad 
\begin{prooftree}
\mathbf{C} \vee [\mathbf{A}^{\sigma\tau} = \mathbf{B}^{\sigma\tau}]^\ffalse
\quad
\mathtt{sk}^\tau \textrm{ Skol. term}
\justifies
\mathbf{C} \vee [\mathbf{A} \, \mathtt{sk} = \mathbf{B} \, \mathtt{sk}]^\ffalse
\thickness=0.08em
\using{\textsc{FuncNeg}}
\end{prooftree}
\quad
\begin{prooftree}
\mathbf{C} \vee [\mathbf{A}^o = \mathbf{B}^o]^\ffalse
\justifies
\mathbf{C} \vee [\mathbf{A}^o \longleftrightarrow \mathbf{B}^o]^\ffalse
\thickness=0.08em
\using{\textsc{BoolNeg}}
\end{prooftree}

\vskip.2em

\quad 
\begin{prooftree}
\mathbf{C} \vee [h^{\sigma\tau}\overline{\mathbf{A}^\alpha}^k \; = \; h^{\sigma\tau}\overline{\mathbf{B}^\alpha}^k]^\ffalse
\justifies
\mathbf{C} \vee \overline{[\mathbf{A}_i \; = \; \mathbf{B}_i]^\ffalse}^{i\leq k}
\thickness=0.08em
\using{\textsc{Dec}}
\end{prooftree}
\quad 
\begin{prooftree}
\mathbf{C} \vee [X=\mathbf{A}]^\ffalse
\quad
X \not\in \mathrm{FV}(\mathbf{A})
\justifies
\mathbf{C}[\mathbf{A}/X]
\thickness=0.08em
\using{\textsc{Subst}}
\end{prooftree}

\vskip.2em

\quad
\begin{prooftree}
\mathbf{C} \vee [\mathbf{A}=\mathbf{A}]^\ffalse
\justifies
\mathbf{C}
\thickness=0.08em
\using{\textsc{Triv}}
\end{prooftree}
\quad 
\begin{prooftree}
\mathbf{C} \vee [F^\tau\overline{\mathbf{A}}^n \; = \; h\overline{\mathbf{B}}^m]^\ffalse
\quad
\mathbf{G} \in \mathcal{AB}^{(h)}_\tau
\justifies
\mathbf{C} \vee [F = \mathbf{G}]^\ffalse \;\vee\; [F\overline{\mathbf{A}}^n = h\overline{\mathbf{B}}^m]^\ffalse
\thickness=0.08em
\using{\textsc{FlexRigid}}
\end{prooftree}

\vskip.3em

Choice 

\vskip.2em

\quad
\begin{prooftree}
\begin{array}{c}
\\
C:= \mathbf{C'} \vee [\mathbf{A}[E_{(\alpha\rightarrow o)\rightarrow \alpha} \mathbf{B}]]^p
\end{array}
 \quad
\begin{array}{c}
\epsilon\in\ChoiceFuns, \; E = \epsilon \textit{ or } E \in \textit{freeVars(C)},
\\ \textit{freeVars}(\mathbf{B})\subseteq \textit{freeVars}(C), Y\,\textit{fresh}
\end{array}
\justifies
  [\mathbf{B} \; Y]^\ffalse \vee [\mathbf{B} \; (\epsilon_{\alpha (\alpha o)}
  \mathbf{B})]^\ttrue
\thickness=0.08em
\using{\rl{choice}}
\label{choice}
\end{prooftree}

\vskip.2em

\qquad \qquad \qquad \qquad 
\begin{prooftree}
[P X]^\ffalse \vee [P (f_{(\alpha\rightarrow o)\rightarrow \alpha} P)]^\ttrue
\justifies
 \ChoiceFuns \longleftarrow \ChoiceFuns\cup\{f_{(\alpha\rightarrow o)\rightarrow \alpha}\}
\thickness=0.08em
\using{\rl{detectChoiceFn}}
\label{detectChoiceFn}
\end{prooftree}


 \vskip.3em

Optional additional rules include (a) exhaustive universal instantion rule for (selective) finite domains, (b) detection and removal of Leibniz equations and Andrews equations, and (c) splitting.  Like \rl{detectChoiceFn} these rules are admissible. 

% \vskip.2em

% \quad
% \begin{prooftree}
% \mathbf{C} \vee [\Pi^\tau \mathbf{A}]^\ttrue
% \quad
% \textrm{canonical } \overline{\mathbf{B}^\tau}^n
% \quad
% n = \mathrm{Card}(\tau)
% \quad
% n \in \omega
% \justifies
% \mathbf{C} \vee \overline{[\mathbf{A} \, \mathbf{B}]^\ttrue}^n
% \thickness=0.08em
% \using{\Pi^\ttrue_\textrm{special}}
% \label{extcnf-forall-special-pos}
% \end{prooftree}

% \begin{prooftree}
% \[
% \mathbf{C} \vee [\mathbf{A}_1 \circ \ldots \circ \mathbf{A}_n]^p
% \using {\rl{split\_conjecture}}
% \justifies
%  \[
%  \[
%  \mathbf{C} \vee [\mathbf{A}_1]^p
%  \justifies
%  \vdots
%  \]
%  \justifies
% [\ffalse]^\ttrue
%  \]
%  \quad
%  \ldots
%  \quad
%  \[
%  \[
%  \mathbf{C} \vee [\mathbf{A}_n]^p
%  \justifies
%  \vdots
%  \]
%  \justifies
% [\ffalse]^\ttrue
%  \]
% \]
% \justifies
% [\ffalse]^\ttrue
% \using {\rl{solved\_all\_splits}}
% \end{prooftree}

\end{calculus}

% The following sections ("clarifications", "history", 
% "technicalities") are optional. If you use them, 
% be very concise and objective. Nevertheless, do write full sentences. 
% Try to have at most one paragraph per section, because line breaks 
% do not look nice in a short entry.

 \begin{clarifications} 
 $\mathbf{A}$ and $\mathbf{B}$ are metavariables ranging over terms of HOL \cite{sep-type-theory-church}; see also \iref{GBetaFB}). The logical connectives are $\neg$, $\vee$, $\Pi^\tau$ (universal quantification over variables of type $\tau$), and $=^\tau$ (equality on terms of type $\tau$). Types are shown only if unclear in context. For example, in  rule choice the variable $E^{\alpha (\alpha o)}$ is of function type, also written as ${(\alpha \rightarrow o) \rightarrow \alpha}$. Variables like $F$ are presented as upper case symbols and constant symbols like $h$  are lower case. $\alpha$ equality and $\beta\eta$-normalisation are treated implicit, meaning that all clauses are implicitly normalised. $\mathbf{C}$ and $\mathbf{D}$ are metavariables ranging over clauses, which are disjunctions of literals. These disjunctions are implicitly assumed associative and commutative; the latter also applies to all equations. Literals are formulas shown in square brackets and labelled with a \emph{polarity} (either $\ttrue$ or $\ffalse$), e.g. $[\neg X]^\ffalse$ denotes the negation of $\neg X$.  $\mathrm{FV}(\mathbf{A})$ denotes the free variables of term $\mathbf{A}$. $\mathcal{AB}^{(h)}_\tau$ is the set of approximating bindings for head $h$ and type $\tau$. $\epsilon_{\alpha (\alpha o)}$ is a choice operator and $\ChoiceFuns$ is a set of dynamically collected choice functions symbols; $\ChoiceFuns$ is initialised with a single choice function.
\end{clarifications}


 \begin{history}
The original calculus (without choice) has been presented in \cite{C5} and \cite{J5}. Recent modifications and extensions (e.g. choice) are discussed in  \cite{W47} and \cite{EasyChair:215}.
The calculus is inspired by and extends Huet's constrained resolution \cite{Huet:amott73,Huet:cracmfhol72} and the extensional resolution calculus in \cite{C2}.
% ToDo: write here short historical remarks about this proof system,
% especially if they relate to other proof systems. 
% Use "\iref{OtherProofSystem}" to refer to another proof system 
% in the Encyclopedia (where "OtherProofSystem" is its ID). 
% Use "\irefmissing{SuggestedIDForOtherProofSystem}" to refer to 
% another proof system that is not yet available in the encyclopedia.
 \end{history}

\begin{technicalities} 
The calculus works for classical higher-order logic with Henkin semantics and choice. Soundness and completeness has been discussed in  \cite{C5} and \cite{J5}. In the prover LEO-II, the factorisation rule is for performance reasons restricted to binary clauses and a (parametrisable) depth limit is employed for pre-unification. Such restrictions are a (deliberate) source for incompleteness.
 \end{technicalities}


% General Instructions:
% =====================

% The preferred length of an entry is 1 page. 
% Do the best you can to fit your proof system in one page.
%
% If you are finding it hard to fit what you want in one page, remember:
%
%   * Your entry needs to be neither self-contained nor fully understandable
%     (the interested reader may consult the cited full paper for details)
%
%   * If you are describing several proof systems in one entry, 
%     consider splitting your entry.
%
%   * You may reduce the size of your entry by ommitting inference rules
%     that are already described in other entries.
%
%   * Cite parsimoniously (see detailed citation instructions below).
%
% 
% If you do not manage to fit everything in one page, 
% it is acceptable for an entry to have 2 pages.
%
% For aesthetical reasons, it is preferable for an entry to have
% 1 full page or 2 full pages, in order to avoid unused blank space.



% Citation Instructions:
% ======================

% Please cite the original paper where the proof system was defined.
% To do so, you may use the \cite command within 
% one of the optional environments above,
% or use the \nocite command otherwise.

% You may also cite a modern paper or book where the 
% proof system is explained in greater depth or clarity.
% Cite parsimoniously.

% Do not cite related work. Instead, use the "\iref" or "\irefmissing" 
% commands to make an internal reference to another entry, 
% as explained within the "history" environment above.

% You do not need to create the "References" section yourself. 
% This is done automatically.




% Leave an empty line above "\end{entry}".

\end{entry}