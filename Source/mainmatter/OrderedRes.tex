


\calculusName{Ordered Resolution}   
\calculusAcronym{}     
\calculusLogic{Classical Logic}  
\calculusLogicOrder{First-Order}
\calculusType{Resolution}   
\calculusYear{1969} 
  
\calculusAuthor{Sergey Maslov}
\calculusAuthor{Robert Kowalski}
\calculusAuthor{Patrick J. Hayes} 


\entryTitle{Ordered Resolution}     
\entryAuthor{Uwe Waldmann}     





\maketitle



\begin{entry}{OrderedRes}  




\begin{calculus}

% Add the inference rules of your proof system here.
% The "proof.sty" and "bussproofs.sty" packages are available.
% If you need any other package, please contact the editor (bruno@logic.at)

\[
\infer[\textit{Resolution}]
{(D \lor C)\sigma}{D \lor B\vphantom{[]}
& C \lor \neg A}
\]
\[
\infer[\textit{Factoring}]
{(C \lor L_1)\sigma}{C \lor L_1 \lor \dots \lor L_n}
\]

\medskip

$C,D$ are (possibly empty) clauses,
$L_1,\dots,L_n$ are literals,
$A,B$ are atoms,
$A$ and $B$, or $L_1,\dots,L_n$, respectively,
are unifiable with most general unifier~$\sigma$.

The literals $\neg A$, $B$, and $L_1$ are maximal in the respective premises.
\end{calculus}



\begin{clarifications}
Ordered resolution is a refutational saturation calculus for
first-order clauses (disjunctions of possibly negated atoms).
It works on a set $N$ of clauses that is saturated
by successively computing inferences
with premises in $N$ and adding the conclusion of the inference to $N$,
until the empty clause (i.\,e., false) is derived.
\end{clarifications}

\begin{history}
The idea to use a syntactic ordering on literals to restrict
the number of possible inferences was developed independently
by Maslov~\cite{Maslov1964,Maslov1968,Maslov1971}\irefmissing{Maslov}
for the \emph{inverse method} (resolution can be seen
as the dual form of a special case of the inverse method)
and by Kowalski and Hayes~\cite{KowalskiHayes1969} for resolution itself
(the requirements for the ordering differ slightly).

\end{history}

\begin{technicalities}
The ordered resolution calculus is refutationally complete for
sets of first-order clauses.
\end{technicalities}













\end{entry}
