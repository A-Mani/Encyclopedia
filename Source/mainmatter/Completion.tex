
\calculusName{(Unfailing) Completion}   % The name of the calculus
\calculusAcronym{}    % The acronym if defined above, or empty otherwise. 
\calculusLogic{Classical Logic}  % Specify the logic (e.g. classical, intuitionistic, ...) for which this calculus is intended.
\calculusLogicOrder{First-Order}
\calculusType{Superposition}   % Specify the calculus type (e.g. Frege-Hilbert style, tableau, sequent calculus, hypersequent calculus, natural deduction, ...)
\calculusYear{1970/1986}   % The year when the calculus was invented.
\calculusAuthor{Donald E. Knuth, Peter B. Bendix, Leo Bachmair, Nachum Dershowitz, Jieh Hsiang} % The name(s) of the author(s) of the calculus.


\entryTitle{(Unfailing) Completion}     % Title of the entry (usually coincides with the name of the calculus).
\entryAuthor{\iA{Uwe Waldmann}}     


% If you wish, use tags to give any other information 
% that might be helpful for classifying and grouping this entry:
% e.g. \tag{Two-Sided Sequents}
% e.g. \tag{Multiset Cedents}
% e.g. \tag{List Cedents}
% You are free to invent your own tags. 
% The Encyclopedia's coordinator will take care of 
% merging semantically similar tags in the future.


\maketitle


% If your files are called "MyProofSystem.tex" and "MyProofSystem.bib", 
% then you should write "\begin{entry}{MyProofSystem}" in the line below
\begin{entry}{Completion}

% Define here any newcommands you may need:
% e.g. \newcommand{\necessarily}{\Box}
% e.g. \newcommand{\possibly}{\Diamond}
\newcommand{\deq}{\buildrel{\mbox{\Large.}}\over\approx}

\begin{calculus}

% Add the inference rules of your proof system here.
% The "proof.sty" and "bussproofs.sty" packages are available.
% If you need any other package, please contact the editor (bruno@logic.at)

Standard Completion:
\[
\begin{array}{l@{\qquad}l}
\begin{array}{ll}
\infer[\textit{Orient}]
  {\strut E, \ \  R \cup \{ s \to t \}}
  {E \cup \{ s \deq t \}, \ \  R} \\
\mbox{if $s \succ t$\rule[-3ex]{0ex}{6ex}}
\end{array}
&
\begin{array}{ll}
\infer[\textit{Simplify-Equation}]
  {\strut E \cup \{ u \deq t \}, \ \  R}
  {E \cup \smash{\{ s \deq t \}}, \ \  R} \\
\mbox{if $s \to_R u$\rule[-3ex]{0ex}{6ex}}
\end{array}
\\
\begin{array}{ll}
\infer[\textit{Deduce}]
  {\strut E \cup \{ s \approx t \}, \ \  R}
  {E, \ \  R} \\
\mbox{if $\langle s,t \rangle \in \mathrm{CP}(R)$\rule[-3ex]{0ex}{6ex}}
\end{array}
&
\begin{array}{ll}
\infer[\textit{Left-Simplify-Rule}]
  {\strut E \cup \{ u \approx t \}, \ \  R}
  {E, \ \  R \cup \{ s \to t \}} \\
\mbox{if $s \to_R u$ using a rule $l \to r \in R$ such that $s \sqsupset l$\rule[-3ex]{0ex}{6ex}}
\end{array}
\\
\begin{array}{ll}
\infer[\textit{Delete}]
  {\strut E, \ \  R}
  {E \cup \{ s \approx s \}, \ \  R} \\
\mbox{\rule[-3ex]{0ex}{4ex}}
\end{array}
&
\begin{array}{ll}
\infer[\textit{Right-Simplify-Rule}]
  {\strut E, \ \  R \cup \{ s \to u \}}
  {E, \ \  R \cup \{ s \to t \}} \\
\mbox{if $t \to_R u$\rule[-2ex]{0ex}{5ex}}
\end{array}
\end{array}
\]
plus, for Unfailing Completion:
\[
\begin{array}{ll}
\infer[\textit{UC-Deduce}]
  {\strut E \cup \{ s \approx t \}, \ \  R}
  {E, \ \  R} \\
\mbox{if $\langle s,t \rangle \in \mathrm{CP}(E \cup R)$\rule[-3ex]{0ex}{6ex}}
\end{array}
\]

$E$ is a set of equations,
$R$ is a set of rewrite rules,
$s,t,u,l,r$ are terms,
$s \deq t$ represents $s \approx t$ or $t \approx s$,
$\mathrm{CP}(\dots)$ denotes the set of (ordered) critical
pairs of a set of (equations and) rules,
$\succ$ is a reduction ordering that is total on ground terms.

\end{calculus}

% The following sections ("clarifications", "history", 
% "technicalities") are optional. If you use them, 
% be very concise and objective. Nevertheless, do write full sentences. 
% Try to have at most one paragraph per section, because line breaks 
% do not look nice in a short entry.

\begin{clarifications}
Standard completion tries to convert a set of equations into
an equivalent terminating and confluent set of rewrite rules;
it may fail, however, for certain inputs $E$ and $\succ$.
Adding the \textit{UC-Deduce} rule turns standard completion into
a refutationally complete calculus for equational theories.
\end{clarifications}

\begin{history}
Standard completion was developed by Knuth and Bendix~\cite{KnuthBendix1970};
the presentation as an inference system given here
and the extension to unfailing completion
are due to
Bachmair, Dershowitz, and Hsiang~\cite{BachmairDershowitzHsiang1986LICS},
see also~\cite{Bachmair1991}.
An approach to extend completion to completion modulo
associativity and/or commutativity was presented in~\cite{PetersonStickel1981}.

\end{history}

\begin{technicalities}
To prove that an equation $s \approx t$ is entailed by $E$,
unfailing completion is
applied to $E \cup \{\mathit{eq}(x,x) \approx \mathit{true}$,
$\mathit{eq}(\hat{s},\hat{t}) \approx \mathit{false}\}$,
where $\hat{s}$ and $\hat{t}$ are skolemized versions of $s$ and $t$.
Unfailing completion derives $\mathit{true} \approx \mathit{false}$
if and only if $E \models s \approx t$.
\end{technicalities}


% General Instructions:
% =====================

% The preferred length of an entry is 1 page. 
% Do the best you can to fit your proof system in one page.
%
% If you are finding it hard to fit what you want in one page, remember:
%
%   * Your entry needs to be neither self-contained nor fully understandable
%     (the interested reader may consult the cited full paper for details)
%
%   * If you are describing several proof systems in one entry, 
%     consider splitting your entry.
%
%   * You may reduce the size of your entry by ommitting inference rules
%     that are already described in other entries.
%
%   * Cite parsimoniously (see detailed citation instructions below).
%
% 
% If you do not manage to fit everything in one page, 
% it is acceptable for an entry to have 2 pages.
%
% For aesthetical reasons, it is preferable for an entry to have
% 1 full page or 2 full pages, in order to avoid unused blank space.



% Citation Instructions:
% ======================

% Please cite the original paper where the proof system was defined.
% To do so, you may use the \cite command within 
% one of the optional environments above,
% or use the \nocite command otherwise.

% You may also cite a modern paper or book where the 
% proof system is explained in greater depth or clarity.
% Cite parsimoniously.

% Do not cite related work. Instead, use the "\iref" or "\irefmissing" 
% commands to make an internal reference to another entry, 
% as explained within the "history" environment above.

% You do not need to create the "References" section yourself. 
% This is done automatically.




% Leave an empty line above "\end{entry}".

\end{entry}
