
\calculusName{Conditional Nested Sequent Calculi $\mathcal{N}S$}   
\calculusAcronym{$\mathcal{N}S$}     
\calculusLogic{Conditional Logics}  
\calculusLogicOrder{Propositional}
\calculusType{Sequent Calculus}   
\calculusYear{2012-2014}   
\calculusAuthor{R\'egis Alenda, Nicola Olivetti, Gian Luca Pozzato} 


\entryTitle{Conditional Nested Sequents $\mathcal{N}S$}     
\entryAuthor{Nicola Olivetti} \entryAuthor{Gian Luca Pozzato}     





\maketitle



\begin{entry}{NestedConditionals}  




\begin{calculus}
\begin{footnotesize}
\[
\begin{array}{lclr}
\Gamma(P, \lnot P) \quad (\mathit{AX})
& 
\Gamma(\top) \quad (\mathit{AX}_\top)
& 
\quad \Gamma(\lnot \bot) \quad (\mathit{AX}_\bot) & \\
\mbox{\tiny $P$ atomic} & & &  \\ \\ 
\begin{prooftree}
\Gamma(A) \justifies \Gamma(\lnot \lnot A) \using (\lnot)
\end{prooftree}
& 
\quad
\begin{prooftree}
   \Gamma(\lnot(A \Rightarrow B),[A': \Delta, \lnot B]) \quad A, \lnot A' \quad A', \lnot A
   \justifies \Gamma(\lnot(A \Rightarrow B),[A': \Delta]) \using (\Rightarrow^-)
\end{prooftree} 
 &
\quad \begin{prooftree}
  \Gamma([A: B]) \justifies \Gamma(A \Rightarrow B) \using (\Rightarrow^+)
\end{prooftree}
&    \\ \\ 
\begin{prooftree}
  \Gamma([A: \Delta, \lnot A]) \justifies \Gamma([A: \Delta]) \using (\mathit{ID})
\end{prooftree}
& 
\quad
\begin{prooftree}
   \Gamma([A: \Delta, \Sigma],[B: \Sigma]) \qquad A, \lnot B \qquad B, \lnot A
   \justifies \Gamma([A: \Delta],[B: \Sigma]) \using (\mathit{CEM})
\end{prooftree}
 &
& \qquad  \\ \\ 
\end{array}
\]
\[
\begin{array}{l}
\begin{prooftree}
   \Gamma(\lnot(A \Rightarrow B),A) \qquad \Gamma(\lnot(A \Rightarrow B),\lnot B)
   \justifies \Gamma(\lnot(A \Rightarrow B)) \using (\mathit{MP})
\end{prooftree}
\\ \\ 
\begin{prooftree}
\Gamma,\lnot(C \Rightarrow D),[A: \Delta, \lnot D] \quad \Gamma, \lnot(C \Rightarrow D), [A: C]
\quad \Gamma, \lnot(C \Rightarrow D), [C: A] \justifies  \Gamma, \lnot(C \Rightarrow D), [A: \Delta]
\using (\mathit{CSO})
\end{prooftree}
\end{array}
\]
\end{footnotesize}
\end{calculus}



 \begin{clarifications}
% ToDo: write here short remarks that may help the reader to understand 
% the inference rules of the proof system.
Conditional logics extend classical logic with formulas of the form $A \Rightarrow B$: intuitively, $A \Rightarrow B$ is true in a world $x$ if $B$ is true in the set of worlds where $A$ is true and that are most similar to $x$. The calculi $\mathcal{N}S$ manipulate \emph{nested} sequents, a generalization of ordinary sequent calculi where sequents are allowed to occur within sequents. A nested sequent $$A_1, \dots, A_m, [B_1: \Gamma_1], \dots, [B_n: \Gamma_n]$$ is inductively defined by the formula $$\mathcal{F}(\Gamma)=A_1 \vee \dots \vee A_m \vee (B_1 \Rightarrow \mathcal{F}(\Gamma_1)) \vee \dots \vee (B_n \Rightarrow \mathcal{F}(\Gamma_n)).$$ $\Gamma(\Delta)$ represents a sequent $\Gamma$ containing a \emph{context} (a unique empty position) filled by the (nested) sequent $\Delta$.
  Besides the rules shown above, the calculi $\mathcal{N}S$ also include standard 
   rules for propositional connectives.
 \end{clarifications}

 \begin{history}
  The calculi $\mathcal{N}S$ have been introduced in 
  \cite{jelia2012pozz} and extended  in \cite{jlcpozz}. The theorem prover NESCOND, implementing $\mathcal{N}S$ in Prolog, has been presented in \cite{ijcarpozz}.
 \end{history}

 \begin{technicalities}
Completeness is a consequence of the admissibility of cut. The calculi $\mathcal{N}S$ can be used to obtain a \textsc{PSpace} decision procedure for the respective conditional logics (optimal for CK and extensions with ID and MP).
 \end{technicalities}













\end{entry}
