% If the calculus has an acronym, define it.
% (e.g. \newcommand{\LK}{\ensuremath{\mathbf{LK}}\xspace})

\calculusName{Untyped Lambda Reduction}         % The name of the calculus
\calculusAcronym{}          % The acronym if defined above, or empty otherwise.
\calculusLogic{}        % Specify the logic (e.g. Classical Logic, Intuitionistic Logic, ...) for which this calculus is intended.
\calculusLogicOrder{Lambda Abstraction, First-Order}   % Specify the order of the logic (e.g. Propositional, Quantified Propositional, First-Order, Higher-Order, ...).
\calculusType{(Term-)Sequent Calculus}         % Specify the calculus type (e.g. Tableau, Sequent Calculus, Hyper-Sequent Calculus, Natural Deduction, ...)
\calculusYear{2011}         % The year when the calculus was published.

\calculusAuthor{Michael Gabbay}       % The name(s) of the author(s) of the calculus.
%\calculusAuthor{ToDo:FullNameAuthor2}
%\calculusAuthor{ToDo:FullNameAuthor3}


\entryTitle{Untyped Lambda Reduction}     % Title of the entry (usually coincides with the name of the calculus).
\entryAuthor{Michael Gabbay}
%\entryAuthor{ToDo:FullNameAuthor2}
%\entryAuthor{ToDo:FullNameAuthor3}

% The encyclopedia's peer-reviewing policy is described here:
% http://proofsystem.github.io/Encyclopedia/
%
% Reviewers of this entry will be acknowledged in the following lines:
% \entryReviewer{Reviewer 1's name}
% \entryReviewer{Reviewer 2's name}
% \entryReviewer{Reviewer 3's name}
%
% The lines above will be filled by the coordinators.
% If you would like to indicate people
% who could review your entry, contact the coordinators.


% If you wish, use tags to give any other information
% that might be helpful for classifying and grouping this entry:
% e.g. \tag{Two-Sided Sequents}
% e.g. \tag{Multiset Cedents}
% e.g. \tag{List Cedents}
% You are free to invent your own tags.
% The Encyclopedia's coordinator will take care of
% merging semantically similar tags in the future.


\maketitle


% If your files are called "MyProofSystem.tex" and "MyProofSystem.bib",
% then you should write "\begin{entry}{MyProofSystem}" in the line below
\begin{entry}{LambdaTermSequent}

\newcommand{\pa}[1]{\langle #1\rangle}
\newcommand{\col}{{{:}\!{-}}}
\newcommand{\ci}{\rightarrow}
\newcommand\aeq{{\rightsquigarrow}}
\newcommand\aep{{\aeq}L}
\newcommand\aer{{\aeq}R}

\begin{calculus}

\textbf{Term-Sequent Rules}
$$
%
\infer[(Ax_{\col})]{\Gamma\vdash {t}\col {t},\Psi,\Delta}{} \qquad
\infer[(Cut_{\lambda})]{\Gamma\vdash\pa{\dots\Theta\dots}\col{t},\Psi,\Delta}{\Gamma\vdash\Theta\col{s},\Psi,\Delta&\Gamma\vdash\pa{\dots{s}\dots}\col{t},\Psi,\Delta}
$$
$$
\infer[(\cdot L)]{\Gamma\vdash\pa{\dots{t}_1\cdot{t}_2\dots}\col
{t},\Psi,\Delta}{\Gamma\vdash\pa{\dots\pa{{t}_1,{t}_2}\dots}\col
{t},\Psi,\Delta} \quad \infer[(\cdot
R)]{\Gamma\vdash\pa{\Theta_1,\Theta_2}\col{t}_1\cdot{t}_2,\Psi,\Delta}{\Gamma\vdash\Theta_1\col{t}_1,\Psi,\Delta&\Gamma\vdash\Theta_2\col{t}_2,\Psi,\Delta
}
$$
$$
%
\infer[(\aep)]{\Gamma,{t}_1\aeq{t}_2\vdash\pa{\dots\Theta\dots}\col{t},\Psi,\Delta}{\Gamma\vdash\Theta\col{t}_1,\Psi,\Delta&\Gamma\vdash\pa{\dots{t}_2\dots}\col{t},\Psi,\Delta}%%\
\qquad
%
\infer[(\aer)]{\Gamma\vdash {t}_1\aeq{t}_2,\Psi,\Delta}{\Gamma\vdash
{t}_1\col{t}_2,\Psi,\Delta}
%
$$
$$
%
\infer[(\lambda
L)]{\Gamma\vdash\pa{\dots\pa{\lambda{x}.{t}_2,\Theta}\dots}\col{t},\Psi,\Delta}{\Gamma\vdash\Theta\col{t}_1,\Psi,\Delta&\Gamma\vdash\pa{\dots{t}_2[{x}/{t}_1]\dots}\col{t},\Psi,\Delta}
%
\qquad
%
\infer[(\lambda
R)]{\Gamma\vdash\Theta\col\lambda{x}.{t},\Psi,\Delta}{\Gamma\vdash\pa{\Theta,{y}}\col{t}[{x}/{y}]
,\Psi,\Delta}
%\ \
%
%
%
$$
\textbf{(Classical) Sequent Rules}
$$
%
\infer[(Cut)]{\Gamma\vdash\Psi,\Delta}{\Gamma\vdash \Psi,{A},\Delta
&\Gamma,{A}\vdash \Psi,\Delta}
%
%
$$
$$
%
%
\infer[(\forall L)]{\Gamma,\forall {x}.{A}\vdash\Psi,\Delta}{\Gamma,{A}[{x}/{t}]\vdash
\Psi,\Delta}
%
\qquad
%
\infer[(\forall R)]{\Gamma\vdash\Psi,\forall {x}.{A},\Delta}{\Gamma\vdash \Psi,{A}[{x}/{y}],\Delta}% \ \
%
%\parbox{11ex}{\smaller ${y}$ not free in$\Gamma,\Delta$ or $\Psi$}
% %\quad
%\infer[(\cn R)]{\Gamma\vdash\cn {A},\Delta}{\Gamma,{A}\vdash\Delta}
%\quad
%\infer[(\cn L)]{\Gamma,\cn {A}\vdash\Delta}{\Gamma\vdash {A},\Delta}
$$
$$
\infer[(\ci L)]{\Gamma,{A}\ci {B}\vdash \Psi,\Delta}{\Gamma\vdash
{A},\Delta&\Gamma,{B}\vdash \Psi,\Delta} \qquad \infer[(\ci
R)]{\Gamma\vdash\Psi, {A}\ci {B},\Delta}{\Gamma,{A}\vdash\Psi,{B},\Delta}
\qquad \infer[(\bot)]{\Gamma,\bot\vdash
\Psi,\Delta}{}%\Gamma,{A}\vdash\Psi,\Delta}
$$
\end{calculus}

% The following sections ("clarifications", "history",
% "technicalities") are optional. If you use them,
% be very concise and objective. Nevertheless, do write full sentences.
% Try to have at most one paragraph per section, because line breaks
% do not look nice in a short entry.

 \begin{clarifications}
% ToDo: write here short remarks that may help the reader to understand
% the inference rules of the proof system.
A {\em term-sequent} is a pair $\Theta \col {t}$ where $\Theta$ is a tree and ${t}$ is a term. Define {\em trees} by: $\Theta::={t} \mid \pa{\Theta_1,\Theta_2}$. If $\Theta'$ occurs as a {\em subtree} of $\Theta$ then we write $\Theta$ as $\pa{\dots\Theta'\dots}$. A {\em sequent} has the form $\Gamma\vdash \Psi,\Delta$ where $\Gamma$ and $\Delta$ are sets of formulae and $\Psi$ is a set of term-sequents. ${y}$ must not be free in the lower term-sequent of $(\lambda R)$ nor the lower sequent of $(\forall R)$.
 \end{clarifications}

% \begin{history}
% ToDo: write here short historical remarks about this proof system,
% especially if they relate to other proof systems.
% Use "\iref{OtherProofSystem}" to refer to another proof system
% in the Encyclopedia (where "OtherProofSystem" is its ID).
% Use "\irefmissing{SuggestedIDForOtherProofSystem}" to refer to
% another proof system that is not yet available in the encyclopedia.
% \end{history}

\begin{technicalities}
Cut elimination --- for both $(Cut)$ and $(Cut_{\lambda})$ --- is proved in~\cite{mgabbay:lambdacut}, as is soundness and completeness of term-sequents with respect to the calculus of untyped lambda reduction with $\beta$-reduction and $\eta$-expansion. Soundness and completeness of the full calculus is shown for an axiomatic presentation of a stronger system in~\cite{gabbay:simcks} (using a model theory of lambda reduction similar to {\em graph models} which is expanded further in~\cite{gabbay:simcmt,Gabbay2016}).
\end{technicalities}

\end{entry} 