
% If the calculus has an acronym, define it.
% (e.g. \newcommand{\LK}{\ensuremath{\mathbf{LK}}\xspace})

\calculusName{LambdaPiModulo}   % The name of the calculus
\calculusAcronym{}    % The acronym if defined above, or empty otherwise.
\calculusLogic{Type Theory}  % Specify the logic (e.g. classical, intuitionistic, ...) for which this calculus is intended.
\calculusLogicOrder{First-Order}
\calculusType{Natural Deduction}   % Specify the calculus type (e.g. Frege-Hilbert style, tableau, sequent calculus, hypersequent calculus, natural deduction, ...)
\calculusYear{2007}   % The year when the calculus was invented.
\calculusAuthor{Denis Cousineau \and Gilles Dowek} % The name(s) of the author(s) of the calculus.


\entryTitle{$\lambda\Pi$-Calculus Modulo}     % Title of the entry (usually coincides with the name of the calculus).
\entryAuthor{Ronan Saillard}    % Your name(s). Separate multiple names with "\and".


\tag{Deduction Modulo}


\maketitle


% If your files are called "MyProofSystem.tex" and "MyProofSystem.bib",
% then you should write "\begin{entry}{MyProofSystem}" in the line below
\begin{entry}{LambdaPiModulo}

% Define here any newcommands you may need:
% e.g. \newcommand{\necessarily}{\Box}
% e.g. \newcommand{\possibly}{\Diamond}
\newcommand\AxCb[1]{\AxiomC{#1}}
\newcommand\UICb[1]{\UnaryInfC{#1}}
\newcommand\BICb[1]{\BinaryInfC{#1}}
\newcommand\TICb[1]{\TrinaryInfC{#1}}
\newcommand\DPb{\DisplayProof}
\newcommand\la{\lambda}
\newcommand\be{\beta}
\newcommand\Ga{\Gamma}
\newcommand\De{\Delta}
\newcommand\eqbg{\equiv_{\be\Ga}}
\newcommand\Type{{\bf Type}}
\newcommand\Kind{{\bf Kind}}
\newcommand\typg[3]{\Ga;#1\vdash#2:#3}
\newcommand\typctx[1]{\Ga\vdash^{ctx}#1}
\newcommand\wf[1]{#1\ {\bf wf}}
\newcommand\red{\rightarrow}
\newcommand\rw{\hookrightarrow}

\begin{calculus}

% Add the inference rules of your proof system here.
% The "proof.sty" and "bussproofs.sty" packages are available.
% If you need any other package, please contact the editor (bruno@logic.at)
\centering
\begin{tabular}{c}
	\multicolumn{1}{l}{\sc Typing Rules for Terms}\\
	% SORT
	\AxCb{$\typctx{\De}$}
	\RightLabel{\bf(Sort)}
	\UICb{$\typg{\De}{\Type}{\Kind}$}
	\DPb{}
	\\[0.5cm]

	% CONSTANT
	\AxCb{$\typctx{\De}$}
	\AxCb{$(c:A)\in\Ga$}
	\RightLabel{\bf(Constant)}
	\BICb{$\typg{\De}{c}{A}$}
	\DPb{}

	% VARIABLE
	\AxCb{$\typctx{\De}$}
	\AxCb{$(x:A)\in\De$}
	\RightLabel{\bf(Variable)}
	\BICb{$\typg{\De}{x}{A}$}
	\DPb{}
	\\[0.5cm]

	% APPLICATION
	\AxCb{$\typg{\De}{t}{\Pi x:A.B}$}
	\AxCb{$\typg{\De}{u}{A}$}
	\RightLabel{\bf(Application)}
	\BICb{$\typg{\De}{tu}{B[x/u]}$}
	\DPb{}
	\\[0.5cm]

	% ABSTRACTION
	\AxCb{$\typg{\De(x:A)}{t}{B}$}
	\AxCb{$\typg{\De}{\Pi x:A.B}{s}$}
	\RightLabel{\bf(Abstraction)}
	\BICb{$\typg{\De}{\la x:A.t}{\Pi x:A.B}$}
	\DPb{}
	\\[0.5cm]

	% PRODUCT
	\AxCb{$\typg{\De}{A}{\Type}$}
	\AxCb{$\typg{\De(x:A)}{B}{s}$}
	\RightLabel{\bf(Product)}
	\BICb{$\typg{\De}{\Pi x:A.B}{s}$}
	\DPb{}
	\\[0.5cm]

	% CONVERSION
	\AxCb{$\typg{\De}{t}{A}$}
	\AxCb{$\typg{\De}{B}{s}$}
	\AxCb{$A \eqbg B$}
	\RightLabel{\bf(Conversion)}
	\TICb{$\typg{\De}{t}{B}$}
	\DPb{}
	\\[0.5cm]

	\multicolumn{1}{l}{\sc Well-Formedness for Local Contexts} \\[0.5cm]

	% EMPTY SIGNATURE
	\AxCb{$\wf{\Ga}$}
	\UICb{$\typctx{\emptyset}$}
	\DPb{}
	\quad

	% OBJECT VARIABLE
	\AxCb{$\typctx{\De}$}
	\AxCb{$\typg{\De}{U}{\Type}$}
	\AxCb{$x \notin dom(\De)$}
	\TICb{$\typctx{\De(x:U)}$}
	\DPb{}
	\\[0.5cm]

	\multicolumn{1}{l}{\sc Well-Formedness Rules for Global Contexts}  \\[0.5cm]
	% EMPTY
	\AxCb{}
	\UICb{$\wf{\emptyset}$}
	\DPb{}
	\quad

	% OBJECT CONSTANT
	\AxCb{$\wf{\Ga}$}
	\AxCb{$\typg{\emptyset}{U}{\Type}$}
	%\AxC{$\ud{c} \notin dom(\Ga)$}
	\BICb{$\wf{\Ga(c:U)}$}
	\DPb{}
	\quad

	% TYPE CONSTANT
	\AxCb{$\wf{\Ga}$}
	\AxCb{$\typg{\emptyset}{K}{\Kind}$}
	%\AxC{${\bf PC}(\Ga(C:K))\footnotemark[1]$}
	%\AxC{$C\notin dom(\Ga)$}
	\BICb{$\wf{\Ga(C:K)}$}
	\DPb{}
	\\[0.5cm]

	% REWRITE RULES
	\AxCb{$\wf{\Ga}$}
	%\AxC{${\bf PC}(\Ga\Xi)$}
	\AxCb{$\red_{\be} \cup \red_{\Ga\Xi}$ is confluent}
	\AxCb{$(\forall i)\Ga \vdash u_i \rw v_i$}
	\noLine{}
	\UICb{$\Xi = (u_1\rw v_1)\ldots(u_n\rw v_n)$}
	\TICb{$\wf{\Ga\Xi}$}
	\DPb{}
\end{tabular}

\end{calculus}

% The following sections ("clarifications", "history",
% "technicalities") are optional. If you use them,
% be very concise and objective. Nevertheless, do write full sentences.
% Try to have at most one paragraph per section, because line breaks
% do not look nice in a short entry.

\begin{clarifications}
	The $\lambda\Pi$-Calculus Modulo is an extension of the $\lambda$-Calculus with dependent types and rewrite rules.
	Computational equivalence is extended from $\be$-equivalence to $\be\Ga$-equivalence ($\equiv_{\be\Ga}$),
	the congruence generated by $\be$-reduction and the rewrite rules $(u\rw v)$ in the global context $\Ga$.
\end{clarifications}

\begin{history}
	The $\lambda\Pi$-Calculus Modulo has been introduced by Cousineau and Dowek~\cite{DBLP:conf/tlca/CousineauD07} as an expressive logical framework.
	It has been used to design \emph{shallow encodings} of many logics and calculus such as
	functional Pure Type Systems~\cite{DBLP:conf/tlca/CousineauD07},
	Higher-Order Logic~\cite{AliHOL},
	the Calculus of Inductive Constructions~\cite{Coqine},
	resolution and superposition~\cite{Resolution},
	or the $\varsigma$-calculus~\cite{RaphaelSigma}.
%
	The well-formedness rules for global contexts were not part of the original type system and have been introduced
	by Saillard~\cite{ModuloBeta}.
%
	The $\lambda\Pi$-Calculus Modulo is implemented in the proof checker Dedukti~\cite{Dedukti}.
\end{history}

\begin{technicalities}
	Confluence of the rewriting relation $\red_{\be\Ga}$ is required to guarantee subject reduction.
	This requirement can be weakened to confluence for a notion of rewriting modulo $\be$~\cite{ModuloBeta}.
	Decidability of type inference depends on strong normalization.
\end{technicalities}


% General Instructions:
% =====================

% The preferred length of an entry is 1 page.
% Do the best you can to fit your proof system in one page.
%
% If you are finding it hard to fit what you want in one page, remember:
%
%   * Your entry needs to be neither self-contained nor fully understandable
%     (the interested reader may consult the cited full paper for details)
%
%   * If you are describing several proof systems in one entry,
%     consider splitting your entry.
%
%   * You may reduce the size of your entry by ommitting inference rules
%     that are already described in other entries.
%
%   * Cite parsimoniously (see detailed citation instructions below).
%
%
% If you do not manage to fit everything in one page,
% it is acceptable for an entry to have 2 pages.
%
% For aesthetical reasons, it is preferable for an entry to have
% 1 full page or 2 full pages, in order to avoid unused blank space.



% Citation Instructions:
% ======================

% Please cite the original paper where the proof system was defined.
% To do so, you may use the \cite command within
% one of the optional environments above,
% or use the \nocite command otherwise.

% You may also cite a modern paper or book where the
% proof system is explained in greater depth or clarity.
% Cite parsimoniously.

% Do not cite related work. Instead, use the "\iref" or "\irefmissing"
% commands to make an internal reference to another entry,
% as explained within the "history" environment above.

% You do not need to create the "References" section yourself.
% This is done automatically.




% Leave an empty line above "\end{entry}".

\end{entry}
