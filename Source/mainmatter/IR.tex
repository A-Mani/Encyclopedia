
\calculusName{IR} 
\calculusAcronym{IR}   
\calculusLogic{False  Quantified Boolean Formulas in Closed Prenex CNF} 
\calculusType{Resolution} 
\calculusYear{2014}  
\calculusAuthor{Olaf Beyersdorff, Leroy Chew, Mikol\'{a}\v{s} Janota } 


\entryTitle{IR} 
\entryAuthor{Leroy Chew \and Mikol\'{a}\v{s} Janota} 
\maketitle

\tag{Resolution}
\tag{Refutation}
\tag{Instantiation}
\tag{CNF}
\tag{Closed Prenex QBF}
\tag{QBF}

\begin{entry}{IR}  

%\DeclareMathOperator*{\lev}{\operatorname{lv}}
%\DeclareMathOperator*{\instantiate}{\textsf{inst}}
%\DeclareMathOperator*{\range}{{\sf rng}}
\newcommand{\comprehension}[2]{\ensuremath{\left\{ {#1} \;|\; {#2}\right\}}}
\newcommand{\lev}{{\sf lv}}
\newcommand{\restr}{{\sf restr}}
\newcommand{\range}{{\sf rng}}
\newcommand{\domain}{{\sf dom}}

\begin{calculus}

    \begin{bussprooftree}
      \AxiomC{}
      \RightLabel{(Ax)}
      \UnaryInfC{
        $\comprehension{x^{\hspace{1pt}\restr(\tau,x)}}
                       {x\in C, x\textrm{ is existential}} $}
    \end{bussprooftree}
    $C$ is a non-tautological clause from the matrix. \\$\tau=\comprehension{0/u}{u\textrm{ is universal in }C}$, where the notation $0/u$ for literals $u$ is shorthand for $0/y$ if $u=y$ and $1/y$ if $u=\neg y$. We define $\restr(\tau, x)$ as ${\comprehension{c/u}{c/u\in \tau, \lev(u)<\lev(x)}}$.
    \begin{bussprooftree}
      \AxiomC{$x^\tau\lor C_1 $ }
      \AxiomC{$\lnot x^\tau\lor C_2 $}
      \RightLabel{(Resolution)}
      \BinaryInfC{$C_1\cup C_2$}
    \end{bussprooftree}
    \begin{bussprooftree}
      \AxiomC{$C$}
      \RightLabel{(Instantiation)}
      \UnaryInfC{$\comprehension{x^{\hspace{1pt}\xi}}{x^{\hspace{1pt}\sigma}\in C, x\textrm{ is existential}}$}
    \end{bussprooftree}
    $\tau$ is a partial assignment to universal variables with $\range(\tau) \subseteq \{0,1\}$.
    $\xi=\sigma\cup\comprehension{c/u}{c/u \in \restr(\tau,x), u \notin \domain(\sigma)}$


    \centerline{The rules of IR~\cite{MFCS14}}
\end{calculus}


\begin{clarifications}
  The calculus aims to refute a quantified Boolean formula (QBF) of the form
  $Q_1 x_1\dots Q_n x_n.\,\phi$ where $Q_i\in\{\forall,\exists\}$
  and $\phi$ is a Boolean formula in conjunctive normal form (CNF).
  The formula $\phi$ is referred to as the \emph{matrix}.
  We write $\lev(x)$ for the \emph{quantification level} of~$x$, i.e.\ $\lev(x_i)=i$.
  A variable~$x_i$ is \emph{existential} (resp.\ \emph{universal}) if $Q_i=\exists$ (resp.\ $Q_i=\forall$).

  The calculus works by introducing clauses as \emph{annotated clauses}, which are sets of annotated literals. Annotated literals consist of an existential literal and an annotation -- a partial assignment to universal variables in $\{0,1\}$. Two literals are identical if and only if both the existential literal and annotation are equal.
%
  The calculus enables deriving the empty clause if and only if the given formula is false.
\end{clarifications}


\begin{technicalities}
Soundness was shown by extracting valid Herbrand functions. Completeness is shown by p-simulation of 
another known QBF system Q-Resolution \irefmissing{QResolution}.
\end{technicalities}


\begin{history}
The name of the calculus comes from the two pivotal operations \emph{instantiation} and \emph{resolution}.
The calculus naturally generalizes an older calculus $\forall$Exp+Res~\cite{JanotaTCS15},
which  requires all clauses to be introduced into the proof by using a complete assignment.
\irefmissing{SuggestedIDForOtherProofSystem}
\end{history}


\end{entry}
