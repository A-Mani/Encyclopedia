
% If the calculus has an acronym, define it.
% (e.g. \newcommand{\LK}{\ensuremath{\mathbf{LK}}\xspace})

\calculusName{Constraint Superposition}   % The name of the calculus
\calculusAcronym{}    % The acronym if defined above, or empty otherwise. 
\calculusLogic{Classical Logic}  % Specify the logic (e.g. classical, intuitionistic, ...) for which this calculus is intended.
\calculusLogicOrder{First-Order}
\calculusType{Superposition}   % Specify the calculus type (e.g. Frege-Hilbert style, tableau, sequent calculus, hypersequent calculus, natural deduction, ...)
\calculusYear{1992/1995}   % The year when the calculus was invented.
\calculusAuthor{Robert Nieuwenhuis, Albert Rubio} % The name(s) of the author(s) of the calculus.


\entryTitle{Constraint Superposition}     % Title of the entry (usually coincides with the name of the calculus).
\entryAuthor{Uwe Waldmann}    % Your name(s). Separate multiple names with "\and".


% If you wish, use tags to give any other information 
% that might be helpful for classifying and grouping this entry:
% e.g. \tag{Two-Sided Sequents}
% e.g. \tag{Multiset Cedents}
% e.g. \tag{List Cedents}
% You are free to invent your own tags. 
% The Encyclopedia's coordinator will take care of 
% merging semantically similar tags in the future.


\maketitle


% If your files are called "MyProofSystem.tex" and "MyProofSystem.bib", 
% then you should write "\begin{entry}{MyProofSystem}" in the line below
\begin{entry}{ConstraintSup}

% Define here any newcommands you may need:
% e.g. \newcommand{\necessarily}{\Box}
% e.g. \newcommand{\possibly}{\Diamond}
\newcommand{\co}[1]{\mathrel{[\kern-0.18em[#1]\kern-0.18em]}}


\begin{calculus}

% Add the inference rules of your proof system here.
% The "proof.sty" and "bussproofs.sty" packages are available.
% If you need any other package, please contact the editor (bruno@logic.at)

\[
\infer[\textit{Equality Resolution}]
{C\co{T \land T''}}{C \lor \neg u \approx v\co{T}}
\]
\[
\infer[\textit{Negative Superposition}]
{D \lor C \lor \neg t[u'] \approx t'\co{T \land T' \land T''}}
{D \lor u \approx u'\co{T'}
& C \lor \neg t[v] \approx t'\co{T}}
\]
\[
\infer[\textit{Positive Superposition}]
{D \lor C \lor t[u'] \approx t'\co{T \land T' \land T''}}
{D \lor u \approx u'\co{T'}
& C \lor t[v] \approx t'\co{T}}
\]
\[
\infer[\textit{Equality Factoring}]
{C \lor \neg u' \approx v' \lor u \approx u'\co{T \land T''}}
{C \lor v \approx v' \lor u \approx u'\co{T}}
\]
$C,D$ are (possibly empty) equational clauses,
$T,T',T''$ are constraints (i.\,e., first-order formulas over
terms and the binary predicate symbols $=$ and $\succ$),
$t,t',u,u',v,v'$ are terms.
In binary inferences, $v$ is not a variable.\\
The constraint $T''$ is the conjunction of the
unifiability constraint $u = v$
and the ordering constraints
that state
that the literals involved in the inference
are maximal in their premises
(except for the last but one literal in
\textit{Equality Factoring} inferences),
that positive literals involved in a
\textit{(Positive or Negative) Superposition} inference
are strictly maximal in the respective premise,
and that
in every literal involved in the inference
(except \textit{Equality Resolution}),
the lhs is strictly maximal.

% \bigskip
% 
% \textbf{Saturation}
% \[
% \infer
% {N \cup \{C\}}{N & C \textrm{~is the conclusion of a Resolution inference
% from clauses in~} N}
% \]
% \mbox{}\quad $N$ is a finite set of clauses,
% $C$ is a clause.
\end{calculus}

% The following sections ("clarifications", "history", 
% "technicalities") are optional. If you use them, 
% be very concise and objective. Nevertheless, do write full sentences. 
% Try to have at most one paragraph per section, because line breaks 
% do not look nice in a short entry.

\begin{clarifications}
Constraint superposition is a refutational saturation calculus for
first-order clauses (disjunctions of possibly negated atoms)
with equality (denoted by~$\approx$).
A constrained clause $C\co{T}$ represents those ground instances $C\theta$
for which $T\theta$ evaluates to $\mathit{true}$;
the initially given clauses are supposed to have a trivial
constraint, that is, $C\co{\mathit{true}}$.
The inference rules are supplemented by a redundancy criterion
that permits to delete constrained clauses that are unnecessary for
deriving a contradiction during the saturation, see \iref{SaturationWithRed}.
In particular, every constrained clause with an unsatisfiable
constraint is redundant.
\end{clarifications}

\begin{history}
The idea to use constrained formulas in automated reasoning
originated in~\cite{KirchnerKirchnerRusinowitch1990RFIA}.
There are several reasons to switch from
standard superposition~\iref{Superposition} to
superposition with constrained clauses~\cite{NieuwenhuisRubio1992ESOP,NieuwenhuisRubio1992CADE,NieuwenhuisRubio1995JSC}.
First,
ordering constraints
make it possible to pass on information about the instances
for which an inference is actually needed to the derived clauses.
Second,
working with unifiability constraints
rather than computing and applying unifiers
avoids future superposition inferences
into the substitution part
(basic strategy).
Finally, in theory calculi, such as~\cite{NieuwenhuisRubio1994CADE},
unifiability constraints allow
to encode a multitude of theory
unifiers compactly.

\end{history}

\begin{technicalities}
The constraint superposition calculus is refutationally complete for
first-order logic with equality,
provided that the initially given clauses have only trivial constraints
(for ordering constraints, this requirement can be relaxed slightly).
\end{technicalities}


% General Instructions:
% =====================

% The preferred length of an entry is 1 page. 
% Do the best you can to fit your proof system in one page.
%
% If you are finding it hard to fit what you want in one page, remember:
%
%   * Your entry needs to be neither self-contained nor fully understandable
%     (the interested reader may consult the cited full paper for details)
%
%   * If you are describing several proof systems in one entry, 
%     consider splitting your entry.
%
%   * You may reduce the size of your entry by ommitting inference rules
%     that are already described in other entries.
%
%   * Cite parsimoniously (see detailed citation instructions below).
%
% 
% If you do not manage to fit everything in one page, 
% it is acceptable for an entry to have 2 pages.
%
% For aesthetical reasons, it is preferable for an entry to have
% 1 full page or 2 full pages, in order to avoid unused blank space.



% Citation Instructions:
% ======================

% Please cite the original paper where the proof system was defined.
% To do so, you may use the \cite command within 
% one of the optional environments above,
% or use the \nocite command otherwise.

% You may also cite a modern paper or book where the 
% proof system is explained in greater depth or clarity.
% Cite parsimoniously.

% Do not cite related work. Instead, use the "\iref" or "\irefmissing" 
% commands to make an internal reference to another entry, 
% as explained within the "history" environment above.

% You do not need to create the "References" section yourself. 
% This is done automatically.




% Leave an empty line above "\end{entry}".

\end{entry}
