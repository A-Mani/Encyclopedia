


\calculusName{Constraint Superposition}   
\calculusAcronym{}     
\calculusLogic{Classical Logic}  
\calculusLogicOrder{First-Order}
\calculusType{Superposition}   
\calculusYear{1992/1995}   
\calculusAuthor{Robert Nieuwenhuis} \calculusAuthor{Albert Rubio} 


\entryTitle{Constraint Superposition}     
\entryAuthor{Uwe Waldmann}     





\maketitle



\begin{entry}{ConstraintSup}


\newcommand{\co}[1]{\mathrel{[\kern-0.18em[#1]\kern-0.18em]}}


\begin{calculus}

% Add the inference rules of your proof system here.
% The "proof.sty" and "bussproofs.sty" packages are available.
% If you need any other package, please contact the editor (bruno@logic.at)

\[
\infer[\textit{Equality Resolution}]
{C\co{T \land T''}}{C \lor \neg u \approx v\co{T}}
\]
\[
\infer[\textit{Negative Superposition}]
{D \lor C \lor \neg t[u'] \approx t'\co{T \land T' \land T''}}
{D \lor u \approx u'\co{T'}
& C \lor \neg t[v] \approx t'\co{T}}
\]
\[
\infer[\textit{Positive Superposition}]
{D \lor C \lor t[u'] \approx t'\co{T \land T' \land T''}}
{D \lor u \approx u'\co{T'}
& C \lor t[v] \approx t'\co{T}}
\]
\[
\infer[\textit{Equality Factoring}]
{C \lor \neg u' \approx v' \lor u \approx u'\co{T \land T''}}
{C \lor v \approx v' \lor u \approx u'\co{T}}
\]
$C,D$ are (possibly empty) equational clauses,
$T,T',T''$ are constraints (i.\,e., first-order formulas over
terms and the binary predicate symbols $=$ and $\succ$),
$t,t',u,u',v,v'$ are terms.
In binary inferences, $v$ is not a variable.\\
The constraint $T''$ is the conjunction of the
unifiability constraint $u = v$
and the ordering constraints
that state
that the literals involved in the inference
are maximal in their premises
(except for the last but one literal in
\textit{Equality Factoring} inferences),
that positive literals involved in a
\textit{(Positive or Negative) Superposition} inference
are strictly maximal in the respective premise,
and that
in every literal involved in the inference
(except \textit{Equality Resolution}),
the lhs is strictly maximal.

% \bigskip
% 
% \textbf{Saturation}
% \[
% \infer
% {N \cup \{C\}}{N & C \textrm{~is the conclusion of a Resolution inference
% from clauses in~} N}
% \]
% \mbox{}\quad $N$ is a finite set of clauses,
% $C$ is a clause.
\end{calculus}



\begin{clarifications}
Constraint superposition is a refutational saturation calculus for
first-order clauses (disjunctions of possibly negated atoms)
with equality (denoted by~$\approx$).
A constrained clause $C\co{T}$ represents those ground instances $C\theta$
for which $T\theta$ evaluates to $\mathit{true}$;
the initially given clauses are supposed to have a trivial
constraint, that is, $C\co{\mathit{true}}$.
The inference rules are supplemented by a redundancy criterion
that permits to delete constrained clauses that are unnecessary for
deriving a contradiction during the saturation, see \iref{SaturationWithRed}.
In particular, every constrained clause with an unsatisfiable
constraint is redundant.
\end{clarifications}

\begin{history}
The idea to use constrained formulas in automated reasoning
originated in~\cite{KirchnerKirchnerRusinowitch1990RFIA}.
There are several reasons to switch from
standard superposition~\iref{Superposition} to
superposition with constrained clauses~\cite{NieuwenhuisRubio1992ESOP,NieuwenhuisRubio1992CADE,NieuwenhuisRubio1995JSC}.
First,
ordering constraints
make it possible to pass on information about the instances
for which an inference is actually needed to the derived clauses.
Second,
working with unifiability constraints
rather than computing and applying unifiers
avoids future superposition inferences
into the substitution part
(basic strategy).
Finally, in theory calculi, such as~\cite{NieuwenhuisRubio1994CADE},
unifiability constraints allow
to encode a multitude of theory
unifiers compactly.

\end{history}

\begin{technicalities}
The constraint superposition calculus is refutationally complete for
first-order logic with equality,
provided that the initially given clauses have only trivial constraints
(for ordering constraints, this requirement can be relaxed slightly).
\end{technicalities}













\end{entry}
