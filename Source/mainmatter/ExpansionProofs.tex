\calculusName{Expansion Proofs}
\calculusAcronym{Expansion Proofs}
\calculusLogic{Classical Logic}
\calculusLogicOrder{First-Order}
\calculusType{Matrix-Based}
\calculusYear{1983}
\calculusAuthor{Dale Miller}

\entryTitle{Expansion Proofs}
\entryAuthor{Dale Miller}

\tag{Matrix-based proof system}

\maketitle

\begin{entry}{ExpansionProofs}  

\newcommand{\Deep}[1]{\mathsf{Dp}(#1)}
\newcommand{\Sh}[1]{\mathsf{Sh}(#1)}
\newcommand{\calE}{\mathcal{E}}

\newcommand\ALL   {}
\def\ALL#1.{\forall{#1}.\,}
\def\ALLx#1.{\forall\mkern -3mu{#1}.\,}
\newcommand\EX    {}
\def\EX#1.{\exists{#1}.\,}

\begin{calculus}

\emph{Expansion trees}, \emph{eigenvariables}, and the function
$\Sh{-}$ (read \emph{shallow formula of}), that maps an expansion
tree to a formula, are defined as follows:
\begin{enumerate} 
\item If $A$ is $\top$ (true), $\bot$ (false), or a literal, then $A$ is an
    expansion tree with top node $A$, and $\Sh{A} = A$.
\item If $E$ is an expansion tree with $\Sh{E}=[y/x]A$ and $y$ is not an
    eigenvariable of any node in $E$, then $E' = \ALLx x. A +^y E$ is
    an expansion tree with top node $\ALLx x. A$ and $\Sh{E'}
    = \ALLx x. A$.  The variable $y$ is called an {\em eigenvariable}
    of (the top node of) $E'$\kern -3pt.  The set of eigenvariables of
    all nodes in an expansion tree is called the \emph{eigenvariables
    of} the tree.
\item If $\{t_1,\ldots,t_n\}$ (with $n \ge 0$) is a set of terms and
    $E_1,\ldots,E_n$ are expansion trees with pairwise disjoint
    eigenvariable sets and with $\Sh{E_i} = [t_i/x]A$ for
    $i\in\{1,\ldots,n\}$, then $E' = \EX x.  A +^{t_1} E_1 \ldots
    +^{t_n} E_n$ is an expansion tree with top node $\EX x.  A$ and
    $\Sh{E'} = \EX x. A$.  The terms $t_1,\ldots,t_n$ are known
    as the \emph{expansion terms} of (the top node of) $E'$.
\item If $E_1$ and $E_2$ are expansion trees that share no eigenvariables and
    $\circ\in\{\land,\lor\}$, then $E_1\circ E_2$ is an expansion tree
    with top node $\circ$ and $\Sh{E_1\circ E_2}
    = \Sh{E_1} \circ \Sh{E_2}$.
\end{enumerate}

\medskip

In the expansion tree $\ALLx x. A +^x E$ (resp. in $\EX x. A +^{t_1} E_1
\ldots +^{t_n} E_n$), we say that $x$ (resp. $t_i$) \emph{labels} the top node
of $E$ (resp. $E_i$, for any $i \in\{1,\ldots,n\}$).  A term
$t$ \emph{dominates} a node in an expansion tree if it labels a parent
node of that node in the tree.

\medskip

For an expansion tree $E$, the quantifier-free formula $\Deep{E}$, 
called the \emph{deep formula of} $E$, is defined as:
\begin{itemize}
\item $\Deep{E} = E$ if $E$ is $\top$, $\bot$, or a literal;
\item $\Deep{E_1 \circ E_2} = \Deep{E_1} \circ \Deep{E_2}$ for
      $\circ\in\{\land, \lor \}$; 
\item $\Deep{\ALLx x. A +^y E} = \Deep{E}$; and
\item $\Deep{\EX x. A +^{t_1} E_1 \cdots +^{t_n} E_n} = 
       \Deep{E_1}\lor\ldots\lor \Deep{E_n}$ if $n > 0$, and $\Deep{\EX
  x. A} = \bot$. 
\end{itemize}

Let $\calE$ be an expansion tree and let $<^0_{\calE}$ be the binary
relation on the occurrences of expansion terms in $\calE$ defined by
$t <^0_{\calE} s$ if there is an $x$ which is free in $s$ and which is
the eigenvariable of a node dominated by $t$.  Then $<_{\calE}$, the
transitive closure of $<^0_{\calE}$, is called the
\emph{dependency relation} of $\calE$.

\medskip

An expansion tree $\calE$ is said to be an \emph{expansion proof} if
$<_\calE$ is acyclic and $\Deep{\calE}$ is a tautology; in particular,
$\calE$ is an \emph{expansion proof of} $\Sh{\calE}$.
\end{calculus}

% The following environments ("clarifications", "history", 
% "technicalities") are optional. If you do use them, 
% be very concise and objective.

\begin{clarifications}
The soundness and completeness theorem for expansion trees is the
following.  A formula $B$ is a theorem of first-order logic if and
only if there is an expansion proof $Q$ such that $\Sh{Q}=B$.
\end{clarifications}

\begin{history}
Expansion trees and expansion proofs \cite{miller87sl,miller83}
provide a simple generalization of both Herbrand's disjunctions and
Gentzen's mid-sequent theorem to formulas that are not necessarily in
prenex-normal form.  These proof structures were originally defined
for higher-order classical logic and used to provide a generalization
of Herbrand's theorem for higher-order logic as well as a soundness
proof for skolemization in the presence of higher-order
quantification.  Expansion trees are an early example of a
matrix-based proof system that emphasizes parallelism within proof
structures in a manner similar to that found in linear logic proof
nets \cite{girard87tcs}.  That parallelism is explicitly analyzed
in \cite{chaudhuri14jlc} using a multi-focused version of
LKF \iref{LKF}.
\end{history}

% \begin{history}
% ToDo: write here short historical remarks about this proof system,
% especially if they relate to other proof systems. 
% Use "\iref{OtherProofSystem}" to refer to another proof system 
% in the Encyclopedia (where "OtherProofSystem" is its ID). 
% Use "\irefmissing{SuggestedIDForOtherProofSystem}" to refer to 
% another proof system that is not yet available in the encyclopedia.
% \end{history}

% \begin{technicalities}
% ToDo: write here remarks about soundness, completeness, decidability...
% \end{technicalities}



% Please cite the original paper where the proof system was defined.
% To do so, you may use the \cite command within 
% one of the optional environments above,
% or use the \nocite command otherwise.

% You may also cite a modern paper or book where the 
% proof system is explained in greater depth or clarity.
% Cite parsimoniously.

% Do not cite related work. Instead, use the "\iref" or "\irefmissing" 
% commands to make an internal reference to another entry, 
% as explained within the "history" environment above.

% You do not need to create the "References" section yourself. 
% This is done automatically.




\end{entry}
