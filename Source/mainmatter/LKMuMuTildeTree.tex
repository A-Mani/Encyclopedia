
\calculusName{\LKmumutildetree} 
\calculusAcronym{} 
\calculusLogic{Classical Logic} 
\calculusLogicOrder{First-Order}
\calculusType{Sequent Calculus} 
\calculusYear{2005} 
\calculusAuthor{Herbelin} 
\entryTitle{LK$\mu\tilde\mu$ in sequent-free tree form} 
\entryAuthor{Hugo Herbelin}  
% If you wish, use tags to give any other information
% that might be helpful for classifying and grouping this entry:
% e.g. \tag{Multiset Cedents}
% You are free to invent your own tags.
% The Encyclopedia's coordinator will take care of
% merging semantically similar tags in the future.
\maketitle
% If your files are called "MyProofSystem.tex" and "MyProofSystem.bib",
% then you should write "\begin{entry}{MyProofSystem}" in the line below
\begin{entry}{LKMuMuTildeTree}

\begin{calculus}
% Add the inference rules of your proof system here.
% The "proof.sty" and "bussproofs.sty" packages are available.
% If you need any other package, please contact the editor (bruno@logic.at)

{\sc Structural subsystem}
\[
\infer[\mathit{Cut}]
      {\vdash}
      {\vdash A & A \vdash}
\]
\[
\infer[\mathit{Focus}_L]
      {A \vdash}
      {\infer*{\vdash}{[\vdash A]}}
\qquad
\infer[\mathit{Focus}_R]
      {\vdash A}
      {\infer*{\vdash}{[A \vdash]}}
\]
{\sc Introduction rules}
\[
\infer[\wedge_L^i]
      {A_1 \wedge A_2 \vdash}
      {A_i \vdash}
\qquad
\infer[\wedge_R]
      {\vdash A_1 \wedge A_2}
      {\vdash A_1 & \vdash A_2}
\]
\[
\infer[\vee_L]
      { A_1 \vee A_2 \vdash}
      { A_1 \vdash & A_2 \vdash}
\qquad
\infer[\vee_R^i]
      {\vdash A_1 \vee A_2 }
      {\vdash A_i }
\]
\[
\infer[\rightarrow_L]
      {A\rightarrow B \vdash}
      {\vdash A & B \vdash}
\qquad
\infer[\rightarrow_R]
      {\vdash A \rightarrow B}
      {\infer*{\vdash B}{[\vdash A]}}
\]
\[
\infer[\exists_L]
      { \exists x\,A[x] \vdash}
      { A[y] \vdash}
\qquad
\infer[\exists_R]
      {\vdash \exists x\,A[x] }
      {\vdash A[t] }
\]
\[
\infer[\forall_L]
      {\forall x\, A[x] \vdash}
      {A[t] \vdash}
\qquad
\infer[\forall_R]
      {\vdash \forall x\,A[x]}
      {\vdash A[y]}
\]
\[
\infer[\bot_L]
      {\bot \vdash}
      {}
\qquad
\infer[\top_R]
      {\vdash \top}
      {}
\]

\end{calculus}
% The following sections ("clarifications", "history",
% "technicalities") are optional. If you use them,
% be very concise and objective. Nevertheless, do write full sentences.
% Try to have at most one paragraph per section, because line breaks
% do not look nice in a short entry.
\begin{clarifications}
There are three kinds of nodes, $\vdash A$ for asserting formulas, $A
\vdash$ for refuting formulas, and $\vdash$ for expressing a
contradiction. Negation $\neg A$ can be defined as $A \rightarrow \bot$.
In the rules $\exists_E$ and $\forall_R$, $y$ is assumed fresh in all
the unbracketed assumption formula upon which that the derivation of
$A(y)$ depends.
\end{clarifications}

\begin{history}
The purpose of this system is to show that the original distinction in
Gentzen~\cite{Gentzen1935} between natural deduction presented as a
tree of formulas and sequent calculus presented as a tree of sequents
is no longer relevant. It is known from at least
Howard~\cite{Howard80} that natural deduction can be presented with
sequents. The above formulation shows that systems based on left and
right introductions (``sequent-calculus style'') can be presented as a
sequent-free tree of formulas~\cite{HerbelinHdR}.

The terminology ``sequent calculus'' seems to have become popular
from~\cite{Prawitz65} followed then e.g. by~\cite{Troelstra73} who
were associating the term ``sequents'' to Gentzen's LJ and LK systems.
%In works prior to Prawitz, wording such as Gentzen-type calculi can be
%found~\cite{Kleene52}.
The terminology having lost the connection to its
etymology, this motivated some authors to use alternative
terminologies such as ``L'' systems~\cite{Munch-Maccagnoni09}.
\end{history}

% Use "\irefmissing{SuggestedIDForOtherProofSystem}" to refer to
% another proof system that is not yet available in the encyclopedia.
% \end{history}
\begin{technicalities}
%The system is obviously logically equivalent to Gentzen's LK when
%equipped with the appropriate connectives.
As pointed out e.g. in~\cite{GeuversPhD} in the context of natural
deduction, to obtain a computationally non-degenerate proof-as-program
correspondence with a presentation of a calculus as a tree of
formulas, the bracketed assumptions have to be annotated with the
exact occurrence of the rule which bracketed them. Then, annotation by
proof-terms can optionally be added as in \iref{LKMuMuTilde}.
\end{technicalities}
% General Instructions:
% =====================
% The preferred length of an entry is 1 page.
% Do the best you can to fit your proof system in one page.
%
% If you are finding it hard to fit what you want in one page, remember:
%
% * Your entry needs to be neither self-contained nor fully understandable
% (the interested reader may consult the cited full paper for details)
%
% * If you are describing several proof systems in one entry,
% consider splitting your entry.
%
% * You may reduce the size of your entry by ommitting inference rules
% that are already described in other entries.
%
% * Cite parsimoniously (see detailed citation instructions below).
%
%
% If you do not manage to fit everything in one page,
% it is acceptable for an entry to have 2 pages.
%
% For aesthetical reasons, it is preferable for an entry to have
% 1 full page or 2 full pages, in order to avoid unused blank space.
% Citation Instructions:
% ======================
% Please cite the original paper where the proof system was defined.
% To do so, you may use the \cite command within
% one of the optional environments above,
% or use the \nocite command otherwise.
% You may also cite a modern paper or book where the
% proof system is explained in greater depth or clarity.
% Cite parsimoniously.
% Do not cite related work. Instead, use the "\iref" or "\irefmissing"
% commands to make an internal reference to another entry,
% as explained within the "history" environment above.
% You do not need to create the "References" section yourself.
% This is done automatically.


\end{entry}
