
% If the calculus has an acronym, define it.
% (e.g. \newcommand{\LK}{\ensuremath{\mathbf{LK}}\xspace})

\newcommand{\Gthree}{\ensuremath{\mathbf{G3}}\xspace}

\calculusName{Kleene's Classical $\Gthree$}   % The name of the calculus
\calculusAcronym{\Gthree}    % The acronym if defined above, or empty otherwise. 
\calculusLogic{Classical Logic}  % Specify the logic (e.g. classical, intuitionistic, ...) for which this calculus is intended.
\calculusLogicOrder{First-Order}
\calculusType{Sequent Calculus}   % Specify the calculus type (e.g. Frege-Hilbert style, tableau, sequent calculus, hypersequent calculus, natural deduction, ...)
\calculusYear{1952}   % The year when the calculus was invented.
\calculusAuthor{Stephen Cole Kleene} % The name(s) of the author(s) of the calculus.


\entryTitle{Kleene's Classical $\Gthree$ System}     % Title of the entry (usually coincides with the name of the calculus).
\entryAuthor{Bj{\"o}rn Lellmann, Valeria de Paiva}    % Your name(s). Separate multiple names with "\and".


% If you wish, use tags to give any other information 
% that might be helpful for classifying and grouping this entry:
% e.g. \tag{Two-Sided Sequents}
% e.g. \tag{Multiset Cedents}
% e.g. \tag{List Cedents}
% You are free to invent your own tags. 
% The Encyclopedia's coordinator will take care of 
% merging semantically similar tags in the future.

\tag{Two-Sided Sequents}
\tag{Symmetric Sequents}
\tag{Multi-succedent Sequents}

\maketitle


% If your files are called "MyProofSystem.tex" and "MyProofSystem.bib", 
% then you should write "\begin{entry}{MyProofSystem}" in the line below
\begin{entry}{KleeneG3classical}  

% Define here any newcommands you may need:
% e.g. \newcommand{\necessarily}{\Box}
% e.g. \newcommand{\possibly}{\Diamond}


\begin{calculus}

% Add the inference rules of your proof system here.
% The "proof.sty" and "bussproofs.sty" packages are available.
% If you need any other package, please contact the editor (bruno@logic.at)
\[
\infer[]{A, \Gamma \seq \Theta,A}{}
\]
\[
\begin{array}{c@{\qquad}c}
  \infer[\to\seq]{A \to B, \Gamma \seq \Theta}{A \to B, \Gamma \seq \Theta, A
  & B, A \to B, \Gamma \seq \Theta} &
  \infer[\seq\to]{\Gamma \seq \Theta, A \to B}{A,\Gamma \seq
                                      \Theta,A\to B, B}\medskip\\
  \infer[\lor\seq]{A \lor B, \Gamma\seq \Theta}{A, A \lor B, \Gamma
  \seq \Theta & B, A \lor B, \Gamma \seq \Theta}
  &
  \infer[\seq\lor_1]{\Gamma \seq \Theta, A \lor B}{\Gamma \seq \Theta, A \lor B,
  A} \quad   \infer[\seq\lor_2]{\Gamma \seq \Theta, A \lor B}{\Gamma \seq \Theta, A \lor B,
      B}\medskip\medskip\\
  \infer[\land\seq_1]{A \land B, \Gamma \seq \Theta}{A, A \land B,
  \Gamma \seq \Theta}
  \quad
  \infer[\land\seq_2]{A \land B, \Gamma \seq \Theta}{B, A \land B,
  \Gamma \seq \Theta}
  &
  \infer[\seq\land]{\Gamma \seq \Theta, A \land B}{\Gamma \seq \Theta,
    A \land B, A & \Gamma \seq \Theta, A \land B, B}\medskip\\
  \infer[\neg\seq]{\neg A, \Gamma\seq \Theta}{\neg A, \Gamma \seq
  \Theta, A}
  &
  \infer[\seq\neg]{\Gamma \seq \Theta, \neg A}{A, \Gamma \seq \Theta,
    \neg A}\medskip\\
  \infer[\forall\seq]{\forall x A(x),\Gamma\seq \Theta}{A(t), \forall
  x A(x), \Gamma \seq \Theta}
  &
  \infer[\seq\forall]{\Gamma \seq \Theta, \forall x A(x)}{\Gamma \seq
    \Theta, \forall x A(x), A(b)}\medskip\\
  \infer[\exists\seq]{\exists x A(x), \Gamma \seq \Theta}{A(b), \exists x
  A(x), \Gamma \seq \Theta}
  &
  \infer[\seq\exists]{\Gamma \seq \Theta, \exists x A(x)}{\Gamma \seq
    \Theta, \exists x A(x), A(t)}
\end{array}
\]
\begin{center}
The term $t$ is free for $x$ in $A(x)$.\\
The variable $b$ is free for $x$ in $A(x)$ and (unless $b$ is $x$)
does not occur in $\Gamma,\Theta,A(x)$.
\end{center}
\end{calculus}

% The following sections ("clarifications", "history", 
% "technicalities") are optional. If you use them, 
% be very concise and objective. Nevertheless, do write full sentences. 
% Try to have at most one paragraph per section, because line breaks 
% do not look nice in a short entry.

\begin{clarifications}
% ToDo: write here short remarks that may help the reader to understand 
% the inference rules of the proof system.
  The $A,B$ are formulae; $\Gamma,\Theta$ are finite (possibly empty)
  sequences of formulae; $x$ is a variable; $A(x)$ is a formula. In
  applications of the rules every sequent $\Gamma \seq \Theta$ can be
  replaced with a \emph{cognate} one, i.e., a sequent
  $\Gamma' \seq \Theta'$ such that the sets of formulae occurring in
  $\Gamma$ and $\Gamma'$ resp.\ $\Theta$ and $\Theta'$ are the same.
\end{clarifications}

\begin{history}
% ToDo: write here short historical remarks about this proof system,
% especially if they relate to other proof systems. 
% Use "\iref{OtherProofSystem}" to refer to another proof system 
% in the Encyclopedia (where "OtherProofSystem" is its ID). 
% Use "\irefmissing{SuggestedIDForOtherProofSystem}" to refer to 
% another proof system that is not yet available in the encyclopedia.
  Kleene's systems introduced in his~1952 monograph were the staple of
  generations of logicians, who learned about sequent calculus from
  his textbooks~\cite{Kleene:1952} and~\cite{Kleene:1967}.
\end{history}

\begin{technicalities}
% ToDo: write here remarks about soundness, completeness, decidability...
  Based on Gentzen's sequent calculus $\LK$~\iref{GentzenLK} (called
  classical $\mathbf{G1}$ in~\cite{Kleene:1952}). Seems to be the
  first system (with~\iref{KleeneG3intuitionistic}) in which
  admissibility of contraction is obtained by copying the principal
  formulae into the premisses (accordingly, this is sometimes called
  \emph{Kleene's Method}). Used together with its single-conclusion
  version for intuitionistic logic~\iref{KleeneG3intuitionistic} to
  uniformly obtain decidability of propositional classical and
  intuitionistic logic via backwards proof search
  in~\cite{Kleene:1952}.
\end{technicalities}


% General Instructions:
% =====================

% The preferred length of an entry is 1 page. 
% Do the best you can to fit your proof system in one page.
%
% If you are finding it hard to fit what you want in one page, remember:
%
%   * Your entry needs to be neither self-contained nor fully understandable
%     (the interested reader may consult the cited full paper for details)
%
%   * If you are describing several proof systems in one entry, 
%     consider splitting your entry.
%
%   * You may reduce the size of your entry by ommitting inference rules
%     that are already described in other entries.
%
%   * Cite parsimoniously (see detailed citation instructions below).
%
% 
% If you do not manage to fit everything in one page, 
% it is acceptable for an entry to have 2 pages.
%
% For aesthetical reasons, it is preferable for an entry to have
% 1 full page or 2 full pages, in order to avoid unused blank space.



% Citation Instructions:
% ======================

% Please cite the original paper where the proof system was defined.
% To do so, you may use the \cite command within 
% one of the optional environments above,
% or use the \nocite command otherwise.

% You may also cite a modern paper or book where the 
% proof system is explained in greater depth or clarity.
% Cite parsimoniously.

% Do not cite related work. Instead, use the "\iref" or "\irefmissing" 
% commands to make an internal reference to another entry, 
% as explained within the "history" environment above.

% You do not need to create the "References" section yourself. 
% This is done automatically.


\nocite{Kleene:1952}

% Leave an empty line above "\end{entry}".

\end{entry}
