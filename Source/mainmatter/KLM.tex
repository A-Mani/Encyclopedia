

\calculusName{Preferential Tableau Calculi \calcoloP}   
\calculusAcronym{\calcoloP}     
\calculusLogic{Preferential Logics} 
%\calculusLogic{KLM Preferential Logics} 
\calculusLogicOrder{Propositional}
\calculusType{Tableau}   
\calculusYear{2005-2009}   
\calculusAuthor{Laura Giordano} \calculusAuthor{Valentina Gliozzi} \calculusAuthor{Nicola Olivetti} \calculusAuthor{Gian Luca Pozzato} 


\entryTitle{Preferential Tableau Calculi \calcoloP}     
\entryAuthor{Laura Giordano} \entryAuthor{Valentina Gliozzi} \entryAuthor{Nicola Olivetti} \entryAuthor{Gian Luca Pozzato}     





\maketitle



\begin{entry}{KLM}  


\newcommand {\bbox} {\square}
\newcommand {\ent} {\mathrel{{\scriptstyle\mid\!\sim}}}
\newcommand {\freccia} {\downarrow}


\begin{calculus}
\begin{footnotesize}
\[
\begin{array}{clcrc}
\quad &
\Gamma, P, \lnot P \ ({\bf AX}) \quad  {\tiny \mbox{with} \ P \ \mbox{atomic}}
& \quad &
\begin{prooftree}
   \Gamma, \lnot \bbox \lnot A; \Sigma
   \justifies 
   A, \bbox \lnot A, \Gamma^\bbox, \Gamma^{\bbox^{\freccia}}, \Gamma^{\ent\pm}, \Sigma; \emptyset \using (\bbox^-)
\end{prooftree}
& 
\quad  \\ \\ 
&
\begin{prooftree}
   \Gamma, \lnot (A \ent B); \Sigma
   \justifies A, \bbox \lnot A, \lnot B, \Gamma^{\ent\pm}; \emptyset \using (\ent^-)
\end{prooftree}
& &
\begin{prooftree}
    \Gamma, A \ent B; \Sigma
    \justifies
    \Gamma, \lnot A; \Sigma, A \ent B
    \qquad
    \Gamma, \lnot \bbox \lnot A; \Sigma, A \ent B
   \qquad
    \Gamma, B; \Sigma, A \ent B   \using (\ent^+)
\end{prooftree}
&
\end{array}
\]
\end{footnotesize}
\end{calculus}



 \begin{clarifications}
% ToDo: write here short remarks that may help the reader to understand 
% the inference rules of the proof system.
  According to Kraus, Lehmann and
Magidor (KLM) \cite{KrausLehmannMagidor:90}, defeasible knowledge is
represented by a  (finite) set of nonmonotonic conditionals 
$A \ent B$ (normally the $A$'s
are $B$'s). 
Models are possible-world structures
equipped with a preference relation (irreflexive and transitive for {\bf P}) among worlds or states.  The meaning of  $A \ent B$ is that
$B$ holds in the worlds/states where $A$
holds and that are \emph{minimal} with respect to the preference relation.

The calculus \calcoloP \ is based on the idea of interpreting the preference relation as an
accessibility relation: a conditional $A \ent B$ holds in a model
if  $B$ is true in all minimal $A$-worlds, where a world $w$ is an $A$-world if it satisfies $A$, and it is a minimal $A$-world if there is no $A$-world $w'$ preferred to $w$.  

Nodes are pairs $\Gamma;
\Sigma$, where $\Gamma$ is a set of formulas and $\Sigma$ is a set
of conditional formulas $A \ent B$. $\Sigma$ is used to keep
track of positive conditionals $A \ent B$ to which the rule
$(\ent^{+})$ has already been applied: the idea
is that one does not need to apply $(\ent^{+})$ on the same
conditional formula $A \ent B$ \emph{more than once in the same
world}. When $(\ent^{+})$ is applied to a formula $A \ent B \in \Gamma$, then $A \ent B$ is moved from $\Gamma$ to $\Sigma$ in the conclusions of the rule, so that it is no longer available for further applications in the current world.
The dynamic rules re-introduce formulas from $\Sigma$ to $\Gamma$ in order to
allow further applications of $(\ent^{+})$ in new  worlds.

Given $\Gamma$, we define:
\begin{itemize}
   \item $\Gamma^{\bbox}=\{ \bbox \lnot A \mid \bbox \lnot A \in \Gamma
    \}$
    \item $\Gamma^{\bbox^{\freccia}}=\{ \lnot A \mid \bbox \lnot A \in \Gamma \}$
  \item     $\Gamma^{\ent^{+}}=\{A \ent B \mid A \ent B \in
    \Gamma\}$
  \item     $\Gamma^{\ent^{-}}=\{\lnot(A \ent B) \mid \lnot(A \ent B) \in
    \Gamma\}$
  \item     $\Gamma^{\ent\pm}=\Gamma^{\ent^{+}} \cup \Gamma^{\ent^{-}}$
  \end{itemize}

Besides the rules shown above, the calculus \calcoloP \ also includes standard 
   rules for propositional connectives.
\end{clarifications}

 \begin{history}
 In \cite{KrausLehmannMagidor:90} Kraus, Lehmann and
Magidor proposed a formalization of nonmonotonic
reasoning that was early recognized as a landmark. According to their framework, defeasible knowledge is
represented by a  (finite) set of nonmonotonic conditionals or
assertions of the form
$A \ent B$, whose reading is \emph{normally (or typically) the $A$'s
are $B$'s}. The operator ``$\ent$'' is nonmonotonic, in the sense
that $A \ent B$ does not imply $A \land C \ent B$. 
 
  The calculus \calcoloP \ and extensions for all the logics of the KLM family are proposed in \cite{toclKLMpozz}. The theorem provers KLMLean and FreeP implementing the tableau calculi have been presented at \cite{tableaux2007pozz,aiia207pozz}.
 \end{history}

 \begin{technicalities}
The calculus \calcoloP \ can be used to define a decision procedure and
obtain  a complexity bound for the preferential logic {\bf P}, namely that it is
 {\bf coNP}-complete.
 \end{technicalities}













\end{entry}
