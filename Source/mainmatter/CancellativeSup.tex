
\calculusName{Cancellative Superposition}   % The name of the calculus
\calculusAcronym{}    % The acronym if defined above, or empty otherwise. 
\calculusLogic{Classical Logic}  % Specify the logic (e.g. classical, intuitionistic, ...) for which this calculus is intended.
\calculusLogicOrder{First-Order}
\calculusType{Superposition}   % Specify the calculus type (e.g. Frege-Hilbert style, tableau, sequent calculus, hypersequent calculus, natural deduction, ...)
\calculusYear{1996}   % The year when the calculus was invented.
\calculusAuthor{Harald Ganzinger, Uwe Waldmann} % The name(s) of the author(s) of the calculus.


\entryTitle{Cancellative Superposition}     % Title of the entry (usually coincides with the name of the calculus).
\entryAuthor{Uwe Waldmann}    % Your name(s). Separate multiple names with "\and".


% If you wish, use tags to give any other information 
% that might be helpful for classifying and grouping this entry:
% e.g. \tag{Two-Sided Sequents}
% e.g. \tag{Multiset Cedents}
% e.g. \tag{List Cedents}
% You are free to invent your own tags. 
% The Encyclopedia's coordinator will take care of 
% merging semantically similar tags in the future.


\maketitle


% If your files are called "MyProofSystem.tex" and "MyProofSystem.bib", 
% then you should write "\begin{entry}{MyProofSystem}" in the line below
\begin{entry}{CancellativeSup}

% Define here any newcommands you may need:
% e.g. \newcommand{\necessarily}{\Box}
% e.g. \newcommand{\possibly}{\Diamond}


\begin{calculus}

% Add the inference rules of your proof system here.
% The "proof.sty" and "bussproofs.sty" packages are available.
% If you need any other package, please contact the editor (bruno@logic.at)

Cancellative rules (for simplicity, the ground versions
are given; the non-ground rules are obtained by lifting):
\[
\infer[\textit{Equality Resolution}]
{C}{C \lor \neg t \approx t}
\]
\[
\infer[\textit{Cancellation}]
{C \lor [\neg] (n{-}m)u + t \approx s}
{C \lor [\neg] nu + t \approx mu + s}
\]
\[
\infer[\textit{Cancellative Superposition}]
{D \lor C \lor [\neg] (n{-}m)u + t + s' \approx t' + s}
{D \lor mu + s \approx s' & C \lor [\neg] nu + t \approx t'}
\]
\[
\infer[\textit{Cancellative Equality Factoring}]
{C \lor \neg s + t' \approx s' + t \lor nu + t \approx t'}
{C \lor nu + s \approx s' \lor nu + t \approx t'}
\]
plus, if there are any non-constant function symbols besides $+$,
the rules of the standard superposition calculus~\iref{Superposition}
and
\[
\infer[\textit{Abstraction}]
{C \lor \neg x \approx nu + t \lor [\neg] w[x] \approx w'}
{C \lor [\neg] w[nu + t] \approx w'}
\]
$C,D$ are (possibly empty) equational clauses,
$s,s',t,t'$ are terms,
$u$ is an atomic term,
$n,m$ are positive integers.
Every literal involved in some inference
is maximal in the respective premise
(except for the last but one literal in
\textit{Equality Factoring} inferences).
A positive literal involved in a
\textit{Superposition} inference
is strictly maximal in the respective clause.
In every literal involved in a cancellative inference
(except \textit{Equality Resolution}),
the term $u$ is the maximal atomic term.

\end{calculus}

% The following sections ("clarifications", "history", 
% "technicalities") are optional. If you use them, 
% be very concise and objective. Nevertheless, do write full sentences. 
% Try to have at most one paragraph per section, because line breaks 
% do not look nice in a short entry.

\begin{clarifications}
Cancellative superposition is a refutational saturation calculus for
first-order clauses containing the axioms of
cancellative abelian monoids or abelian groups.
The inference rules are supplemented by a redundancy criterion
that permits to delete clauses that are unnecessary for
deriving a contradiction during the saturation, see \iref{SaturationWithRed}.
\end{clarifications}

\begin{history}
As a na{\"\i}ve handling of axioms like commutativity or associativity
in an automated theorem prover
leads to an explosion of the search space,
there has been a lot of interest in
incorporating specialized techniques into general proof systems
to work efficiently within standard algebraic theories.
The cancellative superposition calculus~\cite{GanzingerWaldmann1996CADE}
shown above
is one example of a saturation calculus with a built-in algebraic theory.
By using dedicated inference rules,
explicit inferences with the theory axioms become superfluous;
moreover variable elimination techniques
and strengthened ordering restrictions and redundancy criteria
lead to a significant reduction of the search space.
The cancellative superposition calculus is refutationally complete for
first-order logic modulo cancellative abelian monoids.

Other examples for ``white-box'' theory integration include
calculi for dealing with associativity and
commutativity~\cite{Plotkin1972,Slagle1974JACM,RusinowitchVigneron1995,BachmairGanzinger1994CTRS},
% \cite{PetersonStickel1981}
% \cite{Paul1992}
superposition modulo abelian groups~\cite{GodoyNieuwenhuis2004},
chaining calculi~\cite{Slagle1972JACM,Hines1992JAR,BachmairGanzinger1994LICS,BachmairGanzinger1994CADE},
or superposition modulo
divisible torsion-free abelian groups
or ordered divisible abelian groups~\cite{Waldmann2002abJSC,Waldmann2001IJCAR}.

\end{history}

% \begin{technicalities}
% \end{technicalities}


% General Instructions:
% =====================

% The preferred length of an entry is 1 page. 
% Do the best you can to fit your proof system in one page.
%
% If you are finding it hard to fit what you want in one page, remember:
%
%   * Your entry needs to be neither self-contained nor fully understandable
%     (the interested reader may consult the cited full paper for details)
%
%   * If you are describing several proof systems in one entry, 
%     consider splitting your entry.
%
%   * You may reduce the size of your entry by ommitting inference rules
%     that are already described in other entries.
%
%   * Cite parsimoniously (see detailed citation instructions below).
%
% 
% If you do not manage to fit everything in one page, 
% it is acceptable for an entry to have 2 pages.
%
% For aesthetical reasons, it is preferable for an entry to have
% 1 full page or 2 full pages, in order to avoid unused blank space.



% Citation Instructions:
% ======================

% Please cite the original paper where the proof system was defined.
% To do so, you may use the \cite command within 
% one of the optional environments above,
% or use the \nocite command otherwise.

% You may also cite a modern paper or book where the 
% proof system is explained in greater depth or clarity.
% Cite parsimoniously.

% Do not cite related work. Instead, use the "\iref" or "\irefmissing" 
% commands to make an internal reference to another entry, 
% as explained within the "history" environment above.

% You do not need to create the "References" section yourself. 
% This is done automatically.




% Leave an empty line above "\end{entry}".

\end{entry}
