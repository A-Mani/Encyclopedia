
% If the calculus has an acronym, define it.
\newcommand{\Eres}{\mathbf{E}^\mathsf{CPL}_{res}}

\calculusName{Erotetic Dual Resolution for Classical Propositional Logic}   
\calculusAcronym{\Eres}     
\calculusLogic{Classical Logic}  
\calculusLogicOrder{Propositional}
\calculusType{Resolution}   
\calculusYear{2014}   

\calculusAuthor{Szymon Chlebowski}
\calculusAuthor{Dorota Leszczy\'nska-Jasion} 


\entryTitle{Erotetic Dual Resolution for Classical Propositional Logic}     
\entryAuthor{Dorota Leszczy\'nska-Jasion}     




\tag{resolution}
\tag{non-clausal resolution}
\tag{dual resolution}

\maketitle



\begin{entry}{EDResolutionCPL}  




\begin{calculus}

% Add the inference rules of your proof system here.
% The "proof.sty" and "bussproofs.sty" packages are available.
% If you need any other package, please contact the editor (bruno@logic.at)

$$\infer[\textbf{R}_{\beta}]
{?(\Phi\:;\:\dashv S\:'\:\beta_{1}\:'\:T\:;\:\dashv S\:'\:\beta_{2}\:'\:T\:;\:\Psi)}
{?(\Phi\:;\:\dashv S\:'\:\beta\:'\:T\:;\:\Psi)} $$

\vspace{-0.3cm}

$$
\quad
\infer[\textbf{R}_{\alpha}]
{?(\Phi\:;\:\dashv S\:'\:\alpha_{1}\:'\:\alpha_{2}\:'\:T\:;\:\Psi)}
{?(\Phi\:;\:\dashv   S\:'\:\alpha\:'\:T\:;\:\Psi)}
\qquad
\infer[\textbf{R}_{\neg\neg}]
{?(\Phi\:;\:\dashv S\:'\:A\:'\:T\:;\:\Psi)}
{?(\Phi\:;\:\dashv S\:'\:\neg\neg A\:'\:T\:;\:\Psi)}$$

\vspace{-0.3cm}

$$\infer[\textbf{R}_{res}]
{?(\dashv \underline{S}\:'\:\underline{T}\:'\:\underline{U}\:'\:\underline{V}\:;\:\Phi\:;\:\Psi\:;\:\Omega\:;\:\dashv S\:'\: A\:'\:T\:;\:\dashv U\:'\:\overline{A}\:'\:V)}
{?(\Phi\:;\:\dashv S\:'\: A\:'\:T\:;\:\Psi\:;\:\dashv U\:'\:\overline{A}\:'\:V\:;\:\Omega )}$$

In $\mathbf{R}_{res}$: $A$ and $\overline{A}$ must be complementary, that is either $A = \lnot \overline{A}$ or $\overline{A} = \lnot A$. $\underline{S}$, $\underline{T}$ are obtained from $S$, $T$ by deleting all occurrences of $A$ and $\underline{U}$, $\underline{V}$ are obtained from $U$, $V$ by deleting all occurrences of $\lnot A$. For the $\alpha$, $\beta$ notation see entry \iref{SocraticProofsCPL}.
\end{calculus}



\begin{clarifications}
The calculus is built in the framework of Inferential Erotetic Logic (\cite{AW:2013}) and is designed to transform questions concerning refutability (falsifiability) of a formula. However, the rules act upon \textit{reversed sequents}. The sequents are right-sided with sequences of formulas in the succedent. $\Phi$, $\Psi$ are finite (possibly empty) sequences of sequents. $S$, $T$ are finite (possibly empty) sequences of formulas. The semicolon `;' is the concatenation sign for sequences of sequents, whereas `$'$' is the concatenation sign for sequences of formulas. The resolution rule is non-clausal (all the formulas in the premise, $A$ included, may be compound) and dual with respect to the standard resolution (instead of CNF of $\lnot A$, DNF of $A$ is derived). A Socratic refutation of a sequent of the form `$\dashv A$' is a kind of a resolution refutation of $\lnot A$ and it ends when the empty sequent (a counterpart of the empty clause) is arrived at.
\end{clarifications}

\begin{history}
The calculus $\Eres$ has been presented in \cite{SzChDLJ:LFI} together with extensions to some paraconsistent logics (see \iref{EDResolutionmbC}). Compare also \iref{SocraticProofsCPL}.
\end{history}

\begin{technicalities}
A formula $A$ is $\mathsf{CPL}$-valid iff $\dashv A$ has a Socratic refutation in $\Eres$. Similar results are obtained with respect to $\mathsf{CLuN}$, $\mathsf{CLuNs}$ and $\mathsf{mbC}$.
\end{technicalities}













\end{entry}
