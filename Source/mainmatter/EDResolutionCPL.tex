
% If the calculus has an acronym, define it.
\newcommand{\Eres}{\mathbf{E}^\mathsf{CPL}_{res}}

\calculusName{Erotetic Dual Resolution for Classical Propositional Logic}   % The name of the calculus
\calculusAcronym{\Eres}    % The acronym if defined above, or empty otherwise. 
\calculusLogic{Classical Logic}  % Specify the logic (e.g. classical, intuitionistic, ...) for which this calculus is intended.
\calculusLogicOrder{Propositional}
\calculusType{Resolution}   % Specify the calculus type (e.g. Frege-Hilbert style, tableau, sequent calculus, hypersequent calculus, natural deduction, ...)
\calculusYear{2014}   % The year when the calculus was invented.
\calculusAuthor{Szymon Chlebowski, Dorota Leszczy\'nska-Jasion} % The name(s) of the author(s) of the calculus.


\entryTitle{Erotetic Dual Resolution for Classical Propositional Logic}     % Title of the entry (usually coincides with the name of the calculus).
\entryAuthor{Dorota Leszczy\'nska-Jasion}    % Your name(s). Separate multiple names with "\and".


% If you wish, use tags to give any other information 
% that might be helpful for classifying and grouping this entry:
% e.g. \tag{Two-Sided Sequents}
% e.g. \tag{Multiset Cedents}
% e.g. \tag{List Cedents}
% You are free to invent your own tags. 
% The Encyclopedia's coordinator will take care of 
% merging semantically similar tags in the future.

\tag{resolution}
\tag{non-clausal resolution}
\tag{dual resolution}

\maketitle


% If your files are called "MyProofSystem.tex" and "MyProofSystem.bib", 
% then you should write "\begin{entry}{MyProofSystem}" in the line below
\begin{entry}{EDResolutionCPL}  

% Define here any newcommands you may need:
% e.g. \newcommand{\necessarily}{\Box}
% e.g. \newcommand{\possibly}{\Diamond}


\begin{calculus}

% Add the inference rules of your proof system here.
% The "proof.sty" and "bussproofs.sty" packages are available.
% If you need any other package, please contact the editor (bruno@logic.at)

$$\infer[\textbf{R}_{\beta}]
{?(\Phi\:;\:\dashv S\:'\:\beta_{1}\:'\:T\:;\:\dashv S\:'\:\beta_{2}\:'\:T\:;\:\Psi)}
{?(\Phi\:;\:\dashv S\:'\:\beta\:'\:T\:;\:\Psi)} $$

\vspace{-0.3cm}

$$
\quad
\infer[\textbf{R}_{\alpha}]
{?(\Phi\:;\:\dashv S\:'\:\alpha_{1}\:'\:\alpha_{2}\:'\:T\:;\:\Psi)}
{?(\Phi\:;\:\dashv   S\:'\:\alpha\:'\:T\:;\:\Psi)}
\qquad
\infer[\textbf{R}_{\neg\neg}]
{?(\Phi\:;\:\dashv S\:'\:A\:'\:T\:;\:\Psi)}
{?(\Phi\:;\:\dashv S\:'\:\neg\neg A\:'\:T\:;\:\Psi)}$$

\vspace{-0.3cm}

$$\infer[\textbf{R}_{res}]
{?(\dashv \underline{S}\:'\:\underline{T}\:'\:\underline{U}\:'\:\underline{V}\:;\:\Phi\:;\:\Psi\:;\:\Omega\:;\:\dashv S\:'\: A\:'\:T\:;\:\dashv U\:'\:\overline{A}\:'\:V)}
{?(\Phi\:;\:\dashv S\:'\: A\:'\:T\:;\:\Psi\:;\:\dashv U\:'\:\overline{A}\:'\:V\:;\:\Omega )}$$

In $\mathbf{R}_{res}$: $A$ and $\overline{A}$ must be complementary, that is either $A = \lnot \overline{A}$ or $\overline{A} = \lnot A$. $\underline{S}$, $\underline{T}$ are obtained from $S$, $T$ by deleting all occurrences of $A$ and $\underline{U}$, $\underline{V}$ are obtained from $U$, $V$ by deleting all occurrences of $\lnot A$. For the $\alpha$, $\beta$ notation see entry \iref{SocraticProofsCPL}.
\end{calculus}

% The following sections ("clarifications", "history", 
% "technicalities") are optional. If you use them, 
% be very concise and objective. Nevertheless, do write full sentences. 
% Try to have at most one paragraph per section, because line breaks 
% do not look nice in a short entry.

\begin{clarifications}
The calculus is built in the framework of Inferential Erotetic Logic (\cite{AW:2013}) and is designed to transform questions concerning refutability (falsifiability) of a formula. However, the rules act upon \textit{reversed sequents}. The sequents are right-sided with sequences of formulas in the succedent. $\Phi$, $\Psi$ are finite (possibly empty) sequences of sequents. $S$, $T$ are finite (possibly empty) sequences of formulas. The semicolon `;' is the concatenation sign for sequences of sequents, whereas `$'$' is the concatenation sign for sequences of formulas. The resolution rule is non-clausal (all the formulas in the premise, $A$ included, may be compound) and dual with respect to the standard resolution (instead of CNF of $\lnot A$, DNF of $A$ is derived). A Socratic refutation of a sequent of the form `$\dashv A$' is a kind of a resolution refutation of $\lnot A$ and it ends when the empty sequent (a counterpart of the empty clause) is arrived at.
\end{clarifications}

\begin{history}
The calculus $\Eres$ has been presented in \cite{SzChDLJ:LFI} together with extensions to some paraconsistent logics (see \iref{EDResolutionmbC}). Compare also \iref{SocraticProofsCPL}.
\end{history}

\begin{technicalities}
A formula $A$ is $\mathsf{CPL}$-valid iff $\dashv A$ has a Socratic refutation in $\Eres$. Similar results are obtained with respect to $\mathsf{CLuN}$, $\mathsf{CLuNs}$ and $\mathsf{mbC}$.
\end{technicalities}


% General Instructions:
% =====================

% The preferred length of an entry is 1 page. 
% Do the best you can to fit your proof system in one page.
%
% If you are finding it hard to fit what you want in one page, remember:
%
%   * Your entry needs to be neither self-contained nor fully understandable
%     (the interested reader may consult the cited full paper for details)
%
%   * If you are describing several proof systems in one entry, 
%     consider splitting your entry.
%
%   * You may reduce the size of your entry by ommitting inference rules
%     that are already described in other entries.
%
%   * Cite parsimoniously (see detailed citation instructions below).
%
% 
% If you do not manage to fit everything in one page, 
% it is acceptable for an entry to have 2 pages.
%
% For aesthetical reasons, it is preferable for an entry to have
% 1 full page or 2 full pages, in order to avoid unused blank space.



% Citation Instructions:
% ======================

% Please cite the original paper where the proof system was defined.
% To do so, you may use the \cite command within 
% one of the optional environments above,
% or use the \nocite command otherwise.

% You may also cite a modern paper or book where the 
% proof system is explained in greater depth or clarity.
% Cite parsimoniously.

% Do not cite related work. Instead, use the "\iref" or "\irefmissing" 
% commands to make an internal reference to another entry, 
% as explained within the "history" environment above.

% You do not need to create the "References" section yourself. 
% This is done automatically.




% Leave an empty line above "\end{entry}".

\end{entry}
