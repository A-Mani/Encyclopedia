
\calculusName{Conditional Labelled Sequent Calculi SeqS}   % The name of the calculus
\calculusAcronym{SeqS}    % The acronym if defined above, or empty otherwise. 
\calculusLogic{Conditional Logics}  % Specify the logic (e.g. classical, intuitionistic, ...) for which this calculus is intended.
\calculusLogicOrder{Propositional}
\calculusType{Sequent Calculus}   % Specify the calculus type (e.g. Frege-Hilbert style, tableau, sequent calculus, hypersequent calculus, natural deduction, ...)
\calculusYear{2003-2007}   % The year when the calculus was invented.
\calculusAuthor{R\'egis Alenda, Nicola Olivetti, Gian Luca Pozzato} % The name(s) of the author(s) of the calculus.


\entryTitle{Conditional Labelled Sequent Calculi SeqS}     % Title of the entry (usually coincides with the name of the calculus).
\entryAuthor{Nicola Olivetti \and Gian Luca Pozzato \and Camilla Schwind}    % Your name(s). Separate multiple names with "\and".


% If you wish, use tags to give any other information 
% that might be helpful for classifying and grouping this entry:
% e.g. \tag{Two-Sided Sequents}
% e.g. \tag{Multiset Cedents}
% e.g. \tag{List Cedents}
% You are free to invent your own tags. 
% The Encyclopedia's coordinator will take care of 
% merging semantically similar tags in the future.


\maketitle


% If your files are called "MyProofSystem.tex" and "MyProofSystem.bib", 
% then you should write "\begin{entry}{MyProofSystem}" in the line below
\begin{entry}{LabelledConditionals}  

% Define here any newcommands you may need:
% e.g. \newcommand{\necessarily}{\Box}
% e.g. \newcommand{\possibly}{\Diamond}
\newcommand {\trans}[1]{\stackrel{#1}{\longrightarrow}}
\newcommand {\prova} {\vdash}


\begin{calculus}
\begin{footnotesize}
\[
\begin{array}{clcrc}
\ &
({\bf AX}) \ \Gamma, x: P \prova \Delta, x: P \quad (P \ \mbox{atomic})
& \ &
({\bf A} \bot) \ \Gamma, x: \bot \prova \Delta & \ \\ \\
& 
\begin{prooftree}
\Gamma, x \trans{A} y \prova \Delta, y: B
\justifies \Gamma \prova \Delta, x: A \Rightarrow B \using (\Rightarrow {\bf R})  \quad (y \not\in \Gamma, \Delta)
\end{prooftree} 
& & \\ \\
&
\begin{prooftree}
\Gamma, x: A \Rightarrow B \prova x \trans{A} y, \Delta \qquad \Gamma, x: A \Rightarrow B, y: B \prova \Delta
\justifies \Gamma, x: A \Rightarrow B \prova \Delta \using (\Rightarrow {\bf L})
\end{prooftree}
& & \\ \\
& 
\begin{prooftree}
u: A \prova u: B \qquad u: B \prova u: A
\justifies \Gamma, x \trans{A} y \prova x \trans{B} y, \Delta \using ({\bf EQ})
\end{prooftree}
& & \\ \\
& 
\begin{prooftree}
\Gamma, x \trans{A} y, y: A \prova \Delta
\justifies \Gamma, x \trans{A} y \prova \Delta \using ({\bf ID})
\end{prooftree}
& &
\begin{prooftree}
\Gamma \prova  x \trans{A} x, x: A, \Delta
\justifies \Gamma \prova  x \trans{A} x, \Delta \using ({\bf MP})
\end{prooftree}
\end{array}
\] 
\[
\begin{array}{cl}
\ & 
\begin{prooftree}
\Gamma, x \trans{A} y \prova \Delta, x: A \qquad \Gamma[x/u, y/u], u \trans{A} u \prova \Delta[x/u, y/u]
\justifies \Gamma, x \trans{A} y \prova \Delta \using ({\bf CS}) \ (x \not= y, u \not\in \Gamma, \Delta)
\end{prooftree} \\ \\
& 
\begin{prooftree}
\Gamma x \trans{A} y \prova  \Delta, x \trans{A} z \qquad (\Gamma x \trans{A} y \prova  \Delta)[y/u, z/u]
\justifies \Gamma x \trans{A} y \prova  \Delta \using ({\bf CEM})  \ (y \not= z, u \not\in \Gamma, \Delta)
\end{prooftree} 
\end{array}
\]

{\scriptsize Given a sequent $\Gamma$ and labels $x$ and $u$,  $\Gamma[x/u]$ is the sequent obtained by replacing in $\Gamma$ all occurrences of $x$ with $u$.}
\end{footnotesize}
\end{calculus}

% The following sections ("clarifications", "history", 
% "technicalities") are optional. If you use them, 
% be very concise and objective. Nevertheless, do write full sentences. 
% Try to have at most one paragraph per section, because line breaks 
% do not look nice in a short entry.

 \begin{clarifications}
% ToDo: write here short remarks that may help the reader to understand 
% the inference rules of the proof system.
Conditional logics extend classical logic with formulas of the form $A \Rightarrow B$. SeqS considers the \emph{selection function} semantics:  $A \Rightarrow B$ is true in a world $w$ if $B$ is true in the set of worlds selected by the selection function $f$ for $A$ and $w$ (that are most similar to $w$). SeqS manipulate \emph{labelled} formulas, where labels represent worlds, of the form $x: A$ ($A$ is true in $x$) and $x \trans{A} y$ ($y$ belongs to $f(x,A)$).

The calculi SeqS consider \emph{normal} conditional logics, such that if $A$ and $B$ are true in the same worlds, then $f(w,A)=f(w,B)$. The rule $({\bf EQ})$ takes care of normality.

  Besides the rules shown, SeqS also include standard 
   rules for propositional connectives.
 \end{clarifications}

 \begin{history}
  The calculi SeqS have been introduced in 
  \cite{toclpozz}. The theorem prover CondLean, implementing SeqS calculi in Prolog, has been presented in \cite{tab2003pozz,tab2005pozz}.
 \end{history}

 \begin{technicalities}
Completeness is a consequence of the admissibility of cut. The calculi SeqS can be used to obtain a \textsc{PSpace} decision procedure for the respective conditional logics and to develop goal-directed proof procedures.
 \end{technicalities}


% General Instructions:
% =====================

% The preferred length of an entry is 1 page. 
% Do the best you can to fit your proof system in one page.
%
% If you are finding it hard to fit what you want in one page, remember:
%
%   * Your entry needs to be neither self-contained nor fully understandable
%     (the interested reader may consult the cited full paper for details)
%
%   * If you are describing several proof systems in one entry, 
%     consider splitting your entry.
%
%   * You may reduce the size of your entry by ommitting inference rules
%     that are already described in other entries.
%
%   * Cite parsimoniously (see detailed citation instructions below).
%
% 
% If you do not manage to fit everything in one page, 
% it is acceptable for an entry to have 2 pages.
%
% For aesthetical reasons, it is preferable for an entry to have
% 1 full page or 2 full pages, in order to avoid unused blank space.



% Citation Instructions:
% ======================

% Please cite the original paper where the proof system was defined.
% To do so, you may use the \cite command within 
% one of the optional environments above,
% or use the \nocite command otherwise.

% You may also cite a modern paper or book where the 
% proof system is explained in greater depth or clarity.
% Cite parsimoniously.

% Do not cite related work. Instead, use the "\iref" or "\irefmissing" 
% commands to make an internal reference to another entry, 
% as explained within the "history" environment above.

% You do not need to create the "References" section yourself. 
% This is done automatically.




% Leave an empty line above "\end{entry}".

\end{entry}
