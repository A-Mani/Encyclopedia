
\calculusName{Conditional Labelled Sequent Calculi SeqS}   
\calculusAcronym{SeqS}     
\calculusLogic{Conditional Logics}  
\calculusLogicOrder{Propositional}
\calculusType{Sequent Calculus}   
\calculusYear{2003-2007}   
\calculusAuthor{R\'egis Alenda}
\calculusAuthor{Nicola Olivetti}
\calculusAuthor{Gian Luca Pozzato} 


\entryTitle{Conditional Labelled Sequent Calculi SeqS}     
\entryAuthor{Nicola Olivetti} \entryAuthor{Gian Luca Pozzato} \entryAuthor{Camilla Schwind}     





\maketitle



\begin{entry}{LabelledConditionals}  


\newcommand {\trans}[1]{\stackrel{#1}{\longrightarrow}}
\newcommand {\prova} {\vdash}


\begin{calculus}
\begin{footnotesize}
\[
\begin{array}{clcrc}
\ &
({\bf AX}) \ \Gamma, x: P \prova \Delta, x: P \quad (P \ \mbox{atomic})
& \ &
({\bf A} \bot) \ \Gamma, x: \bot \prova \Delta & \ \\ \\
& 
\begin{prooftree}
\Gamma, x \trans{A} y \prova \Delta, y: B
\justifies \Gamma \prova \Delta, x: A \Rightarrow B \using (\Rightarrow {\bf R})  \quad (y \not\in \Gamma, \Delta)
\end{prooftree} 
& & \\ \\
&
\begin{prooftree}
\Gamma, x: A \Rightarrow B \prova x \trans{A} y, \Delta \qquad \Gamma, x: A \Rightarrow B, y: B \prova \Delta
\justifies \Gamma, x: A \Rightarrow B \prova \Delta \using (\Rightarrow {\bf L})
\end{prooftree}
& & \\ \\
& 
\begin{prooftree}
u: A \prova u: B \qquad u: B \prova u: A
\justifies \Gamma, x \trans{A} y \prova x \trans{B} y, \Delta \using ({\bf EQ})
\end{prooftree}
& & \\ \\
& 
\begin{prooftree}
\Gamma, x \trans{A} y, y: A \prova \Delta
\justifies \Gamma, x \trans{A} y \prova \Delta \using ({\bf ID})
\end{prooftree}
& &
\begin{prooftree}
\Gamma \prova  x \trans{A} x, x: A, \Delta
\justifies \Gamma \prova  x \trans{A} x, \Delta \using ({\bf MP})
\end{prooftree}
\end{array}
\] 
\[
\begin{array}{cl}
\ & 
\begin{prooftree}
\Gamma, x \trans{A} y \prova \Delta, x: A \qquad \Gamma[x/u, y/u], u \trans{A} u \prova \Delta[x/u, y/u]
\justifies \Gamma, x \trans{A} y \prova \Delta \using ({\bf CS}) \ (x \not= y, u \not\in \Gamma, \Delta)
\end{prooftree} \\ \\
& 
\begin{prooftree}
\Gamma x \trans{A} y \prova  \Delta, x \trans{A} z \qquad (\Gamma x \trans{A} y \prova  \Delta)[y/u, z/u]
\justifies \Gamma x \trans{A} y \prova  \Delta \using ({\bf CEM})  \ (y \not= z, u \not\in \Gamma, \Delta)
\end{prooftree} 
\end{array}
\]

{\scriptsize Given a sequent $\Gamma$ and labels $x$ and $u$,  $\Gamma[x/u]$ is the sequent obtained by replacing in $\Gamma$ all occurrences of $x$ with $u$.}
\end{footnotesize}
\end{calculus}



 \begin{clarifications}
% ToDo: write here short remarks that may help the reader to understand 
% the inference rules of the proof system.
Conditional logics extend classical logic with formulas of the form $A \Rightarrow B$. SeqS considers the \emph{selection function} semantics:  $A \Rightarrow B$ is true in a world $w$ if $B$ is true in the set of worlds selected by the selection function $f$ for $A$ and $w$ (that are most similar to $w$). SeqS manipulate \emph{labelled} formulas, where labels represent worlds, of the form $x: A$ ($A$ is true in $x$) and $x \trans{A} y$ ($y$ belongs to $f(x,A)$).

The calculi SeqS consider \emph{normal} conditional logics, such that if $A$ and $B$ are true in the same worlds, then $f(w,A)=f(w,B)$. The rule $({\bf EQ})$ takes care of normality.

  Besides the rules shown, SeqS also include standard 
   rules for propositional connectives.
 \end{clarifications}

 \begin{history}
  The calculi SeqS have been introduced in 
  \cite{toclpozz}. The theorem prover CondLean, implementing SeqS calculi in Prolog, has been presented in \cite{tab2003pozz,tab2005pozz}.
 \end{history}

 \begin{technicalities}
Completeness is a consequence of the admissibility of cut. The calculi SeqS can be used to obtain a \textsc{PSpace} decision procedure for the respective conditional logics and to develop goal-directed proof procedures.
 \end{technicalities}













\end{entry}
