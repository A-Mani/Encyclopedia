
% If the calculus has an acronym, define it.
\newcommand{\EmbC}{\mathbf{E}^\mathsf{mbC}_{res}}

\calculusName{Erotetic Dual Resolution for mbC}   % The name of the calculus
\calculusAcronym{\EmbC}    % The acronym if defined above, or empty otherwise. 
\calculusLogic{paraconsistent logics mbC, CLuN}  % Specify the logic (e.g. classical, intuitionistic, ...) for which this calculus is intended.
\calculusType{resolution, other}   % Specify the calculus type (e.g. Frege-Hilbert style, tableau, sequent calculus, hypersequent calculus, natural deduction, ...)
\calculusYear{2014}   % The year when the calculus was invented.
\calculusAuthor{Szymon Chlebowski, Dorota Leszczy\'nska-Jasion} % The name(s) of the author(s) of the calculus.


\entryTitle{Erotetic Dual Resolution for mbC}     % Title of the entry (usually coincides with the name of the calculus).
\entryAuthor{Dorota Leszczy\'nska-Jasion}    % Your name(s). Separate multiple names with "\and".


\tag{resolution}
\tag{non-clausal resolution}
\tag{dual resolution}

\maketitle


% If your files are called "MyProofSystem.tex" and "MyProofSystem.bib", 
% then you should write "\begin{entry}{MyProofSystem}" in the line below
\begin{entry}{EDResolutionmbC}  

% Define here any newcommands you may need:
% e.g. \newcommand{\necessarily}{\Box}
% e.g. \newcommand{\possibly}{\Diamond}


\begin{calculus}

% Add the inference rules of your proof system here.
% The "proof.sty" and "bussproofs.sty" packages are available.
% If you need any other package, please contact the editor (bruno@logic.at)
The rules of $\Eres$ (see \iref{EDResolutionCPL}) and the following rules (`$\lnot$' is used for the classical negation and `$\sim$' for the paraconsistent one):

\vspace{-0.3cm}

$$
\quad
\infer[\textbf{R}_{\sim }]
{?(\Phi;\:\dashv S\:'\:\neg A\:' T;\:\dashv S\:'\:\chi\sim A\:'\:T;\Psi)}
{?(\Phi;\:\dashv S\:' \sim A\:'\:T;\Psi)}
\qquad
\infer[\textbf{R}_{\neg\sim}]
{?(\Phi;\:\dashv S\:'\:A\:'\:\neg\chi\sim A\:'\:T;\Psi)}
{?(\Phi;\:\dashv S\:'\:\neg\sim A\:'\:T;\Psi)}$$

\vspace{-0.5cm}

$$\infer[\textbf{R}_{\circ}]
{?(\Phi\:;\:\dashv S\:'\:\neg A\:'\:\chi\!\circ A\:'\:T\:;\:\dashv S\:'\:\neg\sim A\:'\:\chi\!\circ A\:'\:T\:;\:\Psi)}
{?(\Phi\:;\:\dashv S\:'\:\circ A\:'\:T\:;\:\Psi)}$$

\vspace{-0.3cm}

$$\infer[\textbf{R}_{\neg\circ}]
{?(\Phi\:;\:\dashv S\:'\:A\:'\:\sim A\:'\:T\:;\:\dashv S\:'\:\neg\chi\!\circ A\:'\:T\:;\:\Psi)}
{?(\Phi\:;\:\dashv S\:'\:\neg\circ A\:'\:T\:;\:\Psi)}$$
\end{calculus}

% The following sections ("clarifications", "history", 
% "technicalities") are optional. If you use them, 
% be very concise and objective. Nevertheless, do write full sentences. 
% Try to have at most one paragraph per section, because line breaks 
% do not look nice in a short entry.

\begin{clarifications}
See \iref{EDResolutionCPL} for notational conventions.

The calculus is worded in a language being an extension of the language of $\mathsf{mbC}$, where the role of the additional $\chi$ operator is to syntactically express the fact that certain formulas (e.g. of the form `$\sim A$', `$\circ A$') may have a logical value independent of the value of $A$.
\end{clarifications}

\begin{history}
The calculus $\EmbC$ has been presented in \cite{SzChDLJ:LFI} together with similar calculi for $\mathsf{CLuN}$, $\mathsf{CLuNs}$ and for Classical Propositional Logic. The idea to use $\chi$ operator was taken from \cite{WVL:2005}, where the authors presented erotetic calculi for logics $\mathsf{CLuN}$ and $\mathsf{CLuNs}$ in a non-resolution account.
\end{history}

\begin{technicalities}
A formula $A$ is $\mathsf{mbC}$-valid iff $\dashv A$ has a Socratic refutation in $\EmbC$. A formula $A$ is $\mathsf{CLuN}$-valid iff $\dashv A$ has a Socratic refutation constructed without the use of rules $\mathbf{R}_\circ$, $\mathbf{R}_{\lnot \circ}$. Similar results are obtained with respect to $\mathsf{CLuNs}$ and the classical case.
\end{technicalities}


% General Instructions:
% =====================

% The preferred length of an entry is 1 page. 
% Do the best you can to fit your proof system in one page.
%
% If you are finding it hard to fit what you want in one page, remember:
%
%   * Your entry needs to be neither self-contained nor fully understandable
%     (the interested reader may consult the cited full paper for details)
%
%   * If you are describing several proof systems in one entry, 
%     consider splitting your entry.
%
%   * You may reduce the size of your entry by ommitting inference rules
%     that are already described in other entries.
%
%   * Cite parsimoniously (see detailed citation instructions below).
%
% 
% If you do not manage to fit everything in one page, 
% it is acceptable for an entry to have 2 pages.
%
% For aesthetical reasons, it is preferable for an entry to have
% 1 full page or 2 full pages, in order to avoid unused blank space.



% Citation Instructions:
% ======================

% Please cite the original paper where the proof system was defined.
% To do so, you may use the \cite command within 
% one of the optional environments above,
% or use the \nocite command otherwise.

% You may also cite a modern paper or book where the 
% proof system is explained in greater depth or clarity.
% Cite parsimoniously.

% Do not cite related work. Instead, use the "\iref" or "\irefmissing" 
% commands to make an internal reference to another entry, 
% as explained within the "history" environment above.

% You do not need to create the "References" section yourself. 
% This is done automatically.




% Leave an empty line above "\end{entry}".

\end{entry}
