

\calculusName{Natural Deduction}
\calculusAcronym{\NJ}
\calculusLogic{Intuitionistic Logic}
\calculusLogicOrder{First-Order}
\calculusType{Natural Deduction}
\calculusYear{1935}
\calculusAuthor{\iCA{Gerhard Karl Erich Gentzen}}

\entryTitle{Intuitionistic Natural Deduction \NJ}
\entryAuthor{Bruno Woltzenlogel Paleo}

\maketitle

\begin{entry}{GentzenNJ}

\newcommand{\A}{\mathfrak{A}}
\newcommand{\B}{\mathfrak{B}}
\newcommand{\C}{\mathfrak{C}}
\newcommand{\DD}{\mathfrak{D}}
\newcommand{\F}{\mathfrak{F}}
\newcommand{\x}{\mathfrak{x}}
\newcommand{\ma}{\mathfrak{a}}

\newcommand{\implies}{\supset}
\newcommand{\bottom}{\curlywedge}

\begin{calculus}

\[
\quad
\infer[UE]{\A \& \B}{\A & \B} 
\qquad
\infer[UB]{\A}{\A \& \B}
\qquad
\infer[UB]{\B}{\A \& \B} 
\]

\[
\qquad
\infer[OE]{\A \vee \B}{\A}
\qquad
\infer[OE]{\A \vee \B}{\B}
\qquad
\infer[OB]{\C}{\A \vee \B & 
               \infer*{\C}{[\A]} &
               \infer*{\C}{[\B]}}
\]

\[
\quad
\infer[AE]{\forall \x \F \x}{\F \ma}
\qquad
\infer[AB]{\F \ma}{\forall \x \F \x}
\qquad
\infer[EE]{\exists \x \F \x}{\F \ma}
\qquad
\infer[EB]{\C}{\exists \x \F \x & \infer*{\C}{[\F \ma]}}
\]

\[
\quad
\infer[FE]{\A \implies \B}{\infer*{B}{[A]}}
\qquad
\infer[FB]{\B}{\A & \A \implies \B}
\qquad
\infer[NE]{\neg \A}{\infer*{\bottom}{[\A]}}
\qquad
\infer[NB]{\bottom}{\A & \neg \A}
\qquad
\infer{\DD}{\bottom}
\]

The eigenvariable $\ma$ of an $AE$ must not occur in the formula 
designated in the schema by $\forall \x \F \x$; 
nor in any assumption formula upon which that formula depends. 
The eigenvariable $\ma$ of an $EB$ must not occur in the formula 
designated in the schema by $\exists \x \F \x$; 
nor in any assumption formula upon which that formula depends, 
with the exception of the assumption formulae designated by $\F \aa$.

\end{calculus}

\begin{clarifications}
The names of the rules are those originally given by Gentzen~\cite{Gentzen1935}: \\
$U$ = und (and), $O$ = oder (or), $A$ = all, $E$ = es-gibt (exists), $F$ = folgt (follows), \\ 
$N$ = nicht (not), $E$ = Einf\"uhrung (introduction), $B$ = Beseitigung (elimination).
\end{clarifications}


\begin{history}
The main novelty introduced by Gentzen in this proof system is its 
\emph{assumption} handling mechanism, which allows formal proofs to reflect 
more naturally the logical reasoning involved in mathematical proofs.
\end{history}

\newcommand{\LHJ}{\ensuremath{\mathbf{LHJ}}\xspace}

\begin{technicalities}
In \cite{Gentzen1935}, completeness of \NJ is proven by showing how to translate proofs in the Hilbert-style calculus \LHJ \irefmissing{ToDo} to \NJ-proofs, and soundness is proven by showing how to translate \NJ-proofs to $\mathbf{LJ}$-proofs \iref{GentzenLJ}.
\end{technicalities}


\end{entry}
