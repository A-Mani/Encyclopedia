\calculusName{Syntactic Higher-order Pre-unification}   % The name of the calculus
\calculusAcronym{}    % The acronym if defined above, or empty otherwise.
\calculusLogic{higher-order}  % Specify the logic (e.g. classical, intuitionistic, ...) for which this calculus is intended.
\calculusType{unification}   % Specify the calculus type (e.g. Frege-Hilbert style, tableau, sequent calculus, hypersequent calculus, natural deduction, ...)
\calculusYear{1975}   % The year when the calculus was invented.
\calculusAuthor{Gérard Pierre Huet} % The name(s) of the author(s) of the calculus.


\entryTitle{Syntactic Higher-order Pre-unification}
\entryAuthor{Tomer Libal}    % Your name(s). Separate multiple names with "\and".

\maketitle

\begin{entry}{PreUnif}


% Define here any newcommands you may need:
% e.g. \newcommand{\necessarily}{\Box}
% e.g. \newcommand{\possibly}{\Diamond}
\newcommand{\upair}[2]{\langle#1, #2\rangle}
\newcommand{\spair}[2]{\langle\langle#1, #2\rangle\rangle}

\begin{calculus}

% Add the inference rules of your proof system here.
% The "proof.sty" and "bussproofs.sty" packages are available.
% If you need any other package, please contact the editor (bruno@logic.at)

\[
\infer[delete] {S}
               {\{\upair u u\} \cup S}
\qquad
\infer[decomp] {\{\upair {v_1} {u_1} ,\ldots, \upair {v_n} {u_n}\}  \cup S}
               {\{\upair {f(v_1,\ldots,v_n)} {f(u_1,\ldots,u_n)}\} \cup S}
\qquad
\infer[varelim]{\{\spair x v\} \cup \sigma(S)}
               {\{\upair x v\} \cup S}
\]
Where $x$ does not occur in $v$ and $\sigma = [v/x]$.
\end{calculus}

% The following sections ("clarifications", "history",
% "technicalities") are optional. If you use them,
% be very concise and objective. Nevertheless, do write full sentences.
% Try to have at most one paragraph per section, because line breaks
% do not look nice in a short entry.

 \begin{clarifications}
  $\upair v u$ and $\spair v u$ are unsolved and solved, respectively, pairs of first-order terms,
  $S$ is a set of such pairs and $\sigma$ is a substitution.
   The set $S$ is considered
   solved if it contains only solved pairs. The application of the above rules always terminates on a given set of pairs of terms
   and if, in addition, the set is unifiable, then it terminates in a set $S'$ containing only solved pairs.
   The set $S'$ contains the substitution components \cite{Robinson1965JACM} of a most general unifier of $S$.
\end{clarifications}


 \begin{history}
   The unification principle was first presented by Robinson, in an algorithmtic form,
   in his paper about Resolution \cite{Robinson1965JACM} (see \iref{Resolution}).
   The above set of rules is taken from Snyder and Gallier paper about unification
   as a set of transformation rules \cite{Snyder1989101}.
 \end{history}


% General Instructions:
% =====================

% The preferred length of an entry is 1 page.
% Do the best you can to fit your proof system in one page.
%
% If you are finding it hard to fit what you want in one page, remember:
%
%   * Your entry needs to be neither self-contained nor fully understandable
%     (the interested reader may consult the cited full paper for details)
%
%   * If you are describing several proof systems in one entry,
%     consider splitting your entry.
%
%   * You may reduce the size of your entry by ommitting inference rules
%     that are already described in other entries.
%
%   * Cite parsimoniously (see detailed citation instructions below).
%
%
% If you do not manage to fit everything in one page,
% it is acceptable for an entry to have 2 pages.
%
% For aesthetical reasons, it is preferable for an entry to have
% 1 full page or 2 full pages, in order to avoid unused blank space.



% Citation Instructions:
% ======================

% Please cite the original paper where the proof system was defined.
% To do so, you may use the \cite command within
% one of the optional environments above,
% or use the \nocite command otherwise.

% You may also cite a modern paper or book where the
% proof system is explained in greater depth or clarity.
% Cite parsimoniously.

% Do not cite related work. Instead, use the "\iref" or "\irefmissing"
% commands to make an internal reference to another entry,
% as explained within the "history" environment above.

% You do not need to create the "References" section yourself.
% This is done automatically.

% Leave an empty line above "\end{entry}".

\end{entry}
