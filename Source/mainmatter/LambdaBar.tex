
% If the calculus has an acronym, define it.
% (e.g. \newcommand{\LK}{\ensuremath{\mathbf{LK}}\xspace})


\calculusName{LambdaBar}   % The name of the calculus
\calculusAcronym{$\lambdabar$}    % The acronym if defined above, or empty otherwise. 
\calculusLogic{Classical Logic}  % Specify the logic (e.g. classical, intuitionistic, ...) for which this calculus is intended.
\calculusLogic{Intuitionistic Logic}
\calculusLogicOrder{Propositional}
\calculusType{Sequent Calculus}   % Specify the calculus type (e.g. Frege-Hilbert style, tableau, sequent calculus, hypersequent calculus, natural deduction, ...)
\calculusYear{1994}   % The year when the calculus was invented.
\calculusAuthor{\iCA{Herbelin}} % The name(s) of the author(s) of the calculus.


\entryTitle{$\lambdabar$-calculus}     % Title of the entry (usually coincides with the name of the calculus).
\entryAuthor{Hugo Herbelin}     

\tag{Term-annotated}
% e.g. \tag{Two-Sided Sequents}
% e.g. \tag{Multiset Cedents}
% e.g. \tag{List Cedents}
% You are free to invent your own tags. 
% The Encyclopedia's coordinator will take care of 
% merging semantically similar tags in the future.

\maketitle

% If your files are called "MyProofSystem.tex" and "MyProofSystem.bib", 
% then you should write "\begin{entry}{MyProofSystem}" in the line below
\begin{entry}{LambdaBar}

% Define here any newcommands you may need:
% e.g. \newcommand{\necessarily}{\Box}
% e.g. \newcommand{\possibly}{\Diamond}

\begin{calculus}

% Add the inference rules of your proof system here.
% The "proof.sty" and "bussproofs.sty" packages are available.
% If you need any other package, please contact the editor (bruno@logic.at)

{\sc Cut-free system}
\[
\infer[\mathit{Ax}]
      {\Gamma; \cdot:A \vdash \cdot\,() : A}
      {}
\qquad
\infer[\mathit{Cont}]
      {\Gamma \vdash a(l):C}
      {\Gamma; \cdot :A \vdash \cdot\,(l):C & (a:A) \in \Gamma}
\]
\[
\infer[\rightarrow_L]
      {\Gamma ~|~ (p,l) : A\rightarrow B \vdash C}
      {\Gamma \vdash p:A & \Gamma; \cdot:B \vdash \cdot\,(l):C}
\qquad
\infer[\rightarrow_R]
      {\Gamma \vdash \lambda a.p : A \rightarrow B}
      {\Gamma, a:A \vdash p:B}
\]
{\sc Cut rules}
\[
\infer[\mathit{Cut_H^1}]
      {\Gamma \vdash p(l):C}
      {\Gamma \vdash p:A & \Gamma; \cdot:A \vdash \cdot\,(l):C}
\qquad
\infer[\mathit{Cut_H^2}]
      {\Gamma; \cdot:A \vdash \cdot\,(l@l'):C}
      {\Gamma; \cdot:A \vdash \cdot\,(l):B & \Gamma; \cdot:B \vdash \cdot\,(l'):C}
\]
\[
\infer[\mathit{Cut_M^1}]
      {\Gamma, \Gamma' \vdash q[p/a]:C}
      {\Gamma \vdash p:A & \Gamma, a:A, \Gamma' \vdash q:C}
\qquad
\infer[\mathit{Cut_M^2}]
      {\Gamma, \Gamma'; \cdot:B \vdash \cdot\,(l[p/a])C}
      {\Gamma \vdash p:A & \Gamma, a:A, \Gamma'; \cdot:B \vdash \cdot\,(l):C}
\]
\end{calculus}

% The following sections ("clarifications", "history", 
% "technicalities") are optional. If you use them, 
% be very concise and objective. Nevertheless, do write full sentences. 
% Try to have at most one paragraph per section, because line breaks 
% do not look nice in a short entry.

\begin{clarifications}
This calculus can be seen as an organization of the rules of Gentzen's
intuitionistic sequent calculus in a way such that: there is
computational interpretation of proofs as $\lambda$-calculus-like terms;
there is a simple one-to-one correspondence between cut-free proofs
and normal proofs of natural deduction.

The definition of the calculus is based on two kinds of sequents: the
sequents $\Gamma \vdash p:A$ have a focus on the right and are
annotated by a program $p$; the sequents
$\Gamma; \cdot:A \vdash \cdot(l):B$ have an extra focussed formula on
the left annotated by a placeholder name $\cdot$ while the formula on
the right is annotated by a program referring to this placeholder.
The syntax of the underlying calculus is:
$$
\begin{array}{lll}
(l),(l') & ::= & () ~|~ (p,l) ~|~ (l@l') ~|~ (l[p/a])\\
p,q & ::= & a(l) ~|~ \lambda a.p ~|~ p(l) ~|~ q[p/a]\\
\end{array}
$$
with $()$ and $(p,l)$ denoting lists of arguments, $l@l'$ denoting
concatenation of lists, $l[p/a]$ and $p[q/a]$ denoting explicit
substitution, $x(l)$ and $p(l)$ denoting cut-free and non cut-free
application, respectively. The first two items of each entry
characterize the syntax of cut-free proofs.
\end{clarifications}

\begin{history}
The $\lambdabar$-calculus has been designed
in~\cite{Herbelin94,HerbelinPhD}. It can be seen as the direct counterpart
for sequent calculus of what $\lambda$-calculus is for natural
deduction, along the lines of the Curry-Howard correspondence between
proofs and programs. The idea of focussing a specific formula of the
sequent comes from Girard~\cite{girard91mscs} which himself credits it to
Andreoli~\cite{andreoli92jlc} (see also \iref{LKF}). With proof
annotations removed, the calculus can be seen as the intuitionistic
fragment LJT of the subsystem LKT of LK~\cite{danos93wll}, with LKT and
LKQ representing two dual ways to add asymmetric focus to LK.

Extensions to other connectives than implication can be given.
Extensions to classical logic, namely a computational presentation of
LKT, can be obtained by adding the $\mu$ and bracket operators of
$\lambda\mu$-calculus~\iref{LambdaMu} and by considering instead three
kinds of sequents, $\Gamma \vdash p:A ~|~ \Delta$, or
$\Gamma; \cdot:A \vdash \cdot\,(l):B$, or $c: (\Gamma \vdash \Delta)$
(see~\cite{HerbelinPhD}). A variant with implicit substitution is
possible.

The symmetrization of $\lambdabar$-calculus led to
$\LKmumutilde$~\iref{LKMuMuTilde}.
\end{history}

%\begin{technicalities}
%ToDo: write here remarks about soundness, completeness, decidability...
%\end{technicalities}


% General Instructions:
% =====================

% The preferred length of an entry is 1 page. 
% Do the best you can to fit your proof system in one page.
%
% If you are finding it hard to fit what you want in one page, remember:
%
%   * Your entry needs to be neither self-contained nor fully understandable
%     (the interested reader may consult the cited full paper for details)
%
%   * If you are describing several proof systems in one entry, 
%     consider splitting your entry.
%
%   * You may reduce the size of your entry by ommitting inference rules
%     that are already described in other entries.
%
%   * Cite parsimoniously (see detailed citation instructions below).
%
% 
% If you do not manage to fit everything in one page, 
% it is acceptable for an entry to have 2 pages.
%
% For aesthetical reasons, it is preferable for an entry to have
% 1 full page or 2 full pages, in order to avoid unused blank space.



% Citation Instructions:
% ======================

% Please cite the original paper where the proof system was defined.
% To do so, you may use the \cite command within 
% one of the optional environments above,
% or use the \nocite command otherwise.

% You may also cite a modern paper or book where the 
% proof system is explained in greater depth or clarity.
% Cite parsimoniously.

% Do not cite related work. Instead, use the "\iref" or "\irefmissing" 
% commands to make an internal reference to another entry, 
% as explained within the "history" environment above.

% You do not need to create the "References" section yourself. 
% This is done automatically.

% Leave an empty line above "\end{entry}".

\end{entry}
