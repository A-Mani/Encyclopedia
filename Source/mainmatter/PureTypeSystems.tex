
% If the calculus has an acronym, define it.
% (e.g. \newcommand{\LK}{\ensuremath{\mathbf{LK}}\xspace})

\calculusName{Pure Type Systems}   % The name of the calculus
\calculusAcronym{}    % The acronym if defined above, or empty otherwise.
\calculusLogic{Type Theory}  % Specify the logic (e.g. classical, intuitionistic, ...) for which this calculus is intended.
\calculusLogicOrder{Higher-Order}
\calculusType{Natural Deduction}   % Specify the calculus type (e.g. Frege-Hilbert style, tableau, sequent calculus, hypersequent calculus, natural deduction, ...)
\calculusYear{1989}   % The year when the calculus was invented.
\calculusAuthor{Berardi, Terlouw, Barendregt, Geuvers} % The name(s) of the author(s) of the calculus.


\entryTitle{Pure Type Systems}     % Title of the entry (usually coincides with the name of the calculus).
\entryAuthor{Ali Assaf}    % Your name(s). Separate multiple names with "\and".


% If you wish, use tags to give any other information
% that might be helpful for classifying and grouping this entry:
% e.g. \tag{Two-Sided Sequents}
% e.g. \tag{Multiset Cedents}
% e.g. \tag{List Cedents}
% You are free to invent your own tags.
% The Encyclopedia's coordinator will take care of
% merging semantically similar tags in the future.


\maketitle


% If your files are called "MyProofSystem.tex" and "MyProofSystem.bib",
% then you should write "\begin{entry}{MyProofSystem}" in the line below
\begin{entry}{PureTypeSystems}

% Define here any newcommands you may need:
% e.g. \newcommand{\necessarily}{\Box}
% e.g. \newcommand{\possibly}{\Diamond}

\newcommand{\sorts}{\mathcal{S}}
\newcommand{\axioms}{\mathcal{A}}
\newcommand{\rules}{\mathcal{R}}
\newcommand{\variables}{\mathcal{V}}
\newcommand{\constants}{\mathcal{C}}
\newcommand{\terms}{\mathcal{T}}

\newcommand{\union}{\cup}

\newcommand{\Prop}{\ast}
\newcommand{\Type}{\square}
\newcommand{\Kind}{\triangle}

\newcommand{\entails}{\vdash}

\newcommand{\negvspace}{\vspace{-1ex}}

\begin{calculus}

% Add the inference rules of your proof system here.
% The "proof.sty" and "bussproofs.sty" packages are available.
% If you need any other package, please contact the editor (bruno@logic.at)
\negvspace
\[ \infer[axiom\quad (c : s) \in \axioms]{\entails c : s}{ } \qquad
   \infer[start\quad (x \not\in \Gamma)]{\Gamma, x : A \entails x : A}{\Gamma \entails A : s} \]
\[ \infer[weakening\quad (x\not\in\Gamma)]{\Gamma, x : A \entails M : B}{\Gamma \entails M : B \qquad \Gamma \entails A : s} \]
\[ \infer[product\quad (s_1,s_2,s_3)\in\rules]{\Gamma \entails \Pi x : A. B : s_3}{\Gamma \entails A : s_1 \qquad \Gamma, x : A \entails B : s_2} \]
\[ \infer[abstraction]{\Gamma \entails \lambda x: A. M : \Pi x : A. B}{\Gamma, x : A \entails M : B \qquad \Gamma \entails \Pi x : A. B : s} \]
\[ \infer[application]{\Gamma \entails M~N : B[x \coloneqq N]}{\Gamma \entails M : \Pi x : A. B \qquad \Gamma \entails N : A} \]
\[ \infer[conversion]{\Gamma \entails M : B}{\Gamma \entails M : A \qquad \Gamma \entails B : s \qquad A \equiv_\beta B} \]

\vspace{5pt}

\end{calculus}

% The following sections ("clarifications", "history",
% "technicalities") are optional. If you use them,
% be very concise and objective. Nevertheless, do write full sentences.
% Try to have at most one paragraph per section, because line breaks
% do not look nice in a short entry.

\begin{clarifications}
% ToDo: write here short remarks that may help the reader to understand
% the inference rules of the proof system.
\emph{Pure type systems} (PTS) are a general class of typed $\lambda$ calculus. They represent logical systems through the Curry-Howard correspondence and the "propositions as types" interpretation. The syntax is given by the grammar:
\negvspace\[
  \terms \Coloneqq \variables \mid \constants \mid \Pi \variables : \terms. \terms \mid \lambda \variables : \terms. \terms \mid \terms~\terms
\negvspace\]
where $\variables$ is a set of variables and $\constants$ is a set of constants.
A PTS is parameterized by a \emph{specification} $(\sorts, \axioms, \rules)$ where
$\sorts \subseteq \constants$ is the set of \emph{sorts},
$\axioms \subseteq \constants \times \sorts$ is the set of \emph{axioms}, and
$\rules \subseteq \sorts \times \sorts \times \sorts$ is the set of \emph{rules}.
\end{clarifications}

\begin{history}
% ToDo: write here short historical remarks about this proof system,
% especially if they relate to other proof systems.
% Use "\iref{OtherProofSystem}" to refer to another proof system
% in the Encyclopedia (where "OtherProofSystem" is its ID).
% Use "\irefmissing{SuggestedIDForOtherProofSystem}" to refer to
% another proof system that is not yet available in the encyclopedia.
Pure type systems were independently introduced by Berardi and Terlouw as a generalization of systems of the $\lambda$ cube, and further developed and popularized by Barendregt, Geuvers, Nederhof \cite{barendregt_introduction_1991,geuvers_modular_1991,barendregt_lambda_1992,geuvers_logics_1993}. Many important systems can be expressed as PTSs, including simply typed $\lambda$ calculus ($\lambda{\rightarrow}$), $\lambda\Pi$ calculus \iref{LuoLF} ($\lambda{P}$), system F \iref{fc} ($\lambda{2}$), and the calculus of constructions ($\lambda{C}$):
\negvspace\[
  \begin{array}{cc}
    \sorts = \{\Prop, \Type\} \qquad \axioms = \{(\Prop, \Type)\} \qquad \rules_{\rightarrow} = \{(\Prop, \Prop, \Prop)\} \\[1ex]
    \rules_{P} = \rules_{\rightarrow} \union \{(\Prop,\Type,\Type)\} \qquad
    \rules_{2} = \rules_{\rightarrow} \union \{(\Type,\Prop,\Prop)\} \qquad
    \rules_{C} = \rules_{P} \union \rules_{2} \union \{(\Type,\Type,\Type)\}
  \end{array}
\negvspace\]
as well as intuitionistic higher-order logic ($\lambda{HOL}$). Pure type systems form the basis of many proof assistants such as Automath, Lego, Coq, Agda, and Matita.
\end{history}

\begin{technicalities}
% ToDo: write here remarks about soundness, completeness, decidability...
Soundness and decidability of type checking in PTSs are closely related to \emph{strong normalization} (SN), i.e.~the property that all well-typed terms terminate. Not all pure type systems are SN. Examples of PTSs that are \emph{not} SN (and are therefore inconsistent) are Girard's system U and the universal PTS $\lambda{\Prop}$:
\negvspace\[
  \sorts = \{\Prop\} \qquad \axioms = \{(\Prop, \Prop)\} \qquad \rules = \{(\Prop, \Prop, \Prop)\}
\negvspace\]
\end{technicalities}


% General Instructions:
% =====================

% The preferred length of an entry is 1 page.
% Do the best you can to fit your proof system in one page.
%
% If you are finding it hard to fit what you want in one page, remember:
%
%   * Your entry needs to be neither self-contained nor fully understandable
%     (the interested reader may consult the cited full paper for details)
%
%   * If you are describing several proof systems in one entry,
%     consider splitting your entry.
%
%   * You may reduce the size of your entry by ommitting inference rules
%     that are already described in other entries.
%
%   * Cite parsimoniously (see detailed citation instructions below).
%
%
% If you do not manage to fit everything in one page,
% it is acceptable for an entry to have 2 pages.
%
% For aesthetical reasons, it is preferable for an entry to have
% 1 full page or 2 full pages, in order to avoid unused blank space.



% Citation Instructions:
% ======================

% Please cite the original paper where the proof system was defined.
% To do so, you may use the \cite command within
% one of the optional environments above,
% or use the \nocite command otherwise.

% You may also cite a modern paper or book where the
% proof system is explained in greater depth or clarity.
% Cite parsimoniously.

% Do not cite related work. Instead, use the "\iref" or "\irefmissing"
% commands to make an internal reference to another entry,
% as explained within the "history" environment above.

% You do not need to create the "References" section yourself.
% This is done automatically.




% Leave an empty line above "\end{entry}".

\end{entry}
