
\calculusName{Proof Nets for $\mathbf{MLL}^-$}
\calculusAcronym{}  
\calculusLogic{Linear Logics} 
\calculusLogicOrder{Propositional}
\calculusType{Proof Net} 
\calculusYear{1987} 
\calculusAuthor{Jean-Yves Girard}


\entryTitle{Proof Nets for $\mathbf{MLL}^-$}
\entryAuthor{Gianluigi Bellin}


\maketitle


\begin{entry}{ProofNetMLL}  


\begin{calculus}

%{\bf Grammar of $\mathbf{MLL}^-$\ :}
%$A, B\quad :=\quad p\ |\ p^{\bot}\ |\ A \otimes B\ |\ A \wp B $   \quad (as in LL1986

%{\bf Sequent Calculus for $\mathbf{MLL}^-$:} \hskip 2in{as in LL1986.}
% $\hskip.5in
% (A \otimes B)^{\bot} =_{df} A^{\bot}\wp B^{\bot}\quad 
% (A \wp B)^{\bot} =_{df} A^{\bot}\otimes B^{\bot} (A^{\bot})^{\bot} =_{df} X$\\

{\bf Links :} 
 \AxiomC{\it axiom}
 \UnaryInfC{$A^\bot\quad A$}
\DisplayProof
\quad 
\AxiomC{$A$}
\AxiomC{$A^\bot$}
\BinaryInfC{\it cut}
\DisplayProof
\quad 
\AxiomC{$A$}
\AxiomC{$B$}
\BinaryInfC{$A\otimes B$}
\DisplayProof
\quad 
\AxiomC{$A$}
\AxiomC{$B$}
\BinaryInfC{$A \wp B$}
\DisplayProof

\vspace{10pt}

{\bf Proof Structure $\mathcal{R}(R, L)$ :} a nonempty set $R$ of formula occurrences,
with a set $L$ of links, such that each $A\in R$ is a conclusion of \emph{exactly one}
link and a premise of \emph{at most one} link. If $A$ is not a  premise, then it is a 
{\bf  conclusion of}  $\mathcal{R}$. (\emph{Cut} links behave like \emph{times} links 
with conclusion $A\otimes A^{\bot}$.)

\vspace{10pt}

{\bf Switching} $s$: 
a choice for every {\em par} link $\ell$ of one premise, $s(\ell)$= `{\it left}' or `{\it right}'.

\vspace{10pt}

{\bf Switching graph} $s\mathcal{R}$: an undirected graph $(R, E)$
%whose vertices are the formula occurrences of $\mathcal{R}$ and  \\
with edges $E$ as follows: 
%  (i) $A - A^{\bot}$ in an axiom or cut link,  
%  (ii) $A - A\otimes B$ and $A\otimes B - B$\\  in a {\em times} link and
% (iii) for a par link $\ell$,  $s(\ell) - A\wp B$, i.e.,  either $A - A\wp B$ or $B - A\wp B$.\\

$
\xymatrix@1@R=5pt @C=8pt{% &axiom:&&cut:&&concl.:\\
&A\ar@/^/@{-}[rr]|-{ax}&&A^{\bot}& A^{\bot}\ar@/_/@{-}[rr]|-{cut}&&A&A\ar@{-}[dr]&&B\ar@{-}[dl]& %
A\ar@{-}[dr] \ar@{}[rr]|-{\mathit{left}} && B \ar@{.}[dl] & A\ar@{.}[dr]\ar@{}[rr]|-{\mathit{right}} && B \ar@{-}[dl] \\
&                           &&             &                                      &&   &     &A\otimes B& &     &A\wp B&&  
&A\wp B&\\}
$

\vspace{10pt}

{\bf Proof net:} A proof structure $\mathcal{R}$ such that for every switching $s$
the graph 
$s\mathcal{R}$ is \emph{acyclic} and \emph{connected} 
(Danos Regnier's \emph{correctness criterion} \cite{DanosRegnier}).


\end{calculus}

\begin{clarifications}
\begin{enumerate}
\item The forest of sub-formulas of a multiset $\Gamma =  C_1, \ldots, C_n$, 
with a partition of the leaves in unordered pairs $(p, p^{\bot})$ is a cut-free proof-structure. Also
$
\xymatrix@1@R=5pt @C=15pt{\ar@{-}@<1ex>[r]^{\mathit{axiom}}A& 
A^{\bot} \ar@{-}@<1ex>[l]^{\mathit{cut}} }
$
is a proof structure. 


\item Proof nets are canonical representations of $\mathbf{MLL}^-$ sequent calculus proofs
and solve the proof identity problem for $\mathbf{MLL}^-$ in linear time. The \emph{desequentialization map} $(\ )^{-}$ identifies sequent calculus derivations $d_1$ and $d_2$ that differ only by permutations 
of inferences:\\
\begin{scriptsize}
\ \ 
$(\AxiomC{$\vdash \overline{A, A^{\bot}}$}\DisplayProof)^-$ 
= \AxiomC{\it axiom}\UnaryInfC{$A\quad A^{\bot}$}\DisplayProof 
\quad \ \
 $\Big(
 \AxiomC{$d_1$}
 \noLine
 \UnaryInfC{$\vdash \Gamma,A$}
 \AxiomC{$d_2$}
 \noLine
\UnaryInfC{$\vdash \Delta,B$}
 \BinaryInfC{$\vdash \Gamma, A \otimes B, \Delta$}
 \DisplayProof \Big)^- $
 \ =\  
 \AxiomC{$(d_1)^-\  (d_2)^-$}
 \noLine
 \UnaryInfC{$\underline{A \quad\ B}$}
 \noLine
 \UnaryInfC{$\Gamma\  A \otimes B\ \Delta$}
 \DisplayProof 
\quad \ \
$\Big(
 \AxiomC{$d$}
  \noLine
  \UnaryInfC{$\vdash \Gamma, A, B$}
  \UnaryInfC{$\vdash \Gamma, A \wp B$}
   \DisplayProof
   \Big)^-$ \ =\ 
  \AxiomC{$\quad (d)^-$}
  \noLine
 \UnaryInfC{$\quad \underline{A\quad B}$}
 \noLine
  \UnaryInfC{$\Gamma\ A \wp B$}
 \DisplayProof
\end{scriptsize} \\


\item $\mathcal{R}_1(R_1, L_1)$ is a \emph{subnet} of  $\mathcal{R}_2(R_2, L_2)$ if 
$R'\subseteq R$ and  $L_1$ = $L_2 |_{R_1}$. 

\vspace{1ex}

\noindent
{\bf Lemma 1.} Let $\mathcal{R}_1$ and $\mathcal{R}_2$ be subnets of $\mathcal{R}$
with $\mathcal{R}_1\cap \mathcal{R}_2\neq \emptyset$. Then $\mathcal{S} = \mathcal{R}_1\cup \mathcal{R}_2$ is a subnet of $\mathcal{R}$. 
%
{\bf Proof.} Since any $s\mathcal{R}$ is acyclic, so is its subgraph $s\mathcal{S}$.
Given $A \in {R}_1$, $B \in {R}_2$ and $C \in (R_1\cap R_2)$, $A$ is connected to $C$ in 
$\mathcal{R}_1$ and $C$ is connected to $B$ in $\mathcal{R}_2$, so $A$ is connected to $C$ in 
$\mathcal{S}$. {\bf qed}

\vspace{1ex}

\noindent
The \emph{empire} $eA$, for $A\in \mathcal{R}$, 
is the \emph{largest subnet having $A$ as a conclusion}. 
%
%\noindent
If $s_A\mathcal{R}$ is the subgraph of $s\mathcal{R}$ with the vertex $A$ as root, 
then $eA = \bigcap_s s_{A}\mathcal{R}$.
%
%\vspace{1ex}
%
%\noindent
The \emph{kingdom} $kA$ of $A$ is the \emph{smallest} subnet having $A$ 
as conclusion. 

\vspace{1ex}

\noindent
{\bf Lemma 2.} Let $\ell_1$ and $\ell_2$ be links in $\mathcal{R}$ with conclusions 
$A_0\otimes A_1$ and  $C_0\wp C_1$, respectively.
If $C_i \in eA_j$ but $C_0\wp C_1\notin eA_j$ then  $A_0\otimes A_1\in k(C_0\wp C_1)$. 
\begin{center}
\AxiomC{$\vdots$}
\noLine
\UnaryInfC{$A_0$}
\AxiomC{$\vdots$}
\noLine
\UnaryInfC{$\mathbf{A_1}$}
\LeftLabel{$\ell_1$}
\BinaryInfC{$A_0\otimes A_1$}
\AxiomC{$e\mathbf{A_1}$}
\noLine
\UnaryInfC{\strut}
\noLine
\UnaryInfC{$k(C_0\wp C_1)$}
\noLine
\UnaryInfC{}
\AxiomC{$\vdots\quad \vdots$}
\noLine
\UnaryInfC{$\mathbf{C_0} \quad C_1$}
\LeftLabel{$\ell_2$}
\UnaryInfC{$C_0 \wp C_1$}
\noLine
\TrinaryInfC{\strut}
\DisplayProof
\end{center}

\noindent
{\bf Proof.}
Let $C_0\in eA_1$; clearly $C_0\in k(C_0\wp C_1)$ so $\mathcal{S} = eA_1\cup  k(C_0\wp C_1) \neq \emptyset$ and by Lemma 1 is a subnet. Suppose $A_0\otimes A_1$ does not belong to 
$k(C_0\wp C_1)$; then $\mathcal{S}$ has $A_1$ as conclusion and is larger than $eA_1$, a 
contradiction. {\bf qed}

\vspace{1ex}

\noindent
{\bf Sequentialization Theorem.} 
If $\mathcal{R}$ is a proof net for $\mathbf{MLL}^-$ with conclusions $Gamma$,
then a sequent calculus derivation $d$ of $\vdash Gamma$ can be constructed such that 
$(d)^{-} = \mathcal{R}$. \\
\textbf{Proof sketch.} By induction on the number of links of $eA$. 
Terminal \emph{par} links can be deleted and the result follows from the induction hypothesis. 
Suppose the non-atomic conclusions of $eA$ are $A_0\otimes B_0, \ldots, A_n\otimes B_n$.  
We need to find a \emph{splitting link} $\ell_i$ with conclusion  $A_i \otimes B_i$, such that by removing $\ell_i$ the net splits in two disjoint components  $e(A_i)$ and $e(B_i)$. 
We choose an $\ell_i$ such that  $A_i\otimes B_i$ is not included in $k(A_j\otimes B_j)$ for $j\neq i$. 
%
%\vspace{1ex}
%
%\noindent
If all conclusions of $eA_i$  are conclusions of $eA$, we are done.
Otherwise let $\ell$ be a link such that a premise $C$ is in $eA_i$, but 
the conclusion is not. Then $\ell$ must be a \emph{par} link with conclusion, say, $C \wp D$. 
By Lemma 2 $A_i\otimes B_i \in k(C\wp D)$.  But $C\wp D$ must occur above a link 
$\ell_j$ with conclusion $A_j\otimes B_j$. It follows that 
$(A_i\otimes B_i) \in k(C\wp D) \subset k(A_j\otimes B_j)$
contrary to the choice of $\ell_i$. {\bf qed}

\end{enumerate}

\end{clarifications}

\begin{history}
Proof nets for $\mathbf{MLL}^-$ were introduced by J-Y.Girard in 1987 \cite{Girard}. 
Simplifications were given by Danos and Regnier \cite{DanosRegnier} and in \cite{BellinDeWiele}.  
Since then many systems of proof nets were found for larger fragments of linear logic, with additives (D.~Hughes and R.~van Glabbek) and for variants of linear logic, with \emph{Mix} (A.~Fleury, C.~Retor\'e, G.~Bellin) and F.~Lamarche's \emph{essential nets} for intuitionistic linear logic). 
In 1999 S. Guerrini showed that correctness of multiplicative proof-nets without units is linear.
For {\bf MLL} \emph{with the units} proof-nets are non-canonical with respect to permutation of inferences in the sequent calculus. In 2014 W.~Heijltjes and R.~Houston showed that the identity problem for {\bf MLL} proofs is PSPACE complete. 
\end{history}

% \begin{technicalities}
% ToDo: write here remarks about soundness, completeness, decidability...
% \end{technicalities}


\end{entry}
