
% If the calculus has an acronym, define it.
% (e.g. \newcommand{\LK}{\ensuremath{\mathbf{LK}}\xspace})

\calculusName{Saturation With Redundancy}   % The name of the calculus
\calculusAcronym{}    % The acronym if defined above, or empty otherwise. 
\calculusLogic{Classical Logic}  % Specify the logic (e.g. classical, intuitionistic, ...) for which this calculus is intended.
\calculusType{Meta-Calculus}   % Specify the calculus type (e.g. Frege-Hilbert style, tableau, sequent calculus, hypersequent calculus, natural deduction, ...)
\calculusYear{1990}   % The year when the calculus was invented.
\calculusAuthor{\iCA{Leo Bachmair}, \iCA{Harald Ganzinger}} % The name(s) of the author(s) of the calculus.


\entryTitle{Saturation With Redundancy}     % Title of the entry (usually coincides with the name of the calculus).
\entryAuthor{Uwe Waldmann}     


% If you wish, use tags to give any other information 
% that might be helpful for classifying and grouping this entry:
% e.g. \tag{Two-Sided Sequents}
% e.g. \tag{Multiset Cedents}
% e.g. \tag{List Cedents}
% You are free to invent your own tags. 
% The Encyclopedia's coordinator will take care of 
% merging semantically similar tags in the future.


\maketitle


% If your files are called "MyProofSystem.tex" and "MyProofSystem.bib", 
% then you should write "\begin{entry}{MyProofSystem}" in the line below
\begin{entry}{SaturationWithRed}  

% Define here any newcommands you may need:
% e.g. \newcommand{\necessarily}{\Box}
% e.g. \newcommand{\possibly}{\Diamond}


\begin{calculus}

% Add the inference rules of your proof system here.
% The "proof.sty" and "bussproofs.sty" packages are available.
% If you need any other package, please contact the editor (bruno@logic.at)

\textbf{Primary Rules}

\[
\infer[\textit{Deduction}]
{N \cup \{C\}}{N & N \models C}
\]
\[
\infer[\textit{Deletion}]
{N}{N \cup \{C\} & C \textrm{~$\mathcal{R}$-redundant w.\,r.\,t.~} N}
\]

\textbf{Derived Rules}

\[
\infer[\textit{Simplification}]
{N \cup M}{N \cup \{C\} & N \cup \{C\} \models M & C \textrm{~$\mathcal{R}$-redundant w.\,r.\,t.~} N \cup M}
\]
\mbox{}\quad is a shorthand for
\[
\infer[\textit{Deletion}]
{N \cup M}{\infer[\textit{Deduction}^+]
           {N \cup \{C\} \cup M}{N \cup \{C\} & N \cup \{C\} \models M}
           & C \textrm{~$\mathcal{R}$-redundant w.\,r.\,t.~} N \cup M}
\]

$N$ and $M$ are finite sets of formulas, $C$ is a formula.
\end{calculus}

% The following sections ("clarifications", "history", 
% "technicalities") are optional. If you use them, 
% be very concise and objective. Nevertheless, do write full sentences. 
% Try to have at most one paragraph per section, because line breaks 
% do not look nice in a short entry.

\begin{clarifications}
This is a meta-inference system for refutational calculi
that is parameterized by
(1) an entailment relation~$\models$, (2) an inference system $\mathcal{I}$
and (3) a redundancy criterion $\mathcal{R}$ for formulas and inferences,
such that $\mathcal{I}$-inferences are sound w.\,r.\,t.~$\models$,
and such that $\mathcal{I}$-inferences whose conclusion is contained in
$N$ are $\mathcal{R}$-redundant w.\,r.\,t.~$N$.
Note that the \textit{Deduction} rule is not restricted to adding the conclusions
of $\mathcal{I}$-inferences from $N$;
fairness, however, requires that
every $\mathcal{I}$-inference from persisting formulas must become
$\mathcal{R}$-redundant at some point
(for instance, by adding its conclusion).
\end{clarifications}

\begin{history}
In theorem proving calculi with a redundancy concept,
closure under the inference rules can be replaced by
a refined notion of saturation that allows to
alternate between derivation of new formulas and
elimination of irrelevant formulas
(e.\,g., tautologies and subsumed formulas).
The system was introduced by
Bachmair and Gan\-zin\-ger~\cite{BachmairGanzinger1990CTRS}
for superposition~\iref{Superposition};
it can be used for most other super\-posi\-tion-like calculi,
such as constraint superposition~\iref{ConstraintSup},
superposition modulo theories~\iref{CancellativeSup}, or
hierarchic superposition~\iref{HierarchicSup}, with appropriate choices for
\looseness=-1
$\models$, $\mathcal{I}$, and $\mathcal{R}$.
\end{history}

% \begin{technicalities}
% ToDo: write here remarks about soundness, completeness, decidability...
% \end{technicalities}


% General Instructions:
% =====================

% The preferred length of an entry is 1 page. 
% Do the best you can to fit your proof system in one page.
%
% If you are finding it hard to fit what you want in one page, remember:
%
%   * Your entry needs to be neither self-contained nor fully understandable
%     (the interested reader may consult the cited full paper for details)
%
%   * If you are describing several proof systems in one entry, 
%     consider splitting your entry.
%
%   * You may reduce the size of your entry by ommitting inference rules
%     that are already described in other entries.
%
%   * Cite parsimoniously (see detailed citation instructions below).
%
% 
% If you do not manage to fit everything in one page, 
% it is acceptable for an entry to have 2 pages.
%
% For aesthetical reasons, it is preferable for an entry to have
% 1 full page or 2 full pages, in order to avoid unused blank space.



% Citation Instructions:
% ======================

% Please cite the original paper where the proof system was defined.
% To do so, you may use the \cite command within 
% one of the optional environments above,
% or use the \nocite command otherwise.

% You may also cite a modern paper or book where the 
% proof system is explained in greater depth or clarity.
% Cite parsimoniously.

% Do not cite related work. Instead, use the "\iref" or "\irefmissing" 
% commands to make an internal reference to another entry, 
% as explained within the "history" environment above.

% You do not need to create the "References" section yourself. 
% This is done automatically.




% Leave an empty line above "\end{entry}".

\end{entry}
