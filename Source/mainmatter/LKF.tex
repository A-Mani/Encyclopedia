\calculusName{Focused LK}
\calculusAcronym{\LKF}
\calculusLogic{First-order classical logic}
\calculusType{Focused sequent calculus}
\calculusYear{2007}
\calculusAuthor{Chuck Liang and Dale Miller}

\entryTitle{Focused LK}
\entryAuthor{Dale Miller}

\tag{One-Sided Sequents}
\tag{Focused Proof System}

\maketitle

\begin{entry}{LKF}  

\newcommand{\true}{\hbox{\sl t}}
\newcommand{\false}{\hbox{\sl f}}
\newcommand{\truen}{t^-}
\newcommand{\truep}{t^+}
\newcommand{\falsen}{f^-}
\newcommand{\falsep}{f^+}
\newcommand{\wedgep}{\wedge^{\!+}}
\newcommand{\wedgen}{\wedge^{\!-}}
\newcommand{\veep}{\vee^{\!+}}
\newcommand{\veen}{\vee^{\!-}}

\newcommand{\async}[2]{\vdash#1\mathbin{\Uparrow}   #2}
\newcommand{\sync }[2]{\vdash#1\mathbin{\Downarrow} #2}

\begin{calculus}
{\sc Asynchronous introduction rules}
\[
\infer{\async{\Gamma}{\truen\kern-3pt,\Theta}}{}
\qquad
\infer{\async{\Gamma}{B_1\wedgen B_2,\Theta}}
      {\async{\Gamma}{B_1,\Theta}\quad \async{\Gamma}{B_2,\Theta}}
\qquad
\infer{\async{\Gamma}{\falsen\kern-3pt,\Theta}}{\async{\Gamma}{\Theta}}
\qquad
\infer{\async{\Gamma}{B_1\veen B_1,\Theta}}{\async{\Gamma}{B_1,B_2,\Theta}}
\]
\[
\infer{\async{\Gamma}{\forall x.B,\Theta}}{\async{\Gamma}{[y/x]B,\Theta}}
\]
{\sc Synchronous introduction rules}
\[
\infer{\sync{\Gamma}{\truep}}{}
\qquad
\infer{\sync{\Gamma}{B_1\wedgep B_2}}
      {\sync{\Gamma}{B_1}\quad\sync{\Gamma}{B_2}}
\qquad
\infer[i \in \{1,2\}]{\sync{\Gamma}{B_1\veep B_2}}{\sync{\Gamma}{B_i}}
\qquad
\infer{\sync{\Gamma}{\exists x.B}}{\sync{\Gamma}{[t/x]B}}
\]
{\sc Identity rules}
\[
\infer[init]{\sync{\neg P,\Gamma}{P}}{P \; \mbox{atomic}}
\qquad
\infer[cut]{\async{\Gamma}{\cdot}}{\async{\Gamma}{B}\quad \async{\Gamma}{\neg{B}}}
\]
{\sc Structural rules}
\[
\infer[store]{\async{\Gamma}{C,\Theta}}{\async{\Gamma,C}{\Theta}}
\qquad
\infer[release]{\sync{\Gamma}{N}}{\async{\Gamma}{N}}
\qquad
\infer[decide]{\async{P,\Gamma}{\cdot}}{\sync{P,\Gamma}{P}}
\]
Here,
$\Gamma$ ranges over multisets of polarized formulas;
$\Theta$ ranges over lists of polarized formulas;
$P$ denotes a positive formula;
$N$ denotes a negative formula;
$C$ denotes either a negative formula or a positive atom; and
$B$ denotes an unrestricted polarized formula.
The negation in $\neg B$ denotes the negation normal form of the de
Morgan dual of $B$.
The right introduction rule for $\forall$ has the the usual
eigenvariable restriction that $y$ is not free in any formula in the
conclusion sequent.
\end{calculus}

\begin{clarifications}
This proof system involves \emph{polarized} (negative normal) formulas
of first-order classical logic: in order to polarize a formula $B$,
one must assign the status of ``positive'' or ``negative'' bias to all
atomic formulas and replace all occurrences of truth with either
$\truep$ or $\truen$ and replace all occurrences of conjunctions with
either $\wedgep$ or $\wedgen$; similarly, all occurrences of false and
disjunctions must be polarized into $\falsep$, $\falsen$, $\veep$, and
$\veen$. If there are $n$ occurrences of propositional connectives
in $B$, there are $2^n$ ways to polarize $B$.
The \emph{positive connectives} are $\falsep$, $\veep$, $\truep$,
$\wedgep$, and $\exists$ while the \emph{negative connectives} are
$\truen$, $\wedgen$, $\falsen$, $\veen$, and $\forall$.
A formula is \emph{positive} it is a positive atom or has a top-level
positive connective; similarly a formula is \emph{negative} if it is a
negative atom or has a top-level negative connective.

There are two kinds of sequents in this proof system, namely,
$\async{\Gamma}{\Theta}$ and $\sync{\Gamma}{B}$, where $\Gamma$ is a
multiset of polarized formulas, $B$ is a polarized formula, and
$\Theta$ is a list of polarized formulas.  The list structure of
$\Theta$ can be replaced by a multiset.
\end{clarifications}

\begin{history}
This focused proof system is a slight variation of the proof systems in
\cite{liang09tcs,liang07csl}.  
A multifocus variant of \LKF has been described in \cite{chaudhuri14jlc}.
The design of \LKF borrows strongly from Andreoli's focused proof system for
linear logic \cite{andreoli92jlc} and Girard's LC proof
system \cite{girard91mscs}.  The first-order versions 
of the LKT and LKQ proof systems of \cite{danos93wll} can be seen 
subsystems of \LKF.
\end{history}
% Leave an empty line above "\end{entry}".

\end{entry}
