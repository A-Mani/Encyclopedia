


\calculusName{Lambek Calculus}   
\calculusAcronym{\LambekCalc}     
\calculusLogic{Substructural Logics} 
\calculusLogicOrder{Propositional}
\calculusType{Sequent Calculus}   
\calculusYear{1958}   
\calculusAuthor{Joachim Lambek}

\entryTitle{Lambek Calculus}        
\entryAuthor{Harley Eades III} \entryAuthor{Valeria de Paiva}  


\tag{Non-commutative}

\maketitle

\begin{entry}{LambekCalc}  

\newcommand{\llimp}[0]{\leftharpoonup}
\newcommand{\rlimp}[0]{\rightharpoonup}    
  
\begin{calculus}
\[
\begin{array}{ccccccccccc}
  \infer[ax]{A \vdash A}{}
  & \quad &
  \infer[cut]{\Gamma_1,\Gamma_2,\Gamma_3 \vdash C}{\Gamma_1 \vdash A & \Gamma_2,A,\Gamma_3 \vdash C}
  & \quad &
  \infer[I_r]{\cdot \vdash I}{}
  & \quad &
  \infer[I_l]
        {\Gamma_1,I,\Gamma_2 \vdash A}
        {\Gamma_1,\Gamma_2 \vdash A}
  \\[8pt]  
  \infer[\otimes_r]
        {\Gamma_1,\Gamma_2 \vdash A \otimes B}
        {\Gamma_1 \vdash A & \Gamma_2 \vdash B}
  & \quad &
  \infer[\otimes_l]
        {\Gamma_1,A \otimes B,\Gamma_2 \vdash C}
        {\Gamma_1,A,B,\Gamma_2 \vdash C}
  & \quad &
  \infer[\rlimp_r]{\Gamma \vdash A \rlimp B}{\Gamma,A \vdash B}
  & \quad &
  \infer[\rlimp_l]{\Gamma_1,A \rlimp B,\Gamma_2,\Gamma_3 \vdash C}{\Gamma_1 \vdash A & \Gamma_2, B, \Gamma_3 \vdash C}
  \\[8pt]
  \infer[\llimp_r]{\Gamma \vdash B \llimp A}{A,\Gamma \vdash B}
  & \quad &
  \infer[\rlimp_l]{\Gamma_1,B \llimp A,\Gamma_2,\Gamma_3 \vdash C}{\Gamma_1 \vdash A & \Gamma_2, B, \Gamma_3 \vdash C}
\end{array}
\]
\end{calculus}

\begin{clarifications}
The Lambek Calculus described here was introduced by Joachim Lambek to
study sentence structure in 1958 \cite{lambek1958}.  Actually the
calculus Lambek first introduced, despite being motivated by algebraic
considerations as we are told in \cite{lambek1988}, had no constant
corresponding to the unity of the tensor product $I$. The Lambek
calculus can be seen as the logic one obtains from Gentzen's
Intuitionistic Propositional Logic (LJ) \iref{GentzenLJ} if we remove
the structural rules of contraction, weakening and commutation. Lambek
also introduced another calculus \cite{lambek1961} where even the
associativity of the tensor is not valid.  \end{clarifications}

\begin{history}
%% ToDo: write here short historical remarks about this proof system,
%% especially if they relate to other proof systems. 
%% Use "\iref{OtherProofSystem}" to refer to another proof system 
%% in the Encyclopedia (where "OtherProofSystem" is its ID). 
%% Use "\irefmissing{SuggestedIDForOtherProofSystem}" to refer to 
%% another proof system that is not yet available in the encyclopedia.
The system now known as the basic Lambek Calculus was introduced in
1958 by Joachim Lambek as the "Syntactic Calculus" \cite{lambek1958}.
Lambek's motivation was to ``to obtain an effective rule (or
algorithm) for distinguishing sentences from non-sentences, which
works not only for the formal languages of interest to the
mathematical logician, but also for natural languages $[\ldots]$'', as
explained by Moortgat in \cite{moortgat2010}.  After a long period of
ostracism, around the middle 1980s the Syntactic Calculus, now called
the Lambek Calculus was taken up by logicians interested in
Computational Linguistics, especially van Benthem, Buszkowski and
Moortgat. They realized that a computational semantics for categorical
derivations along the lines of the Curry-Howard proofs-as-programs
interpretation would provide us with a ``parsing-as-deduction"
paradigm and a powerful tool to study ``logical" derivational
semantics. Around the same time, the introduction of Linear Logic
\iref{LL}, by Jean-Yves Girard also gave a new impulse to the work in
Categorical Grammars. This was because of Linear Logic's insight that
even if you had a very weak proof system, you could introduce
structural rules in a controlled fashion and hence obtain more
expressive systems, by the use of the so called modalities. Since no
expressivity is lost in this process, this opened the way for various
types of experiments, trying to make sure that the logical system
could cope with more phenomena from the language, see discussion of
examples in \cite{moortgat2010}.
\end{history}

%% \begin{technicalities}
%% ...
%% \end{technicalities}

\end{entry}
