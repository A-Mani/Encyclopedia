
\calculusName{Contextual Natural Deduction}
\calculusAcronym{\NDc}
\calculusLogic{Minimal Logic}
\calculusType{Natural Deduction}
\calculusYear{2013}
\calculusAuthor{\iCA{Bruno Woltzenlogel Paleo}}

\entryTitle{Contextual Natural Deduction}
\entryAuthor{Bruno Woltzenlogel Paleo}

\maketitle


\begin{entry}{ContextualND}

\newcommand{\lc}{\lambda^c}
\newcommand{\C}{\mathcal{C}}

\begin{calculus}
\centering
$$
\infer{\Gamma, a: A \seq a: A}{}
$$

$$
\infer[\imp_I (\pi)]{\Gamma \seq \lambda_{\pi} a^A. b : \C_{\pi}[A \imp B]}
{\Gamma, a: A \seq b: \C_{\pi}[B]}
$$


$$
\infer[\imp_E^{\rightharpoonup} (\pi_1;\pi_2)]{\Gamma \seq (f \ x)^{\rightharpoonup}_{(\pi_1;\pi_2)} : \C^1_{\pi_1}[\C^2_{\pi_2}[B]]}
{\Gamma \seq f: \C^1_{\pi_1}[A \imp B]  &  \Gamma \seq x: \C^2_{\pi_2}[A] }
$$


$$
\infer[\imp_E^{\leftharpoonup} (\pi_1;\pi_2)]{\Gamma \seq (f \ x)^{\leftharpoonup}_{(\pi_1;\pi_2)} : \C^2_{\pi_1}[\C^1_{\pi_2}[B]]}
{\Gamma \seq f: \C^1_{\pi_1}[A \imp B]  &  \Gamma \seq x: \C^2_{\pi_2}[A] }
$$

\bigskip

\small{$\pi$, $\pi_1$ and $\pi_2$ must be positive positions. 
$a$ is allowed to occur in $b$ only if $\pi$ is strongly positive.}

\smallskip

\end{calculus}

\begin{clarifications}
$\C_{\pi}[F]$ denotes a formula with $F$ occurring in the hole of 
a \emph{context} $\C_{\pi}[]$. $\pi$ is the position of the hole. 
It is: \emph{positive} iff it is in the left side of an even number 
of implications; \emph{strongly positive} iff this number is zero.
\end{clarifications}

\begin{history}
Contextual Natural Deduction~\cite{ContextualND} 
combines the idea of deep inference~\irefmissing{ToDo} with 
Gentzen's natural deduction~\iref{GentzenNJ}. 
\end{history}

\begin{technicalities}
Soundness and completeness w.r.t. minimal logic are 
proven~\cite{ContextualND} by providing translations 
between \NDc and the minimal fragment of \NJ \iref{GentzenNJ}. 
\NDc proofs can be quadratically shorter than 
proofs in the minimal fragment of \NJ.
\end{technicalities}


\end{entry}




