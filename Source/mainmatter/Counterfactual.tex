

\calculusName{Counterfactual Sequent Calculi I} 
\calculusAcronym{} 
%\calculusLogic{Lewis' Counterfactual Logics}
\calculusLogic{Counterfactual Logics}
\calculusLogicOrder{Propositional}
\calculusType{Sequent Calculus}
\calculusYear{1983,1992,2012,2013} 
\calculusAuthor{de Swart} \calculusAuthor{Gent} \calculusAuthor{Lellmann} \calculusAuthor{Pattinson} 

\entryTitle{Counterfactual Sequent Calculi I}
\entryAuthor{Bj{\"o}rn Lellmann} 

\maketitle


\begin{entry}{Counterfactual}


\newcommand{\nc}{\newcommand}
\nc{\cless}{\preccurlyeq}
\nc{\rarr}{\rightarrow}
\nc{\CC}{\mathbb{C}}
\nc{\NN}{\mathbb{N}}
\newcommand{\Rules}{\mathcal{R}}
\nc{\TT}{\mathbb{T}}
\nc{\VA}{\mathbb{VA}}
\nc{\VNA}{\mathbb{VNA}}
\nc{\VV}{\mathbb{V}}
\nc{\WW}{\mathbb{W}}


\begin{calculus}
\begin{small}
\[
  \vcenter{\infer[R_{n,m}]{\Gamma,(C_1\cless D_1),\dots,(C_m\cless
    D_m)\seq\Delta,(A_1\cless B_1),\dots,(A_n\cless B_n)}
    {\begin{array}{c}\{\;B_k\seq A_1,\dots,A_n,D_1,\dots,D_m\mid k\leq n\;\} \\\;\cup\;
    \{\;C_k\seq A_1,\dots,A_n,D_1,\dots,D_{k-1}\mid k\leq m\;\} \end{array}
  }}
\]

\vspace{-5pt}

\[
\infer[T_{m}]{\Gamma, (C_1\cless D_1),\dots, (C_m\cless D_m) \seq \Delta
}
{\{\; C_k \seq D_1,\dots, D_{k-1} \mid k \leq m\;\}  \quad & \quad \Gamma \seq
  \Delta, D_1,\dots, D_m
}
\]

\vspace{-5pt}

\[
  \vcenter{\infer[W_{n,m}]{\Gamma,(C_1\cless
    D_1),\dots,(C_m\cless D_m)\seq \Delta,(A_1\cless
    B_1),\dots,(A_n\cless B_n)}
    {\{\;C_k\seq
    A_1,\dots,A_n,D_1,\dots,D_{k-1}\mid k \leq m\;\} \quad&\quad
    \Gamma\seq\Delta,A_1,\dots,A_n,D_1,\dots,D_m}} 
\]

\vspace{-5pt}

\[
\vcenter{\infer[R_{C1}]{\Gamma\seq\Delta,(A\cless
    B)}{\Gamma\seq\Delta,A}}
%$
\quad
\infer[A_{n,m}]{\Gamma, (C_1\cless D_1), \dots,
  (C_m\cless D_m) \seq \Delta, ( A_1,\cless B_1), \dots,
  (A_n\cless B_n)
}
{\begin{array}{c}\{\;\Gamma^\cless,B_k\seq \Delta^\cless, A_1,\dots,A_n,D_1,\dots,D_m\mid k\leq n\;\} \\\;\cup\;
    \{\; \Gamma^\cless,C_k\seq \Delta^\cless, A_1,\dots,A_n,D_1,\dots,D_{k-1}\mid k\leq m\;\}\end{array}
}
\quad
%$
 \vcenter{\infer[R_{C2}]{\Gamma,(A\cless
     B)\seq\Delta}{\Gamma,A\seq\Delta \quad&\quad \Gamma\seq \Delta,
     B}}
\]
%\smallskip

\begin{center}
\begin{tabular}{c@{\qquad}c}
  \multicolumn{2}{c}{$\Rules_{\VV_\cless} = \{R_{n,m} \mid n\geq 1, m\geq 0\}$
    }\\
\begin{tabular}{lll}
$\Rules_{\VV\NN_\cless}$ & = & $\{ R_{n,m} \mid n+m \geq 1\}$\\
$\Rules_{\VV\TT_\cless}$ & = & $\Rules_{\VV_\cless}\cup \{T_m \mid m \geq 1\}$\\
$\Rules_{\VV\WW_\cless}$ & = &$\Rules_{\VV_\cless}\cup \{W_{n,m}\mid n
+ m \geq 1\}\quad$
\end{tabular} &
\begin{tabular}{lll}
$\Rules_{\VV\CC_\cless}$ & = & $\Rules_\VV\cup\{R_{C1},R_{C2}\}$\\
$\Rules_{\VA_\cless}$ & = & $\{ A_{n,m} \mid n\geq 1, m\geq 0 \}$\\
$\Rules_{\VNA_\cless}$ & = & $\{ A_{n,m} \mid n + m\geq 1 \}$
\end{tabular}
\end{tabular}
\end{center}

\end{small}
\end{calculus}


\begin{clarifications}
  Sequents are based on multisets. The rules $\Rules_{\mathcal{L}_\cless}$ 
  form a calculus for a counterfactual logic $\mathcal{L}$ described 
  in \cite{Lewis:1973uq}, where $\cless$ is the \emph{comparative plausibility} 
  operator. Besides the rules shown above, these calculi also include the 
  propositional rules of
  \Gtc \iref{G3c} and contraction rules. The contexts $\Gamma^\cless$ and \
  $\Delta^\cless$ contain all formulae of resp. $\Gamma$ and \ $\Delta$ of
  the form $A \cless B$.
\end{clarifications}

\begin{history}
  The calculus for $\VV\CC$ was introduced in the tableaux setting
  \cite{Swart:1983uq,Gent:1992p3090}. The remaining calculi were
  introduced in \cite{Lellmann:2012fk,Lellmann:2013fk} and corrected
  in \cite{Lellmann:2013}.
\end{history}

\begin{technicalities}
  Soundness and completeness are shown by proving equivalence to 
  Hilbert-style calculi and (syntactical) cut elimination. 
  These calculi yield $\mathsf{PSPACE}$ decision
  procedures ($\mathsf{EXPTIME}$ for $\VA_\cless$ and
  $\VNA_\cless$) and, in most cases, enjoy Craig Interpolation. 
  Contraction can be made admissible. %\nocite{Lellmann:2012fk,Lellmann:2013}
\end{technicalities}

\vspace{-5pt}

\end{entry}

