\calculusName{Two-sided Linear Sequent Calculus}   
\calculusAcronym{\TLL}     
%\calculusLogic{Classical Linear Logic} 
\calculusLogic{Linear Logics} 
\calculusLogicOrder{Propositional}
\calculusType{Sequent Calculus}   
\calculusYear{1992}   
\calculusAuthor{Jean-Yves Girard} 

\entryTitle{Two-sided Linear Sequent Calculus}        
\entryAuthor{Elaine Pimentel} \entryAuthor{Harley Eades III}  


\tag{Two-Sided Sequents}

\maketitle


\begin{entry}{TLL} 
\newcommand{\one}{\mathbf{1}}
\newcommand{\zero}{\mathbf{0}}
\newcommand\bang{\mathop{!}}
\newcommand\quest{\mathord{?}}
\newcommand\limp{\mathbin{-\hspace{-0.70mm}\circ}}
\newcommand\tensor\otimes
%% \newcommand\with{\mathbin{\&}}



\newcommand{\llimp}{\multimap}

\begin{calculus}
\small
\[
\begin{array}{cccccccc}
  \infer[Init]{B \vdash B}{}
  &
  \quad
  &
  \infer[Cut]{\Gamma, \Gamma' \vdash \Delta,\Delta'}{\Gamma \vdash B \mid \Delta & \Gamma', B \vdash \Delta'}
  &
  \quad
  &
  \infer[\one_L]{\Gamma,\one \vdash\Delta}{\Gamma \vdash \Delta}\\
  \\
  \infer[\one_R]{ \vdash \one}{}
  &&
  \infer[\bot_L]{\bot\vdash}{}
  &&
  \infer[\bot_R]{\Gamma\vdash\bot, \Delta}{\Gamma \vdash  \Delta}\\
  \\
  \infer[\llimp_L]{\Gamma, A \llimp B, \Delta \vdash C}{\Gamma \vdash A \vdash B, \Delta \vdash C}
  &&
  \infer[\llimp_R]{\Gamma \vdash A \llimp B}{\Gamma, A \vdash B}
  &&
  \infer[\otimes_L]{\Gamma,B\otimes C\vdash \Delta}{\Gamma,B,C\vdash \Delta}\\
  \\
  \infer[\otimes_R]{\Gamma_1,\Gamma_2 \vdash B\otimes C, \Delta_1,\Delta_2}{\Gamma_1 \vdash B,\Delta_1\quad\Gamma_2 \vdash C,\Delta_2}
  &&
  \infer[\bindnasrepma_L]{\Gamma_1,\Gamma_2,B\bindnasrepma C \vdash \Delta_1,\Delta_2}{\Gamma_1,B\vdash \Delta_1 \quad \Gamma_2,C\vdash \Delta_2}
  &&
  \infer[\bindnasrepma_R]{\Gamma \vdash B \bindnasrepma C, \Delta}{\Gamma \vdash B,C,\Delta}\\
  \\
  \infer[\zero_L]{\Gamma,\zero \vdash\Delta}{}
  &&
  \infer[\top_R]{\Gamma \vdash\top, \Delta}{}
  &&
  \infer[\with_L\;(i=1,2)]{\Gamma,B_1\with B_2\vdash \Delta}{\Gamma,B_i\vdash \Delta}\\
  \\
  \infer[\with_R]{\Gamma \vdash B\with C, \Delta}{\Gamma \vdash B,\Delta\quad\Gamma \vdash C,\Delta}
  &&
  \infer[\oplus_L]{\Gamma,B\oplus C \vdash \Delta}{\Gamma,B\vdash \Delta \quad \Gamma,C\vdash \Delta}
  &&
  \infer[\oplus_R\;(i=1,2)]{\Gamma \vdash B_1\oplus B_2, \Delta}{\Gamma \vdash B_i,\Delta}\\
  \\
  \infer[\forall_L]{\Gamma,\forall x.B\vdash \Delta}{\Gamma,B[t/x]\vdash \Delta}\quad
  &&
  \infer[\forall_R]{\Gamma \vdash \forall x.B, \Delta}{\Gamma \vdash B[y/x],\Delta}
  &&
  \infer[\exists_L]{\Gamma,\exists x.B\vdash \Delta}{\Gamma,B[y/x]\vdash \Delta}\quad\\
  \\
  \infer[\exists_R]{\Gamma \vdash \exists x.B, \Delta}{\Gamma \vdash B[t/x],\Delta}
  &&
  \infer[\quest_L]{\bang\Gamma,\quest B\vdash \quest\Delta}{\bang\Gamma,B\vdash \quest\Delta}\quad
  &&
  \infer[\bang_R]{\bang\Gamma\vdash \bang B,\quest\Delta}{\bang\Gamma\vdash B, \quest\Delta}\\
  \\
  \infer[\quest_W]{\Gamma\vdash \quest B,\Delta}{\Gamma\vdash \Delta}\quad
  &&
  \infer[\quest_C]{\Gamma\vdash \quest B,\Delta}{\Gamma\vdash \quest B,\quest B, \Delta}\quad
  &&
  \infer[\quest_D]{\Gamma\vdash \quest B,\Delta}{\Gamma\vdash B, \Delta}\\
  \\
  \infer[\bang_W]{\Gamma,\bang B\vdash \Delta}{\Gamma\vdash \Delta}\quad
  &&
  \infer[\bang_C]{\Gamma,\bang B\vdash \Delta}{\Gamma,\bang B,\bang B\vdash \Delta}\quad
  &&
  \infer[\bang_D]{\Gamma,\bang B\vdash \Delta}{\Gamma, B\vdash \Delta}
\end{array}
\]
\end{calculus}



\begin{clarifications}
This is an alternate formalization of the sequent style formalization
of Linear Logic \iref{LL}.
\end{clarifications}

\begin{history}
This formalization first appeared in \cite{Troelstra:1992}.
\end{history}

%% \begin{technicalities}
%%   $\FILL$ enjoys cut elimination.  It also has a categorical model in
%%   dialectica categories \cite{dePaiva:1990}.
%% \end{technicalities}



\end{entry}
