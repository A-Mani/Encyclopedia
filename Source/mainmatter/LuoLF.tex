% If the calculus has an acronym, define it.
	% (e.g. \newcommand{\LK}{\ensuremath{\mathbf{LK}}\xspace})
	\newcommand{\LuoLF}{\ensuremath{\mathbf{LF}}\xspace}
	\calculusName{Typed LF for Type Theories} % The name of the calculus
	\calculusAcronym{LF} % The acronym if defined above, or empty otherwise.
	\calculusLogic{Type Theory} % Specify the logic (e.g. classical, intuitionistic, ...) for which this calculus is intended.
	\calculusLogicOrder{Higher-Order}
	%\calculusType{Logical Framework}
	\calculusType{Natural Deduction}
	\calculusYear{1994} % The year when the calculus was invented.
	\calculusAuthor{\iCA{Zhaohui Luo}} % The name(s) of the author(s) of the calculus.
	
	
	\entryTitle{Typed LF for Type Theories} % Title of the entry (usually coincides with the name of the calculus).
	\entryAuthor{\iA{Zhaohui Luo}, \iA{Sergei Soloviev}}  
	
	
	% If you wish, use tags to give any other information
	% that might be helpful for classifying and grouping this entry:
	% e.g. \tag{Two-Sided Sequents}
	% e.g. \tag{Multiset Cedents}
	% e.g. \tag{List Cedents}
	% You are free to invent your own tags.
	% The Encyclopedia's coordinator will take care of
	% merging semantically similar tags in the future.
	
	
	\maketitle
	
	
	% If your files are called "<ID>.tex" and "<ID>.bib",
	% then you should write "\begin{entry}{<ID>}" in the line below
	\begin{entry}{LuoLF}
	
	% Define here any newcommands you may need:
	% e.g. \newcommand{\necessarily}{\Box}
	% e.g. \newcommand{\possibly}{\Diamond}
	
	
	\begin{calculus}
	
	% Add the inference rules of your proof system here
\centering
$$
\infer{<>\seq {\bf valid}}{}\!\quad\!\infer{\Gamma, x:K\seq {\bf valid}}{\Gamma\seq K\,{\bf kind} & x\notin FV(\Gamma)}\quad\!\infer{\Gamma, x:K, \Gamma'\seq x:K}{\Gamma, x:K, \Gamma'\seq {\bf valid}}\!\quad\! (1)
$$
$$
\infer{\Gamma\seq k:K'}{\Gamma\seq k:K & \Gamma\seq K=K'}\!\quad\!
\infer{\Gamma\seq k=k':K'}{\Gamma\seq k=k':K & \Gamma\seq K=K'}\!\quad\! (2)^*
$$
$$\infer{\Gamma, [k/x]\Gamma'\seq [k/x]J}{\Gamma, x:K, \Gamma'\seq J & 
\Gamma\seq k:K}\!\quad\! (3)^{**}$$
$$
\infer{\Gamma\seq (x:K)K' \, {\bf kind}}{\Gamma\seq K \,{\bf kind} & 
\Gamma, x:K\seq K'\,{\bf kind}}\!\quad\!
\infer{\Gamma\seq (x:K_1)K_1'=(x:K_2)K_2'}{\Gamma\seq K_1=K_2 & 
\Gamma, x:K_1\seq K_1'=K_2'}
$$
$$
\infer{\Gamma\seq [x:K]k:(x:K)K'}{\Gamma, x:K\seq k:K'}\!\quad\!
\infer{\Gamma\seq [x:K_1]k_1=[x:K_2]k_2:(x:K_1)K}{\Gamma\seq K_1=K_2 & 
\Gamma, x:K_1\seq k_1=k_2:K}
$$
$$
\infer{\Gamma\seq f(k):[k/x]K'}{\Gamma\seq f:(x:K)K' & 
\Gamma\seq k:K}\!\quad\!
\infer{\Gamma\seq f(k_1) = f'(k_2):[k_1/x]K'}{\Gamma\seq f=f':(x:K)K' & 
\Gamma\seq k_1=k_2:K}$$ 
$$
\infer{\Gamma\seq ([x:K]k')(k)=[k/x]k':[k/x]K'}{\Gamma, x:K\seq k': K'& 
\Gamma\seq k:K}\!\quad\!
\infer{\Gamma\seq [x:K]f(x)=f:(x:K)K'}{\Gamma\seq f:(x:K)K' & 
x\notin FV(f)}\!\quad\!(4)
$$
$$
\infer{\Gamma\seq {\bf Type\, kind}}{\Gamma\seq{\bf valid}}\!\quad
\!\infer{\Gamma\seq El(A)\,{\bf kind}}{\Gamma\seq A:{\bf Type}}\!\quad\! (5)
$$


	% The "proof.sty" and "bussproofs.sty" packages are available.
	% If you need any other package, please contact the editor (bruno@logic.at)
	
	%ToDo
	
	\end{calculus}
	
	% The following environments ("clarifications", "history",
	% "technicalities") are optional. If you do use them,
	% be very concise and objective.
	
	\begin{clarifications}
We follow~\cite{Luo:94}.  
Terms of $\LuoLF$ are of the forms ${\bf Type}, El(A)$, $(x:K)K'$
(dependent product), $[x:K]K'$ (abstraction), $f(k)$, and judgements
of the forms $\Gamma\seq{\bf valid}$ (validity of context),
$\Gamma\seq K {\bf kind}$, $\Gamma\seq k:K$, $\Gamma\seq k=k':K$, $\Gamma\seq K=K'$. 
Rule groups: (1) rules for contexts and assumptions; (2)* equality rules (reflexivity, symmetry and transitivity rules are ommitted); (3)** substitution rules ($J$ denotes the right side of any of the five forms of judgement); (4) rules for dependent product kinds; (5) and the kind {\bf Type}. 
	\end{clarifications}
	
	\begin{history}
First defined in~\cite{Luo:94}, ch. 9, $\LuoLF$ is a typed version of Martin-L\"of's logical framework~\cite{NPS:90}\irefmissing{Martin-L\"of's LF}. In difference
from Edinburgh LF\irefmissing{Edinburgh LF} it may be used to specify type theories.
{\em E.g.}, theories specified in $\LuoLF$ were used as basis of proof-assistants Lego and Plastic. Later the system was extended to include coercive subtyping~\cite{Luo:99, SolLuo:02, LuoSolXue:13}\irefmissing{LF with coercive subtyping}.
	% ToDo: write here short historical remarks about this proof system,
	% especially if they relate to other proof systems.
	% Use "\iref{OtherProofSystem}" to refer to another proof system
	% in the Encyclopedia (where "OtherProofSystem" is its ID).
	% Use "\irefmissing{SuggestedIDForOtherProofSystem}" to refer to
	% another proof system that is not yet available in the encyclopedia.
	 \end{history}
	
	\begin{technicalities}
	The proof-theoretical analysis of $\LuoLF$ above was used in meta-theoretical studies of larger theories defined on its basis, {\em e.g.}, UTT (Unifying Theory of dependent Types) that includes inductive schemata, second order logic SOL with impredicative type $Prop$ and a hierarchy of predicative universes~\cite{Luo:94}. H. Goguen defined a typed operational semantics for UTT and proved strong normalization theorem~\cite{HG:94}. For $\LuoLF$ with coercive subtyping conservativity results were obtained~\cite{Luo:99, SolLuo:02, LuoSolXue:13}.
	\end{technicalities}
	
	
	
	% Please cite the original paper where the proof system was defined.
	% To do so, you may use the \cite command within
	% one of the optional environments above,
	% or use the \nocite command otherwise.
	
	% You may also cite a modern paper or book where the
	% proof system is explained in greater depth or clarity.
	% Cite parsimoniously.
	
	% Do not cite related work. Instead, use the "\iref" or "\irefmissing"
	% commands to make an internal reference to another entry,
	% as explained within the "history" environment above.
	
	% You do not need to create the "References" section yourself.
	% This is done automatically.
	
	
	% Leave an empty line above "\end{entry}".
	
	\end{entry}
