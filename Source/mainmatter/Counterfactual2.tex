
\calculusName{Counterfactual Sequent Calculi II} 
\calculusAcronym{} 
%\calculusLogic{Lewis' Counterfactual Logics}
\calculusLogic{Counterfactual Logics}
\calculusLogicOrder{Propositional}
\calculusType{Sequent Calculus}
\calculusYear{2012, 2013} 
\calculusAuthor{\iCA{Lellmann}, \iCA{Pattinson}} 

\entryTitle{Counterfactual Sequent Calculi II}
\entryAuthor{\iA{Bj{\"o}rn Lellmann}} 

\maketitle


\begin{entry}{Counterfactual2}


\newcommand{\nc}{\newcommand}
\nc{\rarr}{\rightarrow}
\nc{\scimp}{\boxRight} 
\nc{\CC}{\mathbb{C}}
\nc{\NN}{\mathbb{N}}
\newcommand{\Rules}{\mathcal{R}}
\nc{\TT}{\mathbb{T}}
\nc{\VV}{\mathbb{V}}
\nc{\WW}{\mathbb{W}}


\begin{calculus}
\[
\vcenter{
  \infer[R_{n,m}]
   {\Gamma,(A_1\scimp B_1),\dots,(A_n\scimp B_n)\seq
        \Delta,(C_1\scimp D_1),\dots,(C_m\scimp D_m)}
   {\begin{array}{c}
           \big\{\;C_k, \vec{B}^I \seq \vec{A}^{[n]\smallsetminus I},
           \vec{C}^J, \vec{D}^{[k-1]\smallsetminus J} \mid 1\leq k\leq m,\,
           I\subseteq [n],\, J\subseteq [k-1]\;\big\}\\
           \cup\;\big\{\; A_k,B_k, \vec{B}^I \seq
           \vec{A}^{[n]\smallsetminus I}, \vec{C}^J,
           \vec{D}^{[m]\smallsetminus J}\mid k\leq n, I\subseteq [n],
           J\subseteq [m]\;\big\} 
         \end{array}
  }
}
\]

\[
\vcenter{
  \infer[T_m]
  {\Gamma \seq
        \Delta,(C_1\scimp D_1),\dots,(C_m\scimp D_m)
  }
  {\big\{\;
    \Gamma \seq \Delta,\vec{C}^J, \vec{D}^{[m]\smallsetminus J} \mid J
    \subseteq [m]
    \;\big\}
    \;\cup\;
    \big\{\;
    C_k \seq D_k, \vec{C}^J, \vec{D}^{[k-1]\smallsetminus J} \mid 1
    \leq k \leq m, J \subseteq [k-1]
    \;\big\}
  }
}
\]

\[
\vcenter{
  \infer[W_{n,m}]
    {\Gamma,(A_1\scimp B_1),\dots,(A_n\scimp
      B_n)\seq\Delta,(C_1\scimp D_1),\dots,(C_m\scimp D_m)}
    {\begin{array}{c}
      \big\{\;C_k,\vec{B}^I \seq \vec{A}^{[n]\smallsetminus I},
           \vec{C}^{J},\vec{D}^{[k-1]\smallsetminus J} \mid 1 \leq k\leq m,\,
           I\subseteq [n],\, J\subseteq [k-1]\;\big\} \\
        \cup \; \big\{\;\Gamma,\vec{B}^I \seq
        \vec{A}^{[n]\smallsetminus I},
        \vec{C}^J,\vec{D}^{[m]\smallsetminus J} \mid I\subseteq
        [n], J\subseteq [m]\;\big\}
      \end{array}
  }
}
\]

\[
\vcenter{
  \infer[R_{C1}]
    {\Gamma,(A\scimp B)\seq\Delta
    }
    {\Gamma\seq\Delta,A \quad&\quad \Gamma,B\seq\Delta
    }
}
\quad
\vcenter{
  \infer[R_{C2}]
      {\Gamma\seq\Delta,(A\scimp B)
      }
      {\Gamma\seq\Delta,A\quad &\quad
        \Gamma,A\seq\Delta,B
      }
}
\]
\centerline{\small For $n>0$ the set $[n]$ is $\{1,
  \dots, n \}$ and $[0]$ is $\emptyset $. For a set $I$ of indices,
  $\vec{A}^I$ contains all $A_i$ with $i \in I$.
  }\\

\begin{center}
\begin{tabular}{c@{\qquad}c}
\multicolumn{2}{c}{
    $\Rules_{\VV_\scimp} = \{R_{n,m} \mid n\geq 1, m\geq 0\}$
 }\\
\begin{tabular}{lll}
$\Rules_{\VV\NN_\scimp}$ & = & $\{ R_{n,m} \mid n+m \geq 1\}$\\
$\Rules_{\VV\TT_\scimp}$ & = & $\Rules_{\VV_\scimp} \cup \{ T_m \mid m
\geq 1\}$\\
\end{tabular}
&
\begin{tabular}{lll}
$\Rules_{\VV\WW_\scimp}$ & = &$\Rules_{\VV\TT_\scimp}\cup
\{W_{n,m}\mid n+m \geq 1\}$\\
$\Rules_{\VV\CC_\scimp}$ & = &
$\Rules_{\VV_\scimp}\cup\{R_{C1},R_{C2}\}$\\
\end{tabular}
\end{tabular}
\end{center}


\end{calculus}


\begin{clarifications}
  Sequents are based on multisets. The rules $\Rules_{\mathcal{L}_\scimp}$ 
  form a calculus for a counterfactual logic $\mathcal{L}$ described 
  in \cite{Lewis:1973uq}, where $\scimp$ is the 
  \emph{strong counterfactual implication} operator. 
  Besides the rules shown above, these calculi also include the 
  propositional rules of \Gtc \iref{G3c} and contraction rules.
\end{clarifications}

\begin{history}
  These calculi were introduced in \cite{Lellmann:2012fk} and corrected in
  \cite{Lellmann:2013}.
\end{history}

\begin{technicalities}
  The calculi are translations of the calculi in \iref{Counterfactual} to the language with $\scimp$. They inherit cut elimination and yield $\mathsf{PSPACE}$
  decision procedures. Contraction can be made admissible.
\end{technicalities}


\end{entry}

