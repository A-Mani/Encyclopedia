\calculusName{Sequent Calculus for Superintuitionistic Modal Logic KMlin}
\calculusAcronym{\seqlcnxt}
\calculusLogic{Modal Logics}
\calculusLogicOrder{Propositional}
\calculusType{Sequent Calculus} 
\calculusYear{2014} 

\calculusAuthor{Ranald Clouston}
\calculusAuthor{Rajeev Gor\'e}

\entryTitle{Sequent Calculus $\seqlcnxt$ for Superintuitionistic Modal Logic KMlin} 
\entryAuthor{Rajeev Gor\'e}
\tag{Two-Sided Sequents}
\tag{Set Cedents}
\tag{Multi-Conclusion Succedent}
\tag{Superintuitionistic}


\maketitle
\begin{entry}{SKMlin}

\newcommand{\iimp}{\twoheadrightarrow}
\newcommand{\limp}{\rightarrow}
\newcommand{\opset}[2]{{#2}^{#1}}
\newcommand{\nxt}{\rhd}
\newcommand{\nxtlist}[2]{\nxt {#1}_{1} , \cdots ,  \nxt {#1}_{#2}}
\newcommand{\opsetmi}[3]{\opset{#1}{#2}_{-#3}}
\newcommand{\implist}[5]{#2_{#4} #1 {#3}_{#4} , \cdots ,  #2_{#5}
  #1 {#3}_{#5}}
\newcommand{\condzero}{\mathrm{C0}}
\newcommand{\condone}{\mathrm{C1}}
\newcommand{\condtwo}{\mathrm{C2}}
\newcommand{\oplist}[4]{#1 {#2}_{#3} , \cdots ,  #1 #2_{#4}}

\newcommand{\idrulename}{\small id}
\newcommand{\idrule}{
\AxiomC{}
\LeftLabel{\idrulename}
\UnaryInfC{$\Gamma, \varphi \seq \varphi, \Delta $}
\DisplayProof
}

\newcommand{\toprightrulename}{\small \top \mathrm{R}}
\newcommand{\toprule}{
\AxiomC{}
\LeftLabel{$\toprightrulename$}
\UnaryInfC{$\Gamma \seq \top, \Delta $}
\DisplayProof
}
\newcommand{\botleftrulename}{\small \bot \mathrm{L}}
\newcommand{\botrule}{
\AxiomC{}
\LeftLabel{$\botleftrulename$}
\UnaryInfC{$\Gamma , \bot \seq \Delta $}
\DisplayProof
}
\newcommand{\orleftrulename}{\small \lor \mathrm{L}}
\newcommand{\orleftrule}{
\AxiomC{$\Gamma , \varphi \seq \Delta $}
\AxiomC{$\Gamma , \psi \seq  \Delta $}
\LeftLabel{$\orleftrulename$}
\BinaryInfC{$\Gamma, \varphi \lor \psi \seq \Delta $}
\DisplayProof
}

\newcommand{\orrightrulename}{\small \lor \mathrm{R}}
\newcommand{\orrightrule}{
\AxiomC{$\Gamma \seq  \varphi, \psi , \Delta $}
\LeftLabel{$\orrightrulename$}
\UnaryInfC{$\Gamma  \seq \varphi\lor\psi , \Delta $}
\DisplayProof
}

\newcommand{\andrightrulename}{\small \land \mathrm{R}}
\newcommand{\andrightrule}{
\AxiomC{$\Gamma  \seq \varphi, \Delta $}
\AxiomC{$\Gamma  \seq \psi,  \Delta $}
\LeftLabel{$\andrightrulename$}
\BinaryInfC{$\Gamma \seq \varphi\land\psi , \Delta $}
\DisplayProof
}

\newcommand{\andleftrulename}{\small \land \mathrm{L}}
\newcommand{\andleftrule}{
\AxiomC{$\Gamma , \varphi , \psi \seq \Delta $}
\LeftLabel{$\andleftrulename$}
\UnaryInfC{$\Gamma , \varphi\land\psi \seq \Delta $}
\DisplayProof
}
\newcommand{\impleftrulename}{\small \limp\!\!\mathrm{L}}
\newcommand{\impleftrule}{
\AxiomC{$\Gamma  , \varphi\iimp\psi  \seq \varphi, \Delta $}
\AxiomC{$\Gamma , \varphi\iimp\psi, \psi \seq  \Delta $}
\LeftLabel{$\impleftrulename$}
\BinaryInfC{$\Gamma , \varphi\limp\psi \seq \Delta $}
\DisplayProof
}

\newcommand{\imprightrulename}{\small \limp\!\!\mathrm{R}}
\newcommand{\imprightrule}{
\AxiomC{$\Gamma , \varphi \seq \psi ,  \Delta$}
\AxiomC{$\Gamma  \seq \varphi\iimp\psi ,\Delta$}
\LeftLabel{$\imprightrulename$}
\BinaryInfC{$\Gamma \seq \varphi\limp\psi , \Delta$}
\DisplayProof
}

\newcommand{\steprulename}{\textsc{step}}
\newcommand{\newsteprulenewversion}{
   \begin{tabular}[c]{l@{\extracolsep{2cm}}l}
   \multicolumn{2}{c}{
   \AxiomC{$\mathrm{Prem}_1 
         \quad \cdots \quad 
         \mathrm{Prem}_k 
         \quad 
         \mathrm{Prem}_{k+1} 
         \quad \cdots
         \quad \mathrm{Prem}_{k+n}$}
   \LeftLabel{$\steprulename$}
%   \RightLabel{$\opset{\iimp}{\Delta} \cup \opset{\nxt}{\Phi}  \neq
%   \emptyset$}
   \RightLabel{$\dagger$}
   \UnaryInfC{$\Sigma_l, \opset{\nxt}{\Theta} ,\opset{\iimp}{\Gamma} 
         \seq 
        \opset{\iimp}{\Delta},
        \opset{\nxt}{\Phi},
         \Sigma_r$}
   \noLine
   \UnaryInfC{
     \begin{tabular}[c]{lll}
     \\%[5px]
      $\qquad\qquad
      \mathrm{Prem}_{1 \leq i \leq k}$ 
      & $=$
      & $\Sigma_l, 
     \Theta , 
     \opset{\nxt}{\Theta},
     \opset{\limp}{\Gamma} ,
     \varphi_i \iimp \psi_i , 
     \varphi_i 
     \seq
     \psi_i,
     \opsetmi{\limp}{\Delta}{i},
     \Phi$
   \\[5px]
   $\qquad\qquad
    \mathrm{Prem}_{k+1 \leq i \leq k+n}$
   & $=$
   & $\Sigma_l, 
    \Theta, 
     \opset{\nxt}{\Theta},
    \opset{\limp}{\Gamma},
    \nxt \phi_{i-k} 
    \seq
    \opset{\limp}{\Delta},
    \Phi$
   \end{tabular}
}
\DisplayProof
}
%\end{tabular}
 \\%[5px]
%    \begin{tabular}[c]{ll}
 \\%[5px]
%  --$\quad$
  $\opset{\nxt}{\Theta} = \nxtlist{\theta}{j}$
    &
  $\opset{}{\Theta} = \oplist{}{\theta}{1}{j}$
   \\[2px]
%    --$\quad$
  $\opset{\iimp}{\Gamma} = \{\implist{\iimp}{\alpha}{\beta}{1}{l}\}$
      &
  $\opset{\limp}{\Gamma} = \{\implist{\limp}{\alpha}{\beta}{1}{l}\}$
   \\[2px]
%    --$\quad$
  $\opset{\iimp}{\Delta} = \{\implist{\iimp}{\varphi}{\psi}{1}{k}\}$
      &
  $\opset{\limp}{\Delta} = \{\implist{\limp}{\varphi}{\psi}{1}{k}\}$
\\[2px]
%   --$\quad$
%   $\opset{\iimp}{\Delta} \cup \opset{\nxt}{\Phi}  \neq \emptyset$
%    &
  $\opsetmi{\limp}{\Delta}{i} = 
     \opset{\limp}{\Delta}\setminus 
     \{\varphi_i\limp\psi_i\}
  $
   \\[2px]
%    --$\quad$
  $\opset{\nxt}{\Phi} = \nxt\phi_1 , \cdots , \nxt\phi_n$
    &
  $\opset{}{\Phi} = \phi_1 , \cdots , \phi_n $
   \\[5px]
   \multicolumn{2}{l}{
   where $\dagger$ means that the conditions $\condzero$, $\condone$ 
   and $\condtwo$ below must hold
  }
\\[2px]
   \multicolumn{2}{l}{
($\condzero$)   $\opset{\iimp}{\Delta} \cup \opset{\nxt}{\Phi}  \neq \emptyset$
  }
\\[2px]
   \multicolumn{2}{l}{
($\condone$)    
    $\bot \not\in\Sigma_l$ and $\top\not\in\Sigma_r$ and 
    $(\Sigma_l \cup \opset\nxt\Theta \cup \Gamma^\iimp ) \cap 
       (\opset{\iimp}{\Delta} 
        \cup 
        \opset{\nxt}{\Phi} 
        \cup 
        \Sigma_r)  = \emptyset$
   }
   \\[2px]
   \multicolumn{2}{l}{
($\condtwo$) 
    $\Sigma_l$ and $\Sigma_r$ each contain atomic formulae only 
   }
  \\[5px]
   \multicolumn{2}{l}{
   Explanations for the conditions:}
   \\[2px]
   \multicolumn{2}{l}{
   ($\condzero$) there is at least one $\nxt$- or $\iimp$-formula in the
   succedent of the conclusion}
   \\[2px]
   \multicolumn{2}{l}{
   ($\condone$) none of the rules
       $\botleftrulename, \toprightrulename, \idrulename$ 
       are applicable to the conclusion}
   \\[2px]
   \multicolumn{2}{l}{
   ($\condtwo$) none of the rules $\orleftrulename, \orrightrulename,
   \andleftrulename, \andrightrulename, \impleftrulename,
   \imprightrulename$ are applicable to the conclusion}
    \end{tabular}
}

\begin{calculus}
% Add the inference rules of your proof system here.
% The "proof.sty" and "bussproofs.sty" packages are available.
% If you need any other package, please contact the editor (bruno@logic.at)
  \begin{tabular}[c]{l@{\extracolsep{0.5cm}}l}
    \multicolumn{2}{l}{
      $\quad\toprule \qquad\qquad \idrule \qquad\qquad \botrule$}
\\[15px]
    \orleftrule &  \orrightrule
\\[15px]
    \andleftrule  & \andrightrule
\\[15px]
    \impleftrule & \imprightrule
\\[15px]
    \multicolumn{2}{l}{
    \newsteprulenewversion
    }
  \end{tabular}
\end{calculus}
\begin{clarifications}
  There is a unary modal connective $\nxt$ to be read as ``later''.
  Its semantics are box-like in terms of the underlying intuitionistic
  Kripke relation, which is irreflexive!  There is also a new
  connective $\iimp$ corresponding to an irreflexive version of
  intuitionistic implication, which can be defined as
  $\nxt(\varphi\limp\psi)$.  The $\impleftrulename$ rule is \LK-like
  \iref{GentzenLK} in that it is multi-conclusioned and has one branch
  for each subformula of $\varphi\land\psi$, but it also converts
  $\varphi\limp\psi$ to $\varphi\iimp\psi$, read upwards, building in
  a form of contraction on such formulae.  The $\imprightrulename$
  rule is unusual in that it has two premises: the left one is LK-like
  in that it does not delete $\Delta$, read upwards, while the right
  one converts $\varphi\limp\psi$ to
  $\varphi\iimp\psi$ read upwards building in a form of contraction on
  such formulae. The $\steprulename$ rule has an indeterminate number
  of premises, one for each
  $\varphi_i\iimp\psi_i \in \opset{\iimp}{\Delta}$, and one for each
  $\varphi_i \in \opset{\nxt}{\Phi}$. For each such ``eventuality'',
  the rule creates a premise that contains the subformula on an
  appropriate side, but also creates a copy of the principal formula
  in the antecedent of that premise, thus building in aspects of the
  standard sequent calculus for G\"oedel-L\"ob logic.
\end{clarifications}

\begin{history}
The superintuitionistic modal logic KMlin is obtained from
Kuznetsov-Muravitsky logic KM~\cite{Litak:Constructive} by demanding
that the underlying Kripke frames be linear. The semantics of the
unary modality $\nxt$ becomes ``true in all strict successors''.

Clouston and Gor\'e~\cite{DBLP:conf/fossacs/CloustonG15} defined this
sequent calculus. Their rules are inspired by those of:
Mauro Ferrarri, Camillo Fiorentini and Guido Fiorino for a sequent
calculus with compartments for intuitionistic
logic~\cite{DBLP:journals/jar/FerrariFF13}; Giovanna Corsi for her
sequent calculus for (quantified) G\"odel-Dummett logic
LC~\cite{Corsi89};
and George
Boolos for his sequent calculus for Go\"del-L\"ob logic
GL~\cite{boolos-provability-logic}.

\end{history}

\begin{technicalities}
  Clouston and Gor\'e gave semantic proofs of soundness, cut-free
  completeness and the finite model property, thus giving
  decidability. They showed that the validity problem for this logic
  is coNP-complete. They also showed that all rules are invertible, so
  the sequent calculus can be used for backtrack-free and terminating
  backward proof search via the following strategy for rule
  applications: apply any applicable rule backwards, always preferring
  zero-premise rules if possible!
\end{technicalities}

\end{entry}