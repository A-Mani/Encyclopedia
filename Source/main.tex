\documentclass[graybox, envcountchap]{svmult}

\usepackage{etoolbox}


% Poster Production
% =================
% The following commands should be enabled for poster production only.
% Posters should be printed with 125% scale to trim margins.
\usepackage[hmarginratio=1:1,vmarginratio=1:1]{geometry}
%\pagestyle{empty} % removes page numbers


% Book Production
% ===============
% The following commands should be enabled for book production only.
\newcommand{\pagenumberstyle}{plain} % centers page numbers
\pagestyle{\pagenumberstyle}

% Font Packages
% =============
\usepackage{anyfontsize}     % allows arbitrary font sizes
\usepackage{mathptmx}        % selects Times Roman as basic font
\usepackage{helvet}          % selects Helvetica as sans-serif font
\usepackage{courier}         % selects Courier as typewriter font
\usepackage{txfonts}


% LaTeX Layout and Functionality Packages and Commands
% ====================================================
\usepackage{hyperref}         % standard LaTeX hyperreferencing package
\usepackage{graphicx}         % standard LaTeX graphics package

\usepackage{xspace}           % fixes spacing after LaTeX commands

\usepackage{multicol}         % used for the two-column index
\usepackage[bottom]{footmisc} % places footnotes at page bottom
\DeclareMathVersion{mathhungry} % for entries that need too many math alphabets        


% General Mathematical Symbols
% ============================
\usepackage{ded,stmaryrd}
\usepackage{amssymb}


% Logical Symbols
% ===============
\usepackage{cmll} % symbols for linear logic


% Proof Drawing Packages
% ======================
\usepackage{proof}
\usepackage{qtree}
\usepackage{prooftree}
\usepackage[all]{xypic}
\usepackage{bussproof} \EnableBpAbbreviations 
% "bussproof" is a modified version of "bussproofs". 
% it avoids conflict with the "prooftree" package, 
% by defining the "bussprooftree" environment instead of "prooftree".





% Packages and Commands for Indexes
% =================================
\usepackage{multind}         % allows multiple index generation
\makeindex{logics}           % index for logics
\makeindex{calculusTypes}    % index for calculus types
\makeindex{authors}          % index for entry authors
\makeindex{calculusAuthors}  % index for calculus authors
               

% Packages and Commands for Bibliographies
% ========================================
\usepackage[sorting=ynt,maxnames=99]{biblatex}


% Custom Packages of the Encyclopedia
% ===================================
\usepackage{commands}       % Commands for the entry template
\usepackage{logicalsymbols} % LaTeX commands for logical symbols
\usepackage{acronyms}       % Acronyms for proof systems
\usepackage{bibliographies} % BibLaTeX inclusion statements for entry bibliographies


% Other Packages
% ==============
\usepackage{tikz} \usetikzlibrary{quotes,arrows.meta}
\usepackage{microtype} 





\begin{document}
\frontmatter

  

%\begin{Sbox}
%\begin{minipage}[b]{0.98\textwidth}

\begin{center}

\phantom{.}

\vspace{0pt}




{\fontsize{70}{90}\selectfont


Encyclopaedia\\
of\\
Proof\\ 
Systems\\
}

\vspace{120pt}
%\vfill

\begin{LARGE}
\url{http://ProofSystem.github.io/Encyclopedia/}
\end{LARGE}

\end{center}

%\end{minipage}
%\end{Sbox}
%\fbox{\TheSbox}

  %%%%%%%%%%%%%%%%%%%%%%%% dedic.tex %%%%%%%%%%%%%%%%%%%%%%%%%%
%
% sample dedication
%
% Use this file as a template for your own input.
%
%%%%%%%%%%%%%%%%%%%%%%%% Springer %%%%%%%%%%%%%%%%%%%%%%%%%%

\begin{dedication}
Use the template \emph{dedic.tex} together with the Springer document class SVMono for monograph-type books or SVMult for contributed volumes to style a quotation or a dedication\index{dedication} at the very beginning of your book in the Springer layout
\end{dedication}




  %%%%%%%%%%%%%%%%%%%%%%%foreword.tex%%%%%%%%%%%%%%%%%%%%%%%%%%%
% sample foreword
%
% Use this file as a template for your own input.
%
%%%%%%%%%%%%%%%%%%%%%%%% Springer %%%%%%%%%%%%%%%%%%%%%%%%%%

\foreword

Use the template \textit{foreword.tex} together with the Springer document class SVMono (monograph-type books) or SVMult (edited books) to style your foreword\index{foreword} in the Springer layout. 

The foreword covers introductory remarks preceding the text of a book that are written by a \textit{person other than the author or editor} of the book. If applicable, the foreword precedes the preface which is written by the author or editor of the book.


\vspace{\baselineskip}
\begin{flushright}\noindent
Place, month year\hfill {\it Firstname  Surname}\\
\end{flushright}



  
\preface

The \textbf{Encyclopedia of Proof Systems} aims at providing a reliable, technically informative, historically accurate, concise and convenient central repository of proof systems for various logics. The goal is to facilitate the exchange of information among logicians with mathematical, computational or philosophical backgrounds; in order to foster and accelerate the development of new proof systems and automated deduction tools that rely on them.

Preparatory work for the creation of the Encyclopedia, such as the implementation of the LaTeX template and the setup of the Github repository, started in October 2014, trigerred by the call for workshop proposals for the 25th Conference on Automated Deduction (CADE). Christoph Benzm\"uller, CADE's general chair, and Jasmin Blanchette, CADE's workshop co-chair, encouraged me to submit a workshop proposal and supported my alternative idea to organize instead a special poster session based on encyclopedia entries. I am thankful for their encouragement and support.

In December 2014, Bj\"orn Lellman, Giselle Reis and Martin Riener kindly accepted my request to beta-test the template and the instructions I had created. They submitted the first few example entries to the encyclopedia and provided valuable feedback, for which I am grateful. Their comments were essential for improving the templates and instructions before the public announcement of the encyclopedia.

 

\vspace{\baselineskip}
\begin{flushright}\noindent
January 2015\hfill {\it Bruno Woltzenlogel Paleo}
\end{flushright}



  %%%%%%%%%%%%%%%%%%%%%%%acknow.tex%%%%%%%%%%%%%%%%%%%%%%%%%%%%%%%%%%%%%%%%%
% sample acknowledgement chapter
%
% Use this file as a template for your own input.
%
%%%%%%%%%%%%%%%%%%%%%%%% Springer %%%%%%%%%%%%%%%%%%%%%%%%%%

\extrachap{Acknowledgements}

Use the template \emph{acknow.tex} together with the Springer document class SVMono (monograph-type books) or SVMult (edited books) if you prefer to set your acknowledgement section as a separate chapter instead of including it as last part of your preface.



  \tableofcontents
  


\mainmatter

  
\begin{partbacktext}
\part{\emph{Introductions}}

\noindent \begin{large}\textbf{ToDo:} \end{large}

\noindent We plan to add a few short introductory chapters, to address various aspects that are orthogonal to several entries, such as: 
\begin{itemize}
\item basic technical notions for each type of proof system (e.g. tableaux, natural deduction systems, sequent calculi, resolution calculi \ldots),
\item logical languages
\item logics (classical, intuitionistic, modal, substructural, linear, paraconsistent, \ldots)
\item application domains
\end{itemize}

\noindent
The goals of these introductory chapters will be to:
\begin{itemize} 
\item provide a global technical and historical view of the entries,
\item reduce repetition of basic notions in the entries,
\item increase the understandability of the entries,
\item make the encyclopedia more self-contained.
\end{itemize}

\noindent
The exact structure and content of these chapters will be decided later, after the collection of sufficiently many entries.

For now, entries are sorted in chronological order only, and various indexes are provided in the backmatter. If the need arises, we might consider grouping entries according to various criteria.


\end{partbacktext}

  
\part{\emph{Proof Systems}}

% When adding an entry for a proof system "X" to the encyclopedia, 
% an "\includeProofSystem" statement with the following
% format should be added in this file : 

%   \includeProofSystem{X}


% ======================================================
% Acronyms of Proof Systems in Chronological Order



\newcommand{\NJ}{\ensuremath{\mathbf{NJ}}\xspace} % GentzenNJ 1935
\newcommand{\LK}{\ensuremath{\mathbf{LK}}\xspace} % GentzenLK 1935
\newcommand{\LJ}{\ensuremath{\mathbf{LJ}}\xspace} % GentzenLJ 1935
\newcommand{\LJmc}{\ensuremath{\mathbf{LJ'}}\xspace} % MultiConclusionLJ 1954
\newcommand{\SystemF}{\ensuremath{\mathbf{F}}\xspace} % fc 1971
\newcommand{\Bledsoe}{\ensuremath{\mathbf{Prover}}\xspace} % Bledsoe 1984
\newcommand{\Muscadet}{\ensuremath{\mathbf{Muscadet}}\xspace} % Muscadet 1984
\newcommand{\LF}{\ensuremath{\mathbf{LF}}\xspace} % LuoLF 1994
\renewcommand{\lambdabar}{\overline{\lambda}} % LambdaBar 1994
\newcommand{\LJT}{\ensuremath{\mathbf{LJT}}\xspace} % LambdaBar 1994
\newcommand{\Gtc}{\ensuremath{\mathbf{G3c}}\xspace} % G3c 1996
\newcommand{\LKmumutilde}{\ensuremath{\mathbf{LK}_{\mu\tilde\mu}}\xspace} % LKMuMuTilde 2000
\newcommand{\LKT}{\ensuremath{\mathbf{LKT}\xspace}} % LKMuMuTilde 2000
\newcommand{\LKQ}{\ensuremath{\mathbf{LKQ}\xspace}} % LKMuMuTilde 2000
\newcommand{\LKmumutildetree}{\ensuremath{\mathbf{LK}_{\mu\tilde\mu}-\mathbf{tree}}\xspace} % LKMuMuTildeTree 2005
\newcommand{\calcoloP}{{\bf $\mathcal{T}P^{\bf T}$}} % KLM 2005
\newcommand{\LKF}{\ensuremath{\mathbf{LKF}}\xspace} % LKF 2007
\newcommand{\LJF}{\ensuremath{\mathbf{LJF}}\xspace} % LJF 2007
\newcommand{\FILL}{\ensuremath{\mathbf{FILL}}\xspace} % FILL 1990
\newcommand{\NDc}{\ensuremath{\mathbf{ND}^\mathbf{c}}\xspace} % ContextualND 2013
\newcommand{\seqlcnxt}{\ensuremath{\mathbf{SKM_{lin}}}\xspace}) % SKMlin 2014


% ======================================================
% Proof Systems in Chronological Order

\includeProofSystem{GentzenNJ} % 1935
\includeProofSystem{GentzenLK} % 1935
\includeProofSystem{GentzenLJ} % 1935
\includeProofSystem{MultiConclusionLJ} % 1954
\includeProofSystem{fc} % 1971
\includeProofSystem{ExpansionProofs} % 1983
\includeProofSystem{Bledsoe} % 1984
\includeProofSystem{Muscadet} % 1984
\includeProofSystem{FILL} % 1990
\includeProofSystem{LuoLF} % 1994
\includeProofSystem{LambdaBar} % 1994
\includeProofSystem{G3c} % 1996
\includeProofSystem{LKMuMuTilde} % 2000
\includeProofSystem{LKMuMuTildeTree} % 2005
\includeProofSystem{LabelledConditionals} % 2003-2007
\includeProofSystem{KLM} % 2005-2009
\includeProofSystem{GBetaFB} % 2004-2009

\mathversion{mathhungry}
\includeProofSystem{ExtResLEO2} % 2006-2013
\mathversion{normal}

\includeProofSystem{LKF} % 2007
\includeProofSystem{LJF} % 2007
\includeProofSystem{LambdaPiModulo} % 2007
\includeProofSystem{Counterfactual} % 1983-2013
\includeProofSystem{Counterfactual2} % 2012-2013
\includeProofSystem{NestedConditionals} % 2012-2014
\includeProofSystem{ContextualND} % 2013
\includeProofSystem{IR} % 2014
\includeProofSystem{SKMlin} % 2014



\backmatter

  \part{\emph{Indexes}}

  \appendix

  %%%%%%%%%%%%%%%%%%%%%clist.tex %%%%%%%%%%%%%%%%%%%%%%%%
%                                                    
% sample list of contributors and their addresses    
%                                                    
% Use this file as a template for your own input.    
%                                                    
%%%%%%%%%%%%%%%%%%%%%%%% Springer %%%%%%%%%%%%%%%%%%%%
\contributors

\chapter{Contributors}

\begin{thecontriblist}
ToDo: Use an index instead.
\end{thecontriblist}


\chapter{Authors}

ToDo: Use an index instead.
  %%%%%%%%%%%%%%%%%%%%%%%acronym.tex%%%%%%%%%%%%%%%%%%%%%%%%%%%%%%%%%%%%%%%%%
% sample list of acronyms
%
% Use this file as a template for your own input.
%
%%%%%%%%%%%%%%%%%%%%%%%% Springer %%%%%%%%%%%%%%%%%%%%%%%%%%

\chapter{Acronyms}

Use the template \emph{acronym.tex} together with the Springer document class SVMono (monograph-type books) or SVMult (edited books) to style your list(s) of abbreviations or symbols in the Springer layout.

Lists of abbreviations\index{acronyms, list of}, symbols\index{symbols, list of} and the like are easily formatted with the help of the Springer-enhanced \verb|description| environment.

\begin{description}[CABR]
\item[ABC]{Spelled-out abbreviation and definition}
\item[BABI]{Spelled-out abbreviation and definition}
\item[CABR]{Spelled-out abbreviation and definition}
\end{description}
  %%%%%%%%%%%%%%%%%%%%%%%acronym.tex%%%%%%%%%%%%%%%%%%%%%%%%%%%%%%%%%%%%%%%%%
% sample list of acronyms
%
% Use this file as a template for your own input.
%
%%%%%%%%%%%%%%%%%%%%%%%% Springer %%%%%%%%%%%%%%%%%%%%%%%%%%

\Extrachap{Glossary}


Use the template \emph{glossary.tex} together with the Springer document class SVMono (monograph-type books) or SVMult (edited books) to style your glossary\index{glossary} in the Springer layout.


\runinhead{glossary term} Write here the description of the glossary term. Write here the description of the glossary term. Write here the description of the glossary term.

\runinhead{glossary term} Write here the description of the glossary term. Write here the description of the glossary term. Write here the description of the glossary term.

\runinhead{glossary term} Write here the description of the glossary term. Write here the description of the glossary term. Write here the description of the glossary term.

\runinhead{glossary term} Write here the description of the glossary term. Write here the description of the glossary term. Write here the description of the glossary term.

\runinhead{glossary term} Write here the description of the glossary term. Write here the description of the glossary term. Write here the description of the glossary term.

  \newpage

  \begin{scriptsize}
  \printindex{logics}{Proof Systems Grouped by Logics}
  \printindex{calculusTypes}{Proof Systems Grouped by Type}
  \printindex{authors}{Entry Authors}
  \printindex{calculusAuthors}{Proof Systems' Authors}
  \end{scriptsize}

\end{document}

